\documentclass[
   paper=a4,
   twoside=false,
   parskip=half,
   listof=entryprefix,
   listof=totoc,
   index=totoc,
   bibliography=totoc,
   headsepline,
]{scrbook}

\usepackage{silence}
\WarningFilter{biblatex}{File 'ngerman-iso.lbx'}
\WarningFilter{biblatex}{'\mainlang'}
\WarningFilter{biblatex}{Bibliography string 'online' untranslated}
\WarningFilter{hyperref}{Token not allowed in a PDF string}

%%%%%%%%%%%%%%%%%%%%%%%%%%%%%%%%%%%%%%%%%%%%%%%%%%%%%%%%%%%%%%%%%%%%%%%%%%%%%%
% Fonts Fonts Fonts
%%%%%%%%%%%%%%%%%%%%%%%%%%%%%%%%%%%%%%%%%%%%%%%%%%%%%%%%%%%%%%%%%%%%%%%%%%%%%%
\usepackage[utf8]{inputenc}
\usepackage[ngerman]{babel}
\usepackage[T1]{fontenc}
\usepackage{scrhack}
\usepackage{pdfpages,graphicx,subcaption,lastpage,xspace}
\graphicspath{ {./images} }
\usepackage{float,xcolor,colortbl,csquotes,microtype,etoolbox}
\MakeOuterQuote{"}
\usepackage[automark,markcase=ignoreuppercase,autooneside=false]{scrlayer-scrpage}
\usepackage[official]{eurosym}
\usepackage[breaklinks,colorlinks,linkcolor=black,citecolor=black,filecolor=black,urlcolor=black]{hyperref}

%Requiers Incscape to be installed%
\usepackage[inkscapeformat=png,inkscapedpi=600]{svg}
% Used for Tables spanning across Pages %
\usepackage{longtable}

%%%%%%%%%%%%%%%%%%%%%%%%%%%%%%%%%%%%%%%%%%%%%%%%%%%%%%%%%%%%%%%%%%%%%%%%%%%%%%
% Listings Paket
%%%%%%%%%%%%%%%%%%%%%%%%%%%%%%%%%%%%%%%%%%%%%%%%%%%%%%%%%%%%%%%%%%%%%%%%%%%%%%
\usepackage{listings,caption,pmboxdraw}
\definecolor{codebg}{rgb}{0.95,0.95,0.95}
\definecolor{lightgray}{rgb}{.9,.9,.9}
\definecolor{darkgray}{rgb}{.4,.4,.4}
\definecolor{purple}{rgb}{0.65, 0.12, 0.82}

\lstdefinelanguage{JavaScript}{
  keywords={break, case, catch, continue, debugger, default, delete, do, else, false, finally, for, function, if, in, instanceof, new, null, return, switch, this, throw, true, try, typeof, var, void, while, with, const},
  morecomment=[l]{//},
  morecomment=[s]{/*}{*/},
  morestring=[b]',
  morestring=[b]",
  ndkeywords={class, export, boolean, throw, implements, import, this},
  keywordstyle=\color{blue}\bfseries,
  ndkeywordstyle=\color{darkgray}\bfseries,
  identifierstyle=\color{black},
  commentstyle=\color{purple}\ttfamily,
  stringstyle=\color{red}\ttfamily,
  sensitive=true,
}
\lstset{
   basicstyle =\ttfamily\color{black}\small,
   keywordstyle =,
   commentstyle =\color{teal},
   stringstyle =\itshape,
   tabsize=2,
   breaklines=true,
   captionpos=b,
   breakatwhitespace,
   backgroundcolor={\color{codebg}},
   basewidth=0.5em,
   numbers=left,
   numberstyle=\tiny,
   numbersep=-8pt,
   language=JavaScript,
}

%%%%%%%%%%%%%%%%%%%%%%%%%%%%%%%%%%%%%%%%%%%%%%%%%%%%%%%%%%%%%%%%%%%%%%%%%%%%%%
% Bibliography
%%%%%%%%%%%%%%%%%%%%%%%%%%%%%%%%%%%%%%%%%%%%%%%%%%%%%%%%%%%%%%%%%%%%%%%%%%%%%%
\usepackage[
   backend=biber,
   urldate=long,
   style=iso-authoryear,
   useauthor=true,
   mincitenames=1,
   maxcitenames=3,
   maxbibnames=99,
]{biblatex}
\addbibresource{./bib/online.bib}
\addbibresource{./bib/book.bib}
\DeclareNameAlias{default}{family-given/given-family}

%%%%%%%%%%%%%%%%%%%%%%%%%%%%%%%%%%%%%%%%%%%%%%%%%%%%%%%%%%%%%%%%%%%%%%%%%%%%%%
% Fussnoten
%%%%%%%%%%%%%%%%%%%%%%%%%%%%%%%%%%%%%%%%%%%%%%%%%%%%%%%%%%%%%%%%%%%%%%%%%%%%%%
\deffootnote{1.5em}{1em}{\makebox[1.5em][l]{\thefootnotemark}}
\addtolength{\skip\footins}{\baselineskip}
\setlength{\dimen\footins}{10\baselineskip}
\interfootnotelinepenalty=10000  % Verhindert das Fortsetzen von Fussnoten

%%%%%%%%%%%%%%%%%%%%%%%%%%%%%%%%%%%%%%%%%%%%%%%%%%%%%%%%%%%%%%%%%%%%%%%%%%%%%%
% Commands
%%%%%%%%%%%%%%%%%%%%%%%%%%%%%%%%%%%%%%%%%%%%%%%%%%%%%%%%%%%%%%%%%%%%%%%%%%%%%%
\newcommand{\workDatum}{\today\xspace}
\newcommand{\workDateTime}{\today{} - \thistime\ Uhr}
\newcommand{\workFirma}{AStA\xspace}
\newcommand{\workTitel}{AStA - Prozessdigitalisierung}
\newcommand{\workTyp}{Projektarbeit Softwaretechnik und Medieninformatik\xspace}

\newcommand{\www}[1]{\href{http://#1}{#1}}
\newcommand{\wwwhttp}[1]{\href{#1}{#1}}
\newcommand{\wwwlink}[1]{\footnote{\www{#1}}}

\newcommand{\zB}{\mbox{z.\,B.}\xspace}
\newcommand{\ua}{\mbox{u.\,a.}\xspace}
\newcommand{\dah}{\mbox{d.\,h.}\xspace}
\newcommand{\uAe}{\mbox{u.\,a.}\xspace}

\newcommand{\refp}[1]{Seite~\pageref{#1}\xspace}
\newcommand{\refk}[1]{Kapitel~\ref{#1}\xspace}
\newcommand{\refa}[1]{Abbildung~\ref{#1}\xspace}
\newcommand{\reft}[1]{Tabelle~\ref{#1}\xspace}
\newcommand{\reflst}[1]{Listing~\ref{#1}\xspace}

\newcommand{\engl}[1]{(engl: \textit{#1})\xspace}
\newcommand{\dt}[1]{(dt: \textit{#1})\xspace}

%%%%%%%%%%%%%%%%%%%%%%%%%%%%%%%%%%%%%%%%%%%%%%%%%%%%%%%%%%%%%%%%%%%%%%%%%%%%%%
% Kopf und Fusszeilen
%%%%%%%%%%%%%%%%%%%%%%%%%%%%%%%%%%%%%%%%%%%%%%%%%%%%%%%%%%%%%%%%%%%%%%%%%%%%%%
\usepackage{scrtime}
\pagestyle{scrheadings}
\clearpairofpagestyles
\ihead[]{\leftmark}
\ohead[]{\rightmark}
\counterwithout{footnote}{chapter}
\ifoot[\workDateTime]{\workDateTime}
\ofoot[\pagemark]{\pagemark}

%%%%%%%%%%%%%%%%%%%%%%%%%%%%%%%%%%%%%%%%%%%%%%%%%%%%%%%%%%%%%%%%%%%%%%%%%%%%%%
% Aufzählungen
%%%%%%%%%%%%%%%%%%%%%%%%%%%%%%%%%%%%%%%%%%%%%%%%%%%%%%%%%%%%%%%%%%%%%%%%%%%%%%
\renewcommand{\labelenumi}{\arabic{enumi}}
\renewcommand{\labelenumii}{\arabic{enumi}.\arabic{enumii}}
\renewcommand{\labelenumiii}{\arabic{enumi}.\arabic{enumii}.\arabic{enumiii}}
\renewcommand{\labelenumiv}{\arabic{enumi}.\arabic{enumii}.\arabic{enumiii}.\arabic{enumiv}}

%%%%%%%%%%%%%%%%%%%%%%%%%%%%%%%%%%%%%%%%%%%%%%%%%%%%%%%%%%%%%%%%%%%%%%%%%%%%%%
% Acronyms
%%%%%%%%%%%%%%%%%%%%%%%%%%%%%%%%%%%%%%%%%%%%%%%%%%%%%%%%%%%%%%%%%%%%%%%%%%%%%%
% https://ctan.math.washington.edu/tex-archive/macros/latex/contrib/acro/acro-manual.pdf
\usepackage{acro,supertabular,array}

\acsetup{
   make-links=true,
   list/template=supertabular,
   list/heading=chapter*,
   list/sort=true,
   list/display=used,
   list/name=Abkürzungsverzeichnis,
}

\DeclareAcronym{ex}{short=ex,long=example}
\DeclareAcronym{JSON}{short=JSON,long=JavaScript Object Natation}
\DeclareAcronym{JWT}{short=JWT,long=JSON Web Token}
\DeclareAcronym{REST}{short=REST,long=Representational State Transfer}
\DeclareAcronym{API}{short=API,long=Application Programming Interface}
\DeclareAcronym{PDF}{short=PDF,long=Portable Document Format}
\DeclareAcronym{UI}{short=UI,long=User Interface}
\DeclareAcronym{GUI}{short=GUI,long=Graphical user interface}
\DeclareAcronym{ECTS}{short=ECTS,long=European Credit Transfer System}
\DeclareAcronym{AStA}{short=AStA,long=Allgemeiner Studierenden Ausschuss}
\DeclareAcronym{MB}{short=MB,long=Mega Byte}
\DeclareAcronym{SSPL}{short=SSPL,long=Server Side Public License}
\DeclareAcronym{SQL}{short=SQL,long=Structured Query Language}
\DeclareAcronym{CD}{short=CD,long=Continuous Deployment}
\DeclareAcronym{SSL}{short=SSL,long=Secure Sockets Layer}


\usepackage{todonotes}

%%%%%%%%%%%%%%%%%%%%%%%%%%%%%%%%%%%%%%%%%%%%%%%%%%%%%%%%%%%%%%%%%%%%%%%%%%%%%%
% Glossar
%%%%%%%%%%%%%%%%%%%%%%%%%%%%%%%%%%%%%%%%%%%%%%%%%%%%%%%%%%%%%%%%%%%%%%%%%%%%%%
\usepackage[nonumberlist,toc]{glossaries}
\usepackage{glossary-super}
\setglossarystyle{super}
\makenoidxglossaries
\renewcommand*{\glstextformat}{\textbf}
\renewcommand*{\glsnamefont}{\textbf}
\setlength{\glsdescwidth}{0.6\linewidth}

\newglossaryentry{Spike}
{
   name=Spike,
   description={Ein Spike erlaubt es Teammitgliedern sich während eines Sprints mit einem Thema
   zu beschäftigen, um mehr Kenntnis darüber zu erhalten und zukünftige Risiken zu minieren}
}
\newglossaryentry{TypeScript}
{
   name=TypeScript,
   description={TypeScript ist JavaScript mit Types}
}
\newglossaryentry{Benutzer-Foederation}
{
   name=Benutzer-Föderation,
   description={Die Föderierte Identität, auch bekannt als Benutzer-Föderation,
   bezeichnet die verwendung der Identität eines Benutzers über mehrere Systeme.}
}
\newglossaryentry{enum}
{
   name=enum,
   description={Datentyp welcher einen Wert aus einer vordefinirten Menge anehmen kann.}
}

\newglossaryentry{yaml}
{
    name=yaml,
    description={Yaml ist ein textbasiertes Dateiformat zur Datenserialisierung, welches 
    außerdem für Konfigurationsdateien verbreitet ist. Die Synthax basiert auf Einrückungen}
}


%%%%%%%%%%%%%%%%%%%%%%%%%%%%%%%%%%%%%%%%%%%%%%%%%%%%%%%%%%%%%%%%%%%%%%%%%%%%%%
% Dokument
%%%%%%%%%%%%%%%%%%%%%%%%%%%%%%%%%%%%%%%%%%%%%%%%%%%%%%%%%%%%%%%%%%%%%%%%%%%%%%
\begin{document}

   \newcommand{\HRule}[2]{\noindent\rule[#1]{\linewidth}{#2}}
\newcommand{\vlinespace}[1]{\vspace*{#1\baselineskip}}
\newcommand{\titleemph}[1]{\textbf{#1}}
\begin{titlepage}
    \sffamily
    \includegraphics[width=5cm]{AstaRgb}
    \hfill
    \includegraphics[width=5cm]{Hochschule_Esslingen}
    \HRule{13pt}{1pt}
    \centering
    \vlinespace{3}\\
    \workTyp\\
    \begin{Large}
        \textbf{\workTitel}\\
    \end{Large}
    \vlinespace{4}
    am \workDatum\\
    \vlinespace{1}
    Bearbeitungszeitraum:\\
    15.03.2024 -- 25.06.2024\\
    \vlinespace{4}
    vorgelegt von\\
    \begin{Large}
        \textbf{Max Trautwein}\\
        \textbf{Ayhan Yasar}\\
        \textbf{Tobias Bührle}\\
    \end{Large}
    \vlinespace{1}
    \vfill
    \raggedright{}
    \HRule{13pt}{1pt} \\
    \titleemph{Prüfer:} Prof. Dr. Jörg Nitzsche\\
    \titleemph{Betreuer:} Andreas Heinrich\\
\end{titlepage}


   \tableofcontents
   \newpage
   \printacronyms[heading=addchap]
   \printnoidxglossary

   \chapter{Einführung und Ziele}\label{ch:einfuhrung-und-ziele}

\section{Aufgabenstellung}\label{sec:aufgabenstellung}
Die Aufgabe besteht in der Entwicklung einer Web Applikation, welche ein nahezu vollständig digitales
Ausfüllen von Anträgen ermöglicht. Dies soll das bisherige, analoge Ausfüllen von Anträgen ersetzen und so zu einer vereinfachten und beschleunigteren Verwaltung beitragen. Dies beinhaltet das Entwerfen, Programmieren und Testen der 
Software, sowie eine ausführliche Planung und Dokumentation des gesammten Projekts.

\section{Qualitätsziele}\label{sec:qualitatsziele}
Die von uns entwickelte Software soll möglichst alle Merkmale hochqualitativer Software erfüllen, wie 
sie in der ISO/IEC 25010 festgehalten sind. Besonderes Augenmerk legen wir auf die Usability, 
Vollständigkeit, sowie leichte Portierbarkeit und Verlässlichkeit. Dieser Fokus erwächst aus der 
Aufgabenstellung, da eine in der Verwaltung eingesetzte Software vor allem verlässlich und vollständig 
sein muss. Da sie das Ausfüllen von Anträgen vereinfachen soll, ist zudem eine gute Usability und
Portierbarkeit nötig, um Anträge schnell, einfach und von überall aus ausfüllen zu können.
\section{Stakeholder}\label{sec:stakeholder}
   \chapter{Randbedingungen}\label{ch:randbedingungen}


\section{Technische Randbedingungen}\label{sec:technische-randbedingungen}

Die Applikation soll auf bewährte Technologien setzen, um die Wartbarkeit in der Zukunft zu erhöhen und Fehler zu vermeiden.\\

Um ein Deployment auf verschiedenen System einfach zu unterstützen, soll die Applikation dockerized werden
\dah alle Komponenten sollten in Docker Containern lauffähig sein.


\section{Organisatorische Randbedingungen}\label{sec:organisatorische-randbedingungen}

Das Projekt \workTitel~wird als Teil des Moduls \workTyp durchgeführt.
Dies führt zu bestimmten organisatorischen Bedingungen.\\
Das Projekt wurde nach der Bekanntgabe der Einteilungen am 15.03.2024 begonnen und muss am 25.06.2024 abgeschlossen sein.
Da dem Modul mit 10 \ac{ECTS} Punkten bewertet wird, liegt der Arbeitsaufwand pro Teammitglied bei \(30\,Stunden * 10 \ac{ECTS} = 300\,Stunden\) also insgesamt 900 Stunden.
Davon abzuziehen sind noch die zwei verpflichtenden Seminare "Teambildung und Konfliktlösung" sowie "Präsentation und Disputation".
Dementsprechend sind auch die Requirements für das Projekt einzugrenzen.\\

Das Modul fordert einige vordefinierte Abgaben ein, welche teils der agilen Vorgehensweise, welche angestrebt wird, widersprechen.
Diese führt dazu, dass nicht immer alle Teile der angeilen Methodik befolgt werden können.\\

Die Tatsache, dass es sich um ein Projekt handelt, das während des Studiums durchgeführt wird, führt gelegentlich zu Komplikationen.
Neben dem Projekt sollten die Teammitglieder natürlich auch nicht ihre anderen Module vernachlässigen, dies schränkt unter anderem die Verfügbarkeit ein.
Ferner belegen nicht immer alle Teammitglieder dieselben Module, was die Terminfindung / Verfügbarkeit komplexer macht.
   \chapter{Funktionsumfang}\label{ch:funktionsumfang}


   \chapter{Userstories}\label{ch:userstories}
Im Folgenden werden die Anwendungsfälle dieser Applikation anhand von Userstories präzisiert.
\section{Dockerized}
Als Kunde möchte ich, dass die Applikation in Dockerr Containern bereitgestellt werden kann, damit 
sichergestellt ist, dass die Anwendung in verschiedenen Umgebungen konsistent funktioniert.

\section{Technologie}
Als Kunde möchte ich, dass die Applikation auf bewährten Technologien basiert, damit eine hohe
Zuverlässigkeit und Stabilität, sowie eine gute Wartbarkeit gewährleistet ist.

\section{Datenbank}
Als Nutzer möchte ich, dass ich entscheiden kann, ob meine Daten in einer Datenbank der Applikation oder an einem anderen Ort gespeichert werden.

\section{Bedienung/Layout}
Als Nutzer möchte ich, dass die Bedienung der Applikation keine Fachkenntnisse vorraussetzt, damit 
ich mich nicht lange einarbeiten muss.

\section{Persistenz}
Als Nutzer möchte ich, dass der Ausfüll Fortschritt eines Antrags, geräteübergreifend gespeichert wird
um es mir so zu ermöglichen, das Ausfüllen an einem anderen Gerät fortzusetzen.




\section{Login}
Als Nutzer möchte ich mich in der Applikation mit einem Nutzernamen bzw. E-mail und einem Passwort
einloggen können, damit persöhnliche Daten gut geschützt sind.
\section{Aantrrag finden}
Als Nutzer möchte ich Zugriff auf eine Funktionalität haben, die mir hilft den passenden 
Antrag zu finden, um so Zeit zu spären und Verwechslungen zu vermeiden.
\section{Antragsbeschreibung}
Als Nutzer möchte ich, dass die Anträge neben einem Namen, auch über eine Beschreibung verfügen, damit
ich als Nutzer den Zweck des Antrags besser verstehen kann.
\section{Filter}
Als Nutzer möchte ich die Möglichkeit haben, Anträge nach bestimmten Kriterien zu filtern, 
damit ich die relevanten Anträge schneller finden kann.
\section{Antrags Kategorien}
Als Kunde möchte ich in der Lage sein, Anträge bestimmten Kategorien wie Beantragungen und 
Abrechnungen zuteilen zu können, um so Anträge besser organisiern und schneller auf benötigte 
Informationen zugreifen zu können.
\section{Favoriten}
Als Nutzer möchte ich in der Lage sein, Anträge als Favoriten zu markieren um sie so schneller zu 
finden und schneller auf sie zugreifen zu können.
\section{Fertige Anträge}
Als Nutzer möchte ich in der Lage sein, von mir bereits fertig ausgefüllte Anträge zu überarbeiten, um 
so Zeit zu sparen.
   \chapter{Aufwandsschätzung}\label{ch:aufwandsschatzung}

   \chapter{Projektmanagement}\label{ch:projektmanagement}

\section{Methode}\label{sec:methode}

In diesem Projekt wird mit einer agilen Arbeitsweise gearbeitet.\\
Mit einer agilen Arbeitsweise ist es möglich Fehler bereits in der Ausführung zu erkennen und eine schnelle Lösung zu finden.
Ein Beispiel ist Scrum. Scrum ist iterativ und agil. Das bedeutet es gibt wiederholende Zyklen, bei der das klassische Waterfall-Prinzip geeint wird.
Ein Sprint Zyklus unter Scrum besteht aus: Sprint Planung, Sprint, Sprint Review und der Retrospektive.

\section{Lizenz}\label{sec:lizenz}

Der Code wird unter der GNU General Public License v3.0 veröffentlicht.\\
Die Lizenz ist eine Open-Source Lizenz, die die Innovation und Zusammenarbeit fördert.\\
Durch die GNU GPL v3.0 bleibt der Code frei zugänglich und schützt gleichzeitig die Rechte der Nutzer und Entwickler.  


\section{Github Flow}\label{sec:github-flow}

Das Projekt wird mit einem GitHub-Flow realisiert.\\
In einem GitHub-Flow wird für jede Teilaufgabe eine eigene Branch erstellt, da diese eine Einsicht der Problemlösung erfordert.
Dadurch wird verhindert, dass versehentlich fehlerhafter Code im Master landet. Nach erfolgreicher Einsicht des Codes kann dieser auf den Master gemerged werden \dah{} Die Lösung ist valide und kann in den vorerst finalen Code integriert werden.



\section{Definition of Done}\label{sec:dod}

Der Begriff Definition of Done kommt aus einer agilen Arbeitsweise wie Scrum.\\
Die Definition of Done beschreibt Kriterien, die abgedeckt sein müssen, damit das Projekt als abgeschlossen gilt.
Die Kriterien dieses Projekts bestehen zusammengefasst aus einem voll funktionsfähigen Code, sinnvoll kommentierten Code, Erstellung von Tests, erfolgreichen Tests und einer vollständigen Dokumentation des Codes.\\
Der Code muss, die in den Requierements beschriebenen Funktionen lauffähig beinhalten.\\
Komplexer Code muss sinnvoll und nachvollziehbar kommentiert sein.\\
Für jede Funktion muss es einen Test geben. Ein Test kann auch mehrere Funktionen abdecken. Alle Tests müssen bestanden sein.
Das letzte Kriterium wäre eine vollständige Dokumentation des Codes, welche die Vorgehensweise für die Lösung begründet.
   \chapter{Architektur}\label{ch:architektur}
Wie in \refa{fig:HighLevelArch} zu sehen ist, wird hier eine Web-App Architektur angewendet,
welche um einen Authentifizierungsdienst erweitert wird.\\
Das Frontend übernimmt die Aufgabe der Darstellung und Interaktion mit dem Benutzer.
Dabei interagiert dieses mit dem Backend über eine \ac{REST} Schnittstelle, sowie dem Authentifizierungsdienst.\\
Das Backend stellt dem Frontend Daten bereit und interagiert mit einem Datenspeicher.\\
Der Authentifizierungsdienst ist ein externer Dienst, um die Nutzeraccounts zu managen.
Über diesen erhält das Frontend einen \ac{JWT}, welcher den Nutzer gegenüber dem Backend authentifiziert.
Dazu stellt das Backend eine entsprechende Anfrage an den Authentifizierungsdienst mit dem \ac{JWT}.

\begin{figure}[h]
    \centering
    \includesvg[width=8cm]{images/HighLevelArch}
    \caption{Architektur Überblick}\label{fig:HighLevelArch}
\end{figure}

\section{Frontend}\label{sec:frontend}

Die innere Struktur des Frontends ist stark vom verwendeten Framework abhängig.
Daher ist an dieser Stelle keine näher Beschreibung möglich.

\section{Backend}\label{sec:fokus:-backend}

\refa{fig:BackendArch} zeigt den strukturellen Aufbau des Backends näher, dabei kann die Grafik in vier Zonen unterteilt werden:
\begin{itemize}
    \item Links (grün): Kommunikation mit Webdiensten sowie dem eigenen Frontend.
    \item Unten (gelb): Interface mit Datenspeicher.
    \item Rechts (blau): Zugriff auf Konfigurationsdateien.
    \item Zentral (lila): Verarbeitung aller Teilkomponenten
\end{itemize}

\begin{figure}[h]
    \includesvg[width=17cm]{images/BackendArch}
    \caption{Backend Architektur}\label{fig:BackendArch}
\end{figure}

Die Kommunikation mit den Webdiensten lässt sich in zwei Teilmodule unterteilen:
\begin{itemize}
    \item \ac{REST} \ac{API}
    \item Authentifizierungsprüfung
\end{itemize}
Dabei stellt die \ac{REST} \ac{API} eine klare Schnittstelle bereit und greift auf die Authentifizierungsprüfung zu,
um die Berechtigungen des Nutzers zu prüfen.

Das Interface mit den Datenspeichern entkoppelt die Anwendung sowie deren Daten von dem verwendeten Speicherdienst.
Dabei wird zwischen "Core Daten", welche für die Funktionalität zwingend benötigt werden
und "User Daten", welche lediglich die Bedingung sowie Handhabung des Nutzers verbessern, unterschieden.\\
"Core Daten" werden dabei immer auf einer eigenen Datenbank gespeichert.
Hierbei ist die Datenbanktechnologie selbst durch das Datenbankinterface entkoppelt.\\
"User Daten" hingegen könnten an anderer Stelle gespeichert werden.
Diese Möglichkeit wird hier zwar eindeutig berücksichtigt, jedoch wird ein dementsprechendes Interface nicht explizit entwickelt.

Prozesskonfigurationen beinhalten all jene, welche den Fluss und Inhalt eines Antrags sowie dessen Export beschreiben.
Diese weisen eine hohe Komplexität auf und können nicht unbedingt ohne weiteres bearbeitet werden.
Regelkonfigurationen hingegen sind einfach gestaltet und sollten nach einer Erklärung aus der Anleitung leicht anpassbar sein.
Diese enthalten beispielsweise Parameter zur Berechnung von Reisekosten.

Zentral steht die Verarbeitung der außenstehenden Module.
Dabei werden die Informationen der Konfigurationsdateien ausgewertet und kombiniert.
   \chapter{Architekturschichten}\label{ch:arrchitekturschichten}
Im Folgenden wird die Architektur der Applikation genauer beschrieben. Dabei liegt das Augenmerk 
auf der Struktur, dem Verhalten und der Verteilung der einzelnen Komponenten in der Architektur.

%Einzelne Unterkategorien werden ergänzt, wenn die Implementierung der Applikation voran schreitet.

\section{Strukturschicht}\label{sec:strukturschicht}
Die genaue Struktur unserer Applikation sieht wie folgt aus:

\section{Verhaltensschicht}\label{sec:verhaltensschicht}
Im Folgenden wird das Verhalten der Applikation bei verschiedenen Usecases beschrieben.

\section{Verteilungsschicht}\label{sec:verteilungsschicht}
Die feingranulare Verteilung von Funktionalitäten in unserer Applikation sieht wie folgt aus:
   \chapter{Technologien & Frameworks}\label{ch:technologien-&-frameworks}

In diesem Kapitel sind die verwendeten Technologien und Frameworks benannt, erklärt
sowie begründet, weshalb diese verwendet werden.

\section{Angular}\label{sec:angular}

Im Frontend wir Angular für die komponentenbasierte Darstellung verwendet.
Da in Angular die einzelnen Komponenten direkt in solche unterteilt sind,
eignet es sich für den angestrebten modularen Aufbau besonders.

Angular ist eine auf \gls{TypeScript} basierende Entwicklungsplattform, die folgende Funktionalitäten umfasst:
\begin{itemize}
    \item Komponenten basiertes Framework
    \item Eine Vielzahl von stark integrierten Bibliotheken für Routing, Formularmanagement, Client-Server Kommunikation
    \item Entwicklertools zum Entwickeln, Testen und Updaten des Codes
\end{itemize}
\cite{about-angular}

Angular ist MIT lizenziert.
Für mehr Details siehe \refk{sec:mit}.

\subsection{Direkte Abhänigkeiten}

\paragraph{angular-oauth2-oidc}

\paragraph{bootstrap}

\paragraph{cors}

\paragraph{ngx-bootstrap}

\paragraph{rxjs}

\paragraph{tslib}

\paragraph{zone.js}

\paragraph{jasmine-core}

\paragraph{karma}

\paragraph{karma-chrome-launcher}

\paragraph{karma-coverage}

\paragraph{karma-jasmine}

\paragraph{karma-jasmine-html-reporter}

\paragraph{typescript}


\subsection{Alle Abhänigkeiten}
Eine vollständige Liste aller Frontend Abhängigkeiten findet sich im Anhang unter \refk{sec:frontend---abhangigkeitsbaum}.
Diese sind wie folgt lizenziert:
% TODO should we add descriptions to the Ones we Don't have currently?
\begin{itemize}
    \item 672x MIT
    \item 103x ISC
    \item 27x Apache-2.0
    \item 23x BSD-2-Clause
    \item 16x BSD-3-Clause
    \item 2x Unlicense
    \item 2x 0BSD
    \item 1x BlueOak-1.0.0
    \item 1x Python-2.0
    \item 1x CC-BY-4.0
    \item 1x CC-BY-3.0
    \item 1x BSD-3-Clause / GPL-2.0
    \item 1x CC0-1.0:
    \item 1x MIT / CC0-1.0
\end{itemize}



\section{Spring Boot}\label{sec:spring-boot}

Im Backend kommt die bewährte Technologie Spring Boot zu Einsatz.

Spring ermöglicht es Java schneller, leichter und sicherer zu programmieren.
Aufgrund seiner Geschwindigkeit, Einfachheit und Produktivität gilt Spring als am meisten geschätztes Java-Framework weltweit.
\cite{about-springboot}

Spring Boot ist Apache-2.0 Lizenziert.
Für mehr Details siehe \refk{sec:apache-2.0}.

\subsection{Direkte Abhänigkeiten}

\paragraph{Spring Boot Starters}
\begin{itemize}
    \item Thymeleaf
    \item Data MongoDB
    \item Web
    \item Web Services
    \item Test
    \item OAuth2 Resource Server
\end{itemize}
Sind Teil von Spring Boot und demnach ebenfalls Apache-2.0 Lizenziert.
Für mehr Details siehe \refk{sec:apache-2.0}.

\paragraph{lombok}
Erspart boilerplate Code und ist MIT lizenziert.
Für mehr Details siehe \refk{sec:mit}.\\
% TODO Should we add a segment on BSD-3-Clause
Verwendet komponenten welch unter MIT und der BSD-3-Clause lizenziert sind.

\paragraph{commons-lang3}
Java Utility welche Apache-2.0 Lizenziert ist.
Für mehr Details siehe \refk{sec:apache-2.0}.

\paragraph{OpenAPI}
\begin{itemize}
    \item JsonNullable Jackson Module
    \item swagger
    \item OpenAPI Generator
\end{itemize}
Apache License, Version 2.0
https://github.com/OpenAPITools/jackson-databind-nullable?tab=Apache-2.0-1-ov-file

\paragraph{jakarta.validation-api}
Bean Validation Modell.
Ist Apache-2.0 Lizenziert.
Für mehr Details siehe \refk{sec:apache-2.0}.

\paragraph{jakarta.annotation-api}
Jakarta Annotations API.
% TODO Should we add a segment on EPL 2.0?
% TODO Should we add a segment on GNU GPL V2 with the GNU Classpath Exception
Ist primär unter der EPL 2.0 (Eclipse Public License 2.0)
Sowie sekundär unter der GNU General Public License, version 2 with the GNU Classpath Exception
https://mvnrepository.com/artifact/jakarta.annotation/jakarta.annotation-api

\paragraph{de.flapdoodle.embed.mongo}
Eine Embedded MongoDB für Test.
Ist Apache-2.0 Lizenziert.
Für mehr Details siehe \refk{sec:apache-2.0}.

\subsection{Alle Abhänigkeiten}
Ein vollständiger Abhängigkeitsbaum ist im Anhang unter \refk{sec:backend-abhangigkeitsbaum} aufgeführt.

\section{Keycloak}\label{sec:keycloak}

Keycloak dient als einfache Möglichkeit die Applikation abzusichern, ohne selbst Passwörter zu managen.
Zusätzlich wird \gls{Benutzer-Foederation}, starke Authentifizierung sowie eine feine Autorisierung bereitgestellt.
\cite{about-keycloak}

Keycloak ist Apache-2.0 Lizenziert.
Für mehr Details siehe \refk{sec:apache-2.0}.

\section{Postgres}\label{sec:postgres}

Postgres dient in diesem Projekt als Datenspeicher für Keycloak.

PostgreSQL ist ein leistungsfähiges, objektrelationales Open-Source-Datenbanksystem.
Es verwendet eine erweiterte \ac{SQL} Syntax und verfügt über zahlreiche Features,
die es ermöglichen, selbst die komplexesten Daten-Workloads sicher zu speichern und zu skalieren.
\cite{about-postgres}

Postgres steht unter der The PostgreSQL Licence.
Für mehr Details siehe \refk{sec:the-postgresql-licence}.


\section{MongoDB}\label{sec:mongodb}

MongoDB wird zum Speichern der Anträge sowie der Verknüpfung zu den Keycloak Accounts genutzt.
Im Gegensatz zu den klassischen, relationalen Datenbanken zählt MongoDB als Dokumenten basierte NoSQL Datenbank.
Dabei werden Information nicht streng auf einzelne Tabellen verteilt, sondern in Form von \ac{JSON} Objekten in Dokumenten abgelegt.

MongoDB steht unter der \acl{SSPL} zur Verfügung.
Für mehr Details siehe \refk{sec:server-side-public-license}.

\section{Docker}\label{sec:docker}

Docker bietet eine unkomplizierte Option, Container zu konfigurieren und zu teilen.
Dabei kann auf Datenbanken mit Basis Images zugegriffen werden, welche bei Bedarf
einfach in einem Dockerfile angepasst werden können.

Docker selbst ist nicht Open Source, jedoch für die meisten kostenlos verfügbar.
Davon ausgeschlossen sind lediglich Firmen mit mehr als 250 Mitarbeitern oder mindestens 10 Millionen Dollar Einkommen.

\section{Docker Compose}\label{sec:docker-compose}

Docker Compose ist eine Erweiterung von Docker.
Diese ermöglicht es direkt mehrere Container in einer Datei zu managen.
Ein weiterer Vorteil stellt die einfache Konfiguration von Docker-Secrets dar, wodurch \ua Passwörter sicher übergeben werden können.

Docker Compose ist Apache-2.0 Lizenziert.
Für mehr Details siehe \refk{sec:apache-2.0}.


   \chapter{\ac{API}s}\label{ch:apis}
In diesem Kapitel wird detailierter auf die Verwendung von verschiedenen \ac{API}s in unserem Projekt
eingegangen.

\section{Guidelines}
In dieses Projekt orientiert sich im Bezug auf \ac{API}s an den Zalando API Guidelines\footnote{https://opensource.zalando.com/restful-api-guidelines/}. 
Diese werden, so weit sie für das Projekt förderlich und sinnvoll sind, bei der Implementierung 
berücksichtigt.
\section{Verwendete \ac{API}s}
In unserem Projekt finden vollgende \ac{API}s, mit den jeweiligen Endpunkten anwendung:

\section{Endpoints}
Folgende Endpunkte sind in der verwendeten API ansprechbar.

\subsection{GET /templates}

Dieser API Endpunkt liefert alle in der Datenbank gespeicherten Templates zurück. Diese
 werden genutzt, um die Buttons auf der Main Page dynamisch zu generieren.
%Dieser Abschnitt wird weiter ergänzt, wenn die Implementierung der Applikation voranschreitet.

\section{Schemata}
Die Schemata, die die API verwendet, sehen wie folgt aus.

\subsection{FormElement}
   \chapter{UI Design}\label{ch:ui-design}
Das UI Design wird in mehreren Schritten erstellt und im Laufe des Projekts immer weiter
verfeinert. Am Anfang dieses Prozesses steht ein Low Fidelity Prototyp der Applikation.
Dieser dient dazu das ungefähre Layout zu visualisieren und ein Gefühl für den Aufbau
des Frontend zu schaffen. Für die Erstellung dieser Prototypen wird die Prototyping- und
Designsoftware Figma verwendet.

\section{Low Fidelity}\label{Low Fidelity}
\begin{figure}[h]
  \centering
    \includegraphics[width=1.0\textwidth]{Doc/images/Antragshelfer.png}
    \caption{Antragshelfer}
\end{figure}
Das Layout des Prototypen lässt sich in drei Bereiche einteilen. Die Kopfzeile, den
Hauptinhalt der Seite sowie die Fußzeile.

In der Kopfzeile befindet sich das AStA Logo, sowie eine Suchleiste um manuell nach
Anträgen zu suchen. Auch der jeweiligen Nutzerr Account lässt sich über die Schaltfläche
am rerchten Rand erreichen. Außerdem befindet sich am linken Rand ein Burger Menü,
welches das Hinzufügen von Funktionalitäten in zukünftigen Projekten erleichtern soll.

Der Hauptinhalt der Seite stellt die Hauptfunktionalitäten unserer Applikation dar und
ist daher in der fertigen Applikation dynamisch generiert. Bei den Prototypen wird dies
durch ein statisches Layout simuliert.

Durch die Fußzeile soll der Nutzer die Möglichkeit erhalten, schnell und einfach auf von
ihm zuvor festgelegten Anträge zugreifen zu können, was eine gute User Experience
gewährleisten soll. Zu erwähnen ist, dass sowohl Kopf- als auch Fußzeile auf jeder Seite
der Applikation identisch sind. Lediglich die Hauptinhalte unterscheiden sich.

Da der Antrags Helper die Hauptfunktionalität unserer Applikation darstellt, fungiert
dieser auch als Landing Page, also der Seite, welche der Nutzer direkt nach dem Login
sieht, wie in Abb.7.1 dargestellt. Hier soll der Nutzer durch Klicken mehrerer Buttons die
ihm, von der Applikation gestelllten, Fragen beantowrten, um so zum richtigen Antrag zu
gelangen

\begin{figure}[h]
  \centering
    \includegraphics[width=1.0\textwidth]{Doc/images/Reisekostenhelper.png}
    \caption{Reisekostenhelfer}
\end{figure}


   \chapter{Datenbank}
Da in diesem Projekt eine NoSQL Datenbank verwendet wird, gibt es kein Datenbanklayout im klassischen Sinne.
Regeln für die Strukturierung von Daten wie Normalformen bei Relationalen Datenbanken gibt es bei MongoDB nicht.

Dennoch gibt es einige Punkte die zu beachten sind:
\begin{itemize}
    \item Limit von 16\ac{MB} Pro Dokument
    \item Einbetten ist bei NoSQL Datenbanken zu bevorzugen, jedoch nicht immer passend
    \item Arrays sollten nicht unendlich wachsen können
\end{itemize}

Es folgt die vorläufige Struktur für die verschiedenen Dokumente.
Der genaue Inhalt wird angefügt, sobald dieser bekannt wird.

\section{Settings}

\begin{lstlisting}[label={lst:lstlistingusers}]
    // Settings Document
    {
    id: string
    settings:
        {
            ...
        }
    }

\end{lstlisting}

\section{Autofill}
\begin{lstlisting}[label={lst:lstlistingauto}]
    // Autofill Document
    {
    Userid: ref
    pairs:
        [
            {
                key:value
            },
            {
            ...
            }
        ]
    }

\end{lstlisting}

\section{Form}\label{sec:form}
Das Form-Dokument definiert die Struktur, in welcher Formulare gespeichert werden.
Hierbei werden verschiedene \gls{enum}s verwendet, welche im Folgenden näher beschrieben werden.


'EFormElement' beschreibt die verschiedenen Feldtypen, welche dargestellt werden sollen.
\begin{lstlisting}[label={lst:EFormElement}]
    public enum EFormElement {
        text,
        address,
        iban,
        date,
        money,
        TextMultiLine,
        bool
    }
\end{lstlisting}


'ECategory' ermöglicht die Unterteilung eines Formulars in fest definierte Kategorien
\begin{lstlisting}[label={lst:ECategory}]
    public enum ECategory {
        Antrag,
        Abrechnung
    }
\end{lstlisting}

Das Form-Dokument selbst ist im Folgenden beschrieben.
Der Boolean "Template" dient hierbei zur Unterscheidung zwischen
Form-Dokumenten, welche beschrieben, wie das Formular dem Nutzer
dargestellt werden soll, sowie jenen, welche Daten enthalten.

Sollte es sich nicht um eine Vorlage zum Ausfüllen handeln, so werden nur die notwendigen Informationen erfasst.
\begin{lstlisting}[label={lst:lstlistingdoc}]
    {
        "id": string,
        "owner": string,
        "Title": string,
        "Template": bool,
        "Description": string,
        "Category": ECategory
        "form":[
            {
                "order": integer,
                "section": string,
                "items": [
                    {
                        "Description": string,
                        "type": EFormElement,
                        "id": string,
                        "value" string
                    },
                    {
                        "Description": string,
                        "type": EFormElement,
                        "id": string,
                        "value" string
                    }
                ]
            },
            {
                "order": integer,
                "section": string,
                "items": [
                    {
                        "Description": string,
                        "type": EFormElement,
                        "id": string,
                        "value" string
                    },
                    ...
                ]
            },
            {
                ...
            }
        ],
        "Attachments":[
            {
                "id": string,
                "Description": string
            }
        ]
    }
\end{lstlisting}
   \chapter{Lizenzen}\label{ch:lizenzen}

In diesem Kapitel finden sich Informationen zu den verwendeten Lizenzen in dem Projekt.

\section{MIT}\label{sec:mit}
Die MIT Lizenz ist ein kurze und offene Open Source Lizenz mit nur minimalen Bedingungen.

\paragraph{Rechte}
\begin{itemize}
    \item kommerzielle Nutzung
    \item Weitergabe
    \item Anpassung
    \item private Nutzung und Modifikation
\end{itemize}

\paragraph{Bedingungen}
\begin{itemize}
    \item Die Lizenz und Urheberrechte müssen mit verteilt werden.
\end{itemize}

\paragraph{Limitierung}
\begin{itemize}
    \item Haftungsausschluss
    \item keinerlei Garantie
\end{itemize}
Vgl. \cite{choosealicense-com}

\section{Apache-2.0}\label{sec:apache-2.0}

Die Apache-2.0 ist eine Open Source Lizenz welche Nutzer vor Patentrechten schützt,
explizit Rechte auf Warenzeichen ausschließt und fordert, dass Änderungen dokumentiert werden.

\paragraph{Rechte}
\begin{itemize}
    \item kommerzielle Nutzung
    \item Weitergabe
    \item Anpassung
    \item Patent Nutzung % Nach meinem Verständnis schützt es vor Royalty Zahlungs Anforderungen
    \item private Nutzung und Modifikation
\end{itemize}

\paragraph{Bedingungen}
\begin{itemize}
    \item Die Lizenz und Urheberrechte müssen mit verteilt werden.
    \item Änderungen müssen dokumentiert werden.
\end{itemize}

\paragraph{Limitierung}
\begin{itemize}
    \item Haftungsausschluss
    \item keinerlei Garantie
    \item explizit keine Rechte auf Warenzeichen
\end{itemize}
Vgl. \cite{choosealicense-com}

\section{The PostgreSQL Licence}\label{sec:the-postgresql-licence}
%https://opensource.org/license/postgresql
% ähnlich MIT aber nicht Exakt
Eine eigene Open Source Lizenz von PostgreSQL.
Diese ähnelt der MIT Lizenz, ist jedoch nicht gleich im Wortlaut.

\paragraph{Rechte}
\begin{itemize}
    \item kommerzielle Nutzung
    \item Weitergabe
    \item Anpassung
    \item private Nutzung und Modifikation
\end{itemize}

\paragraph{Bedingungen}
\begin{itemize}
    \item Die Lizenz und Urheberrechte müssen mit verteilt werden.
\end{itemize}

\paragraph{Limitierung}
\begin{itemize}
    \item Haftungsausschluss
    \item keinerlei Garantie
\end{itemize}


\section{\acf{SSPL}}\label{sec:server-side-public-license}
Bei dieser Lizenz handelt es sich um eine modifizierte GNU AGPLv3.
Durch diese Anpassung wird die \ac{SSPL} nicht mehr als Open Source angesehen.\cite{osi-sspl}

\paragraph{Rechte}
\begin{itemize}
    \item kommerzielle Nutzung
    \item Weitergabe
    \item Anpassung
    \item Patent Nutzung
    \item private Nutzung und Modifikation
\end{itemize}
\paragraph{Bedingungen}
\begin{itemize}
    \item Anpassungen müssen unter derselben Lizenz bereitgestellt werden
    \item Änderungen Dokumentieren
    \item Die Lizenz und Urheberrechte müssen mit verteilt werden.
    \item Quellcode muss bei Weitergabe offengelegt werden
    \item wenn MongoDB als Service angeboten wird, dann muss der gesamte Quellcode unter der
    \ac{SSPL} frei zur Verfügung gestellt werden.
    Dies umfasst auch jegliche andere Software, welche benötigt wird, sodass ein Nutzer denselben Dienst anbieten kann.
\end{itemize}

\paragraph{Limitierung}
\begin{itemize}
    \item Haftungsausschluss
    \item keinerlei Garantie
\end{itemize}
Vgl. \cite{choosealicense-com}

\section{GNU GPLv3}\label{sec:gnu-gplv3}
Die GNU GPLv3 Lizenz ist eine Weiterentwicklung der GNU GPLv2 Lizenz. In ihr wurden Schwachstellen der
GPLv2 Lizenz im Bezug auf Patentrechte behoben und eine Kompartibilität mit der weit verbreiteten
Apache-2.0 Lizenz integriert.

\paragraph{Rechte}
\begin{itemize}
    \item kommerzielle Nutzung
    \item Weitergabe
    \item Anpassung
    \item private Nutzung und Modifikation
    \item Patent Nutzung
\end{itemize}
\paragraph{Bedingungen}
\begin{itemize}
    \item Bei Veröffentlichung muss der Quellcode frei zugänglich sein.
    \item Die Lizenz und Urheberrechte müssen mit verteilt werden.
    \item Änderungen müssen dokumentiert werden.
    \item Änderungen müssen unter der selben Lizenz veröffentlicht werden.
\end{itemize}

\paragraph{Limitierung}
\begin{itemize}
    \item keinerlei Garantie
    \item eingeschränkte Haftung
\end{itemize}
Vgl. \cite{choosealicense-com}

\section{0BSD(Zero-Clause BSD)}


\paragraph{Rechte}
\begin{itemize}
    \item
\end{itemize}
\paragraph{Bedingungen}
\begin{itemize}
    \item
\end{itemize}

\paragraph{Limitierung}
\begin{itemize}
    \item
\end{itemize}

\section{BSD-3-Clause}


\paragraph{Rechte}
\begin{itemize}
    \item
\end{itemize}
\paragraph{Bedingungen}
\begin{itemize}
    \item
\end{itemize}

\paragraph{Limitierung}
\begin{itemize}
    \item
\end{itemize}

\section{EPL 2.0}


\paragraph{Rechte}
\begin{itemize}
    \item
\end{itemize}
\paragraph{Bedingungen}
\begin{itemize}
    \item
\end{itemize}

\paragraph{Limitierung}
\begin{itemize}
    \item
\end{itemize}

\section{GNU GPL V2 mit GNU Classpath Exception}\label{sec:gnu-gpl-v2-mit-gnu-classpath-exception}
Die GNU GPL V2 ist die Vorgänger version der GNU GPLv3 siehe \refk{sec:gnu-gplv3}.
Wie schon dort beschrieben hat die V2 einige Probleme, welche mit der V3 behoben wurden.

Angesehen davon wird hier noch zusätzlich die Classpath Exception verwendet.
Diese Ermöglicht die nutzung mit anderen Lizenzen wie zum beispiel der Apache-2.0,
sofern diese module unabhängig agieren.

Vgl. \cite{gnu-classPath}

\paragraph{Rechte}
\begin{itemize}
    \item kommerzielle Nutzung
    \item Weitergabe
    \item Anpassung
    \item private Nutzung und Modifikation
\end{itemize}
\paragraph{Bedingungen}
\begin{itemize}
    \item Bei Veröffentlichung muss der Quellcode frei zugänglich sein.
    \item Die Lizenz und Urheberrechte müssen mit verteilt werden.
    \item Änderungen müssen dokumentiert werden.
    \item Änderungen müssen unter der selben Lizenz veröffentlicht werden.
\end{itemize}

\paragraph{Limitierung}
\begin{itemize}
    \item keinerlei Garantie
    \item eingeschränkte Haftung
\end{itemize}
Vgl. \cite{choosealicense-com}, \cite{gnu-why-upgrade-gplv3}




   \printbibheading[title=Externe Quellen]
   \printbibliography[type=book,heading=subbibliography,title=Buch-Quellen]
   \printbibliography[type=online,heading=subbibliography,title=Online-Quellen]
   \listoffigures

   \chapter{Appendix}\label{ch:appendix}

\section{Installations- und Administrationshandbuch}\label{sec:installations--und-administrationshandbuch}
Auch verfügbar unter: https://github.com/MaxTrautwein/AStA-Digital-Forms/wiki
\includepdf[pages=-]{images/guide.pdf}

\section{OpenAPI~Spec}\label{sec:openapi-spec}
\includepdf[pages=-]{images/API\_Spec.pdf}



\section{Backend Abhängigkeitsbaum}\label{sec:backend-abhangigkeitsbaum}
de.PSWTM:DigitalForms:jar:0.0.1-SNAPSHOT\\
+-~org.springframework.boot:spring-boot-starter-data-mongodb:jar:3.2.5:compile\\
|~~+-~org.springframework.boot:spring-boot-starter:jar:3.2.5:compile\\
|~~|~~+-~org.springframework.boot:spring-boot:jar:3.2.5:compile\\
|~~|~~+-~org.springframework.boot:spring-boot-autoconfigure:jar:3.2.5:compile\\
|~~|~~+-~org.springframework.boot:spring-boot-starter-logging:jar:3.2.5:compile\\
|~~|~~|~~+-~ch.qos.logback:logback-classic:jar:1.4.14:compile\\
|~~|~~|~~|~~\textbackslash-~ch.qos.logback:logback-core:jar:1.4.14:compile\\
|~~|~~|~~+-~org.apache.logging.log4j:log4j-to-slf4j:jar:2.21.1:compile\\
|~~|~~|~~|~~\textbackslash-~org.apache.logging.log4j:log4j-api:jar:2.21.1:compile\\
|~~|~~|~~\textbackslash-~org.slf4j:jul-to-slf4j:jar:2.0.13:compile\\
|~~|~~\textbackslash-~org.yaml:snakeyaml:jar:2.2:compile\\
|~~+-~org.mongodb:mongodb-driver-sync:jar:4.11.2:compile\\
|~~|~~+-~org.mongodb:bson:jar:4.11.2:compile\\
|~~|~~\textbackslash-~org.mongodb:mongodb-driver-core:jar:4.11.2:compile\\
|~~|~~~~~\textbackslash-~org.mongodb:bson-record-codec:jar:4.11.2:runtime\\
|~~\textbackslash-~org.springframework.data:spring-data-mongodb:jar:4.2.5:compile\\
|~~~~~+-~org.springframework:spring-tx:jar:6.1.6:compile\\
|~~~~~+-~org.springframework:spring-context:jar:6.1.6:compile\\
|~~~~~+-~org.springframework:spring-beans:jar:6.1.6:compile\\
|~~~~~+-~org.springframework:spring-expression:jar:6.1.6:compile\\
|~~~~~+-~org.springframework.data:spring-data-commons:jar:3.2.5:compile\\
|~~~~~\textbackslash-~org.slf4j:slf4j-api:jar:2.0.13:compile\\
+-~org.springframework.boot:spring-boot-starter-web:jar:3.2.5:compile\\
|~~+-~org.springframework.boot:spring-boot-starter-json:jar:3.2.5:compile\\
|~~|~~+-~com.fasterxml.jackson.datatype:jackson-datatype-jdk8:jar:2.15.4:compile\\
|~~|~~+-~com.fasterxml.jackson.datatype:jackson-datatype-jsr310:jar:2.15.4:compile\\
|~~|~~\textbackslash-~com.fasterxml.jackson.module:jackson-module-parameter-names:jar:2.15.4:compile\\
|~~+-~org.springframework.boot:spring-boot-starter-tomcat:jar:3.2.5:compile\\
|~~|~~+-~org.apache.tomcat.embed:tomcat-embed-core:jar:10.1.20:compile\\
|~~|~~+-~org.apache.tomcat.embed:tomcat-embed-el:jar:10.1.20:compile\\
|~~|~~\textbackslash-~org.apache.tomcat.embed:tomcat-embed-websocket:jar:10.1.20:compile\\
|~~+-~org.springframework:spring-web:jar:6.1.6:compile\\
|~~|~~\textbackslash-~io.micrometer:micrometer-observation:jar:1.12.5:compile\\
|~~|~~~~~\textbackslash-~io.micrometer:micrometer-commons:jar:1.12.5:compile\\
|~~\textbackslash-~org.springframework:spring-webmvc:jar:6.1.6:compile\\
|~~~~~\textbackslash-~org.springframework:spring-aop:jar:6.1.6:compile\\
+-~org.springframework.boot:spring-boot-starter-web-services:jar:3.2.5:compile\\
|~~+-~com.sun.xml.messaging.saaj:saaj-impl:jar:3.0.3:compile\\
|~~|~~+-~jakarta.xml.soap:jakarta.xml.soap-api:jar:3.0.1:compile\\
|~~|~~+-~org.jvnet.staxex:stax-ex:jar:2.1.0:compile\\
|~~|~~+-~jakarta.activation:jakarta.activation-api:jar:2.1.3:compile\\
|~~|~~\textbackslash-~org.eclipse.angus:angus-activation:jar:2.0.2:runtime\\
|~~+-~jakarta.xml.ws:jakarta.xml.ws-api:jar:4.0.1:compile\\
|~~+-~org.springframework:spring-oxm:jar:6.1.6:compile\\
|~~\textbackslash-~org.springframework.ws:spring-ws-core:jar:4.0.10:compile\\
|~~~~~+-~org.springframework.ws:spring-xml:jar:4.0.10:compile\\
|~~~~~\textbackslash-~org.glassfish.jaxb:jaxb-runtime:jar:4.0.5:runtime\\
|~~~~~~~~\textbackslash-~org.glassfish.jaxb:jaxb-core:jar:4.0.5:runtime\\
|~~~~~~~~~~~+-~org.glassfish.jaxb:txw2:jar:4.0.5:runtime\\
|~~~~~~~~~~~\textbackslash-~com.sun.istack:istack-commons-runtime:jar:4.1.2:runtime\\
+-~org.springframework.boot:spring-boot-starter-test:jar:3.2.5:test\\
|~~+-~org.springframework.boot:spring-boot-test:jar:3.2.5:test\\
|~~+-~org.springframework.boot:spring-boot-test-autoconfigure:jar:3.2.5:test\\
|~~+-~com.jayway.jsonpath:json-path:jar:2.9.0:test\\
|~~+-~jakarta.xml.bind:jakarta.xml.bind-api:jar:4.0.2:compile\\
|~~+-~net.minidev:json-smart:jar:2.5.1:test\\
|~~|~~\textbackslash-~net.minidev:accessors-smart:jar:2.5.1:test\\
|~~|~~~~~\textbackslash-~org.ow2.asm:asm:jar:9.6:test\\
|~~+-~org.assertj:assertj-core:jar:3.24.2:test\\
|~~|~~\textbackslash-~net.bytebuddy:byte-buddy:jar:1.14.13:test\\
|~~+-~org.awaitility:awaitility:jar:4.2.1:test\\
|~~+-~org.hamcrest:hamcrest:jar:2.2:test\\
|~~+-~org.junit.jupiter:junit-jupiter:jar:5.10.2:test\\
|~~|~~+-~org.junit.jupiter:junit-jupiter-api:jar:5.10.2:test\\
|~~|~~|~~+-~org.opentest4j:opentest4j:jar:1.3.0:test\\
|~~|~~|~~+-~org.junit.platform:junit-platform-commons:jar:1.10.2:test\\
|~~|~~|~~\textbackslash-~org.apiguardian:apiguardian-api:jar:1.1.2:test\\
|~~|~~+-~org.junit.jupiter:junit-jupiter-params:jar:5.10.2:test\\
|~~|~~\textbackslash-~org.junit.jupiter:junit-jupiter-engine:jar:5.10.2:test\\
|~~|~~~~~\textbackslash-~org.junit.platform:junit-platform-engine:jar:1.10.2:test\\
|~~+-~org.mockito:mockito-core:jar:5.7.0:test\\
|~~|~~+-~net.bytebuddy:byte-buddy-agent:jar:1.14.13:test\\
|~~|~~\textbackslash-~org.objenesis:objenesis:jar:3.3:test\\
|~~+-~org.mockito:mockito-junit-jupiter:jar:5.7.0:test\\
|~~+-~org.skyscreamer:jsonassert:jar:1.5.1:test\\
|~~|~~\textbackslash-~com.vaadin.external.google:android-json:jar:0.0.20131108.vaadin1:test\\
|~~+-~org.springframework:spring-core:jar:6.1.6:compile\\
|~~|~~\textbackslash-~org.springframework:spring-jcl:jar:6.1.6:compile\\
|~~+-~org.springframework:spring-test:jar:6.1.6:test\\
|~~\textbackslash-~org.xmlunit:xmlunit-core:jar:2.9.1:test\\
+-~org.projectlombok:lombok:jar:1.18.32:annotationProcessor\\
+-~org.apache.commons:commons-lang3:jar:3.13.0:compile\\
+-~org.openapitools:jackson-databind-nullable:jar:0.2.6:compile\\
|~~\textbackslash-~com.fasterxml.jackson.core:jackson-databind:jar:2.15.4:compile\\
|~~~~~+-~com.fasterxml.jackson.core:jackson-annotations:jar:2.15.4:compile\\
|~~~~~\textbackslash-~com.fasterxml.jackson.core:jackson-core:jar:2.15.4:compile\\
+-~io.swagger.core.v3:swagger-annotations:jar:2.2.18:compile\\
+-~jakarta.validation:jakarta.validation-api:jar:3.0.2:compile\\
+-~jakarta.annotation:jakarta.annotation-api:jar:2.1.1:compile\\
+-~org.springframework.boot:spring-boot-starter-oauth2-resource-server:jar:3.2.5:compile\\
|~~+-~org.springframework.security:spring-security-config:jar:6.2.4:compile\\
|~~+-~org.springframework.security:spring-security-core:jar:6.2.4:compile\\
|~~|~~\textbackslash-~org.springframework.security:spring-security-crypto:jar:6.2.4:compile\\
|~~+-~org.springframework.security:spring-security-oauth2-resource-server:jar:6.2.4:compile\\
|~~|~~+-~org.springframework.security:spring-security-oauth2-core:jar:6.2.4:compile\\
|~~|~~\textbackslash-~org.springframework.security:spring-security-web:jar:6.2.4:compile\\
|~~\textbackslash-~org.springframework.security:spring-security-oauth2-jose:jar:6.2.4:compile\\
|~~~~~\textbackslash-~com.nimbusds:nimbus-jose-jwt:jar:9.24.4:compile\\
|~~~~~~~~\textbackslash-~com.github.stephenc.jcip:jcip-annotations:jar:1.0-1:compile\\
+-~de.flapdoodle.embed:de.flapdoodle.embed.mongo:jar:4.13.1:compile\\
|~~+-~de.flapdoodle.embed:de.flapdoodle.embed.process:jar:4.11.0:compile\\
|~~|~~+-~de.flapdoodle.reverse:de.flapdoodle.reverse:jar:1.7.2:compile\\
|~~|~~|~~+-~de.flapdoodle.graph:de.flapdoodle.graph:jar:1.3.3:compile\\
|~~|~~|~~|~~\textbackslash-~org.jgrapht:jgrapht-core:jar:1.4.0:compile\\
|~~|~~|~~|~~~~~\textbackslash-~org.jheaps:jheaps:jar:0.11:compile\\
|~~|~~|~~\textbackslash-~de.flapdoodle.java8:de.flapdoodle.java8:jar:1.4.2:compile\\
|~~|~~+-~org.apache.commons:commons-compress:jar:1.26.1:compile\\
|~~|~~|~~+-~commons-codec:commons-codec:jar:1.16.1:compile\\
|~~|~~|~~\textbackslash-~commons-io:commons-io:jar:2.15.1:compile\\
|~~|~~+-~net.java.dev.jna:jna:jar:5.14.0:compile\\
|~~|~~+-~net.java.dev.jna:jna-platform:jar:5.14.0:compile\\
|~~|~~\textbackslash-~de.flapdoodle:de.flapdoodle.os-api:jar:1.4.1:compile\\
|~~\textbackslash-~de.flapdoodle.embed:de.flapdoodle.embed.mongo.packageresolver:jar:4.12.0:compile\\
|~~~~~\textbackslash-~de.flapdoodle:de.flapdoodle.os:jar:1.5.5:compile\\
\textbackslash-~org.springframework.boot:spring-boot-starter-thymeleaf:jar:3.2.5:compile\\
~~~\textbackslash-~org.thymeleaf:thymeleaf-spring6:jar:3.1.2.RELEASE:compile\\
~~~~~~\textbackslash-~org.thymeleaf:thymeleaf:jar:3.1.2.RELEASE:compile\\
~~~~~~~~~+-~org.attoparser:attoparser:jar:2.0.7.RELEASE:compile\\
~~~~~~~~~\textbackslash-~org.unbescape:unbescape:jar:1.1.6.RELEASE:compile\\

\section{Frontend - Abhängigkeitsbaum}\label{sec:frontend---abhangigkeitsbaum}
├─ @ampproject/remapping@2.3.0\\
│  ├─ licenses: Apache-2.0\\
│  ├─ repository: https://github.com/ampproject/remapping\\
│  ├─ publisher: Justin Ridgewell\\
│  ├─ email: jridgewell@google.com\\
│  ├─ path: ./Frontend/node\_modules/@ampproject/remapping\\
│  └─ licenseFile: ./Frontend/node\_modules/@ampproject/remapping/LICENSE\\
├─ @angular-devkit/architect@0.1703.7\\
│  ├─ licenses: MIT\\
│  ├─ repository: https://github.com/angular/angular-cli\\
│  ├─ publisher: Angular Authors\\
│  ├─ path: ./Frontend/node\_modules/@angular-devkit/architect\\
│  └─ licenseFile: ./Frontend/node\_modules/@angular-devkit/architect/LICENSE\\
├─ @angular-devkit/build-angular@17.3.7\\
│  ├─ licenses: MIT\\
│  ├─ repository: https://github.com/angular/angular-cli\\
│  ├─ publisher: Angular Authors\\
│  ├─ path: ./Frontend/node\_modules/@angular-devkit/build-angular\\
│  └─ licenseFile: ./Frontend/node\_modules/@angular-devkit/build-angular/LICENSE\\
├─ @angular-devkit/build-webpack@0.1703.7\\
│  ├─ licenses: MIT\\
│  ├─ repository: https://github.com/angular/angular-cli\\
│  ├─ publisher: Angular Authors\\
│  ├─ path: ./Frontend/node\_modules/@angular-devkit/build-webpack\\
│  └─ licenseFile: ./Frontend/node\_modules/@angular-devkit/build-webpack/LICENSE\\
├─ @angular-devkit/core@17.3.7\\
│  ├─ licenses: MIT\\
│  ├─ repository: https://github.com/angular/angular-cli\\
│  ├─ publisher: Angular Authors\\
│  ├─ path: ./Frontend/node\_modules/@angular-devkit/core\\
│  └─ licenseFile: ./Frontend/node\_modules/@angular-devkit/core/LICENSE\\
├─ @angular-devkit/schematics@17.3.7\\
│  ├─ licenses: MIT\\
│  ├─ repository: https://github.com/angular/angular-cli\\
│  ├─ publisher: Angular Authors\\
│  ├─ path: ./Frontend/node\_modules/@angular-devkit/schematics\\
│  └─ licenseFile: ./Frontend/node\_modules/@angular-devkit/schematics/LICENSE\\
├─ @angular/animations@17.3.8\\
│  ├─ licenses: MIT\\
│  ├─ repository: https://github.com/angular/angular\\
│  ├─ publisher: angular\\
│  ├─ path: ./Frontend/node\_modules/@angular/animations\\
│  └─ licenseFile: ./Frontend/node\_modules/@angular/animations/README.md\\
├─ @angular/cli@17.3.7\\
│  ├─ licenses: MIT\\
│  ├─ repository: https://github.com/angular/angular-cli\\
│  ├─ publisher: Angular Authors\\
│  ├─ path: ./Frontend/node\_modules/@angular/cli\\
│  └─ licenseFile: ./Frontend/node\_modules/@angular/cli/LICENSE\\
├─ @angular/common@17.3.8\\
│  ├─ licenses: MIT\\
│  ├─ repository: https://github.com/angular/angular\\
│  ├─ publisher: angular\\
│  ├─ path: ./Frontend/node\_modules/@angular/common\\
│  └─ licenseFile: ./Frontend/node\_modules/@angular/common/README.md\\
├─ @angular/compiler-cli@17.3.8\\
│  ├─ licenses: MIT\\
│  ├─ repository: https://github.com/angular/angular\\
│  └─ path: ./Frontend/node\_modules/@angular/compiler-cli\\
├─ @angular/compiler@17.3.8\\
│  ├─ licenses: MIT\\
│  ├─ repository: https://github.com/angular/angular\\
│  ├─ publisher: angular\\
│  ├─ path: ./Frontend/node\_modules/@angular/compiler\\
│  └─ licenseFile: ./Frontend/node\_modules/@angular/compiler/README.md\\
├─ @angular/core@17.3.8\\
│  ├─ licenses: MIT\\
│  ├─ repository: https://github.com/angular/angular\\
│  ├─ publisher: angular\\
│  ├─ path: ./Frontend/node\_modules/@angular/core\\
│  └─ licenseFile: ./Frontend/node\_modules/@angular/core/README.md\\
├─ @angular/forms@17.3.8\\
│  ├─ licenses: MIT\\
│  ├─ repository: https://github.com/angular/angular\\
│  ├─ publisher: angular\\
│  ├─ path: ./Frontend/node\_modules/@angular/forms\\
│  └─ licenseFile: ./Frontend/node\_modules/@angular/forms/README.md\\
├─ @angular/platform-browser-dynamic@17.3.8\\
│  ├─ licenses: MIT\\
│  ├─ repository: https://github.com/angular/angular\\
│  ├─ publisher: angular\\
│  ├─ path: ./Frontend/node\_modules/@angular/platform-browser-dynamic\\
│  └─ licenseFile: ./Frontend/node\_modules/@angular/platform-browser-dynamic/README.md\\
├─ @angular/platform-browser@17.3.8\\
│  ├─ licenses: MIT\\
│  ├─ repository: https://github.com/angular/angular\\
│  ├─ publisher: angular\\
│  ├─ path: ./Frontend/node\_modules/@angular/platform-browser\\
│  └─ licenseFile: ./Frontend/node\_modules/@angular/platform-browser/README.md\\
├─ @angular/router@17.3.8\\
│  ├─ licenses: MIT\\
│  ├─ repository: https://github.com/angular/angular\\
│  ├─ publisher: angular\\
│  ├─ path: ./Frontend/node\_modules/@angular/router\\
│  └─ licenseFile: ./Frontend/node\_modules/@angular/router/README.md\\
├─ @babel/code-frame@7.24.2\\
│  ├─ licenses: MIT\\
│  ├─ repository: https://github.com/babel/babel\\
│  ├─ publisher: The Babel Team\\
│  ├─ url: https://babel.dev/team\\
│  ├─ path: ./Frontend/node\_modules/@babel/code-frame\\
│  └─ licenseFile: ./Frontend/node\_modules/@babel/code-frame/LICENSE\\
├─ @babel/compat-data@7.24.4\\
│  ├─ licenses: MIT\\
│  ├─ repository: https://github.com/babel/babel\\
│  ├─ publisher: The Babel Team\\
│  ├─ url: https://babel.dev/team\\
│  ├─ path: ./Frontend/node\_modules/@babel/compat-data\\
│  └─ licenseFile: ./Frontend/node\_modules/@babel/compat-data/LICENSE\\
├─ @babel/core@7.23.9\\
│  ├─ licenses: MIT\\
│  ├─ repository: https://github.com/babel/babel\\
│  ├─ publisher: The Babel Team\\
│  ├─ url: https://babel.dev/team\\
│  ├─ path: ./Frontend/node\_modules/@angular/compiler-cli/node\_modules/@babel/core\\
│  └─ licenseFile: ./Frontend/node\_modules/@angular/compiler-cli/node\_modules/@babel/core/LICENSE\\
├─ @babel/core@7.24.0\\
│  ├─ licenses: MIT\\
│  ├─ repository: https://github.com/babel/babel\\
│  ├─ publisher: The Babel Team\\
│  ├─ url: https://babel.dev/team\\
│  ├─ path: ./Frontend/node\_modules/@babel/core\\
│  └─ licenseFile: ./Frontend/node\_modules/@babel/core/LICENSE\\
├─ @babel/generator@7.23.6\\
│  ├─ licenses: MIT\\
│  ├─ repository: https://github.com/babel/babel\\
│  ├─ publisher: The Babel Team\\
│  ├─ url: https://babel.dev/team\\
│  ├─ path: ./Frontend/node\_modules/@babel/generator\\
│  └─ licenseFile: ./Frontend/node\_modules/@babel/generator/LICENSE\\
├─ @babel/generator@7.24.5\\
│  ├─ licenses: MIT\\
│  ├─ repository: https://github.com/babel/babel\\
│  ├─ publisher: The Babel Team\\
│  ├─ url: https://babel.dev/team\\
│  ├─ path: ./Frontend/node\_modules/@babel/traverse/node\_modules/@babel/generator\\
│  └─ licenseFile: ./Frontend/node\_modules/@babel/traverse/node\_modules/@babel/generator/LICENSE\\
├─ @babel/helper-annotate-as-pure@7.22.5\\
│  ├─ licenses: MIT\\
│  ├─ repository: https://github.com/babel/babel\\
│  ├─ publisher: The Babel Team\\
│  ├─ url: https://babel.dev/team\\
│  ├─ path: ./Frontend/node\_modules/@babel/helper-annotate-as-pure\\
│  └─ licenseFile: ./Frontend/node\_modules/@babel/helper-annotate-as-pure/LICENSE\\
├─ @babel/helper-builder-binary-assignment-operator-visitor@7.22.15\\
│  ├─ licenses: MIT\\
│  ├─ repository: https://github.com/babel/babel\\
│  ├─ publisher: The Babel Team\\
│  ├─ url: https://babel.dev/team\\
│  ├─ path: ./Frontend/node\_modules/@babel/helper-builder-binary-assignment-operator-visitor\\
│  └─ licenseFile: ./Frontend/node\_modules/@babel/helper-builder-binary-assignment-operator-visitor/LICENSE\\
├─ @babel/helper-compilation-targets@7.23.6\\
│  ├─ licenses: MIT\\
│  ├─ repository: https://github.com/babel/babel\\
│  ├─ publisher: The Babel Team\\
│  ├─ url: https://babel.dev/team\\
│  ├─ path: ./Frontend/node\_modules/@babel/helper-compilation-targets\\
│  └─ licenseFile: ./Frontend/node\_modules/@babel/helper-compilation-targets/LICENSE\\
├─ @babel/helper-create-class-features-plugin@7.24.5\\
│  ├─ licenses: MIT\\
│  ├─ repository: https://github.com/babel/babel\\
│  ├─ publisher: The Babel Team\\
│  ├─ url: https://babel.dev/team\\
│  ├─ path: ./Frontend/node\_modules/@babel/helper-create-class-features-plugin\\
│  └─ licenseFile: ./Frontend/node\_modules/@babel/helper-create-class-features-plugin/LICENSE\\
├─ @babel/helper-create-regexp-features-plugin@7.22.15\\
│  ├─ licenses: MIT\\
│  ├─ repository: https://github.com/babel/babel\\
│  ├─ publisher: The Babel Team\\
│  ├─ url: https://babel.dev/team\\
│  ├─ path: ./Frontend/node\_modules/@babel/helper-create-regexp-features-plugin\\
│  └─ licenseFile: ./Frontend/node\_modules/@babel/helper-create-regexp-features-plugin/LICENSE\\
├─ @babel/helper-define-polyfill-provider@0.5.0\\
│  ├─ licenses: MIT\\
│  ├─ repository: https://github.com/babel/babel-polyfills\\
│  ├─ path: ./Frontend/node\_modules/babel-plugin-polyfill-corejs3/node\_modules/@babel/helper-define-polyfill-provider\\
│  └─ licenseFile: ./Frontend/node\_modules/babel-plugin-polyfill-corejs3/node\_modules/@babel/helper-define-polyfill-provider/LICENSE\\
├─ @babel/helper-define-polyfill-provider@0.6.2\\
│  ├─ licenses: MIT\\
│  ├─ repository: https://github.com/babel/babel-polyfills\\
│  ├─ path: ./Frontend/node\_modules/@babel/helper-define-polyfill-provider\\
│  └─ licenseFile: ./Frontend/node\_modules/@babel/helper-define-polyfill-provider/LICENSE\\
├─ @babel/helper-environment-visitor@7.22.20\\
│  ├─ licenses: MIT\\
│  ├─ repository: https://github.com/babel/babel\\
│  ├─ publisher: The Babel Team\\
│  ├─ url: https://babel.dev/team\\
│  ├─ path: ./Frontend/node\_modules/@babel/helper-environment-visitor\\
│  └─ licenseFile: ./Frontend/node\_modules/@babel/helper-environment-visitor/LICENSE\\
├─ @babel/helper-function-name@7.23.0\\
│  ├─ licenses: MIT\\
│  ├─ repository: https://github.com/babel/babel\\
│  ├─ publisher: The Babel Team\\
│  ├─ url: https://babel.dev/team\\
│  ├─ path: ./Frontend/node\_modules/@babel/helper-function-name\\
│  └─ licenseFile: ./Frontend/node\_modules/@babel/helper-function-name/LICENSE\\
├─ @babel/helper-hoist-variables@7.22.5\\
│  ├─ licenses: MIT\\
│  ├─ repository: https://github.com/babel/babel\\
│  ├─ publisher: The Babel Team\\
│  ├─ url: https://babel.dev/team\\
│  ├─ path: ./Frontend/node\_modules/@babel/helper-hoist-variables\\
│  └─ licenseFile: ./Frontend/node\_modules/@babel/helper-hoist-variables/LICENSE\\
├─ @babel/helper-member-expression-to-functions@7.24.5\\
│  ├─ licenses: MIT\\
│  ├─ repository: https://github.com/babel/babel\\
│  ├─ publisher: The Babel Team\\
│  ├─ url: https://babel.dev/team\\
│  ├─ path: ./Frontend/node\_modules/@babel/helper-member-expression-to-functions\\
│  └─ licenseFile: ./Frontend/node\_modules/@babel/helper-member-expression-to-functions/LICENSE\\
├─ @babel/helper-module-imports@7.24.3\\
│  ├─ licenses: MIT\\
│  ├─ repository: https://github.com/babel/babel\\
│  ├─ publisher: The Babel Team\\
│  ├─ url: https://babel.dev/team\\
│  ├─ path: ./Frontend/node\_modules/@babel/helper-module-imports\\
│  └─ licenseFile: ./Frontend/node\_modules/@babel/helper-module-imports/LICENSE\\
├─ @babel/helper-module-transforms@7.24.5\\
│  ├─ licenses: MIT\\
│  ├─ repository: https://github.com/babel/babel\\
│  ├─ publisher: The Babel Team\\
│  ├─ url: https://babel.dev/team\\
│  ├─ path: ./Frontend/node\_modules/@babel/helper-module-transforms\\
│  └─ licenseFile: ./Frontend/node\_modules/@babel/helper-module-transforms/LICENSE\\
├─ @babel/helper-optimise-call-expression@7.22.5\\
│  ├─ licenses: MIT\\
│  ├─ repository: https://github.com/babel/babel\\
│  ├─ publisher: The Babel Team\\
│  ├─ url: https://babel.dev/team\\
│  ├─ path: ./Frontend/node\_modules/@babel/helper-optimise-call-expression\\
│  └─ licenseFile: ./Frontend/node\_modules/@babel/helper-optimise-call-expression/LICENSE\\
├─ @babel/helper-plugin-utils@7.24.5\\
│  ├─ licenses: MIT\\
│  ├─ repository: https://github.com/babel/babel\\
│  ├─ publisher: The Babel Team\\
│  ├─ url: https://babel.dev/team\\
│  ├─ path: ./Frontend/node\_modules/@babel/helper-plugin-utils\\
│  └─ licenseFile: ./Frontend/node\_modules/@babel/helper-plugin-utils/LICENSE\\
├─ @babel/helper-remap-async-to-generator@7.22.20\\
│  ├─ licenses: MIT\\
│  ├─ repository: https://github.com/babel/babel\\
│  ├─ publisher: The Babel Team\\
│  ├─ url: https://babel.dev/team\\
│  ├─ path: ./Frontend/node\_modules/@babel/helper-remap-async-to-generator\\
│  └─ licenseFile: ./Frontend/node\_modules/@babel/helper-remap-async-to-generator/LICENSE\\
├─ @babel/helper-replace-supers@7.24.1\\
│  ├─ licenses: MIT\\
│  ├─ repository: https://github.com/babel/babel\\
│  ├─ publisher: The Babel Team\\
│  ├─ url: https://babel.dev/team\\
│  ├─ path: ./Frontend/node\_modules/@babel/helper-replace-supers\\
│  └─ licenseFile: ./Frontend/node\_modules/@babel/helper-replace-supers/LICENSE\\
├─ @babel/helper-simple-access@7.24.5\\
│  ├─ licenses: MIT\\
│  ├─ repository: https://github.com/babel/babel\\
│  ├─ publisher: The Babel Team\\
│  ├─ url: https://babel.dev/team\\
│  ├─ path: ./Frontend/node\_modules/@babel/helper-simple-access\\
│  └─ licenseFile: ./Frontend/node\_modules/@babel/helper-simple-access/LICENSE\\
├─ @babel/helper-skip-transparent-expression-wrappers@7.22.5\\
│  ├─ licenses: MIT\\
│  ├─ repository: https://github.com/babel/babel\\
│  ├─ publisher: The Babel Team\\
│  ├─ url: https://babel.dev/team\\
│  ├─ path: ./Frontend/node\_modules/@babel/helper-skip-transparent-expression-wrappers\\
│  └─ licenseFile: ./Frontend/node\_modules/@babel/helper-skip-transparent-expression-wrappers/LICENSE\\
├─ @babel/helper-split-export-declaration@7.22.6\\
│  ├─ licenses: MIT\\
│  ├─ repository: https://github.com/babel/babel\\
│  ├─ publisher: The Babel Team\\
│  ├─ url: https://babel.dev/team\\
│  ├─ path: ./Frontend/node\_modules/@babel/helper-split-export-declaration\\
│  └─ licenseFile: ./Frontend/node\_modules/@babel/helper-split-export-declaration/LICENSE\\
├─ @babel/helper-split-export-declaration@7.24.5\\
│  ├─ licenses: MIT\\
│  ├─ repository: https://github.com/babel/babel\\
│  ├─ publisher: The Babel Team\\
│  ├─ url: https://babel.dev/team\\
│  ├─ path: ./Frontend/node\_modules/@babel/helper-module-transforms/node\_modules/@babel/helper-split-export-declaration\\
│  └─ licenseFile: ./Frontend/node\_modules/@babel/helper-module-transforms/node\_modules/@babel/helper-split-export-declaration/LICENSE\\
├─ @babel/helper-string-parser@7.24.1\\
│  ├─ licenses: MIT\\
│  ├─ repository: https://github.com/babel/babel\\
│  ├─ publisher: The Babel Team\\
│  ├─ url: https://babel.dev/team\\
│  ├─ path: ./Frontend/node\_modules/@babel/helper-string-parser\\
│  └─ licenseFile: ./Frontend/node\_modules/@babel/helper-string-parser/LICENSE\\
├─ @babel/helper-validator-identifier@7.24.5\\
│  ├─ licenses: MIT\\
│  ├─ repository: https://github.com/babel/babel\\
│  ├─ publisher: The Babel Team\\
│  ├─ url: https://babel.dev/team\\
│  ├─ path: ./Frontend/node\_modules/@babel/helper-validator-identifier\\
│  └─ licenseFile: ./Frontend/node\_modules/@babel/helper-validator-identifier/LICENSE\\
├─ @babel/helper-validator-option@7.23.5\\
│  ├─ licenses: MIT\\
│  ├─ repository: https://github.com/babel/babel\\
│  ├─ publisher: The Babel Team\\
│  ├─ url: https://babel.dev/team\\
│  ├─ path: ./Frontend/node\_modules/@babel/helper-validator-option\\
│  └─ licenseFile: ./Frontend/node\_modules/@babel/helper-validator-option/LICENSE\\
├─ @babel/helper-wrap-function@7.24.5\\
│  ├─ licenses: MIT\\
│  ├─ repository: https://github.com/babel/babel\\
│  ├─ publisher: The Babel Team\\
│  ├─ url: https://babel.dev/team\\
│  ├─ path: ./Frontend/node\_modules/@babel/helper-wrap-function\\
│  └─ licenseFile: ./Frontend/node\_modules/@babel/helper-wrap-function/LICENSE\\
├─ @babel/helpers@7.24.5\\
│  ├─ licenses: MIT\\
│  ├─ repository: https://github.com/babel/babel\\
│  ├─ publisher: The Babel Team\\
│  ├─ url: https://babel.dev/team\\
│  ├─ path: ./Frontend/node\_modules/@babel/helpers\\
│  └─ licenseFile: ./Frontend/node\_modules/@babel/helpers/LICENSE\\
├─ @babel/highlight@7.24.5\\
│  ├─ licenses: MIT\\
│  ├─ repository: https://github.com/babel/babel\\
│  ├─ publisher: The Babel Team\\
│  ├─ url: https://babel.dev/team\\
│  ├─ path: ./Frontend/node\_modules/@babel/highlight\\
│  └─ licenseFile: ./Frontend/node\_modules/@babel/highlight/LICENSE\\
├─ @babel/parser@7.24.5\\
│  ├─ licenses: MIT\\
│  ├─ repository: https://github.com/babel/babel\\
│  ├─ publisher: The Babel Team\\
│  ├─ url: https://babel.dev/team\\
│  ├─ path: ./Frontend/node\_modules/@babel/parser\\
│  └─ licenseFile: ./Frontend/node\_modules/@babel/parser/LICENSE\\
├─ @babel/plugin-bugfix-safari-id-destructuring-collision-in-function-expression@7.24.1\\
│  ├─ licenses: MIT\\
│  ├─ repository: https://github.com/babel/babel\\
│  ├─ publisher: The Babel Team\\
│  ├─ url: https://babel.dev/team\\
│  ├─ path: ./Frontend/node\_modules/@babel/plugin-bugfix-safari-id-destructuring-collision-in-function-expression\\
│  └─ licenseFile: ./Frontend/node\_modules/@babel/plugin-bugfix-safari-id-destructuring-collision-in-function-expression/LICENSE\\
├─ @babel/plugin-bugfix-v8-spread-parameters-in-optional-chaining@7.24.1\\
│  ├─ licenses: MIT\\
│  ├─ repository: https://github.com/babel/babel\\
│  ├─ publisher: The Babel Team\\
│  ├─ url: https://babel.dev/team\\
│  ├─ path: ./Frontend/node\_modules/@babel/plugin-bugfix-v8-spread-parameters-in-optional-chaining\\
│  └─ licenseFile: ./Frontend/node\_modules/@babel/plugin-bugfix-v8-spread-parameters-in-optional-chaining/LICENSE\\
├─ @babel/plugin-bugfix-v8-static-class-fields-redefine-readonly@7.24.1\\
│  ├─ licenses: MIT\\
│  ├─ repository: https://github.com/babel/babel\\
│  ├─ publisher: The Babel Team\\
│  ├─ url: https://babel.dev/team\\
│  ├─ path: ./Frontend/node\_modules/@babel/plugin-bugfix-v8-static-class-fields-redefine-readonly\\
│  └─ licenseFile: ./Frontend/node\_modules/@babel/plugin-bugfix-v8-static-class-fields-redefine-readonly/LICENSE\\
├─ @babel/plugin-proposal-private-property-in-object@7.21.0-placeholder-for-preset-env.2\\
│  ├─ licenses: MIT\\
│  ├─ repository: https://github.com/babel/babel-plugin-proposal-private-property-in-object\\
│  ├─ publisher: The Babel Team\\
│  ├─ url: https://babel.dev/team\\
│  ├─ path: ./Frontend/node\_modules/@babel/plugin-proposal-private-property-in-object\\
│  └─ licenseFile: ./Frontend/node\_modules/@babel/plugin-proposal-private-property-in-object/LICENSE\\
├─ @babel/plugin-syntax-async-generators@7.8.4\\
│  ├─ licenses: MIT\\
│  ├─ repository: https://github.com/babel/babel/tree/master/packages/babel-plugin-syntax-async-generators\\
│  ├─ path: ./Frontend/node\_modules/@babel/plugin-syntax-async-generators\\
│  └─ licenseFile: ./Frontend/node\_modules/@babel/plugin-syntax-async-generators/LICENSE\\
├─ @babel/plugin-syntax-class-properties@7.12.13\\
│  ├─ licenses: MIT\\
│  ├─ repository: https://github.com/babel/babel\\
│  ├─ path: ./Frontend/node\_modules/@babel/plugin-syntax-class-properties\\
│  └─ licenseFile: ./Frontend/node\_modules/@babel/plugin-syntax-class-properties/LICENSE\\
├─ @babel/plugin-syntax-class-static-block@7.14.5\\
│  ├─ licenses: MIT\\
│  ├─ repository: https://github.com/babel/babel\\
│  ├─ publisher: The Babel Team\\
│  ├─ url: https://babel.dev/team\\
│  ├─ path: ./Frontend/node\_modules/@babel/plugin-syntax-class-static-block\\
│  └─ licenseFile: ./Frontend/node\_modules/@babel/plugin-syntax-class-static-block/LICENSE\\
├─ @babel/plugin-syntax-dynamic-import@7.8.3\\
│  ├─ licenses: MIT\\
│  ├─ repository: https://github.com/babel/babel/tree/master/packages/babel-plugin-syntax-dynamic-import\\
│  ├─ path: ./Frontend/node\_modules/@babel/plugin-syntax-dynamic-import\\
│  └─ licenseFile: ./Frontend/node\_modules/@babel/plugin-syntax-dynamic-import/LICENSE\\
├─ @babel/plugin-syntax-export-namespace-from@7.8.3\\
│  ├─ licenses: MIT\\
│  ├─ repository: https://github.com/babel/babel/tree/master/packages/babel-plugin-syntax-export-namespace-from\\
│  ├─ path: ./Frontend/node\_modules/@babel/plugin-syntax-export-namespace-from\\
│  └─ licenseFile: ./Frontend/node\_modules/@babel/plugin-syntax-export-namespace-from/LICENSE\\
├─ @babel/plugin-syntax-import-assertions@7.24.1\\
│  ├─ licenses: MIT\\
│  ├─ repository: https://github.com/babel/babel\\
│  ├─ publisher: The Babel Team\\
│  ├─ url: https://babel.dev/team\\
│  ├─ path: ./Frontend/node\_modules/@babel/plugin-syntax-import-assertions\\
│  └─ licenseFile: ./Frontend/node\_modules/@babel/plugin-syntax-import-assertions/LICENSE\\
├─ @babel/plugin-syntax-import-attributes@7.24.1\\
│  ├─ licenses: MIT\\
│  ├─ repository: https://github.com/babel/babel\\
│  ├─ publisher: The Babel Team\\
│  ├─ url: https://babel.dev/team\\
│  ├─ path: ./Frontend/node\_modules/@babel/plugin-syntax-import-attributes\\
│  └─ licenseFile: ./Frontend/node\_modules/@babel/plugin-syntax-import-attributes/LICENSE\\
├─ @babel/plugin-syntax-import-meta@7.10.4\\
│  ├─ licenses: MIT\\
│  ├─ repository: https://github.com/babel/babel\\
│  ├─ path: ./Frontend/node\_modules/@babel/plugin-syntax-import-meta\\
│  └─ licenseFile: ./Frontend/node\_modules/@babel/plugin-syntax-import-meta/LICENSE\\
├─ @babel/plugin-syntax-json-strings@7.8.3\\
│  ├─ licenses: MIT\\
│  ├─ repository: https://github.com/babel/babel/tree/master/packages/babel-plugin-syntax-json-strings\\
│  ├─ path: ./Frontend/node\_modules/@babel/plugin-syntax-json-strings\\
│  └─ licenseFile: ./Frontend/node\_modules/@babel/plugin-syntax-json-strings/LICENSE\\
├─ @babel/plugin-syntax-logical-assignment-operators@7.10.4\\
│  ├─ licenses: MIT\\
│  ├─ repository: https://github.com/babel/babel\\
│  ├─ path: ./Frontend/node\_modules/@babel/plugin-syntax-logical-assignment-operators\\
│  └─ licenseFile: ./Frontend/node\_modules/@babel/plugin-syntax-logical-assignment-operators/LICENSE\\
├─ @babel/plugin-syntax-nullish-coalescing-operator@7.8.3\\
│  ├─ licenses: MIT\\
│  ├─ repository: https://github.com/babel/babel/tree/master/packages/babel-plugin-syntax-nullish-coalescing-operator\\
│  ├─ path: ./Frontend/node\_modules/@babel/plugin-syntax-nullish-coalescing-operator\\
│  └─ licenseFile: ./Frontend/node\_modules/@babel/plugin-syntax-nullish-coalescing-operator/LICENSE\\
├─ @babel/plugin-syntax-numeric-separator@7.10.4\\
│  ├─ licenses: MIT\\
│  ├─ repository: https://github.com/babel/babel\\
│  ├─ path: ./Frontend/node\_modules/@babel/plugin-syntax-numeric-separator\\
│  └─ licenseFile: ./Frontend/node\_modules/@babel/plugin-syntax-numeric-separator/LICENSE\\
├─ @babel/plugin-syntax-object-rest-spread@7.8.3\\
│  ├─ licenses: MIT\\
│  ├─ repository: https://github.com/babel/babel/tree/master/packages/babel-plugin-syntax-object-rest-spread\\
│  ├─ path: ./Frontend/node\_modules/@babel/plugin-syntax-object-rest-spread\\
│  └─ licenseFile: ./Frontend/node\_modules/@babel/plugin-syntax-object-rest-spread/LICENSE\\
├─ @babel/plugin-syntax-optional-catch-binding@7.8.3\\
│  ├─ licenses: MIT\\
│  ├─ repository: https://github.com/babel/babel/tree/master/packages/babel-plugin-syntax-optional-catch-binding\\
│  ├─ path: ./Frontend/node\_modules/@babel/plugin-syntax-optional-catch-binding\\
│  └─ licenseFile: ./Frontend/node\_modules/@babel/plugin-syntax-optional-catch-binding/LICENSE\\
├─ @babel/plugin-syntax-optional-chaining@7.8.3\\
│  ├─ licenses: MIT\\
│  ├─ repository: https://github.com/babel/babel/tree/master/packages/babel-plugin-syntax-optional-chaining\\
│  ├─ path: ./Frontend/node\_modules/@babel/plugin-syntax-optional-chaining\\
│  └─ licenseFile: ./Frontend/node\_modules/@babel/plugin-syntax-optional-chaining/LICENSE\\
├─ @babel/plugin-syntax-private-property-in-object@7.14.5\\
│  ├─ licenses: MIT\\
│  ├─ repository: https://github.com/babel/babel\\
│  ├─ publisher: The Babel Team\\
│  ├─ url: https://babel.dev/team\\
│  ├─ path: ./Frontend/node\_modules/@babel/plugin-syntax-private-property-in-object\\
│  └─ licenseFile: ./Frontend/node\_modules/@babel/plugin-syntax-private-property-in-object/LICENSE\\
├─ @babel/plugin-syntax-top-level-await@7.14.5\\
│  ├─ licenses: MIT\\
│  ├─ repository: https://github.com/babel/babel\\
│  ├─ publisher: The Babel Team\\
│  ├─ url: https://babel.dev/team\\
│  ├─ path: ./Frontend/node\_modules/@babel/plugin-syntax-top-level-await\\
│  └─ licenseFile: ./Frontend/node\_modules/@babel/plugin-syntax-top-level-await/LICENSE\\
├─ @babel/plugin-syntax-unicode-sets-regex@7.18.6\\
│  ├─ licenses: MIT\\
│  ├─ repository: https://github.com/babel/babel\\
│  ├─ publisher: The Babel Team\\
│  ├─ url: https://babel.dev/team\\
│  ├─ path: ./Frontend/node\_modules/@babel/plugin-syntax-unicode-sets-regex\\
│  └─ licenseFile: ./Frontend/node\_modules/@babel/plugin-syntax-unicode-sets-regex/LICENSE\\
├─ @babel/plugin-transform-arrow-functions@7.24.1\\
│  ├─ licenses: MIT\\
│  ├─ repository: https://github.com/babel/babel\\
│  ├─ publisher: The Babel Team\\
│  ├─ url: https://babel.dev/team\\
│  ├─ path: ./Frontend/node\_modules/@babel/plugin-transform-arrow-functions\\
│  └─ licenseFile: ./Frontend/node\_modules/@babel/plugin-transform-arrow-functions/LICENSE\\
├─ @babel/plugin-transform-async-generator-functions@7.23.9\\
│  ├─ licenses: MIT\\
│  ├─ repository: https://github.com/babel/babel\\
│  ├─ publisher: The Babel Team\\
│  ├─ url: https://babel.dev/team\\
│  ├─ path: ./Frontend/node\_modules/@babel/plugin-transform-async-generator-functions\\
│  └─ licenseFile: ./Frontend/node\_modules/@babel/plugin-transform-async-generator-functions/LICENSE\\
├─ @babel/plugin-transform-async-to-generator@7.23.3\\
│  ├─ licenses: MIT\\
│  ├─ repository: https://github.com/babel/babel\\
│  ├─ publisher: The Babel Team\\
│  ├─ url: https://babel.dev/team\\
│  ├─ path: ./Frontend/node\_modules/@babel/plugin-transform-async-to-generator\\
│  └─ licenseFile: ./Frontend/node\_modules/@babel/plugin-transform-async-to-generator/LICENSE\\
├─ @babel/plugin-transform-block-scoped-functions@7.24.1\\
│  ├─ licenses: MIT\\
│  ├─ repository: https://github.com/babel/babel\\
│  ├─ publisher: The Babel Team\\
│  ├─ url: https://babel.dev/team\\
│  ├─ path: ./Frontend/node\_modules/@babel/plugin-transform-block-scoped-functions\\
│  └─ licenseFile: ./Frontend/node\_modules/@babel/plugin-transform-block-scoped-functions/LICENSE\\
├─ @babel/plugin-transform-block-scoping@7.24.5\\
│  ├─ licenses: MIT\\
│  ├─ repository: https://github.com/babel/babel\\
│  ├─ publisher: The Babel Team\\
│  ├─ url: https://babel.dev/team\\
│  ├─ path: ./Frontend/node\_modules/@babel/plugin-transform-block-scoping\\
│  └─ licenseFile: ./Frontend/node\_modules/@babel/plugin-transform-block-scoping/LICENSE\\
├─ @babel/plugin-transform-class-properties@7.24.1\\
│  ├─ licenses: MIT\\
│  ├─ repository: https://github.com/babel/babel\\
│  ├─ publisher: The Babel Team\\
│  ├─ url: https://babel.dev/team\\
│  ├─ path: ./Frontend/node\_modules/@babel/plugin-transform-class-properties\\
│  └─ licenseFile: ./Frontend/node\_modules/@babel/plugin-transform-class-properties/LICENSE\\
├─ @babel/plugin-transform-class-static-block@7.24.4\\
│  ├─ licenses: MIT\\
│  ├─ repository: https://github.com/babel/babel\\
│  ├─ publisher: The Babel Team\\
│  ├─ url: https://babel.dev/team\\
│  ├─ path: ./Frontend/node\_modules/@babel/plugin-transform-class-static-block\\
│  └─ licenseFile: ./Frontend/node\_modules/@babel/plugin-transform-class-static-block/LICENSE\\
├─ @babel/plugin-transform-classes@7.24.5\\
│  ├─ licenses: MIT\\
│  ├─ repository: https://github.com/babel/babel\\
│  ├─ publisher: The Babel Team\\
│  ├─ url: https://babel.dev/team\\
│  ├─ path: ./Frontend/node\_modules/@babel/plugin-transform-classes\\
│  └─ licenseFile: ./Frontend/node\_modules/@babel/plugin-transform-classes/LICENSE\\
├─ @babel/plugin-transform-computed-properties@7.24.1\\
│  ├─ licenses: MIT\\
│  ├─ repository: https://github.com/babel/babel\\
│  ├─ publisher: The Babel Team\\
│  ├─ url: https://babel.dev/team\\
│  ├─ path: ./Frontend/node\_modules/@babel/plugin-transform-computed-properties\\
│  └─ licenseFile: ./Frontend/node\_modules/@babel/plugin-transform-computed-properties/LICENSE\\
├─ @babel/plugin-transform-destructuring@7.24.5\\
│  ├─ licenses: MIT\\
│  ├─ repository: https://github.com/babel/babel\\
│  ├─ publisher: The Babel Team\\
│  ├─ url: https://babel.dev/team\\
│  ├─ path: ./Frontend/node\_modules/@babel/plugin-transform-destructuring\\
│  └─ licenseFile: ./Frontend/node\_modules/@babel/plugin-transform-destructuring/LICENSE\\
├─ @babel/plugin-transform-dotall-regex@7.24.1\\
│  ├─ licenses: MIT\\
│  ├─ repository: https://github.com/babel/babel\\
│  ├─ publisher: The Babel Team\\
│  ├─ url: https://babel.dev/team\\
│  ├─ path: ./Frontend/node\_modules/@babel/plugin-transform-dotall-regex\\
│  └─ licenseFile: ./Frontend/node\_modules/@babel/plugin-transform-dotall-regex/LICENSE\\
├─ @babel/plugin-transform-duplicate-keys@7.24.1\\
│  ├─ licenses: MIT\\
│  ├─ repository: https://github.com/babel/babel\\
│  ├─ publisher: The Babel Team\\
│  ├─ url: https://babel.dev/team\\
│  ├─ path: ./Frontend/node\_modules/@babel/plugin-transform-duplicate-keys\\
│  └─ licenseFile: ./Frontend/node\_modules/@babel/plugin-transform-duplicate-keys/LICENSE\\
├─ @babel/plugin-transform-dynamic-import@7.24.1\\
│  ├─ licenses: MIT\\
│  ├─ repository: https://github.com/babel/babel\\
│  ├─ publisher: The Babel Team\\
│  ├─ url: https://babel.dev/team\\
│  ├─ path: ./Frontend/node\_modules/@babel/plugin-transform-dynamic-import\\
│  └─ licenseFile: ./Frontend/node\_modules/@babel/plugin-transform-dynamic-import/LICENSE\\
├─ @babel/plugin-transform-exponentiation-operator@7.24.1\\
│  ├─ licenses: MIT\\
│  ├─ repository: https://github.com/babel/babel\\
│  ├─ publisher: The Babel Team\\
│  ├─ url: https://babel.dev/team\\
│  ├─ path: ./Frontend/node\_modules/@babel/plugin-transform-exponentiation-operator\\
│  └─ licenseFile: ./Frontend/node\_modules/@babel/plugin-transform-exponentiation-operator/LICENSE\\
├─ @babel/plugin-transform-export-namespace-from@7.24.1\\
│  ├─ licenses: MIT\\
│  ├─ repository: https://github.com/babel/babel\\
│  ├─ publisher: The Babel Team\\
│  ├─ url: https://babel.dev/team\\
│  ├─ path: ./Frontend/node\_modules/@babel/plugin-transform-export-namespace-from\\
│  └─ licenseFile: ./Frontend/node\_modules/@babel/plugin-transform-export-namespace-from/LICENSE\\
├─ @babel/plugin-transform-for-of@7.24.1\\
│  ├─ licenses: MIT\\
│  ├─ repository: https://github.com/babel/babel\\
│  ├─ publisher: The Babel Team\\
│  ├─ url: https://babel.dev/team\\
│  ├─ path: ./Frontend/node\_modules/@babel/plugin-transform-for-of\\
│  └─ licenseFile: ./Frontend/node\_modules/@babel/plugin-transform-for-of/LICENSE\\
├─ @babel/plugin-transform-function-name@7.24.1\\
│  ├─ licenses: MIT\\
│  ├─ repository: https://github.com/babel/babel\\
│  ├─ publisher: The Babel Team\\
│  ├─ url: https://babel.dev/team\\
│  ├─ path: ./Frontend/node\_modules/@babel/plugin-transform-function-name\\
│  └─ licenseFile: ./Frontend/node\_modules/@babel/plugin-transform-function-name/LICENSE\\
├─ @babel/plugin-transform-json-strings@7.24.1\\
│  ├─ licenses: MIT\\
│  ├─ repository: https://github.com/babel/babel\\
│  ├─ publisher: The Babel Team\\
│  ├─ url: https://babel.dev/team\\
│  ├─ path: ./Frontend/node\_modules/@babel/plugin-transform-json-strings\\
│  └─ licenseFile: ./Frontend/node\_modules/@babel/plugin-transform-json-strings/LICENSE\\
├─ @babel/plugin-transform-literals@7.24.1\\
│  ├─ licenses: MIT\\
│  ├─ repository: https://github.com/babel/babel\\
│  ├─ publisher: The Babel Team\\
│  ├─ url: https://babel.dev/team\\
│  ├─ path: ./Frontend/node\_modules/@babel/plugin-transform-literals\\
│  └─ licenseFile: ./Frontend/node\_modules/@babel/plugin-transform-literals/LICENSE\\
├─ @babel/plugin-transform-logical-assignment-operators@7.24.1\\
│  ├─ licenses: MIT\\
│  ├─ repository: https://github.com/babel/babel\\
│  ├─ publisher: The Babel Team\\
│  ├─ url: https://babel.dev/team\\
│  ├─ path: ./Frontend/node\_modules/@babel/plugin-transform-logical-assignment-operators\\
│  └─ licenseFile: ./Frontend/node\_modules/@babel/plugin-transform-logical-assignment-operators/LICENSE\\
├─ @babel/plugin-transform-member-expression-literals@7.24.1\\
│  ├─ licenses: MIT\\
│  ├─ repository: https://github.com/babel/babel\\
│  ├─ publisher: The Babel Team\\
│  ├─ url: https://babel.dev/team\\
│  ├─ path: ./Frontend/node\_modules/@babel/plugin-transform-member-expression-literals\\
│  └─ licenseFile: ./Frontend/node\_modules/@babel/plugin-transform-member-expression-literals/LICENSE\\
├─ @babel/plugin-transform-modules-amd@7.24.1\\
│  ├─ licenses: MIT\\
│  ├─ repository: https://github.com/babel/babel\\
│  ├─ publisher: The Babel Team\\
│  ├─ url: https://babel.dev/team\\
│  ├─ path: ./Frontend/node\_modules/@babel/plugin-transform-modules-amd\\
│  └─ licenseFile: ./Frontend/node\_modules/@babel/plugin-transform-modules-amd/LICENSE\\
├─ @babel/plugin-transform-modules-commonjs@7.24.1\\
│  ├─ licenses: MIT\\
│  ├─ repository: https://github.com/babel/babel\\
│  ├─ publisher: The Babel Team\\
│  ├─ url: https://babel.dev/team\\
│  ├─ path: ./Frontend/node\_modules/@babel/plugin-transform-modules-commonjs\\
│  └─ licenseFile: ./Frontend/node\_modules/@babel/plugin-transform-modules-commonjs/LICENSE\\
├─ @babel/plugin-transform-modules-systemjs@7.24.1\\
│  ├─ licenses: MIT\\
│  ├─ repository: https://github.com/babel/babel\\
│  ├─ publisher: The Babel Team\\
│  ├─ url: https://babel.dev/team\\
│  ├─ path: ./Frontend/node\_modules/@babel/plugin-transform-modules-systemjs\\
│  └─ licenseFile: ./Frontend/node\_modules/@babel/plugin-transform-modules-systemjs/LICENSE\\
├─ @babel/plugin-transform-modules-umd@7.24.1\\
│  ├─ licenses: MIT\\
│  ├─ repository: https://github.com/babel/babel\\
│  ├─ publisher: The Babel Team\\
│  ├─ url: https://babel.dev/team\\
│  ├─ path: ./Frontend/node\_modules/@babel/plugin-transform-modules-umd\\
│  └─ licenseFile: ./Frontend/node\_modules/@babel/plugin-transform-modules-umd/LICENSE\\
├─ @babel/plugin-transform-named-capturing-groups-regex@7.22.5\\
│  ├─ licenses: MIT\\
│  ├─ repository: https://github.com/babel/babel\\
│  ├─ publisher: The Babel Team\\
│  ├─ url: https://babel.dev/team\\
│  ├─ path: ./Frontend/node\_modules/@babel/plugin-transform-named-capturing-groups-regex\\
│  └─ licenseFile: ./Frontend/node\_modules/@babel/plugin-transform-named-capturing-groups-regex/LICENSE\\
├─ @babel/plugin-transform-new-target@7.24.1\\
│  ├─ licenses: MIT\\
│  ├─ repository: https://github.com/babel/babel\\
│  ├─ publisher: The Babel Team\\
│  ├─ url: https://babel.dev/team\\
│  ├─ path: ./Frontend/node\_modules/@babel/plugin-transform-new-target\\
│  └─ licenseFile: ./Frontend/node\_modules/@babel/plugin-transform-new-target/LICENSE\\
├─ @babel/plugin-transform-nullish-coalescing-operator@7.24.1\\
│  ├─ licenses: MIT\\
│  ├─ repository: https://github.com/babel/babel\\
│  ├─ publisher: The Babel Team\\
│  ├─ url: https://babel.dev/team\\
│  ├─ path: ./Frontend/node\_modules/@babel/plugin-transform-nullish-coalescing-operator\\
│  └─ licenseFile: ./Frontend/node\_modules/@babel/plugin-transform-nullish-coalescing-operator/LICENSE\\
├─ @babel/plugin-transform-numeric-separator@7.24.1\\
│  ├─ licenses: MIT\\
│  ├─ repository: https://github.com/babel/babel\\
│  ├─ publisher: The Babel Team\\
│  ├─ url: https://babel.dev/team\\
│  ├─ path: ./Frontend/node\_modules/@babel/plugin-transform-numeric-separator\\
│  └─ licenseFile: ./Frontend/node\_modules/@babel/plugin-transform-numeric-separator/LICENSE\\
├─ @babel/plugin-transform-object-rest-spread@7.24.5\\
│  ├─ licenses: MIT\\
│  ├─ repository: https://github.com/babel/babel\\
│  ├─ publisher: The Babel Team\\
│  ├─ url: https://babel.dev/team\\
│  ├─ path: ./Frontend/node\_modules/@babel/plugin-transform-object-rest-spread\\
│  └─ licenseFile: ./Frontend/node\_modules/@babel/plugin-transform-object-rest-spread/LICENSE\\
├─ @babel/plugin-transform-object-super@7.24.1\\
│  ├─ licenses: MIT\\
│  ├─ repository: https://github.com/babel/babel\\
│  ├─ publisher: The Babel Team\\
│  ├─ url: https://babel.dev/team\\
│  ├─ path: ./Frontend/node\_modules/@babel/plugin-transform-object-super\\
│  └─ licenseFile: ./Frontend/node\_modules/@babel/plugin-transform-object-super/LICENSE\\
├─ @babel/plugin-transform-optional-catch-binding@7.24.1\\
│  ├─ licenses: MIT\\
│  ├─ repository: https://github.com/babel/babel\\
│  ├─ publisher: The Babel Team\\
│  ├─ url: https://babel.dev/team\\
│  ├─ path: ./Frontend/node\_modules/@babel/plugin-transform-optional-catch-binding\\
│  └─ licenseFile: ./Frontend/node\_modules/@babel/plugin-transform-optional-catch-binding/LICENSE\\
├─ @babel/plugin-transform-optional-chaining@7.24.5\\
│  ├─ licenses: MIT\\
│  ├─ repository: https://github.com/babel/babel\\
│  ├─ publisher: The Babel Team\\
│  ├─ url: https://babel.dev/team\\
│  ├─ path: ./Frontend/node\_modules/@babel/plugin-transform-optional-chaining\\
│  └─ licenseFile: ./Frontend/node\_modules/@babel/plugin-transform-optional-chaining/LICENSE\\
├─ @babel/plugin-transform-parameters@7.24.5\\
│  ├─ licenses: MIT\\
│  ├─ repository: https://github.com/babel/babel\\
│  ├─ publisher: The Babel Team\\
│  ├─ url: https://babel.dev/team\\
│  ├─ path: ./Frontend/node\_modules/@babel/plugin-transform-parameters\\
│  └─ licenseFile: ./Frontend/node\_modules/@babel/plugin-transform-parameters/LICENSE\\
├─ @babel/plugin-transform-private-methods@7.24.1\\
│  ├─ licenses: MIT\\
│  ├─ repository: https://github.com/babel/babel\\
│  ├─ publisher: The Babel Team\\
│  ├─ url: https://babel.dev/team\\
│  ├─ path: ./Frontend/node\_modules/@babel/plugin-transform-private-methods\\
│  └─ licenseFile: ./Frontend/node\_modules/@babel/plugin-transform-private-methods/LICENSE\\
├─ @babel/plugin-transform-private-property-in-object@7.24.5\\
│  ├─ licenses: MIT\\
│  ├─ repository: https://github.com/babel/babel\\
│  ├─ publisher: The Babel Team\\
│  ├─ url: https://babel.dev/team\\
│  ├─ path: ./Frontend/node\_modules/@babel/plugin-transform-private-property-in-object\\
│  └─ licenseFile: ./Frontend/node\_modules/@babel/plugin-transform-private-property-in-object/LICENSE\\
├─ @babel/plugin-transform-property-literals@7.24.1\\
│  ├─ licenses: MIT\\
│  ├─ repository: https://github.com/babel/babel\\
│  ├─ publisher: The Babel Team\\
│  ├─ url: https://babel.dev/team\\
│  ├─ path: ./Frontend/node\_modules/@babel/plugin-transform-property-literals\\
│  └─ licenseFile: ./Frontend/node\_modules/@babel/plugin-transform-property-literals/LICENSE\\
├─ @babel/plugin-transform-regenerator@7.24.1\\
│  ├─ licenses: MIT\\
│  ├─ repository: https://github.com/babel/babel\\
│  ├─ publisher: The Babel Team\\
│  ├─ url: https://babel.dev/team\\
│  ├─ path: ./Frontend/node\_modules/@babel/plugin-transform-regenerator\\
│  └─ licenseFile: ./Frontend/node\_modules/@babel/plugin-transform-regenerator/LICENSE\\
├─ @babel/plugin-transform-reserved-words@7.24.1\\
│  ├─ licenses: MIT\\
│  ├─ repository: https://github.com/babel/babel\\
│  ├─ publisher: The Babel Team\\
│  ├─ url: https://babel.dev/team\\
│  ├─ path: ./Frontend/node\_modules/@babel/plugin-transform-reserved-words\\
│  └─ licenseFile: ./Frontend/node\_modules/@babel/plugin-transform-reserved-words/LICENSE\\
├─ @babel/plugin-transform-runtime@7.24.0\\
│  ├─ licenses: MIT\\
│  ├─ repository: https://github.com/babel/babel\\
│  ├─ publisher: The Babel Team\\
│  ├─ url: https://babel.dev/team\\
│  ├─ path: ./Frontend/node\_modules/@babel/plugin-transform-runtime\\
│  └─ licenseFile: ./Frontend/node\_modules/@babel/plugin-transform-runtime/LICENSE\\
├─ @babel/plugin-transform-shorthand-properties@7.24.1\\
│  ├─ licenses: MIT\\
│  ├─ repository: https://github.com/babel/babel\\
│  ├─ publisher: The Babel Team\\
│  ├─ url: https://babel.dev/team\\
│  ├─ path: ./Frontend/node\_modules/@babel/plugin-transform-shorthand-properties\\
│  └─ licenseFile: ./Frontend/node\_modules/@babel/plugin-transform-shorthand-properties/LICENSE\\
├─ @babel/plugin-transform-spread@7.24.1\\
│  ├─ licenses: MIT\\
│  ├─ repository: https://github.com/babel/babel\\
│  ├─ publisher: The Babel Team\\
│  ├─ url: https://babel.dev/team\\
│  ├─ path: ./Frontend/node\_modules/@babel/plugin-transform-spread\\
│  └─ licenseFile: ./Frontend/node\_modules/@babel/plugin-transform-spread/LICENSE\\
├─ @babel/plugin-transform-sticky-regex@7.24.1\\
│  ├─ licenses: MIT\\
│  ├─ repository: https://github.com/babel/babel\\
│  ├─ publisher: The Babel Team\\
│  ├─ url: https://babel.dev/team\\
│  ├─ path: ./Frontend/node\_modules/@babel/plugin-transform-sticky-regex\\
│  └─ licenseFile: ./Frontend/node\_modules/@babel/plugin-transform-sticky-regex/LICENSE\\
├─ @babel/plugin-transform-template-literals@7.24.1\\
│  ├─ licenses: MIT\\
│  ├─ repository: https://github.com/babel/babel\\
│  ├─ publisher: The Babel Team\\
│  ├─ url: https://babel.dev/team\\
│  ├─ path: ./Frontend/node\_modules/@babel/plugin-transform-template-literals\\
│  └─ licenseFile: ./Frontend/node\_modules/@babel/plugin-transform-template-literals/LICENSE\\
├─ @babel/plugin-transform-typeof-symbol@7.24.5\\
│  ├─ licenses: MIT\\
│  ├─ repository: https://github.com/babel/babel\\
│  ├─ publisher: The Babel Team\\
│  ├─ url: https://babel.dev/team\\
│  ├─ path: ./Frontend/node\_modules/@babel/plugin-transform-typeof-symbol\\
│  └─ licenseFile: ./Frontend/node\_modules/@babel/plugin-transform-typeof-symbol/LICENSE\\
├─ @babel/plugin-transform-unicode-escapes@7.24.1\\
│  ├─ licenses: MIT\\
│  ├─ repository: https://github.com/babel/babel\\
│  ├─ publisher: The Babel Team\\
│  ├─ url: https://babel.dev/team\\
│  ├─ path: ./Frontend/node\_modules/@babel/plugin-transform-unicode-escapes\\
│  └─ licenseFile: ./Frontend/node\_modules/@babel/plugin-transform-unicode-escapes/LICENSE\\
├─ @babel/plugin-transform-unicode-property-regex@7.24.1\\
│  ├─ licenses: MIT\\
│  ├─ repository: https://github.com/babel/babel\\
│  ├─ publisher: The Babel Team\\
│  ├─ url: https://babel.dev/team\\
│  ├─ path: ./Frontend/node\_modules/@babel/plugin-transform-unicode-property-regex\\
│  └─ licenseFile: ./Frontend/node\_modules/@babel/plugin-transform-unicode-property-regex/LICENSE\\
├─ @babel/plugin-transform-unicode-regex@7.24.1\\
│  ├─ licenses: MIT\\
│  ├─ repository: https://github.com/babel/babel\\
│  ├─ publisher: The Babel Team\\
│  ├─ url: https://babel.dev/team\\
│  ├─ path: ./Frontend/node\_modules/@babel/plugin-transform-unicode-regex\\
│  └─ licenseFile: ./Frontend/node\_modules/@babel/plugin-transform-unicode-regex/LICENSE\\
├─ @babel/plugin-transform-unicode-sets-regex@7.24.1\\
│  ├─ licenses: MIT\\
│  ├─ repository: https://github.com/babel/babel\\
│  ├─ publisher: The Babel Team\\
│  ├─ url: https://babel.dev/team\\
│  ├─ path: ./Frontend/node\_modules/@babel/plugin-transform-unicode-sets-regex\\
│  └─ licenseFile: ./Frontend/node\_modules/@babel/plugin-transform-unicode-sets-regex/LICENSE\\
├─ @babel/preset-env@7.24.0\\
│  ├─ licenses: MIT\\
│  ├─ repository: https://github.com/babel/babel\\
│  ├─ publisher: The Babel Team\\
│  ├─ url: https://babel.dev/team\\
│  ├─ path: ./Frontend/node\_modules/@babel/preset-env\\
│  └─ licenseFile: ./Frontend/node\_modules/@babel/preset-env/LICENSE\\
├─ @babel/preset-modules@0.1.6-no-external-plugins\\
│  ├─ licenses: MIT\\
│  ├─ repository: https://github.com/babel/preset-modules\\
│  ├─ path: ./Frontend/node\_modules/@babel/preset-modules\\
│  └─ licenseFile: ./Frontend/node\_modules/@babel/preset-modules/LICENSE\\
├─ @babel/regjsgen@0.8.0\\
│  ├─ licenses: MIT\\
│  ├─ repository: https://github.com/bnjmnt4n/regjsgen\\
│  ├─ publisher: Benjamin Tan\\
│  ├─ url: https://ofcr.se/\\
│  ├─ path: ./Frontend/node\_modules/@babel/regjsgen\\
│  └─ licenseFile: ./Frontend/node\_modules/@babel/regjsgen/LICENSE-MIT.txt\\
├─ @babel/runtime@7.24.0\\
│  ├─ licenses: MIT\\
│  ├─ repository: https://github.com/babel/babel\\
│  ├─ publisher: The Babel Team\\
│  ├─ url: https://babel.dev/team\\
│  ├─ path: ./Frontend/node\_modules/@babel/runtime\\
│  └─ licenseFile: ./Frontend/node\_modules/@babel/runtime/LICENSE\\
├─ @babel/template@7.24.0\\
│  ├─ licenses: MIT\\
│  ├─ repository: https://github.com/babel/babel\\
│  ├─ publisher: The Babel Team\\
│  ├─ url: https://babel.dev/team\\
│  ├─ path: ./Frontend/node\_modules/@babel/template\\
│  └─ licenseFile: ./Frontend/node\_modules/@babel/template/LICENSE\\
├─ @babel/traverse@7.24.5\\
│  ├─ licenses: MIT\\
│  ├─ repository: https://github.com/babel/babel\\
│  ├─ publisher: The Babel Team\\
│  ├─ url: https://babel.dev/team\\
│  ├─ path: ./Frontend/node\_modules/@babel/traverse\\
│  └─ licenseFile: ./Frontend/node\_modules/@babel/traverse/LICENSE\\
├─ @babel/types@7.24.5\\
│  ├─ licenses: MIT\\
│  ├─ repository: https://github.com/babel/babel\\
│  ├─ publisher: The Babel Team\\
│  ├─ url: https://babel.dev/team\\
│  ├─ path: ./Frontend/node\_modules/@babel/types\\
│  └─ licenseFile: ./Frontend/node\_modules/@babel/types/LICENSE\\
├─ @colors/colors@1.5.0\\
│  ├─ licenses: MIT\\
│  ├─ repository: https://github.com/DABH/colors.js\\
│  ├─ publisher: DABH\\
│  ├─ path: ./Frontend/node\_modules/@colors/colors\\
│  └─ licenseFile: ./Frontend/node\_modules/@colors/colors/LICENSE\\
├─ @discoveryjs/json-ext@0.5.7\\
│  ├─ licenses: MIT\\
│  ├─ repository: https://github.com/discoveryjs/json-ext\\
│  ├─ publisher: Roman Dvornov\\
│  ├─ email: rdvornov@gmail.com\\
│  ├─ url: https://github.com/lahmatiy\\
│  ├─ path: ./Frontend/node\_modules/@discoveryjs/json-ext\\
│  └─ licenseFile: ./Frontend/node\_modules/@discoveryjs/json-ext/LICENSE\\
├─ @esbuild/linux-x64@0.19.12\\
│  ├─ licenses: MIT\\
│  ├─ repository: https://github.com/evanw/esbuild\\
│  ├─ path: ./Frontend/node\_modules/vite/node\_modules/@esbuild/linux-x64\\
│  └─ licenseFile: ./Frontend/node\_modules/vite/node\_modules/@esbuild/linux-x64/README.md\\
├─ @esbuild/linux-x64@0.20.1\\
│  ├─ licenses: MIT\\
│  ├─ repository: https://github.com/evanw/esbuild\\
│  ├─ path: ./Frontend/node\_modules/@esbuild/linux-x64\\
│  └─ licenseFile: ./Frontend/node\_modules/@esbuild/linux-x64/README.md\\
├─ @isaacs/cliui@8.0.2\\
│  ├─ licenses: ISC\\
│  ├─ repository: https://github.com/yargs/cliui\\
│  ├─ publisher: Ben Coe\\
│  ├─ email: ben@npmjs.com\\
│  ├─ path: ./Frontend/node\_modules/@isaacs/cliui\\
│  └─ licenseFile: ./Frontend/node\_modules/@isaacs/cliui/LICENSE.txt\\
├─ @istanbuljs/load-nyc-config@1.1.0\\
│  ├─ licenses: ISC\\
│  ├─ repository: https://github.com/istanbuljs/load-nyc-config\\
│  ├─ path: ./Frontend/node\_modules/@istanbuljs/load-nyc-config\\
│  └─ licenseFile: ./Frontend/node\_modules/@istanbuljs/load-nyc-config/LICENSE\\
├─ @istanbuljs/schema@0.1.3\\
│  ├─ licenses: MIT\\
│  ├─ repository: https://github.com/istanbuljs/schema\\
│  ├─ publisher: Corey Farrell\\
│  ├─ path: ./Frontend/node\_modules/@istanbuljs/schema\\
│  └─ licenseFile: ./Frontend/node\_modules/@istanbuljs/schema/LICENSE\\
├─ @jridgewell/gen-mapping@0.3.5\\
│  ├─ licenses: MIT\\
│  ├─ repository: https://github.com/jridgewell/gen-mapping\\
│  ├─ publisher: Justin Ridgewell\\
│  ├─ email: justin@ridgewell.name\\
│  ├─ path: ./Frontend/node\_modules/@jridgewell/gen-mapping\\
│  └─ licenseFile: ./Frontend/node\_modules/@jridgewell/gen-mapping/LICENSE\\
├─ @jridgewell/resolve-uri@3.1.2\\
│  ├─ licenses: MIT\\
│  ├─ repository: https://github.com/jridgewell/resolve-uri\\
│  ├─ publisher: Justin Ridgewell\\
│  ├─ email: justin@ridgewell.name\\
│  ├─ path: ./Frontend/node\_modules/@jridgewell/resolve-uri\\
│  └─ licenseFile: ./Frontend/node\_modules/@jridgewell/resolve-uri/LICENSE\\
├─ @jridgewell/set-array@1.2.1\\
│  ├─ licenses: MIT\\
│  ├─ repository: https://github.com/jridgewell/set-array\\
│  ├─ publisher: Justin Ridgewell\\
│  ├─ email: justin@ridgewell.name\\
│  ├─ path: ./Frontend/node\_modules/@jridgewell/set-array\\
│  └─ licenseFile: ./Frontend/node\_modules/@jridgewell/set-array/LICENSE\\
├─ @jridgewell/source-map@0.3.6\\
│  ├─ licenses: MIT\\
│  ├─ repository: https://github.com/jridgewell/source-map\\
│  ├─ publisher: Justin Ridgewell\\
│  ├─ email: justin@ridgewell.name\\
│  ├─ path: ./Frontend/node\_modules/@jridgewell/source-map\\
│  └─ licenseFile: ./Frontend/node\_modules/@jridgewell/source-map/LICENSE\\
├─ @jridgewell/sourcemap-codec@1.4.15\\
│  ├─ licenses: MIT\\
│  ├─ repository: https://github.com/jridgewell/sourcemap-codec\\
│  ├─ publisher: Rich Harris\\
│  ├─ path: ./Frontend/node\_modules/@jridgewell/sourcemap-codec\\
│  └─ licenseFile: ./Frontend/node\_modules/@jridgewell/sourcemap-codec/LICENSE\\
├─ @jridgewell/trace-mapping@0.3.25\\
│  ├─ licenses: MIT\\
│  ├─ repository: https://github.com/jridgewell/trace-mapping\\
│  ├─ publisher: Justin Ridgewell\\
│  ├─ email: justin@ridgewell.name\\
│  ├─ path: ./Frontend/node\_modules/@jridgewell/trace-mapping\\
│  └─ licenseFile: ./Frontend/node\_modules/@jridgewell/trace-mapping/LICENSE\\
├─ @leichtgewicht/ip-codec@2.0.5\\
│  ├─ licenses: MIT\\
│  ├─ repository: https://github.com/martinheidegger/ip-codec\\
│  ├─ publisher: Martin Heidegger\\
│  ├─ path: ./Frontend/node\_modules/@leichtgewicht/ip-codec\\
│  └─ licenseFile: ./Frontend/node\_modules/@leichtgewicht/ip-codec/LICENSE\\
├─ @ljharb/through@2.3.13\\
│  ├─ licenses: MIT\\
│  ├─ repository: https://github.com/ljharb/through\\
│  ├─ publisher: Dominic Tarr\\
│  ├─ email: dominic.tarr@gmail.com\\
│  ├─ url: dominictarr.com\\
│  ├─ path: ./Frontend/node\_modules/@ljharb/through\\
│  └─ licenseFile: ./Frontend/node\_modules/@ljharb/through/LICENSE\\
├─ @lukeed/csprng@1.1.0\\
│  ├─ licenses: MIT\\
│  ├─ repository: https://github.com/lukeed/csprng\\
│  ├─ publisher: Luke Edwards\\
│  ├─ email: luke.edwards05@gmail.com\\
│  ├─ url: https://lukeed.com\\
│  ├─ path: ./Frontend/node\_modules/@lukeed/csprng\\
│  └─ licenseFile: ./Frontend/node\_modules/@lukeed/csprng/license\\
├─ @nestjs/axios@3.0.2\\
│  ├─ licenses: MIT\\
│  ├─ repository: https://github.com/nestjs/axios\\
│  ├─ publisher: Kamil Mysliwiec\\
│  ├─ path: ./Frontend/node\_modules/@openapitools/openapi-generator-cli/node\_modules/@nestjs/axios\\
│  └─ licenseFile: ./Frontend/node\_modules/@openapitools/openapi-generator-cli/node\_modules/@nestjs/axios/LICENSE\\
├─ @nestjs/common@10.3.0\\
│  ├─ licenses: MIT\\
│  ├─ repository: https://github.com/nestjs/nest\\
│  ├─ publisher: Kamil Mysliwiec\\
│  ├─ path: ./Frontend/node\_modules/@openapitools/openapi-generator-cli/node\_modules/@nestjs/common\\
│  └─ licenseFile: ./Frontend/node\_modules/@openapitools/openapi-generator-cli/node\_modules/@nestjs/common/LICENSE\\
├─ @nestjs/core@10.3.0\\
│  ├─ licenses: MIT\\
│  ├─ repository: https://github.com/nestjs/nest\\
│  ├─ publisher: Kamil Mysliwiec\\
│  ├─ path: ./Frontend/node\_modules/@openapitools/openapi-generator-cli/node\_modules/@nestjs/core\\
│  └─ licenseFile: ./Frontend/node\_modules/@openapitools/openapi-generator-cli/node\_modules/@nestjs/core/LICENSE\\
├─ @ngtools/webpack@17.3.7\\
│  ├─ licenses: MIT\\
│  ├─ repository: https://github.com/angular/angular-cli\\
│  ├─ publisher: Angular Authors\\
│  ├─ path: ./Frontend/node\_modules/@ngtools/webpack\\
│  └─ licenseFile: ./Frontend/node\_modules/@ngtools/webpack/LICENSE\\
├─ @nodelib/fs.scandir@2.1.5\\
│  ├─ licenses: MIT\\
│  ├─ repository: https://github.com/nodelib/nodelib/tree/master/packages/fs/fs.scandir\\
│  ├─ path: ./Frontend/node\_modules/@nodelib/fs.scandir\\
│  └─ licenseFile: ./Frontend/node\_modules/@nodelib/fs.scandir/LICENSE\\
├─ @nodelib/fs.stat@2.0.5\\
│  ├─ licenses: MIT\\
│  ├─ repository: https://github.com/nodelib/nodelib/tree/master/packages/fs/fs.stat\\
│  ├─ path: ./Frontend/node\_modules/@nodelib/fs.stat\\
│  └─ licenseFile: ./Frontend/node\_modules/@nodelib/fs.stat/LICENSE\\
├─ @nodelib/fs.walk@1.2.8\\
│  ├─ licenses: MIT\\
│  ├─ repository: https://github.com/nodelib/nodelib/tree/master/packages/fs/fs.walk\\
│  ├─ path: ./Frontend/node\_modules/@nodelib/fs.walk\\
│  └─ licenseFile: ./Frontend/node\_modules/@nodelib/fs.walk/LICENSE\\
├─ @npmcli/agent@2.2.2\\
│  ├─ licenses: ISC\\
│  ├─ repository: https://github.com/npm/agent\\
│  ├─ publisher: GitHub Inc.\\
│  ├─ path: ./Frontend/node\_modules/@npmcli/agent\\
│  └─ licenseFile: ./Frontend/node\_modules/@npmcli/agent/README.md\\
├─ @npmcli/fs@3.1.1\\
│  ├─ licenses: ISC\\
│  ├─ repository: https://github.com/npm/fs\\
│  ├─ publisher: GitHub Inc.\\
│  ├─ path: ./Frontend/node\_modules/@npmcli/fs\\
│  └─ licenseFile: ./Frontend/node\_modules/@npmcli/fs/LICENSE.md\\
├─ @npmcli/git@5.0.7\\
│  ├─ licenses: ISC\\
│  ├─ repository: https://github.com/npm/git\\
│  ├─ publisher: GitHub Inc.\\
│  ├─ path: ./Frontend/node\_modules/@npmcli/git\\
│  └─ licenseFile: ./Frontend/node\_modules/@npmcli/git/LICENSE\\
├─ @npmcli/installed-package-contents@2.1.0\\
│  ├─ licenses: ISC\\
│  ├─ repository: https://github.com/npm/installed-package-contents\\
│  ├─ publisher: GitHub Inc.\\
│  ├─ path: ./Frontend/node\_modules/@npmcli/installed-package-contents\\
│  └─ licenseFile: ./Frontend/node\_modules/@npmcli/installed-package-contents/LICENSE\\
├─ @npmcli/node-gyp@3.0.0\\
│  ├─ licenses: ISC\\
│  ├─ repository: https://github.com/npm/node-gyp\\
│  ├─ publisher: GitHub Inc.\\
│  ├─ path: ./Frontend/node\_modules/@npmcli/node-gyp\\
│  └─ licenseFile: ./Frontend/node\_modules/@npmcli/node-gyp/README.md\\
├─ @npmcli/package-json@5.1.0\\
│  ├─ licenses: ISC\\
│  ├─ repository: https://github.com/npm/package-json\\
│  ├─ publisher: GitHub Inc.\\
│  ├─ path: ./Frontend/node\_modules/@npmcli/package-json\\
│  └─ licenseFile: ./Frontend/node\_modules/@npmcli/package-json/LICENSE\\
├─ @npmcli/promise-spawn@7.0.2\\
│  ├─ licenses: ISC\\
│  ├─ repository: https://github.com/npm/promise-spawn\\
│  ├─ publisher: GitHub Inc.\\
│  ├─ path: ./Frontend/node\_modules/@npmcli/promise-spawn\\
│  └─ licenseFile: ./Frontend/node\_modules/@npmcli/promise-spawn/LICENSE\\
├─ @npmcli/redact@1.1.0\\
│  ├─ licenses: ISC\\
│  ├─ repository: https://github.com/npm/redact\\
│  ├─ publisher: GitHub Inc.\\
│  ├─ path: ./Frontend/node\_modules/@npmcli/redact\\
│  └─ licenseFile: ./Frontend/node\_modules/@npmcli/redact/LICENSE\\
├─ @npmcli/run-script@7.0.4\\
│  ├─ licenses: ISC\\
│  ├─ repository: https://github.com/npm/run-script\\
│  ├─ publisher: GitHub Inc.\\
│  ├─ path: ./Frontend/node\_modules/@npmcli/run-script\\
│  └─ licenseFile: ./Frontend/node\_modules/@npmcli/run-script/LICENSE\\
├─ @nuxtjs/opencollective@0.3.2\\
│  ├─ licenses: MIT\\
│  ├─ repository: https://github.com/nuxt-contrib/opencollective\\
│  ├─ path: ./Frontend/node\_modules/@nuxtjs/opencollective\\
│  └─ licenseFile: ./Frontend/node\_modules/@nuxtjs/opencollective/LICENSE\\
├─ @openapitools/openapi-generator-cli@2.13.4\\
│  ├─ licenses: Apache-2.0\\
│  ├─ repository: https://github.com/OpenAPITools/openapi-generator-cli\\
│  ├─ publisher: OpenAPI Tools\\
│  ├─ email: team@openapitools.org\\
│  ├─ url: https://openapitools.org/\\
│  ├─ path: ./Frontend/node\_modules/@openapitools/openapi-generator-cli\\
│  └─ licenseFile: ./Frontend/node\_modules/@openapitools/openapi-generator-cli/README.md\\
├─ @pkgjs/parseargs@0.11.0\\
│  ├─ licenses: MIT\\
│  ├─ repository: https://github.com/pkgjs/parseargs\\
│  ├─ path: ./Frontend/node\_modules/@pkgjs/parseargs\\
│  └─ licenseFile: ./Frontend/node\_modules/@pkgjs/parseargs/LICENSE\\
├─ @popperjs/core@2.11.8\\
│  ├─ licenses: MIT\\
│  ├─ repository: https://github.com/popperjs/popper-core\\
│  ├─ publisher: Federico Zivolo\\
│  ├─ email: federico.zivolo@gmail.com\\
│  ├─ path: ./Frontend/node\_modules/@popperjs/core\\
│  └─ licenseFile: ./Frontend/node\_modules/@popperjs/core/LICENSE.md\\
├─ @rollup/rollup-linux-x64-gnu@4.17.2\\
│  ├─ licenses: MIT\\
│  ├─ repository: https://github.com/rollup/rollup\\
│  ├─ publisher: Lukas Taegert-Atkinson\\
│  ├─ path: ./Frontend/node\_modules/@rollup/rollup-linux-x64-gnu\\
│  └─ licenseFile: ./Frontend/node\_modules/@rollup/rollup-linux-x64-gnu/README.md\\
├─ @rollup/rollup-linux-x64-musl@4.17.2\\
│  ├─ licenses: MIT\\
│  ├─ repository: https://github.com/rollup/rollup\\
│  ├─ publisher: Lukas Taegert-Atkinson\\
│  ├─ path: ./Frontend/node\_modules/@rollup/rollup-linux-x64-musl\\
│  └─ licenseFile: ./Frontend/node\_modules/@rollup/rollup-linux-x64-musl/README.md\\
├─ @schematics/angular@17.3.7\\
│  ├─ licenses: MIT\\
│  ├─ repository: https://github.com/angular/angular-cli\\
│  ├─ publisher: Angular Authors\\
│  ├─ path: ./Frontend/node\_modules/@schematics/angular\\
│  └─ licenseFile: ./Frontend/node\_modules/@schematics/angular/LICENSE\\
├─ @sigstore/bundle@2.3.1\\
│  ├─ licenses: Apache-2.0\\
│  ├─ repository: https://github.com/sigstore/sigstore-js\\
│  ├─ publisher: bdehamer@github.com\\
│  ├─ path: ./Frontend/node\_modules/@sigstore/bundle\\
│  └─ licenseFile: ./Frontend/node\_modules/@sigstore/bundle/LICENSE\\
├─ @sigstore/core@1.1.0\\
│  ├─ licenses: Apache-2.0\\
│  ├─ repository: https://github.com/sigstore/sigstore-js\\
│  ├─ publisher: bdehamer@github.com\\
│  ├─ path: ./Frontend/node\_modules/@sigstore/core\\
│  └─ licenseFile: ./Frontend/node\_modules/@sigstore/core/LICENSE\\
├─ @sigstore/protobuf-specs@0.3.2\\
│  ├─ licenses: Apache-2.0\\
│  ├─ repository: https://github.com/sigstore/protobuf-specs\\
│  ├─ publisher: bdehamer@github.com\\
│  ├─ path: ./Frontend/node\_modules/@sigstore/protobuf-specs\\
│  └─ licenseFile: ./Frontend/node\_modules/@sigstore/protobuf-specs/LICENSE\\
├─ @sigstore/sign@2.3.1\\
│  ├─ licenses: Apache-2.0\\
│  ├─ repository: https://github.com/sigstore/sigstore-js\\
│  ├─ publisher: bdehamer@github.com\\
│  ├─ path: ./Frontend/node\_modules/@sigstore/sign\\
│  └─ licenseFile: ./Frontend/node\_modules/@sigstore/sign/LICENSE\\
├─ @sigstore/tuf@2.3.3\\
│  ├─ licenses: Apache-2.0\\
│  ├─ repository: https://github.com/sigstore/sigstore-js\\
│  ├─ publisher: bdehamer@github.com\\
│  ├─ path: ./Frontend/node\_modules/@sigstore/tuf\\
│  └─ licenseFile: ./Frontend/node\_modules/@sigstore/tuf/LICENSE\\
├─ @sigstore/verify@1.2.0\\
│  ├─ licenses: Apache-2.0\\
│  ├─ repository: https://github.com/sigstore/sigstore-js\\
│  ├─ publisher: bdehamer@github.com\\
│  ├─ path: ./Frontend/node\_modules/@sigstore/verify\\
│  └─ licenseFile: ./Frontend/node\_modules/@sigstore/verify/README.md\\
├─ @socket.io/component-emitter@3.1.2\\
│  ├─ licenses: MIT\\
│  ├─ repository: https://github.com/socketio/emitter\\
│  ├─ path: ./Frontend/node\_modules/@socket.io/component-emitter\\
│  └─ licenseFile: ./Frontend/node\_modules/@socket.io/component-emitter/LICENSE\\
├─ @tufjs/canonical-json@2.0.0\\
│  ├─ licenses: MIT\\
│  ├─ repository: https://github.com/theupdateframework/tuf-js\\
│  ├─ publisher: bdehamer@github.com\\
│  ├─ path: ./Frontend/node\_modules/@tufjs/canonical-json\\
│  └─ licenseFile: ./Frontend/node\_modules/@tufjs/canonical-json/LICENSE\\
├─ @tufjs/models@2.0.1\\
│  ├─ licenses: MIT\\
│  ├─ repository: https://github.com/theupdateframework/tuf-js\\
│  ├─ publisher: bdehamer@github.com\\
│  ├─ path: ./Frontend/node\_modules/@tufjs/models\\
│  └─ licenseFile: ./Frontend/node\_modules/@tufjs/models/LICENSE\\
├─ @types/body-parser@1.19.5\\
│  ├─ licenses: MIT\\
│  ├─ repository: https://github.com/DefinitelyTyped/DefinitelyTyped\\
│  ├─ path: ./Frontend/node\_modules/@types/body-parser\\
│  └─ licenseFile: ./Frontend/node\_modules/@types/body-parser/LICENSE\\
├─ @types/bonjour@3.5.13\\
│  ├─ licenses: MIT\\
│  ├─ repository: https://github.com/DefinitelyTyped/DefinitelyTyped\\
│  ├─ path: ./Frontend/node\_modules/@types/bonjour\\
│  └─ licenseFile: ./Frontend/node\_modules/@types/bonjour/LICENSE\\
├─ @types/connect-history-api-fallback@1.5.4\\
│  ├─ licenses: MIT\\
│  ├─ repository: https://github.com/DefinitelyTyped/DefinitelyTyped\\
│  ├─ path: ./Frontend/node\_modules/@types/connect-history-api-fallback\\
│  └─ licenseFile: ./Frontend/node\_modules/@types/connect-history-api-fallback/LICENSE\\
├─ @types/connect@3.4.38\\
│  ├─ licenses: MIT\\
│  ├─ repository: https://github.com/DefinitelyTyped/DefinitelyTyped\\
│  ├─ path: ./Frontend/node\_modules/@types/connect\\
│  └─ licenseFile: ./Frontend/node\_modules/@types/connect/LICENSE\\
├─ @types/cookie@0.4.1\\
│  ├─ licenses: MIT\\
│  ├─ repository: https://github.com/DefinitelyTyped/DefinitelyTyped\\
│  ├─ path: ./Frontend/node\_modules/@types/cookie\\
│  └─ licenseFile: ./Frontend/node\_modules/@types/cookie/LICENSE\\
├─ @types/cors@2.8.17\\
│  ├─ licenses: MIT\\
│  ├─ repository: https://github.com/DefinitelyTyped/DefinitelyTyped\\
│  ├─ path: ./Frontend/node\_modules/@types/cors\\
│  └─ licenseFile: ./Frontend/node\_modules/@types/cors/LICENSE\\
├─ @types/eslint-scope@3.7.7\\
│  ├─ licenses: MIT\\
│  ├─ repository: https://github.com/DefinitelyTyped/DefinitelyTyped\\
│  ├─ path: ./Frontend/node\_modules/@types/eslint-scope\\
│  └─ licenseFile: ./Frontend/node\_modules/@types/eslint-scope/LICENSE\\
├─ @types/eslint@8.56.10\\
│  ├─ licenses: MIT\\
│  ├─ repository: https://github.com/DefinitelyTyped/DefinitelyTyped\\
│  ├─ path: ./Frontend/node\_modules/@types/eslint\\
│  └─ licenseFile: ./Frontend/node\_modules/@types/eslint/LICENSE\\
├─ @types/estree@1.0.5\\
│  ├─ licenses: MIT\\
│  ├─ repository: https://github.com/DefinitelyTyped/DefinitelyTyped\\
│  ├─ path: ./Frontend/node\_modules/@types/estree\\
│  └─ licenseFile: ./Frontend/node\_modules/@types/estree/LICENSE\\
├─ @types/express-serve-static-core@4.19.0\\
│  ├─ licenses: MIT\\
│  ├─ repository: https://github.com/DefinitelyTyped/DefinitelyTyped\\
│  ├─ path: ./Frontend/node\_modules/@types/express-serve-static-core\\
│  └─ licenseFile: ./Frontend/node\_modules/@types/express-serve-static-core/LICENSE\\
├─ @types/express@4.17.21\\
│  ├─ licenses: MIT\\
│  ├─ repository: https://github.com/DefinitelyTyped/DefinitelyTyped\\
│  ├─ path: ./Frontend/node\_modules/@types/express\\
│  └─ licenseFile: ./Frontend/node\_modules/@types/express/LICENSE\\
├─ @types/http-errors@2.0.4\\
│  ├─ licenses: MIT\\
│  ├─ repository: https://github.com/DefinitelyTyped/DefinitelyTyped\\
│  ├─ path: ./Frontend/node\_modules/@types/http-errors\\
│  └─ licenseFile: ./Frontend/node\_modules/@types/http-errors/LICENSE\\
├─ @types/http-proxy@1.17.14\\
│  ├─ licenses: MIT\\
│  ├─ repository: https://github.com/DefinitelyTyped/DefinitelyTyped\\
│  ├─ path: ./Frontend/node\_modules/@types/http-proxy\\
│  └─ licenseFile: ./Frontend/node\_modules/@types/http-proxy/LICENSE\\
├─ @types/jasmine@5.1.4\\
│  ├─ licenses: MIT\\
│  ├─ repository: https://github.com/DefinitelyTyped/DefinitelyTyped\\
│  ├─ path: ./Frontend/node\_modules/@types/jasmine\\
│  └─ licenseFile: ./Frontend/node\_modules/@types/jasmine/LICENSE\\
├─ @types/json-schema@7.0.15\\
│  ├─ licenses: MIT\\
│  ├─ repository: https://github.com/DefinitelyTyped/DefinitelyTyped\\
│  ├─ path: ./Frontend/node\_modules/@types/json-schema\\
│  └─ licenseFile: ./Frontend/node\_modules/@types/json-schema/LICENSE\\
├─ @types/mime@1.3.5\\
│  ├─ licenses: MIT\\
│  ├─ repository: https://github.com/DefinitelyTyped/DefinitelyTyped\\
│  ├─ path: ./Frontend/node\_modules/@types/mime\\
│  └─ licenseFile: ./Frontend/node\_modules/@types/mime/LICENSE\\
├─ @types/node-forge@1.3.11\\
│  ├─ licenses: MIT\\
│  ├─ repository: https://github.com/DefinitelyTyped/DefinitelyTyped\\
│  ├─ path: ./Frontend/node\_modules/@types/node-forge\\
│  └─ licenseFile: ./Frontend/node\_modules/@types/node-forge/LICENSE\\
├─ @types/node@20.12.12\\
│  ├─ licenses: MIT\\
│  ├─ repository: https://github.com/DefinitelyTyped/DefinitelyTyped\\
│  ├─ path: ./Frontend/node\_modules/@types/node\\
│  └─ licenseFile: ./Frontend/node\_modules/@types/node/LICENSE\\
├─ @types/qs@6.9.15\\
│  ├─ licenses: MIT\\
│  ├─ repository: https://github.com/DefinitelyTyped/DefinitelyTyped\\
│  ├─ path: ./Frontend/node\_modules/@types/qs\\
│  └─ licenseFile: ./Frontend/node\_modules/@types/qs/LICENSE\\
├─ @types/range-parser@1.2.7\\
│  ├─ licenses: MIT\\
│  ├─ repository: https://github.com/DefinitelyTyped/DefinitelyTyped\\
│  ├─ path: ./Frontend/node\_modules/@types/range-parser\\
│  └─ licenseFile: ./Frontend/node\_modules/@types/range-parser/LICENSE\\
├─ @types/retry@0.12.0\\
│  ├─ licenses: MIT\\
│  ├─ repository: https://github.com/DefinitelyTyped/DefinitelyTyped\\
│  ├─ path: ./Frontend/node\_modules/@types/retry\\
│  └─ licenseFile: ./Frontend/node\_modules/@types/retry/LICENSE\\
├─ @types/send@0.17.4\\
│  ├─ licenses: MIT\\
│  ├─ repository: https://github.com/DefinitelyTyped/DefinitelyTyped\\
│  ├─ path: ./Frontend/node\_modules/@types/send\\
│  └─ licenseFile: ./Frontend/node\_modules/@types/send/LICENSE\\
├─ @types/serve-index@1.9.4\\
│  ├─ licenses: MIT\\
│  ├─ repository: https://github.com/DefinitelyTyped/DefinitelyTyped\\
│  ├─ path: ./Frontend/node\_modules/@types/serve-index\\
│  └─ licenseFile: ./Frontend/node\_modules/@types/serve-index/LICENSE\\
├─ @types/serve-static@1.15.7\\
│  ├─ licenses: MIT\\
│  ├─ repository: https://github.com/DefinitelyTyped/DefinitelyTyped\\
│  ├─ path: ./Frontend/node\_modules/@types/serve-static\\
│  └─ licenseFile: ./Frontend/node\_modules/@types/serve-static/LICENSE\\
├─ @types/sockjs@0.3.36\\
│  ├─ licenses: MIT\\
│  ├─ repository: https://github.com/DefinitelyTyped/DefinitelyTyped\\
│  ├─ path: ./Frontend/node\_modules/@types/sockjs\\
│  └─ licenseFile: ./Frontend/node\_modules/@types/sockjs/LICENSE\\
├─ @types/ws@8.5.10\\
│  ├─ licenses: MIT\\
│  ├─ repository: https://github.com/DefinitelyTyped/DefinitelyTyped\\
│  ├─ path: ./Frontend/node\_modules/@types/ws\\
│  └─ licenseFile: ./Frontend/node\_modules/@types/ws/LICENSE\\
├─ @vitejs/plugin-basic-ssl@1.1.0\\
│  ├─ licenses: MIT\\
│  ├─ repository: https://github.com/vitejs/vite-plugin-basic-ssl\\
│  ├─ publisher: Evan You and Vite Contributors\\
│  ├─ path: ./Frontend/node\_modules/@vitejs/plugin-basic-ssl\\
│  └─ licenseFile: ./Frontend/node\_modules/@vitejs/plugin-basic-ssl/LICENSE\\
├─ @webassemblyjs/ast@1.12.1\\
│  ├─ licenses: MIT\\
│  ├─ repository: https://github.com/xtuc/webassemblyjs\\
│  ├─ publisher: Sven Sauleau\\
│  ├─ path: ./Frontend/node\_modules/@webassemblyjs/ast\\
│  └─ licenseFile: ./Frontend/node\_modules/@webassemblyjs/ast/LICENSE\\
├─ @webassemblyjs/floating-point-hex-parser@1.11.6\\
│  ├─ licenses: MIT\\
│  ├─ repository: https://github.com/xtuc/webassemblyjs\\
│  ├─ publisher: Mauro Bringolf\\
│  ├─ path: ./Frontend/node\_modules/@webassemblyjs/floating-point-hex-parser\\
│  └─ licenseFile: ./Frontend/node\_modules/@webassemblyjs/floating-point-hex-parser/LICENSE\\
├─ @webassemblyjs/helper-api-error@1.11.6\\
│  ├─ licenses: MIT\\
│  ├─ repository: https://github.com/xtuc/webassemblyjs\\
│  ├─ publisher: Sven Sauleau\\
│  └─ path: ./Frontend/node\_modules/@webassemblyjs/helper-api-error\\
├─ @webassemblyjs/helper-buffer@1.12.1\\
│  ├─ licenses: MIT\\
│  ├─ repository: https://github.com/xtuc/webassemblyjs\\
│  ├─ publisher: Sven Sauleau\\
│  ├─ path: ./Frontend/node\_modules/@webassemblyjs/helper-buffer\\
│  └─ licenseFile: ./Frontend/node\_modules/@webassemblyjs/helper-buffer/LICENSE\\
├─ @webassemblyjs/helper-numbers@1.11.6\\
│  ├─ licenses: MIT\\
│  ├─ repository: https://github.com/xtuc/webassemblyjs\\
│  ├─ publisher: Sven Sauleau\\
│  └─ path: ./Frontend/node\_modules/@webassemblyjs/helper-numbers\\
├─ @webassemblyjs/helper-wasm-bytecode@1.11.6\\
│  ├─ licenses: MIT\\
│  ├─ repository: https://github.com/xtuc/webassemblyjs\\
│  ├─ publisher: Sven Sauleau\\
│  └─ path: ./Frontend/node\_modules/@webassemblyjs/helper-wasm-bytecode\\
├─ @webassemblyjs/helper-wasm-section@1.12.1\\
│  ├─ licenses: MIT\\
│  ├─ repository: https://github.com/xtuc/webassemblyjs\\
│  ├─ publisher: Sven Sauleau\\
│  ├─ path: ./Frontend/node\_modules/@webassemblyjs/helper-wasm-section\\
│  └─ licenseFile: ./Frontend/node\_modules/@webassemblyjs/helper-wasm-section/LICENSE\\
├─ @webassemblyjs/ieee754@1.11.6\\
│  ├─ licenses: MIT\\
│  ├─ repository: https://github.com/xtuc/webassemblyjs\\
│  └─ path: ./Frontend/node\_modules/@webassemblyjs/ieee754\\
├─ @webassemblyjs/leb128@1.11.6\\
│  ├─ licenses: Apache-2.0\\
│  ├─ repository: https://github.com/xtuc/webassemblyjs\\
│  ├─ path: ./Frontend/node\_modules/@webassemblyjs/leb128\\
│  └─ licenseFile: ./Frontend/node\_modules/@webassemblyjs/leb128/LICENSE.txt\\
├─ @webassemblyjs/utf8@1.11.6\\
│  ├─ licenses: MIT\\
│  ├─ repository: https://github.com/xtuc/webassemblyjs\\
│  ├─ publisher: Sven Sauleau\\
│  └─ path: ./Frontend/node\_modules/@webassemblyjs/utf8\\
├─ @webassemblyjs/wasm-edit@1.12.1\\
│  ├─ licenses: MIT\\
│  ├─ repository: https://github.com/xtuc/webassemblyjs\\
│  ├─ publisher: Sven Sauleau\\
│  ├─ path: ./Frontend/node\_modules/@webassemblyjs/wasm-edit\\
│  └─ licenseFile: ./Frontend/node\_modules/@webassemblyjs/wasm-edit/LICENSE\\
├─ @webassemblyjs/wasm-gen@1.12.1\\
│  ├─ licenses: MIT\\
│  ├─ repository: https://github.com/xtuc/webassemblyjs\\
│  ├─ publisher: Sven Sauleau\\
│  ├─ path: ./Frontend/node\_modules/@webassemblyjs/wasm-gen\\
│  └─ licenseFile: ./Frontend/node\_modules/@webassemblyjs/wasm-gen/LICENSE\\
├─ @webassemblyjs/wasm-opt@1.12.1\\
│  ├─ licenses: MIT\\
│  ├─ repository: https://github.com/xtuc/webassemblyjs\\
│  ├─ publisher: Sven Sauleau\\
│  ├─ path: ./Frontend/node\_modules/@webassemblyjs/wasm-opt\\
│  └─ licenseFile: ./Frontend/node\_modules/@webassemblyjs/wasm-opt/LICENSE\\
├─ @webassemblyjs/wasm-parser@1.12.1\\
│  ├─ licenses: MIT\\
│  ├─ repository: https://github.com/xtuc/webassemblyjs\\
│  ├─ publisher: Sven Sauleau\\
│  ├─ path: ./Frontend/node\_modules/@webassemblyjs/wasm-parser\\
│  └─ licenseFile: ./Frontend/node\_modules/@webassemblyjs/wasm-parser/LICENSE\\
├─ @webassemblyjs/wast-printer@1.12.1\\
│  ├─ licenses: MIT\\
│  ├─ repository: https://github.com/xtuc/webassemblyjs\\
│  ├─ publisher: Sven Sauleau\\
│  ├─ path: ./Frontend/node\_modules/@webassemblyjs/wast-printer\\
│  └─ licenseFile: ./Frontend/node\_modules/@webassemblyjs/wast-printer/LICENSE\\
├─ @xtuc/ieee754@1.2.0\\
│  ├─ licenses: BSD-3-Clause\\
│  ├─ repository: https://github.com/feross/ieee754\\
│  ├─ publisher: Feross Aboukhadijeh\\
│  ├─ email: feross@feross.org\\
│  ├─ url: http://feross.org\\
│  ├─ path: ./Frontend/node\_modules/@xtuc/ieee754\\
│  └─ licenseFile: ./Frontend/node\_modules/@xtuc/ieee754/LICENSE\\
├─ @xtuc/long@4.2.2\\
│  ├─ licenses: Apache-2.0\\
│  ├─ repository: https://github.com/dcodeIO/long.js\\
│  ├─ publisher: Daniel Wirtz\\
│  ├─ email: dcode@dcode.io\\
│  ├─ path: ./Frontend/node\_modules/@xtuc/long\\
│  └─ licenseFile: ./Frontend/node\_modules/@xtuc/long/LICENSE\\
├─ @yarnpkg/lockfile@1.1.0\\
│  ├─ licenses: BSD-2-Clause\\
│  ├─ repository: https://github.com/yarnpkg/yarn/blob/master/packages/lockfile\\
│  ├─ path: ./Frontend/node\_modules/@yarnpkg/lockfile\\
│  └─ licenseFile: ./Frontend/node\_modules/@yarnpkg/lockfile/README.md\\
├─ abbrev@2.0.0\\
│  ├─ licenses: ISC\\
│  ├─ repository: https://github.com/npm/abbrev-js\\
│  ├─ publisher: GitHub Inc.\\
│  ├─ path: ./Frontend/node\_modules/abbrev\\
│  └─ licenseFile: ./Frontend/node\_modules/abbrev/LICENSE\\
├─ accepts@1.3.8\\
│  ├─ licenses: MIT\\
│  ├─ repository: https://github.com/jshttp/accepts\\
│  ├─ path: ./Frontend/node\_modules/accepts\\
│  └─ licenseFile: ./Frontend/node\_modules/accepts/LICENSE\\
├─ acorn-import-assertions@1.9.0\\
│  ├─ licenses: MIT\\
│  ├─ repository: https://github.com/xtuc/acorn-import-assertions\\
│  ├─ publisher: Sven Sauleau\\
│  ├─ email: sven@sauleau.com\\
│  ├─ path: ./Frontend/node\_modules/acorn-import-assertions\\
│  └─ licenseFile: ./Frontend/node\_modules/acorn-import-assertions/README.md\\
├─ acorn@8.11.3\\
│  ├─ licenses: MIT\\
│  ├─ repository: https://github.com/acornjs/acorn\\
│  ├─ path: ./Frontend/node\_modules/acorn\\
│  └─ licenseFile: ./Frontend/node\_modules/acorn/LICENSE\\
├─ adjust-sourcemap-loader@4.0.0\\
│  ├─ licenses: MIT\\
│  ├─ repository: https://github.com/bholloway/adjust-sourcemap-loader\\
│  ├─ publisher: bholloway\\
│  ├─ path: ./Frontend/node\_modules/adjust-sourcemap-loader\\
│  └─ licenseFile: ./Frontend/node\_modules/adjust-sourcemap-loader/LICENSE\\
├─ agent-base@7.1.1\\
│  ├─ licenses: MIT\\
│  ├─ repository: https://github.com/TooTallNate/proxy-agents\\
│  ├─ publisher: Nathan Rajlich\\
│  ├─ email: nathan@tootallnate.net\\
│  ├─ url: http://n8.io/\\
│  ├─ path: ./Frontend/node\_modules/agent-base\\
│  └─ licenseFile: ./Frontend/node\_modules/agent-base/LICENSE\\
├─ aggregate-error@3.1.0\\
│  ├─ licenses: MIT\\
│  ├─ repository: https://github.com/sindresorhus/aggregate-error\\
│  ├─ publisher: Sindre Sorhus\\
│  ├─ email: sindresorhus@gmail.com\\
│  ├─ url: sindresorhus.com\\
│  ├─ path: ./Frontend/node\_modules/aggregate-error\\
│  └─ licenseFile: ./Frontend/node\_modules/aggregate-error/license\\
├─ ajv-formats@2.1.1\\
│  ├─ licenses: MIT\\
│  ├─ repository: https://github.com/ajv-validator/ajv-formats\\
│  ├─ publisher: Evgeny Poberezkin\\
│  ├─ path: ./Frontend/node\_modules/ajv-formats\\
│  └─ licenseFile: ./Frontend/node\_modules/ajv-formats/LICENSE\\
├─ ajv-keywords@3.5.2\\
│  ├─ licenses: MIT\\
│  ├─ repository: https://github.com/epoberezkin/ajv-keywords\\
│  ├─ publisher: Evgeny Poberezkin\\
│  ├─ path: ./Frontend/node\_modules/webpack/node\_modules/ajv-keywords\\
│  └─ licenseFile: ./Frontend/node\_modules/webpack/node\_modules/ajv-keywords/LICENSE\\
├─ ajv-keywords@5.1.0\\
│  ├─ licenses: MIT\\
│  ├─ repository: https://github.com/epoberezkin/ajv-keywords\\
│  ├─ publisher: Evgeny Poberezkin\\
│  ├─ path: ./Frontend/node\_modules/ajv-keywords\\
│  └─ licenseFile: ./Frontend/node\_modules/ajv-keywords/LICENSE\\
├─ ajv@6.12.6\\
│  ├─ licenses: MIT\\
│  ├─ repository: https://github.com/ajv-validator/ajv\\
│  ├─ publisher: Evgeny Poberezkin\\
│  ├─ path: ./Frontend/node\_modules/webpack/node\_modules/ajv\\
│  └─ licenseFile: ./Frontend/node\_modules/webpack/node\_modules/ajv/LICENSE\\
├─ ajv@8.12.0\\
│  ├─ licenses: MIT\\
│  ├─ repository: https://github.com/ajv-validator/ajv\\
│  ├─ publisher: Evgeny Poberezkin\\
│  ├─ path: ./Frontend/node\_modules/ajv\\
│  └─ licenseFile: ./Frontend/node\_modules/ajv/LICENSE\\
├─ angular-iban@17.0.0\\
│  ├─ licenses: MIT\\
│  ├─ repository: https://github.com/fundsaccess/angular-iban\\
│  ├─ publisher: fundsaccess AG\\
│  ├─ path: ./Frontend/node\_modules/angular-iban\\
│  └─ licenseFile: ./Frontend/node\_modules/angular-iban/LICENSE\\
├─ angular-oauth2-oidc@17.0.2\\
│  ├─ licenses: MIT\\
│  ├─ repository: https://github.com/manfredsteyer/angular-oauth2-oidc\\
│  ├─ publisher: Manfred Steyer\\
│  ├─ path: ./Frontend/node\_modules/angular-oauth2-oidc\\
│  └─ licenseFile: ./Frontend/node\_modules/angular-oauth2-oidc/LICENSE\\
├─ ansi-colors@4.1.3\\
│  ├─ licenses: MIT\\
│  ├─ repository: https://github.com/doowb/ansi-colors\\
│  ├─ publisher: Brian Woodward\\
│  ├─ url: https://github.com/doowb\\
│  ├─ path: ./Frontend/node\_modules/ansi-colors\\
│  └─ licenseFile: ./Frontend/node\_modules/ansi-colors/LICENSE\\
├─ ansi-escapes@4.3.2\\
│  ├─ licenses: MIT\\
│  ├─ repository: https://github.com/sindresorhus/ansi-escapes\\
│  ├─ publisher: Sindre Sorhus\\
│  ├─ email: sindresorhus@gmail.com\\
│  ├─ url: https://sindresorhus.com\\
│  ├─ path: ./Frontend/node\_modules/ansi-escapes\\
│  └─ licenseFile: ./Frontend/node\_modules/ansi-escapes/license\\
├─ ansi-html-community@0.0.8\\
│  ├─ licenses: Apache-2.0\\
│  ├─ repository: https://github.com/mahdyar/ansi-html-community\\
│  ├─ publisher: mahdyar\\
│  ├─ path: ./Frontend/node\_modules/ansi-html-community\\
│  └─ licenseFile: ./Frontend/node\_modules/ansi-html-community/LICENSE\\
├─ ansi-regex@5.0.1\\
│  ├─ licenses: MIT\\
│  ├─ repository: https://github.com/chalk/ansi-regex\\
│  ├─ publisher: Sindre Sorhus\\
│  ├─ email: sindresorhus@gmail.com\\
│  ├─ url: sindresorhus.com\\
│  ├─ path: ./Frontend/node\_modules/ansi-regex\\
│  └─ licenseFile: ./Frontend/node\_modules/ansi-regex/license\\
├─ ansi-regex@6.0.1\\
│  ├─ licenses: MIT\\
│  ├─ repository: https://github.com/chalk/ansi-regex\\
│  ├─ publisher: Sindre Sorhus\\
│  ├─ email: sindresorhus@gmail.com\\
│  ├─ url: https://sindresorhus.com\\
│  ├─ path: ./Frontend/node\_modules/@isaacs/cliui/node\_modules/ansi-regex\\
│  └─ licenseFile: ./Frontend/node\_modules/@isaacs/cliui/node\_modules/ansi-regex/license\\
├─ ansi-styles@3.2.1\\
│  ├─ licenses: MIT\\
│  ├─ repository: https://github.com/chalk/ansi-styles\\
│  ├─ publisher: Sindre Sorhus\\
│  ├─ email: sindresorhus@gmail.com\\
│  ├─ url: sindresorhus.com\\
│  ├─ path: ./Frontend/node\_modules/ansi-styles\\
│  └─ licenseFile: ./Frontend/node\_modules/ansi-styles/license\\
├─ ansi-styles@4.3.0\\
│  ├─ licenses: MIT\\
│  ├─ repository: https://github.com/chalk/ansi-styles\\
│  ├─ publisher: Sindre Sorhus\\
│  ├─ email: sindresorhus@gmail.com\\
│  ├─ url: sindresorhus.com\\
│  ├─ path: ./Frontend/node\_modules/critters/node\_modules/ansi-styles\\
│  └─ licenseFile: ./Frontend/node\_modules/critters/node\_modules/ansi-styles/license\\
├─ ansi-styles@6.2.1\\
│  ├─ licenses: MIT\\
│  ├─ repository: https://github.com/chalk/ansi-styles\\
│  ├─ publisher: Sindre Sorhus\\
│  ├─ email: sindresorhus@gmail.com\\
│  ├─ url: https://sindresorhus.com\\
│  ├─ path: ./Frontend/node\_modules/@isaacs/cliui/node\_modules/ansi-styles\\
│  └─ licenseFile: ./Frontend/node\_modules/@isaacs/cliui/node\_modules/ansi-styles/license\\
├─ anymatch@3.1.3\\
│  ├─ licenses: ISC\\
│  ├─ repository: https://github.com/micromatch/anymatch\\
│  ├─ publisher: Elan Shanker\\
│  ├─ url: https://github.com/es128\\
│  ├─ path: ./Frontend/node\_modules/anymatch\\
│  └─ licenseFile: ./Frontend/node\_modules/anymatch/LICENSE\\
├─ argparse@1.0.10\\
│  ├─ licenses: MIT\\
│  ├─ repository: https://github.com/nodeca/argparse\\
│  ├─ path: ./Frontend/node\_modules/argparse\\
│  └─ licenseFile: ./Frontend/node\_modules/argparse/LICENSE\\
├─ argparse@2.0.1\\
│  ├─ licenses: Python-2.0\\
│  ├─ repository: https://github.com/nodeca/argparse\\
│  ├─ path: ./Frontend/node\_modules/cosmiconfig/node\_modules/argparse\\
│  └─ licenseFile: ./Frontend/node\_modules/cosmiconfig/node\_modules/argparse/LICENSE\\
├─ array-flatten@1.1.1\\
│  ├─ licenses: MIT\\
│  ├─ repository: https://github.com/blakeembrey/array-flatten\\
│  ├─ publisher: Blake Embrey\\
│  ├─ email: hello@blakeembrey.com\\
│  ├─ url: http://blakeembrey.me\\
│  ├─ path: ./Frontend/node\_modules/array-flatten\\
│  └─ licenseFile: ./Frontend/node\_modules/array-flatten/LICENSE\\
├─ asynckit@0.4.0\\
│  ├─ licenses: MIT\\
│  ├─ repository: https://github.com/alexindigo/asynckit\\
│  ├─ publisher: Alex Indigo\\
│  ├─ email: iam@alexindigo.com\\
│  ├─ path: ./Frontend/node\_modules/asynckit\\
│  └─ licenseFile: ./Frontend/node\_modules/asynckit/LICENSE\\
├─ autoprefixer@10.4.18\\
│  ├─ licenses: MIT\\
│  ├─ repository: https://github.com/postcss/autoprefixer\\
│  ├─ publisher: Andrey Sitnik\\
│  ├─ email: andrey@sitnik.ru\\
│  ├─ path: ./Frontend/node\_modules/autoprefixer\\
│  └─ licenseFile: ./Frontend/node\_modules/autoprefixer/LICENSE\\
├─ axios@1.6.8\\
│  ├─ licenses: MIT\\
│  ├─ repository: https://github.com/axios/axios\\
│  ├─ publisher: Matt Zabriskie\\
│  ├─ path: ./Frontend/node\_modules/axios\\
│  └─ licenseFile: ./Frontend/node\_modules/axios/LICENSE\\
├─ babel-loader@9.1.3\\
│  ├─ licenses: MIT\\
│  ├─ repository: https://github.com/babel/babel-loader\\
│  ├─ publisher: Luis Couto\\
│  ├─ email: hello@luiscouto.pt\\
│  ├─ path: ./Frontend/node\_modules/babel-loader\\
│  └─ licenseFile: ./Frontend/node\_modules/babel-loader/LICENSE\\
├─ babel-plugin-istanbul@6.1.1\\
│  ├─ licenses: BSD-3-Clause\\
│  ├─ repository: https://github.com/istanbuljs/babel-plugin-istanbul\\
│  ├─ publisher: Thai Pangsakulyanont @dtinth\\
│  ├─ path: ./Frontend/node\_modules/babel-plugin-istanbul\\
│  └─ licenseFile: ./Frontend/node\_modules/babel-plugin-istanbul/LICENSE\\
├─ babel-plugin-polyfill-corejs2@0.4.11\\
│  ├─ licenses: MIT\\
│  ├─ repository: https://github.com/babel/babel-polyfills\\
│  ├─ path: ./Frontend/node\_modules/babel-plugin-polyfill-corejs2\\
│  └─ licenseFile: ./Frontend/node\_modules/babel-plugin-polyfill-corejs2/LICENSE\\
├─ babel-plugin-polyfill-corejs3@0.9.0\\
│  ├─ licenses: MIT\\
│  ├─ repository: https://github.com/babel/babel-polyfills\\
│  ├─ path: ./Frontend/node\_modules/babel-plugin-polyfill-corejs3\\
│  └─ licenseFile: ./Frontend/node\_modules/babel-plugin-polyfill-corejs3/LICENSE\\
├─ babel-plugin-polyfill-regenerator@0.5.5\\
│  ├─ licenses: MIT\\
│  ├─ repository: https://github.com/babel/babel-polyfills\\
│  ├─ path: ./Frontend/node\_modules/babel-plugin-polyfill-regenerator\\
│  └─ licenseFile: ./Frontend/node\_modules/babel-plugin-polyfill-regenerator/LICENSE\\
├─ balanced-match@1.0.2\\
│  ├─ licenses: MIT\\
│  ├─ repository: https://github.com/juliangruber/balanced-match\\
│  ├─ publisher: Julian Gruber\\
│  ├─ email: mail@juliangruber.com\\
│  ├─ url: http://juliangruber.com\\
│  ├─ path: ./Frontend/node\_modules/balanced-match\\
│  └─ licenseFile: ./Frontend/node\_modules/balanced-match/LICENSE.md\\
├─ base64-js@1.5.1\\
│  ├─ licenses: MIT\\
│  ├─ repository: https://github.com/beatgammit/base64-js\\
│  ├─ publisher: T. Jameson Little\\
│  ├─ email: t.jameson.little@gmail.com\\
│  ├─ path: ./Frontend/node\_modules/base64-js\\
│  └─ licenseFile: ./Frontend/node\_modules/base64-js/LICENSE\\
├─ base64id@2.0.0\\
│  ├─ licenses: MIT\\
│  ├─ repository: https://github.com/faeldt/base64id\\
│  ├─ publisher: Kristian Faeldt\\
│  ├─ email: faeldt\_kristian@cyberagent.co.jp\\
│  ├─ path: ./Frontend/node\_modules/base64id\\
│  └─ licenseFile: ./Frontend/node\_modules/base64id/LICENSE\\
├─ batch@0.6.1\\
│  ├─ licenses: MIT\\
│  ├─ repository: https://github.com/visionmedia/batch\\
│  ├─ publisher: TJ Holowaychuk\\
│  ├─ email: tj@vision-media.ca\\
│  ├─ path: ./Frontend/node\_modules/batch\\
│  └─ licenseFile: ./Frontend/node\_modules/batch/LICENSE\\
├─ big.js@5.2.2\\
│  ├─ licenses: MIT\\
│  ├─ repository: https://github.com/MikeMcl/big.js\\
│  ├─ publisher: Michael Mclaughlin\\
│  ├─ email: M8ch88l@gmail.com\\
│  ├─ path: ./Frontend/node\_modules/big.js\\
│  └─ licenseFile: ./Frontend/node\_modules/big.js/LICENCE\\
├─ binary-extensions@2.3.0\\
│  ├─ licenses: MIT\\
│  ├─ repository: https://github.com/sindresorhus/binary-extensions\\
│  ├─ publisher: Sindre Sorhus\\
│  ├─ email: sindresorhus@gmail.com\\
│  ├─ url: https://sindresorhus.com\\
│  ├─ path: ./Frontend/node\_modules/binary-extensions\\
│  └─ licenseFile: ./Frontend/node\_modules/binary-extensions/license\\
├─ bl@4.1.0\\
│  ├─ licenses: MIT\\
│  ├─ repository: https://github.com/rvagg/bl\\
│  ├─ path: ./Frontend/node\_modules/bl\\
│  └─ licenseFile: ./Frontend/node\_modules/bl/LICENSE.md\\
├─ body-parser@1.20.2\\
│  ├─ licenses: MIT\\
│  ├─ repository: https://github.com/expressjs/body-parser\\
│  ├─ path: ./Frontend/node\_modules/body-parser\\
│  └─ licenseFile: ./Frontend/node\_modules/body-parser/LICENSE\\
├─ bonjour-service@1.2.1\\
│  ├─ licenses: MIT\\
│  ├─ repository: https://github.com/onlxltd/bonjour-service\\
│  ├─ publisher: ON LX Lited\\
│  ├─ email: team@onlx.ltd\\
│  ├─ url: https://labs.onlx.ltd\\
│  ├─ path: ./Frontend/node\_modules/bonjour-service\\
│  └─ licenseFile: ./Frontend/node\_modules/bonjour-service/LICENSE\\
├─ boolbase@1.0.0\\
│  ├─ licenses: ISC\\
│  ├─ repository: https://github.com/fb55/boolbase\\
│  ├─ publisher: Felix Boehm\\
│  ├─ email: me@feedic.com\\
│  ├─ path: ./Frontend/node\_modules/boolbase\\
│  └─ licenseFile: ./Frontend/node\_modules/boolbase/README.md\\
├─ bootstrap@5.3.3\\
│  ├─ licenses: MIT\\
│  ├─ repository: https://github.com/twbs/bootstrap\\
│  ├─ publisher: The Bootstrap Authors\\
│  ├─ url: https://github.com/twbs/bootstrap/graphs/contributors\\
│  ├─ path: ./Frontend/node\_modules/bootstrap\\
│  └─ licenseFile: ./Frontend/node\_modules/bootstrap/LICENSE\\
├─ brace-expansion@1.1.11\\
│  ├─ licenses: MIT\\
│  ├─ repository: https://github.com/juliangruber/brace-expansion\\
│  ├─ publisher: Julian Gruber\\
│  ├─ email: mail@juliangruber.com\\
│  ├─ url: http://juliangruber.com\\
│  ├─ path: ./Frontend/node\_modules/brace-expansion\\
│  └─ licenseFile: ./Frontend/node\_modules/brace-expansion/LICENSE\\
├─ brace-expansion@2.0.1\\
│  ├─ licenses: MIT\\
│  ├─ repository: https://github.com/juliangruber/brace-expansion\\
│  ├─ publisher: Julian Gruber\\
│  ├─ email: mail@juliangruber.com\\
│  ├─ url: http://juliangruber.com\\
│  ├─ path: ./Frontend/node\_modules/@tufjs/models/node\_modules/brace-expansion\\
│  └─ licenseFile: ./Frontend/node\_modules/@tufjs/models/node\_modules/brace-expansion/LICENSE\\
├─ braces@3.0.2\\
│  ├─ licenses: MIT\\
│  ├─ repository: https://github.com/micromatch/braces\\
│  ├─ publisher: Jon Schlinkert\\
│  ├─ url: https://github.com/jonschlinkert\\
│  ├─ path: ./Frontend/node\_modules/braces\\
│  └─ licenseFile: ./Frontend/node\_modules/braces/LICENSE\\
├─ browserslist@4.23.0\\
│  ├─ licenses: MIT\\
│  ├─ repository: https://github.com/browserslist/browserslist\\
│  ├─ publisher: Andrey Sitnik\\
│  ├─ email: andrey@sitnik.ru\\
│  ├─ path: ./Frontend/node\_modules/browserslist\\
│  └─ licenseFile: ./Frontend/node\_modules/browserslist/LICENSE\\
├─ buffer-from@1.1.2\\
│  ├─ licenses: MIT\\
│  ├─ repository: https://github.com/LinusU/buffer-from\\
│  ├─ path: ./Frontend/node\_modules/buffer-from\\
│  └─ licenseFile: ./Frontend/node\_modules/buffer-from/LICENSE\\
├─ buffer@5.7.1\\
│  ├─ licenses: MIT\\
│  ├─ repository: https://github.com/feross/buffer\\
│  ├─ publisher: Feross Aboukhadijeh\\
│  ├─ email: feross@feross.org\\
│  ├─ url: https://feross.org\\
│  ├─ path: ./Frontend/node\_modules/buffer\\
│  └─ licenseFile: ./Frontend/node\_modules/buffer/LICENSE\\
├─ bytes@3.0.0\\
│  ├─ licenses: MIT\\
│  ├─ repository: https://github.com/visionmedia/bytes.js\\
│  ├─ publisher: TJ Holowaychuk\\
│  ├─ email: tj@vision-media.ca\\
│  ├─ url: http://tjholowaychuk.com\\
│  ├─ path: ./Frontend/node\_modules/compression/node\_modules/bytes\\
│  └─ licenseFile: ./Frontend/node\_modules/compression/node\_modules/bytes/LICENSE\\
├─ bytes@3.1.2\\
│  ├─ licenses: MIT\\
│  ├─ repository: https://github.com/visionmedia/bytes.js\\
│  ├─ publisher: TJ Holowaychuk\\
│  ├─ email: tj@vision-media.ca\\
│  ├─ url: http://tjholowaychuk.com\\
│  ├─ path: ./Frontend/node\_modules/bytes\\
│  └─ licenseFile: ./Frontend/node\_modules/bytes/LICENSE\\
├─ cacache@18.0.3\\
│  ├─ licenses: ISC\\
│  ├─ repository: https://github.com/npm/cacache\\
│  ├─ publisher: GitHub Inc.\\
│  ├─ path: ./Frontend/node\_modules/cacache\\
│  └─ licenseFile: ./Frontend/node\_modules/cacache/LICENSE.md\\
├─ call-bind@1.0.7\\
│  ├─ licenses: MIT\\
│  ├─ repository: https://github.com/ljharb/call-bind\\
│  ├─ publisher: Jordan Harband\\
│  ├─ email: ljharb@gmail.com\\
│  ├─ path: ./Frontend/node\_modules/call-bind\\
│  └─ licenseFile: ./Frontend/node\_modules/call-bind/LICENSE\\
├─ callsites@3.1.0\\
│  ├─ licenses: MIT\\
│  ├─ repository: https://github.com/sindresorhus/callsites\\
│  ├─ publisher: Sindre Sorhus\\
│  ├─ email: sindresorhus@gmail.com\\
│  ├─ url: sindresorhus.com\\
│  ├─ path: ./Frontend/node\_modules/callsites\\
│  └─ licenseFile: ./Frontend/node\_modules/callsites/license\\
├─ camelcase@5.3.1\\
│  ├─ licenses: MIT\\
│  ├─ repository: https://github.com/sindresorhus/camelcase\\
│  ├─ publisher: Sindre Sorhus\\
│  ├─ email: sindresorhus@gmail.com\\
│  ├─ url: sindresorhus.com\\
│  ├─ path: ./Frontend/node\_modules/camelcase\\
│  └─ licenseFile: ./Frontend/node\_modules/camelcase/license\\
├─ caniuse-lite@1.0.30001618\\
│  ├─ licenses: CC-BY-4.0\\
│  ├─ repository: https://github.com/browserslist/caniuse-lite\\
│  ├─ publisher: Ben Briggs\\
│  ├─ email: beneb.info@gmail.com\\
│  ├─ url: http://beneb.info\\
│  ├─ path: ./Frontend/node\_modules/caniuse-lite\\
│  └─ licenseFile: ./Frontend/node\_modules/caniuse-lite/LICENSE\\
├─ chalk@2.4.2\\
│  ├─ licenses: MIT\\
│  ├─ repository: https://github.com/chalk/chalk\\
│  ├─ path: ./Frontend/node\_modules/chalk\\
│  └─ licenseFile: ./Frontend/node\_modules/chalk/license\\
├─ chalk@4.1.2\\
│  ├─ licenses: MIT\\
│  ├─ repository: https://github.com/chalk/chalk\\
│  ├─ path: ./Frontend/node\_modules/critters/node\_modules/chalk\\
│  └─ licenseFile: ./Frontend/node\_modules/critters/node\_modules/chalk/license\\
├─ chalk@5.3.0\\
│  ├─ licenses: MIT\\
│  ├─ repository: https://github.com/chalk/chalk\\
│  ├─ path: ./Frontend/node\_modules/inquirer/node\_modules/chalk\\
│  └─ licenseFile: ./Frontend/node\_modules/inquirer/node\_modules/chalk/license\\
├─ chardet@0.7.0\\
│  ├─ licenses: MIT\\
│  ├─ repository: https://github.com/runk/node-chardet\\
│  ├─ publisher: Dmitry Shirokov\\
│  ├─ email: deadrunk@gmail.com\\
│  ├─ path: ./Frontend/node\_modules/chardet\\
│  └─ licenseFile: ./Frontend/node\_modules/chardet/LICENSE\\
├─ chokidar@3.6.0\\
│  ├─ licenses: MIT\\
│  ├─ repository: https://github.com/paulmillr/chokidar\\
│  ├─ publisher: Paul Miller\\
│  ├─ url: https://paulmillr.com\\
│  ├─ path: ./Frontend/node\_modules/chokidar\\
│  └─ licenseFile: ./Frontend/node\_modules/chokidar/LICENSE\\
├─ chownr@2.0.0\\
│  ├─ licenses: ISC\\
│  ├─ repository: https://github.com/isaacs/chownr\\
│  ├─ publisher: Isaac Z. Schlueter\\
│  ├─ email: i@izs.me\\
│  ├─ url: http://blog.izs.me/\\
│  ├─ path: ./Frontend/node\_modules/chownr\\
│  └─ licenseFile: ./Frontend/node\_modules/chownr/LICENSE\\
├─ chrome-trace-event@1.0.3\\
│  ├─ licenses: MIT\\
│  ├─ repository: https://github.com/samccone/chrome-trace-event\\
│  ├─ publisher: Trent Mick, Sam Saccone\\
│  ├─ path: ./Frontend/node\_modules/chrome-trace-event\\
│  └─ licenseFile: ./Frontend/node\_modules/chrome-trace-event/LICENSE.txt\\
├─ clean-stack@2.2.0\\
│  ├─ licenses: MIT\\
│  ├─ repository: https://github.com/sindresorhus/clean-stack\\
│  ├─ publisher: Sindre Sorhus\\
│  ├─ email: sindresorhus@gmail.com\\
│  ├─ url: sindresorhus.com\\
│  ├─ path: ./Frontend/node\_modules/clean-stack\\
│  └─ licenseFile: ./Frontend/node\_modules/clean-stack/license\\
├─ cli-cursor@3.1.0\\
│  ├─ licenses: MIT\\
│  ├─ repository: https://github.com/sindresorhus/cli-cursor\\
│  ├─ publisher: Sindre Sorhus\\
│  ├─ email: sindresorhus@gmail.com\\
│  ├─ url: sindresorhus.com\\
│  ├─ path: ./Frontend/node\_modules/cli-cursor\\
│  └─ licenseFile: ./Frontend/node\_modules/cli-cursor/license\\
├─ cli-spinners@2.9.2\\
│  ├─ licenses: MIT\\
│  ├─ repository: https://github.com/sindresorhus/cli-spinners\\
│  ├─ publisher: Sindre Sorhus\\
│  ├─ email: sindresorhus@gmail.com\\
│  ├─ url: https://sindresorhus.com\\
│  ├─ path: ./Frontend/node\_modules/cli-spinners\\
│  └─ licenseFile: ./Frontend/node\_modules/cli-spinners/license\\
├─ cli-width@3.0.0\\
│  ├─ licenses: ISC\\
│  ├─ repository: https://github.com/knownasilya/cli-width\\
│  ├─ publisher: Ilya Radchenko\\
│  ├─ email: knownasilya@gmail.com\\
│  ├─ path: ./Frontend/node\_modules/@openapitools/openapi-generator-cli/node\_modules/cli-width\\
│  └─ licenseFile: ./Frontend/node\_modules/@openapitools/openapi-generator-cli/node\_modules/cli-width/LICENSE\\
├─ cli-width@4.1.0\\
│  ├─ licenses: ISC\\
│  ├─ repository: https://github.com/knownasilya/cli-width\\
│  ├─ publisher: Ilya Radchenko\\
│  ├─ email: knownasilya@gmail.com\\
│  ├─ path: ./Frontend/node\_modules/cli-width\\
│  └─ licenseFile: ./Frontend/node\_modules/cli-width/LICENSE\\
├─ cliui@7.0.4\\
│  ├─ licenses: ISC\\
│  ├─ repository: https://github.com/yargs/cliui\\
│  ├─ publisher: Ben Coe\\
│  ├─ email: ben@npmjs.com\\
│  ├─ path: ./Frontend/node\_modules/karma/node\_modules/cliui\\
│  └─ licenseFile: ./Frontend/node\_modules/karma/node\_modules/cliui/LICENSE.txt\\
├─ cliui@8.0.1\\
│  ├─ licenses: ISC\\
│  ├─ repository: https://github.com/yargs/cliui\\
│  ├─ publisher: Ben Coe\\
│  ├─ email: ben@npmjs.com\\
│  ├─ path: ./Frontend/node\_modules/cliui\\
│  └─ licenseFile: ./Frontend/node\_modules/cliui/LICENSE.txt\\
├─ clone-deep@4.0.1\\
│  ├─ licenses: MIT\\
│  ├─ repository: https://github.com/jonschlinkert/clone-deep\\
│  ├─ publisher: Jon Schlinkert\\
│  ├─ url: https://github.com/jonschlinkert\\
│  ├─ path: ./Frontend/node\_modules/clone-deep\\
│  └─ licenseFile: ./Frontend/node\_modules/clone-deep/LICENSE\\
├─ clone@1.0.4\\
│  ├─ licenses: MIT\\
│  ├─ repository: https://github.com/pvorb/node-clone\\
│  ├─ publisher: Paul Vorbach\\
│  ├─ email: paul@vorba.ch\\
│  ├─ url: http://paul.vorba.ch/\\
│  ├─ path: ./Frontend/node\_modules/clone\\
│  └─ licenseFile: ./Frontend/node\_modules/clone/LICENSE\\
├─ color-convert@1.9.3\\
│  ├─ licenses: MIT\\
│  ├─ repository: https://github.com/Qix-/color-convert\\
│  ├─ publisher: Heather Arthur\\
│  ├─ email: fayearthur@gmail.com\\
│  ├─ path: ./Frontend/node\_modules/color-convert\\
│  └─ licenseFile: ./Frontend/node\_modules/color-convert/LICENSE\\
├─ color-convert@2.0.1\\
│  ├─ licenses: MIT\\
│  ├─ repository: https://github.com/Qix-/color-convert\\
│  ├─ publisher: Heather Arthur\\
│  ├─ email: fayearthur@gmail.com\\
│  ├─ path: ./Frontend/node\_modules/critters/node\_modules/color-convert\\
│  └─ licenseFile: ./Frontend/node\_modules/critters/node\_modules/color-convert/LICENSE\\
├─ color-name@1.1.3\\
│  ├─ licenses: MIT\\
│  ├─ repository: https://github.com/dfcreative/color-name\\
│  ├─ publisher: DY\\
│  ├─ email: dfcreative@gmail.com\\
│  ├─ path: ./Frontend/node\_modules/color-name\\
│  └─ licenseFile: ./Frontend/node\_modules/color-name/LICENSE\\
├─ color-name@1.1.4\\
│  ├─ licenses: MIT\\
│  ├─ repository: https://github.com/colorjs/color-name\\
│  ├─ publisher: DY\\
│  ├─ email: dfcreative@gmail.com\\
│  ├─ path: ./Frontend/node\_modules/critters/node\_modules/color-name\\
│  └─ licenseFile: ./Frontend/node\_modules/critters/node\_modules/color-name/LICENSE\\
├─ colorette@2.0.20\\
│  ├─ licenses: MIT\\
│  ├─ repository: https://github.com/jorgebucaran/colorette\\
│  ├─ publisher: Jorge Bucaran\\
│  ├─ path: ./Frontend/node\_modules/colorette\\
│  └─ licenseFile: ./Frontend/node\_modules/colorette/LICENSE.md\\
├─ combined-stream@1.0.8\\
│  ├─ licenses: MIT\\
│  ├─ repository: https://github.com/felixge/node-combined-stream\\
│  ├─ publisher: Felix Geisendörfer\\
│  ├─ email: felix@debuggable.com\\
│  ├─ url: http://debuggable.com/\\
│  ├─ path: ./Frontend/node\_modules/combined-stream\\
│  └─ licenseFile: ./Frontend/node\_modules/combined-stream/License\\
├─ commander@2.20.3\\
│  ├─ licenses: MIT\\
│  ├─ repository: https://github.com/tj/commander.js\\
│  ├─ publisher: TJ Holowaychuk\\
│  ├─ email: tj@vision-media.ca\\
│  ├─ path: ./Frontend/node\_modules/commander\\
│  └─ licenseFile: ./Frontend/node\_modules/commander/LICENSE\\
├─ commander@8.3.0\\
│  ├─ licenses: MIT\\
│  ├─ repository: https://github.com/tj/commander.js\\
│  ├─ publisher: TJ Holowaychuk\\
│  ├─ email: tj@vision-media.ca\\
│  ├─ path: ./Frontend/node\_modules/@openapitools/openapi-generator-cli/node\_modules/commander\\
│  └─ licenseFile: ./Frontend/node\_modules/@openapitools/openapi-generator-cli/node\_modules/commander/LICENSE\\
├─ common-path-prefix@3.0.0\\
│  ├─ licenses: ISC\\
│  ├─ repository: https://github.com/novemberborn/common-path-prefix\\
│  ├─ publisher: Mark Wubben\\
│  ├─ url: https://novemberborn.net/\\
│  ├─ path: ./Frontend/node\_modules/common-path-prefix\\
│  └─ licenseFile: ./Frontend/node\_modules/common-path-prefix/LICENSE\\
├─ compare-versions@4.1.4\\
│  ├─ licenses: MIT\\
│  ├─ repository: https://github.com/omichelsen/compare-versions\\
│  ├─ publisher: Ole Michelsen\\
│  ├─ path: ./Frontend/node\_modules/compare-versions\\
│  └─ licenseFile: ./Frontend/node\_modules/compare-versions/LICENSE\\
├─ compressible@2.0.18\\
│  ├─ licenses: MIT\\
│  ├─ repository: https://github.com/jshttp/compressible\\
│  ├─ path: ./Frontend/node\_modules/compressible\\
│  └─ licenseFile: ./Frontend/node\_modules/compressible/LICENSE\\
├─ compression@1.7.4\\
│  ├─ licenses: MIT\\
│  ├─ repository: https://github.com/expressjs/compression\\
│  ├─ path: ./Frontend/node\_modules/compression\\
│  └─ licenseFile: ./Frontend/node\_modules/compression/LICENSE\\
├─ concat-map@0.0.1\\
│  ├─ licenses: MIT\\
│  ├─ repository: https://github.com/substack/node-concat-map\\
│  ├─ publisher: James Halliday\\
│  ├─ email: mail@substack.net\\
│  ├─ url: http://substack.net\\
│  ├─ path: ./Frontend/node\_modules/concat-map\\
│  └─ licenseFile: ./Frontend/node\_modules/concat-map/LICENSE\\
├─ concurrently@6.5.1\\
│  ├─ licenses: MIT\\
│  ├─ repository: https://github.com/open-cli-tools/concurrently\\
│  ├─ publisher: Kimmo Brunfeldt\\
│  ├─ path: ./Frontend/node\_modules/concurrently\\
│  └─ licenseFile: ./Frontend/node\_modules/concurrently/LICENSE\\
├─ connect-history-api-fallback@2.0.0\\
│  ├─ licenses: MIT\\
│  ├─ repository: https://github.com/bripkens/connect-history-api-fallback\\
│  ├─ publisher: Ben Ripkens\\
│  ├─ email: bripkens@gmail.com\\
│  ├─ path: ./Frontend/node\_modules/connect-history-api-fallback\\
│  └─ licenseFile: ./Frontend/node\_modules/connect-history-api-fallback/LICENSE\\
├─ connect@3.7.0\\
│  ├─ licenses: MIT\\
│  ├─ repository: https://github.com/senchalabs/connect\\
│  ├─ publisher: TJ Holowaychuk\\
│  ├─ email: tj@vision-media.ca\\
│  ├─ url: http://tjholowaychuk.com\\
│  ├─ path: ./Frontend/node\_modules/connect\\
│  └─ licenseFile: ./Frontend/node\_modules/connect/LICENSE\\
├─ consola@2.15.3\\
│  ├─ licenses: MIT\\
│  ├─ repository: https://github.com/nuxt/consola\\
│  ├─ path: ./Frontend/node\_modules/consola\\
│  └─ licenseFile: ./Frontend/node\_modules/consola/README.md\\
├─ console.table@0.10.0\\
│  ├─ licenses: MIT\\
│  ├─ repository: https://github.com/bahmutov/console.table\\
│  ├─ publisher: Gleb Bahmutov\\
│  ├─ email: gleb.bahmutov@gmail.com\\
│  ├─ path: ./Frontend/node\_modules/console.table\\
│  └─ licenseFile: ./Frontend/node\_modules/console.table/LICENSE-MIT\\
├─ content-disposition@0.5.4\\
│  ├─ licenses: MIT\\
│  ├─ repository: https://github.com/jshttp/content-disposition\\
│  ├─ publisher: Douglas Christopher Wilson\\
│  ├─ email: doug@somethingdoug.com\\
│  ├─ path: ./Frontend/node\_modules/content-disposition\\
│  └─ licenseFile: ./Frontend/node\_modules/content-disposition/LICENSE\\
├─ content-type@1.0.5\\
│  ├─ licenses: MIT\\
│  ├─ repository: https://github.com/jshttp/content-type\\
│  ├─ publisher: Douglas Christopher Wilson\\
│  ├─ email: doug@somethingdoug.com\\
│  ├─ path: ./Frontend/node\_modules/content-type\\
│  └─ licenseFile: ./Frontend/node\_modules/content-type/LICENSE\\
├─ convert-source-map@1.9.0\\
│  ├─ licenses: MIT\\
│  ├─ repository: https://github.com/thlorenz/convert-source-map\\
│  ├─ publisher: Thorsten Lorenz\\
│  ├─ email: thlorenz@gmx.de\\
│  ├─ url: http://thlorenz.com\\
│  ├─ path: ./Frontend/node\_modules/convert-source-map\\
│  └─ licenseFile: ./Frontend/node\_modules/convert-source-map/LICENSE\\
├─ convert-source-map@2.0.0\\
│  ├─ licenses: MIT\\
│  ├─ repository: https://github.com/thlorenz/convert-source-map\\
│  ├─ publisher: Thorsten Lorenz\\
│  ├─ email: thlorenz@gmx.de\\
│  ├─ url: http://thlorenz.com\\
│  ├─ path: ./Frontend/node\_modules/@babel/core/node\_modules/convert-source-map\\
│  └─ licenseFile: ./Frontend/node\_modules/@babel/core/node\_modules/convert-source-map/LICENSE\\
├─ cookie-signature@1.0.6\\
│  ├─ licenses: MIT\\
│  ├─ repository: https://github.com/visionmedia/node-cookie-signature\\
│  ├─ publisher: TJ Holowaychuk\\
│  ├─ email: tj@learnboost.com\\
│  ├─ path: ./Frontend/node\_modules/cookie-signature\\
│  └─ licenseFile: ./Frontend/node\_modules/cookie-signature/Readme.md\\
├─ cookie@0.4.2\\
│  ├─ licenses: MIT\\
│  ├─ repository: https://github.com/jshttp/cookie\\
│  ├─ publisher: Roman Shtylman\\
│  ├─ email: shtylman@gmail.com\\
│  ├─ path: ./Frontend/node\_modules/cookie\\
│  └─ licenseFile: ./Frontend/node\_modules/cookie/LICENSE\\
├─ cookie@0.6.0\\
│  ├─ licenses: MIT\\
│  ├─ repository: https://github.com/jshttp/cookie\\
│  ├─ publisher: Roman Shtylman\\
│  ├─ email: shtylman@gmail.com\\
│  ├─ path: ./Frontend/node\_modules/express/node\_modules/cookie\\
│  └─ licenseFile: ./Frontend/node\_modules/express/node\_modules/cookie/LICENSE\\
├─ copy-anything@2.0.6\\
│  ├─ licenses: MIT\\
│  ├─ repository: https://github.com/mesqueeb/copy-anything\\
│  ├─ publisher: Luca Ban - Mesqueeb\\
│  ├─ path: ./Frontend/node\_modules/copy-anything\\
│  └─ licenseFile: ./Frontend/node\_modules/copy-anything/LICENSE\\
├─ copy-webpack-plugin@11.0.0\\
│  ├─ licenses: MIT\\
│  ├─ repository: https://github.com/webpack-contrib/copy-webpack-plugin\\
│  ├─ publisher: Len Boyette\\
│  ├─ path: ./Frontend/node\_modules/copy-webpack-plugin\\
│  └─ licenseFile: ./Frontend/node\_modules/copy-webpack-plugin/LICENSE\\
├─ core-js-compat@3.37.1\\
│  ├─ licenses: MIT\\
│  ├─ repository: https://github.com/zloirock/core-js\\
│  ├─ publisher: Denis Pushkarev\\
│  ├─ email: zloirock@zloirock.ru\\
│  ├─ url: http://zloirock.ru\\
│  ├─ path: ./Frontend/node\_modules/core-js-compat\\
│  └─ licenseFile: ./Frontend/node\_modules/core-js-compat/LICENSE\\
├─ core-util-is@1.0.3\\
│  ├─ licenses: MIT\\
│  ├─ repository: https://github.com/isaacs/core-util-is\\
│  ├─ publisher: Isaac Z. Schlueter\\
│  ├─ email: i@izs.me\\
│  ├─ url: http://blog.izs.me/\\
│  ├─ path: ./Frontend/node\_modules/core-util-is\\
│  └─ licenseFile: ./Frontend/node\_modules/core-util-is/LICENSE\\
├─ cors@2.8.5\\
│  ├─ licenses: MIT\\
│  ├─ repository: https://github.com/expressjs/cors\\
│  ├─ publisher: Troy Goode\\
│  ├─ email: troygoode@gmail.com\\
│  ├─ url: https://github.com/troygoode/\\
│  ├─ path: ./Frontend/node\_modules/cors\\
│  └─ licenseFile: ./Frontend/node\_modules/cors/LICENSE\\
├─ cosmiconfig@9.0.0\\
│  ├─ licenses: MIT\\
│  ├─ repository: https://github.com/cosmiconfig/cosmiconfig\\
│  ├─ publisher: Daniel Fischer\\
│  ├─ email: daniel@d-fischer.dev\\
│  ├─ path: ./Frontend/node\_modules/cosmiconfig\\
│  └─ licenseFile: ./Frontend/node\_modules/cosmiconfig/LICENSE\\
├─ critters@0.0.22\\
│  ├─ licenses: Apache-2.0\\
│  ├─ repository: https://github.com/GoogleChromeLabs/critters\\
│  ├─ publisher: The Chromium Authors\\
│  ├─ path: ./Frontend/node\_modules/critters\\
│  └─ licenseFile: ./Frontend/node\_modules/critters/README.md\\
├─ cross-spawn@7.0.3\\
│  ├─ licenses: MIT\\
│  ├─ repository: https://github.com/moxystudio/node-cross-spawn\\
│  ├─ publisher: André Cruz\\
│  ├─ email: andre@moxy.studio\\
│  ├─ path: ./Frontend/node\_modules/cross-spawn\\
│  └─ licenseFile: ./Frontend/node\_modules/cross-spawn/LICENSE\\
├─ css-loader@6.10.0\\
│  ├─ licenses: MIT\\
│  ├─ repository: https://github.com/webpack-contrib/css-loader\\
│  ├─ publisher: Tobias Koppers @sokra\\
│  ├─ path: ./Frontend/node\_modules/css-loader\\
│  └─ licenseFile: ./Frontend/node\_modules/css-loader/LICENSE\\
├─ css-select@5.1.0\\
│  ├─ licenses: BSD-2-Clause\\
│  ├─ repository: https://github.com/fb55/css-select\\
│  ├─ publisher: Felix Boehm\\
│  ├─ email: me@feedic.com\\
│  ├─ path: ./Frontend/node\_modules/css-select\\
│  └─ licenseFile: ./Frontend/node\_modules/css-select/LICENSE\\
├─ css-what@6.1.0\\
│  ├─ licenses: BSD-2-Clause\\
│  ├─ repository: https://github.com/fb55/css-what\\
│  ├─ publisher: Felix Böhm\\
│  ├─ email: me@feedic.com\\
│  ├─ url: http://feedic.com\\
│  ├─ path: ./Frontend/node\_modules/css-what\\
│  └─ licenseFile: ./Frontend/node\_modules/css-what/LICENSE\\
├─ cssesc@3.0.0\\
│  ├─ licenses: MIT\\
│  ├─ repository: https://github.com/mathiasbynens/cssesc\\
│  ├─ publisher: Mathias Bynens\\
│  ├─ url: https://mathiasbynens.be/\\
│  ├─ path: ./Frontend/node\_modules/cssesc\\
│  └─ licenseFile: ./Frontend/node\_modules/cssesc/LICENSE-MIT.txt\\
├─ custom-event@1.0.1\\
│  ├─ licenses: MIT\\
│  ├─ repository: https://github.com/webmodules/custom-event\\
│  ├─ publisher: Nathan Rajlich\\
│  ├─ email: nathan@tootallnate.net\\
│  ├─ url: http://n8.io/\\
│  ├─ path: ./Frontend/node\_modules/custom-event\\
│  └─ licenseFile: ./Frontend/node\_modules/custom-event/LICENSE\\
├─ date-fns@2.30.0\\
│  ├─ licenses: MIT\\
│  ├─ repository: https://github.com/date-fns/date-fns\\
│  ├─ path: ./Frontend/node\_modules/date-fns\\
│  └─ licenseFile: ./Frontend/node\_modules/date-fns/LICENSE.md\\
├─ date-format@4.0.14\\
│  ├─ licenses: MIT\\
│  ├─ repository: https://github.com/nomiddlename/date-format\\
│  ├─ publisher: Gareth Jones\\
│  ├─ email: gareth.nomiddlename@gmail.com\\
│  ├─ path: ./Frontend/node\_modules/date-format\\
│  └─ licenseFile: ./Frontend/node\_modules/date-format/LICENSE\\
├─ debug@2.6.9\\
│  ├─ licenses: MIT\\
│  ├─ repository: https://github.com/visionmedia/debug\\
│  ├─ publisher: TJ Holowaychuk\\
│  ├─ email: tj@vision-media.ca\\
│  ├─ path: ./Frontend/node\_modules/compression/node\_modules/debug\\
│  └─ licenseFile: ./Frontend/node\_modules/compression/node\_modules/debug/LICENSE\\
├─ debug@4.3.4\\
│  ├─ licenses: MIT\\
│  ├─ repository: https://github.com/debug-js/debug\\
│  ├─ publisher: Josh Junon\\
│  ├─ email: josh.junon@protonmail.com\\
│  ├─ path: ./Frontend/node\_modules/debug\\
│  └─ licenseFile: ./Frontend/node\_modules/debug/LICENSE\\
├─ default-gateway@6.0.3\\
│  ├─ licenses: BSD-2-Clause\\
│  ├─ repository: https://github.com/silverwind/default-gateway\\
│  ├─ publisher: silverwind\\
│  ├─ path: ./Frontend/node\_modules/default-gateway\\
│  └─ licenseFile: ./Frontend/node\_modules/default-gateway/LICENSE\\
├─ defaults@1.0.4\\
│  ├─ licenses: MIT\\
│  ├─ repository: https://github.com/sindresorhus/node-defaults\\
│  ├─ publisher: Elijah Insua\\
│  ├─ email: tmpvar@gmail.com\\
│  ├─ path: ./Frontend/node\_modules/defaults\\
│  └─ licenseFile: ./Frontend/node\_modules/defaults/LICENSE\\
├─ define-data-property@1.1.4\\
│  ├─ licenses: MIT\\
│  ├─ repository: https://github.com/ljharb/define-data-property\\
│  ├─ publisher: Jordan Harband\\
│  ├─ email: ljharb@gmail.com\\
│  ├─ path: ./Frontend/node\_modules/define-data-property\\
│  └─ licenseFile: ./Frontend/node\_modules/define-data-property/LICENSE\\
├─ define-lazy-prop@2.0.0\\
│  ├─ licenses: MIT\\
│  ├─ repository: https://github.com/sindresorhus/define-lazy-prop\\
│  ├─ publisher: Sindre Sorhus\\
│  ├─ email: sindresorhus@gmail.com\\
│  ├─ url: sindresorhus.com\\
│  ├─ path: ./Frontend/node\_modules/define-lazy-prop\\
│  └─ licenseFile: ./Frontend/node\_modules/define-lazy-prop/license\\
├─ delayed-stream@1.0.0\\
│  ├─ licenses: MIT\\
│  ├─ repository: https://github.com/felixge/node-delayed-stream\\
│  ├─ publisher: Felix Geisendörfer\\
│  ├─ email: felix@debuggable.com\\
│  ├─ url: http://debuggable.com/\\
│  ├─ path: ./Frontend/node\_modules/delayed-stream\\
│  └─ licenseFile: ./Frontend/node\_modules/delayed-stream/License\\
├─ depd@1.1.2\\
│  ├─ licenses: MIT\\
│  ├─ repository: https://github.com/dougwilson/nodejs-depd\\
│  ├─ publisher: Douglas Christopher Wilson\\
│  ├─ email: doug@somethingdoug.com\\
│  ├─ path: ./Frontend/node\_modules/serve-index/node\_modules/depd\\
│  └─ licenseFile: ./Frontend/node\_modules/serve-index/node\_modules/depd/LICENSE\\
├─ depd@2.0.0\\
│  ├─ licenses: MIT\\
│  ├─ repository: https://github.com/dougwilson/nodejs-depd\\
│  ├─ publisher: Douglas Christopher Wilson\\
│  ├─ email: doug@somethingdoug.com\\
│  ├─ path: ./Frontend/node\_modules/depd\\
│  └─ licenseFile: ./Frontend/node\_modules/depd/LICENSE\\
├─ destroy@1.2.0\\
│  ├─ licenses: MIT\\
│  ├─ repository: https://github.com/stream-utils/destroy\\
│  ├─ publisher: Jonathan Ong\\
│  ├─ email: me@jongleberry.com\\
│  ├─ url: http://jongleberry.com\\
│  ├─ path: ./Frontend/node\_modules/destroy\\
│  └─ licenseFile: ./Frontend/node\_modules/destroy/LICENSE\\
├─ detect-node@2.1.0\\
│  ├─ licenses: MIT\\
│  ├─ repository: https://github.com/iliakan/detect-node\\
│  ├─ publisher: Ilya Kantor\\
│  ├─ path: ./Frontend/node\_modules/detect-node\\
│  └─ licenseFile: ./Frontend/node\_modules/detect-node/LICENSE\\
├─ di@0.0.1\\
│  ├─ licenses: MIT\\
│  ├─ repository: https://github.com/vojtajina/node-di\\
│  ├─ publisher: Vojta Jina\\
│  ├─ email: vojta.jina@gmail.com\\
│  ├─ path: ./Frontend/node\_modules/di\\
│  └─ licenseFile: ./Frontend/node\_modules/di/LICENSE\\
├─ dir-glob@3.0.1\\
│  ├─ licenses: MIT\\
│  ├─ repository: https://github.com/kevva/dir-glob\\
│  ├─ publisher: Kevin Mårtensson\\
│  ├─ email: kevinmartensson@gmail.com\\
│  ├─ url: github.com/kevva\\
│  ├─ path: ./Frontend/node\_modules/dir-glob\\
│  └─ licenseFile: ./Frontend/node\_modules/dir-glob/license\\
├─ dns-packet@5.6.1\\
│  ├─ licenses: MIT\\
│  ├─ repository: https://github.com/mafintosh/dns-packet\\
│  ├─ publisher: Mathias Buus\\
│  ├─ path: ./Frontend/node\_modules/dns-packet\\
│  └─ licenseFile: ./Frontend/node\_modules/dns-packet/LICENSE\\
├─ dom-serialize@2.2.1\\
│  ├─ licenses: MIT\\
│  ├─ repository: https://github.com/webmodules/dom-serialize\\
│  ├─ publisher: Nathan Rajlich\\
│  ├─ email: nathan@tootallnate.net\\
│  ├─ url: http://n8.io/\\
│  ├─ path: ./Frontend/node\_modules/dom-serialize\\
│  └─ licenseFile: ./Frontend/node\_modules/dom-serialize/README.md\\
├─ dom-serializer@2.0.0\\
│  ├─ licenses: MIT\\
│  ├─ repository: https://github.com/cheeriojs/dom-serializer\\
│  ├─ publisher: Felix Boehm\\
│  ├─ email: me@feedic.com\\
│  ├─ path: ./Frontend/node\_modules/dom-serializer\\
│  └─ licenseFile: ./Frontend/node\_modules/dom-serializer/LICENSE\\
├─ domelementtype@2.3.0\\
│  ├─ licenses: BSD-2-Clause\\
│  ├─ repository: https://github.com/fb55/domelementtype\\
│  ├─ publisher: Felix Boehm\\
│  ├─ email: me@feedic.com\\
│  ├─ path: ./Frontend/node\_modules/domelementtype\\
│  └─ licenseFile: ./Frontend/node\_modules/domelementtype/LICENSE\\
├─ domhandler@5.0.3\\
│  ├─ licenses: BSD-2-Clause\\
│  ├─ repository: https://github.com/fb55/domhandler\\
│  ├─ publisher: Felix Boehm\\
│  ├─ email: me@feedic.com\\
│  ├─ path: ./Frontend/node\_modules/domhandler\\
│  └─ licenseFile: ./Frontend/node\_modules/domhandler/LICENSE\\
├─ domutils@3.1.0\\
│  ├─ licenses: BSD-2-Clause\\
│  ├─ repository: https://github.com/fb55/domutils\\
│  ├─ publisher: Felix Boehm\\
│  ├─ email: me@feedic.com\\
│  ├─ path: ./Frontend/node\_modules/domutils\\
│  └─ licenseFile: ./Frontend/node\_modules/domutils/LICENSE\\
├─ eastasianwidth@0.2.0\\
│  ├─ licenses: MIT\\
│  ├─ repository: https://github.com/komagata/eastasianwidth\\
│  ├─ publisher: Masaki Komagata\\
│  ├─ path: ./Frontend/node\_modules/eastasianwidth\\
│  └─ licenseFile: ./Frontend/node\_modules/eastasianwidth/README.md\\
├─ easy-table@1.1.0\\
│  ├─ licenses: MIT\\
│  ├─ repository: git+https://eldargab@github.com/eldargab/easy-table\\
│  ├─ publisher: Eldar Gabdullin\\
│  ├─ email: eldargab@gmail.com\\
│  ├─ path: ./Frontend/node\_modules/easy-table\\
│  └─ licenseFile: ./Frontend/node\_modules/easy-table/README.md\\
├─ ee-first@1.1.1\\
│  ├─ licenses: MIT\\
│  ├─ repository: https://github.com/jonathanong/ee-first\\
│  ├─ publisher: Jonathan Ong\\
│  ├─ email: me@jongleberry.com\\
│  ├─ url: http://jongleberry.com\\
│  ├─ path: ./Frontend/node\_modules/ee-first\\
│  └─ licenseFile: ./Frontend/node\_modules/ee-first/LICENSE\\
├─ electron-to-chromium@1.4.767\\
│  ├─ licenses: ISC\\
│  ├─ repository: https://github.com/kilian/electron-to-chromium\\
│  ├─ publisher: Kilian Valkhof\\
│  ├─ path: ./Frontend/node\_modules/electron-to-chromium\\
│  └─ licenseFile: ./Frontend/node\_modules/electron-to-chromium/LICENSE\\
├─ emoji-regex@8.0.0\\
│  ├─ licenses: MIT\\
│  ├─ repository: https://github.com/mathiasbynens/emoji-regex\\
│  ├─ publisher: Mathias Bynens\\
│  ├─ url: https://mathiasbynens.be/\\
│  ├─ path: ./Frontend/node\_modules/emoji-regex\\
│  └─ licenseFile: ./Frontend/node\_modules/emoji-regex/LICENSE-MIT.txt\\
├─ emoji-regex@9.2.2\\
│  ├─ licenses: MIT\\
│  ├─ repository: https://github.com/mathiasbynens/emoji-regex\\
│  ├─ publisher: Mathias Bynens\\
│  ├─ url: https://mathiasbynens.be/\\
│  ├─ path: ./Frontend/node\_modules/@isaacs/cliui/node\_modules/emoji-regex\\
│  └─ licenseFile: ./Frontend/node\_modules/@isaacs/cliui/node\_modules/emoji-regex/LICENSE-MIT.txt\\
├─ emojis-list@3.0.0\\
│  ├─ licenses: MIT\\
│  ├─ repository: https://github.com/kikobeats/emojis-list\\
│  ├─ publisher: Kiko Beats\\
│  ├─ email: josefrancisco.verdu@gmail.com\\
│  ├─ url: https://github.com/Kikobeats\\
│  ├─ path: ./Frontend/node\_modules/emojis-list\\
│  └─ licenseFile: ./Frontend/node\_modules/emojis-list/LICENSE.md\\
├─ encodeurl@1.0.2\\
│  ├─ licenses: MIT\\
│  ├─ repository: https://github.com/pillarjs/encodeurl\\
│  ├─ path: ./Frontend/node\_modules/encodeurl\\
│  └─ licenseFile: ./Frontend/node\_modules/encodeurl/LICENSE\\
├─ encoding@0.1.13\\
│  ├─ licenses: MIT\\
│  ├─ repository: https://github.com/andris9/encoding\\
│  ├─ publisher: Andris Reinman\\
│  ├─ path: ./Frontend/node\_modules/encoding\\
│  └─ licenseFile: ./Frontend/node\_modules/encoding/LICENSE\\
├─ engine.io-parser@5.2.2\\
│  ├─ licenses: MIT\\
│  ├─ repository: https://github.com/socketio/engine.io-parser\\
│  ├─ path: ./Frontend/node\_modules/engine.io-parser\\
│  └─ licenseFile: ./Frontend/node\_modules/engine.io-parser/LICENSE\\
├─ engine.io@6.5.4\\
│  ├─ licenses: MIT\\
│  ├─ repository: https://github.com/socketio/engine.io\\
│  ├─ publisher: Guillermo Rauch\\
│  ├─ email: guillermo@learnboost.com\\
│  ├─ path: ./Frontend/node\_modules/engine.io\\
│  └─ licenseFile: ./Frontend/node\_modules/engine.io/LICENSE\\
├─ enhanced-resolve@5.16.1\\
│  ├─ licenses: MIT\\
│  ├─ repository: https://github.com/webpack/enhanced-resolve\\
│  ├─ publisher: Tobias Koppers @sokra\\
│  ├─ path: ./Frontend/node\_modules/enhanced-resolve\\
│  └─ licenseFile: ./Frontend/node\_modules/enhanced-resolve/LICENSE\\
├─ ent@2.2.0\\
│  ├─ licenses: MIT\\
│  ├─ repository: https://github.com/substack/node-ent\\
│  ├─ publisher: James Halliday\\
│  ├─ email: mail@substack.net\\
│  ├─ url: http://substack.net\\
│  ├─ path: ./Frontend/node\_modules/ent\\
│  └─ licenseFile: ./Frontend/node\_modules/ent/LICENSE\\
├─ entities@4.5.0\\
│  ├─ licenses: BSD-2-Clause\\
│  ├─ repository: https://github.com/fb55/entities\\
│  ├─ publisher: Felix Boehm\\
│  ├─ email: me@feedic.com\\
│  ├─ path: ./Frontend/node\_modules/entities\\
│  └─ licenseFile: ./Frontend/node\_modules/entities/LICENSE\\
├─ env-paths@2.2.1\\
│  ├─ licenses: MIT\\
│  ├─ repository: https://github.com/sindresorhus/env-paths\\
│  ├─ publisher: Sindre Sorhus\\
│  ├─ email: sindresorhus@gmail.com\\
│  ├─ url: sindresorhus.com\\
│  ├─ path: ./Frontend/node\_modules/env-paths\\
│  └─ licenseFile: ./Frontend/node\_modules/env-paths/license\\
├─ err-code@2.0.3\\
│  ├─ licenses: MIT\\
│  ├─ repository: https://github.com/IndigoUnited/js-err-code\\
│  ├─ publisher: IndigoUnited\\
│  ├─ email: hello@indigounited.com\\
│  ├─ url: http://indigounited.com\\
│  ├─ path: ./Frontend/node\_modules/err-code\\
│  └─ licenseFile: ./Frontend/node\_modules/err-code/README.md\\
├─ errno@0.1.8\\
│  ├─ licenses: MIT\\
│  ├─ repository: https://github.com/rvagg/node-errno\\
│  ├─ path: ./Frontend/node\_modules/errno\\
│  └─ licenseFile: ./Frontend/node\_modules/errno/README.md\\
├─ error-ex@1.3.2\\
│  ├─ licenses: MIT\\
│  ├─ repository: https://github.com/qix-/node-error-ex\\
│  ├─ path: ./Frontend/node\_modules/error-ex\\
│  └─ licenseFile: ./Frontend/node\_modules/error-ex/LICENSE\\
├─ es-define-property@1.0.0\\
│  ├─ licenses: MIT\\
│  ├─ repository: https://github.com/ljharb/es-define-property\\
│  ├─ publisher: Jordan Harband\\
│  ├─ email: ljharb@gmail.com\\
│  ├─ path: ./Frontend/node\_modules/es-define-property\\
│  └─ licenseFile: ./Frontend/node\_modules/es-define-property/LICENSE\\
├─ es-errors@1.3.0\\
│  ├─ licenses: MIT\\
│  ├─ repository: https://github.com/ljharb/es-errors\\
│  ├─ publisher: Jordan Harband\\
│  ├─ email: ljharb@gmail.com\\
│  ├─ path: ./Frontend/node\_modules/es-errors\\
│  └─ licenseFile: ./Frontend/node\_modules/es-errors/LICENSE\\
├─ es-module-lexer@1.5.2\\
│  ├─ licenses: MIT\\
│  ├─ repository: https://github.com/guybedford/es-module-lexer\\
│  ├─ publisher: Guy Bedford\\
│  ├─ path: ./Frontend/node\_modules/es-module-lexer\\
│  └─ licenseFile: ./Frontend/node\_modules/es-module-lexer/LICENSE\\
├─ esbuild-wasm@0.20.1\\
│  ├─ licenses: MIT\\
│  ├─ repository: https://github.com/evanw/esbuild\\
│  ├─ path: ./Frontend/node\_modules/esbuild-wasm\\
│  └─ licenseFile: ./Frontend/node\_modules/esbuild-wasm/LICENSE.md\\
├─ esbuild@0.19.12\\
│  ├─ licenses: MIT\\
│  ├─ repository: https://github.com/evanw/esbuild\\
│  ├─ path: ./Frontend/node\_modules/vite/node\_modules/esbuild\\
│  └─ licenseFile: ./Frontend/node\_modules/vite/node\_modules/esbuild/LICENSE.md\\
├─ esbuild@0.20.1\\
│  ├─ licenses: MIT\\
│  ├─ repository: https://github.com/evanw/esbuild\\
│  ├─ path: ./Frontend/node\_modules/esbuild\\
│  └─ licenseFile: ./Frontend/node\_modules/esbuild/LICENSE.md\\
├─ escalade@3.1.2\\
│  ├─ licenses: MIT\\
│  ├─ repository: https://github.com/lukeed/escalade\\
│  ├─ publisher: Luke Edwards\\
│  ├─ email: luke.edwards05@gmail.com\\
│  ├─ url: https://lukeed.com\\
│  ├─ path: ./Frontend/node\_modules/escalade\\
│  └─ licenseFile: ./Frontend/node\_modules/escalade/license\\
├─ escape-html@1.0.3\\
│  ├─ licenses: MIT\\
│  ├─ repository: https://github.com/component/escape-html\\
│  ├─ path: ./Frontend/node\_modules/escape-html\\
│  └─ licenseFile: ./Frontend/node\_modules/escape-html/LICENSE\\
├─ escape-string-regexp@1.0.5\\
│  ├─ licenses: MIT\\
│  ├─ repository: https://github.com/sindresorhus/escape-string-regexp\\
│  ├─ publisher: Sindre Sorhus\\
│  ├─ email: sindresorhus@gmail.com\\
│  ├─ url: sindresorhus.com\\
│  ├─ path: ./Frontend/node\_modules/escape-string-regexp\\
│  └─ licenseFile: ./Frontend/node\_modules/escape-string-regexp/license\\
├─ eslint-scope@5.1.1\\
│  ├─ licenses: BSD-2-Clause\\
│  ├─ repository: https://github.com/eslint/eslint-scope\\
│  ├─ path: ./Frontend/node\_modules/eslint-scope\\
│  └─ licenseFile: ./Frontend/node\_modules/eslint-scope/LICENSE\\
├─ esprima@4.0.1\\
│  ├─ licenses: BSD-2-Clause\\
│  ├─ repository: https://github.com/jquery/esprima\\
│  ├─ publisher: Ariya Hidayat\\
│  ├─ email: ariya.hidayat@gmail.com\\
│  ├─ path: ./Frontend/node\_modules/esprima\\
│  └─ licenseFile: ./Frontend/node\_modules/esprima/LICENSE.BSD\\
├─ esrecurse@4.3.0\\
│  ├─ licenses: BSD-2-Clause\\
│  ├─ repository: https://github.com/estools/esrecurse\\
│  ├─ path: ./Frontend/node\_modules/esrecurse\\
│  └─ licenseFile: ./Frontend/node\_modules/esrecurse/README.md\\
├─ estraverse@4.3.0\\
│  ├─ licenses: BSD-2-Clause\\
│  ├─ repository: https://github.com/estools/estraverse\\
│  ├─ path: ./Frontend/node\_modules/estraverse\\
│  └─ licenseFile: ./Frontend/node\_modules/estraverse/LICENSE.BSD\\
├─ estraverse@5.3.0\\
│  ├─ licenses: BSD-2-Clause\\
│  ├─ repository: https://github.com/estools/estraverse\\
│  ├─ path: ./Frontend/node\_modules/esrecurse/node\_modules/estraverse\\
│  └─ licenseFile: ./Frontend/node\_modules/esrecurse/node\_modules/estraverse/LICENSE.BSD\\
├─ esutils@2.0.3\\
│  ├─ licenses: BSD-2-Clause\\
│  ├─ repository: https://github.com/estools/esutils\\
│  ├─ path: ./Frontend/node\_modules/esutils\\
│  └─ licenseFile: ./Frontend/node\_modules/esutils/LICENSE.BSD\\
├─ etag@1.8.1\\
│  ├─ licenses: MIT\\
│  ├─ repository: https://github.com/jshttp/etag\\
│  ├─ path: ./Frontend/node\_modules/etag\\
│  └─ licenseFile: ./Frontend/node\_modules/etag/LICENSE\\
├─ eventemitter3@4.0.7\\
│  ├─ licenses: MIT\\
│  ├─ repository: https://github.com/primus/eventemitter3\\
│  ├─ publisher: Arnout Kazemier\\
│  ├─ path: ./Frontend/node\_modules/eventemitter3\\
│  └─ licenseFile: ./Frontend/node\_modules/eventemitter3/LICENSE\\
├─ events@3.3.0\\
│  ├─ licenses: MIT\\
│  ├─ repository: https://github.com/Gozala/events\\
│  ├─ publisher: Irakli Gozalishvili\\
│  ├─ email: rfobic@gmail.com\\
│  ├─ url: http://jeditoolkit.com\\
│  ├─ path: ./Frontend/node\_modules/events\\
│  └─ licenseFile: ./Frontend/node\_modules/events/LICENSE\\
├─ execa@5.1.1\\
│  ├─ licenses: MIT\\
│  ├─ repository: https://github.com/sindresorhus/execa\\
│  ├─ publisher: Sindre Sorhus\\
│  ├─ email: sindresorhus@gmail.com\\
│  ├─ url: https://sindresorhus.com\\
│  ├─ path: ./Frontend/node\_modules/execa\\
│  └─ licenseFile: ./Frontend/node\_modules/execa/license\\
├─ exponential-backoff@3.1.1\\
│  ├─ licenses: Apache-2.0\\
│  ├─ repository: https://github.com/coveo/exponential-backoff\\
│  ├─ publisher: Sami Sayegh\\
│  ├─ path: ./Frontend/node\_modules/exponential-backoff\\
│  └─ licenseFile: ./Frontend/node\_modules/exponential-backoff/LICENSE\\
├─ express@4.19.2\\
│  ├─ licenses: MIT\\
│  ├─ repository: https://github.com/expressjs/express\\
│  ├─ publisher: TJ Holowaychuk\\
│  ├─ email: tj@vision-media.ca\\
│  ├─ path: ./Frontend/node\_modules/express\\
│  └─ licenseFile: ./Frontend/node\_modules/express/LICENSE\\
├─ extend@3.0.2\\
│  ├─ licenses: MIT\\
│  ├─ repository: https://github.com/justmoon/node-extend\\
│  ├─ publisher: Stefan Thomas\\
│  ├─ email: justmoon@members.fsf.org\\
│  ├─ url: http://www.justmoon.net\\
│  ├─ path: ./Frontend/node\_modules/extend\\
│  └─ licenseFile: ./Frontend/node\_modules/extend/LICENSE\\
├─ external-editor@3.1.0\\
│  ├─ licenses: MIT\\
│  ├─ repository: https://github.com/mrkmg/node-external-editor\\
│  ├─ publisher: Kevin Gravier\\
│  ├─ email: kevin@mrkmg.com\\
│  ├─ url: https://mrkmg.com\\
│  ├─ path: ./Frontend/node\_modules/external-editor\\
│  └─ licenseFile: ./Frontend/node\_modules/external-editor/LICENSE\\
├─ fast-deep-equal@3.1.3\\
│  ├─ licenses: MIT\\
│  ├─ repository: https://github.com/epoberezkin/fast-deep-equal\\
│  ├─ publisher: Evgeny Poberezkin\\
│  ├─ path: ./Frontend/node\_modules/fast-deep-equal\\
│  └─ licenseFile: ./Frontend/node\_modules/fast-deep-equal/LICENSE\\
├─ fast-glob@3.3.2\\
│  ├─ licenses: MIT\\
│  ├─ repository: https://github.com/mrmlnc/fast-glob\\
│  ├─ publisher: Denis Malinochkin\\
│  ├─ url: https://mrmlnc.com\\
│  ├─ path: ./Frontend/node\_modules/fast-glob\\
│  └─ licenseFile: ./Frontend/node\_modules/fast-glob/LICENSE\\
├─ fast-json-stable-stringify@2.1.0\\
│  ├─ licenses: MIT\\
│  ├─ repository: https://github.com/epoberezkin/fast-json-stable-stringify\\
│  ├─ publisher: James Halliday\\
│  ├─ email: mail@substack.net\\
│  ├─ url: http://substack.net\\
│  ├─ path: ./Frontend/node\_modules/fast-json-stable-stringify\\
│  └─ licenseFile: ./Frontend/node\_modules/fast-json-stable-stringify/LICENSE\\
├─ fast-safe-stringify@2.1.1\\
│  ├─ licenses: MIT\\
│  ├─ repository: https://github.com/davidmarkclements/fast-safe-stringify\\
│  ├─ publisher: David Mark Clements\\
│  ├─ path: ./Frontend/node\_modules/fast-safe-stringify\\
│  └─ licenseFile: ./Frontend/node\_modules/fast-safe-stringify/LICENSE\\
├─ fastq@1.17.1\\
│  ├─ licenses: ISC\\
│  ├─ repository: https://github.com/mcollina/fastq\\
│  ├─ publisher: Matteo Collina\\
│  ├─ email: hello@matteocollina.com\\
│  ├─ path: ./Frontend/node\_modules/fastq\\
│  └─ licenseFile: ./Frontend/node\_modules/fastq/LICENSE\\
├─ faye-websocket@0.11.4\\
│  ├─ licenses: Apache-2.0\\
│  ├─ repository: https://github.com/faye/faye-websocket-node\\
│  ├─ publisher: James Coglan\\
│  ├─ email: jcoglan@gmail.com\\
│  ├─ url: http://jcoglan.com/\\
│  ├─ path: ./Frontend/node\_modules/faye-websocket\\
│  └─ licenseFile: ./Frontend/node\_modules/faye-websocket/LICENSE.md\\
├─ figures@3.2.0\\
│  ├─ licenses: MIT\\
│  ├─ repository: https://github.com/sindresorhus/figures\\
│  ├─ publisher: Sindre Sorhus\\
│  ├─ email: sindresorhus@gmail.com\\
│  ├─ url: https://sindresorhus.com\\
│  ├─ path: ./Frontend/node\_modules/figures\\
│  └─ licenseFile: ./Frontend/node\_modules/figures/license\\
├─ fill-range@7.0.1\\
│  ├─ licenses: MIT\\
│  ├─ repository: https://github.com/jonschlinkert/fill-range\\
│  ├─ publisher: Jon Schlinkert\\
│  ├─ url: https://github.com/jonschlinkert\\
│  ├─ path: ./Frontend/node\_modules/fill-range\\
│  └─ licenseFile: ./Frontend/node\_modules/fill-range/LICENSE\\
├─ finalhandler@1.1.2\\
│  ├─ licenses: MIT\\
│  ├─ repository: https://github.com/pillarjs/finalhandler\\
│  ├─ publisher: Douglas Christopher Wilson\\
│  ├─ email: doug@somethingdoug.com\\
│  ├─ path: ./Frontend/node\_modules/finalhandler\\
│  └─ licenseFile: ./Frontend/node\_modules/finalhandler/LICENSE\\
├─ finalhandler@1.2.0\\
│  ├─ licenses: MIT\\
│  ├─ repository: https://github.com/pillarjs/finalhandler\\
│  ├─ publisher: Douglas Christopher Wilson\\
│  ├─ email: doug@somethingdoug.com\\
│  ├─ path: ./Frontend/node\_modules/express/node\_modules/finalhandler\\
│  └─ licenseFile: ./Frontend/node\_modules/express/node\_modules/finalhandler/LICENSE\\
├─ find-cache-dir@4.0.0\\
│  ├─ licenses: MIT\\
│  ├─ repository: https://github.com/sindresorhus/find-cache-dir\\
│  ├─ publisher: Sindre Sorhus\\
│  ├─ email: sindresorhus@gmail.com\\
│  ├─ url: https://sindresorhus.com\\
│  ├─ path: ./Frontend/node\_modules/find-cache-dir\\
│  └─ licenseFile: ./Frontend/node\_modules/find-cache-dir/license\\
├─ find-up@4.1.0\\
│  ├─ licenses: MIT\\
│  ├─ repository: https://github.com/sindresorhus/find-up\\
│  ├─ publisher: Sindre Sorhus\\
│  ├─ email: sindresorhus@gmail.com\\
│  ├─ url: sindresorhus.com\\
│  ├─ path: ./Frontend/node\_modules/find-up\\
│  └─ licenseFile: ./Frontend/node\_modules/find-up/license\\
├─ find-up@6.3.0\\
│  ├─ licenses: MIT\\
│  ├─ repository: https://github.com/sindresorhus/find-up\\
│  ├─ publisher: Sindre Sorhus\\
│  ├─ email: sindresorhus@gmail.com\\
│  ├─ url: https://sindresorhus.com\\
│  ├─ path: ./Frontend/node\_modules/pkg-dir/node\_modules/find-up\\
│  └─ licenseFile: ./Frontend/node\_modules/pkg-dir/node\_modules/find-up/license\\
├─ flat@5.0.2\\
│  ├─ licenses: BSD-3-Clause\\
│  ├─ repository: https://github.com/hughsk/flat\\
│  ├─ publisher: Hugh Kennedy\\
│  ├─ email: hughskennedy@gmail.com\\
│  ├─ url: http://hughskennedy.com\\
│  ├─ path: ./Frontend/node\_modules/flat\\
│  └─ licenseFile: ./Frontend/node\_modules/flat/LICENSE\\
├─ flatted@3.3.1\\
│  ├─ licenses: ISC\\
│  ├─ repository: https://github.com/WebReflection/flatted\\
│  ├─ publisher: Andrea Giammarchi\\
│  ├─ path: ./Frontend/node\_modules/flatted\\
│  └─ licenseFile: ./Frontend/node\_modules/flatted/LICENSE\\
├─ follow-redirects@1.15.6\\
│  ├─ licenses: MIT\\
│  ├─ repository: https://github.com/follow-redirects/follow-redirects\\
│  ├─ publisher: Ruben Verborgh\\
│  ├─ email: ruben@verborgh.org\\
│  ├─ url: https://ruben.verborgh.org/\\
│  ├─ path: ./Frontend/node\_modules/follow-redirects\\
│  └─ licenseFile: ./Frontend/node\_modules/follow-redirects/LICENSE\\
├─ foreground-child@3.1.1\\
│  ├─ licenses: ISC\\
│  ├─ repository: https://github.com/tapjs/foreground-child\\
│  ├─ publisher: Isaac Z. Schlueter\\
│  ├─ email: i@izs.me\\
│  ├─ url: http://blog.izs.me/\\
│  ├─ path: ./Frontend/node\_modules/foreground-child\\
│  └─ licenseFile: ./Frontend/node\_modules/foreground-child/LICENSE\\
├─ form-data@4.0.0\\
│  ├─ licenses: MIT\\
│  ├─ repository: https://github.com/form-data/form-data\\
│  ├─ publisher: Felix Geisendörfer\\
│  ├─ email: felix@debuggable.com\\
│  ├─ url: http://debuggable.com/\\
│  ├─ path: ./Frontend/node\_modules/form-data\\
│  └─ licenseFile: ./Frontend/node\_modules/form-data/License\\
├─ forwarded@0.2.0\\
│  ├─ licenses: MIT\\
│  ├─ repository: https://github.com/jshttp/forwarded\\
│  ├─ path: ./Frontend/node\_modules/forwarded\\
│  └─ licenseFile: ./Frontend/node\_modules/forwarded/LICENSE\\
├─ fraction.js@4.3.7\\
│  ├─ licenses: MIT\\
│  ├─ repository: https://github.com/rawify/Fraction.js\\
│  ├─ publisher: Robert Eisele\\
│  ├─ email: robert@raw.org\\
│  ├─ url: https://raw.org/\\
│  ├─ path: ./Frontend/node\_modules/fraction.js\\
│  └─ licenseFile: ./Frontend/node\_modules/fraction.js/LICENSE\\
├─ fresh@0.5.2\\
│  ├─ licenses: MIT\\
│  ├─ repository: https://github.com/jshttp/fresh\\
│  ├─ publisher: TJ Holowaychuk\\
│  ├─ email: tj@vision-media.ca\\
│  ├─ url: http://tjholowaychuk.com\\
│  ├─ path: ./Frontend/node\_modules/fresh\\
│  └─ licenseFile: ./Frontend/node\_modules/fresh/LICENSE\\
├─ fs-extra@10.1.0\\
│  ├─ licenses: MIT\\
│  ├─ repository: https://github.com/jprichardson/node-fs-extra\\
│  ├─ publisher: JP Richardson\\
│  ├─ email: jprichardson@gmail.com\\
│  ├─ path: ./Frontend/node\_modules/@openapitools/openapi-generator-cli/node\_modules/fs-extra\\
│  └─ licenseFile: ./Frontend/node\_modules/@openapitools/openapi-generator-cli/node\_modules/fs-extra/LICENSE\\
├─ fs-extra@8.1.0\\
│  ├─ licenses: MIT\\
│  ├─ repository: https://github.com/jprichardson/node-fs-extra\\
│  ├─ publisher: JP Richardson\\
│  ├─ email: jprichardson@gmail.com\\
│  ├─ path: ./Frontend/node\_modules/fs-extra\\
│  └─ licenseFile: ./Frontend/node\_modules/fs-extra/LICENSE\\
├─ fs-minipass@2.1.0\\
│  ├─ licenses: ISC\\
│  ├─ repository: https://github.com/npm/fs-minipass\\
│  ├─ publisher: Isaac Z. Schlueter\\
│  ├─ email: i@izs.me\\
│  ├─ url: http://blog.izs.me/\\
│  ├─ path: ./Frontend/node\_modules/tar/node\_modules/fs-minipass\\
│  └─ licenseFile: ./Frontend/node\_modules/tar/node\_modules/fs-minipass/LICENSE\\
├─ fs-minipass@3.0.3\\
│  ├─ licenses: ISC\\
│  ├─ repository: https://github.com/npm/fs-minipass\\
│  ├─ publisher: GitHub Inc.\\
│  ├─ path: ./Frontend/node\_modules/fs-minipass\\
│  └─ licenseFile: ./Frontend/node\_modules/fs-minipass/LICENSE\\
├─ fs-monkey@1.0.6\\
│  ├─ licenses: Unlicense\\
│  ├─ repository: https://github.com/streamich/fs-monkey\\
│  ├─ path: ./Frontend/node\_modules/fs-monkey\\
│  └─ licenseFile: ./Frontend/node\_modules/fs-monkey/LICENSE\\
├─ fs.realpath@1.0.0\\
│  ├─ licenses: ISC\\
│  ├─ repository: https://github.com/isaacs/fs.realpath\\
│  ├─ publisher: Isaac Z. Schlueter\\
│  ├─ email: i@izs.me\\
│  ├─ url: http://blog.izs.me/\\
│  ├─ path: ./Frontend/node\_modules/fs.realpath\\
│  └─ licenseFile: ./Frontend/node\_modules/fs.realpath/LICENSE\\
├─ function-bind@1.1.2\\
│  ├─ licenses: MIT\\
│  ├─ repository: https://github.com/Raynos/function-bind\\
│  ├─ publisher: Raynos\\
│  ├─ email: raynos2@gmail.com\\
│  ├─ path: ./Frontend/node\_modules/function-bind\\
│  └─ licenseFile: ./Frontend/node\_modules/function-bind/LICENSE\\
├─ gensync@1.0.0-beta.2\\
│  ├─ licenses: MIT\\
│  ├─ repository: https://github.com/loganfsmyth/gensync\\
│  ├─ publisher: Logan Smyth\\
│  ├─ email: loganfsmyth@gmail.com\\
│  ├─ path: ./Frontend/node\_modules/gensync\\
│  └─ licenseFile: ./Frontend/node\_modules/gensync/LICENSE\\
├─ get-caller-file@2.0.5\\
│  ├─ licenses: ISC\\
│  ├─ repository: https://github.com/stefanpenner/get-caller-file\\
│  ├─ publisher: Stefan Penner\\
│  ├─ path: ./Frontend/node\_modules/get-caller-file\\
│  └─ licenseFile: ./Frontend/node\_modules/get-caller-file/LICENSE.md\\
├─ get-intrinsic@1.2.4\\
│  ├─ licenses: MIT\\
│  ├─ repository: https://github.com/ljharb/get-intrinsic\\
│  ├─ publisher: Jordan Harband\\
│  ├─ email: ljharb@gmail.com\\
│  ├─ path: ./Frontend/node\_modules/get-intrinsic\\
│  └─ licenseFile: ./Frontend/node\_modules/get-intrinsic/LICENSE\\
├─ get-package-type@0.1.0\\
│  ├─ licenses: MIT\\
│  ├─ repository: https://github.com/cfware/get-package-type\\
│  ├─ publisher: Corey Farrell\\
│  ├─ path: ./Frontend/node\_modules/get-package-type\\
│  └─ licenseFile: ./Frontend/node\_modules/get-package-type/LICENSE\\
├─ get-stream@6.0.1\\
│  ├─ licenses: MIT\\
│  ├─ repository: https://github.com/sindresorhus/get-stream\\
│  ├─ publisher: Sindre Sorhus\\
│  ├─ email: sindresorhus@gmail.com\\
│  ├─ url: https://sindresorhus.com\\
│  ├─ path: ./Frontend/node\_modules/get-stream\\
│  └─ licenseFile: ./Frontend/node\_modules/get-stream/license\\
├─ glob-parent@5.1.2\\
│  ├─ licenses: ISC\\
│  ├─ repository: https://github.com/gulpjs/glob-parent\\
│  ├─ publisher: Gulp Team\\
│  ├─ email: team@gulpjs.com\\
│  ├─ url: https://gulpjs.com/\\
│  ├─ path: ./Frontend/node\_modules/glob-parent\\
│  └─ licenseFile: ./Frontend/node\_modules/glob-parent/LICENSE\\
├─ glob-parent@6.0.2\\
│  ├─ licenses: ISC\\
│  ├─ repository: https://github.com/gulpjs/glob-parent\\
│  ├─ publisher: Gulp Team\\
│  ├─ email: team@gulpjs.com\\
│  ├─ url: https://gulpjs.com/\\
│  ├─ path: ./Frontend/node\_modules/copy-webpack-plugin/node\_modules/glob-parent\\
│  └─ licenseFile: ./Frontend/node\_modules/copy-webpack-plugin/node\_modules/glob-parent/LICENSE\\
├─ glob-to-regexp@0.4.1\\
│  ├─ licenses: BSD-2-Clause\\
│  ├─ repository: https://github.com/fitzgen/glob-to-regexp\\
│  ├─ publisher: Nick Fitzgerald\\
│  ├─ email: fitzgen@gmail.com\\
│  ├─ path: ./Frontend/node\_modules/glob-to-regexp\\
│  └─ licenseFile: ./Frontend/node\_modules/glob-to-regexp/README.md\\
├─ glob@10.3.15\\
│  ├─ licenses: ISC\\
│  ├─ repository: https://github.com/isaacs/node-glob\\
│  ├─ publisher: Isaac Z. Schlueter\\
│  ├─ email: i@izs.me\\
│  ├─ url: https://blog.izs.me/\\
│  ├─ path: ./Frontend/node\_modules/cacache/node\_modules/glob\\
│  └─ licenseFile: ./Frontend/node\_modules/cacache/node\_modules/glob/LICENSE\\
├─ glob@7.2.3\\
│  ├─ licenses: ISC\\
│  ├─ repository: https://github.com/isaacs/node-glob\\
│  ├─ publisher: Isaac Z. Schlueter\\
│  ├─ email: i@izs.me\\
│  ├─ url: http://blog.izs.me/\\
│  ├─ path: ./Frontend/node\_modules/glob\\
│  └─ licenseFile: ./Frontend/node\_modules/glob/LICENSE\\
├─ globals@11.12.0\\
│  ├─ licenses: MIT\\
│  ├─ repository: https://github.com/sindresorhus/globals\\
│  ├─ publisher: Sindre Sorhus\\
│  ├─ email: sindresorhus@gmail.com\\
│  ├─ url: sindresorhus.com\\
│  ├─ path: ./Frontend/node\_modules/globals\\
│  └─ licenseFile: ./Frontend/node\_modules/globals/license\\
├─ globby@13.2.2\\
│  ├─ licenses: MIT\\
│  ├─ repository: https://github.com/sindresorhus/globby\\
│  ├─ publisher: Sindre Sorhus\\
│  ├─ email: sindresorhus@gmail.com\\
│  ├─ url: https://sindresorhus.com\\
│  ├─ path: ./Frontend/node\_modules/globby\\
│  └─ licenseFile: ./Frontend/node\_modules/globby/license\\
├─ gopd@1.0.1\\
│  ├─ licenses: MIT\\
│  ├─ repository: https://github.com/ljharb/gopd\\
│  ├─ publisher: Jordan Harband\\
│  ├─ email: ljharb@gmail.com\\
│  ├─ path: ./Frontend/node\_modules/gopd\\
│  └─ licenseFile: ./Frontend/node\_modules/gopd/LICENSE\\
├─ graceful-fs@4.2.11\\
│  ├─ licenses: ISC\\
│  ├─ repository: https://github.com/isaacs/node-graceful-fs\\
│  ├─ path: ./Frontend/node\_modules/graceful-fs\\
│  └─ licenseFile: ./Frontend/node\_modules/graceful-fs/LICENSE\\
├─ handle-thing@2.0.1\\
│  ├─ licenses: MIT\\
│  ├─ repository: https://github.com/indutny/handle-thing\\
│  ├─ publisher: Fedor Indutny\\
│  ├─ email: fedor@indutny.com\\
│  ├─ path: ./Frontend/node\_modules/handle-thing\\
│  └─ licenseFile: ./Frontend/node\_modules/handle-thing/README.md\\
├─ has-flag@3.0.0\\
│  ├─ licenses: MIT\\
│  ├─ repository: https://github.com/sindresorhus/has-flag\\
│  ├─ publisher: Sindre Sorhus\\
│  ├─ email: sindresorhus@gmail.com\\
│  ├─ url: sindresorhus.com\\
│  ├─ path: ./Frontend/node\_modules/has-flag\\
│  └─ licenseFile: ./Frontend/node\_modules/has-flag/license\\
├─ has-flag@4.0.0\\
│  ├─ licenses: MIT\\
│  ├─ repository: https://github.com/sindresorhus/has-flag\\
│  ├─ publisher: Sindre Sorhus\\
│  ├─ email: sindresorhus@gmail.com\\
│  ├─ url: sindresorhus.com\\
│  ├─ path: ./Frontend/node\_modules/critters/node\_modules/has-flag\\
│  └─ licenseFile: ./Frontend/node\_modules/critters/node\_modules/has-flag/license\\
├─ has-property-descriptors@1.0.2\\
│  ├─ licenses: MIT\\
│  ├─ repository: https://github.com/inspect-js/has-property-descriptors\\
│  ├─ publisher: Jordan Harband\\
│  ├─ email: ljharb@gmail.com\\
│  ├─ path: ./Frontend/node\_modules/has-property-descriptors\\
│  └─ licenseFile: ./Frontend/node\_modules/has-property-descriptors/LICENSE\\
├─ has-proto@1.0.3\\
│  ├─ licenses: MIT\\
│  ├─ repository: https://github.com/inspect-js/has-proto\\
│  ├─ publisher: Jordan Harband\\
│  ├─ email: ljharb@gmail.com\\
│  ├─ path: ./Frontend/node\_modules/has-proto\\
│  └─ licenseFile: ./Frontend/node\_modules/has-proto/LICENSE\\
├─ has-symbols@1.0.3\\
│  ├─ licenses: MIT\\
│  ├─ repository: https://github.com/inspect-js/has-symbols\\
│  ├─ publisher: Jordan Harband\\
│  ├─ email: ljharb@gmail.com\\
│  ├─ url: http://ljharb.codes\\
│  ├─ path: ./Frontend/node\_modules/has-symbols\\
│  └─ licenseFile: ./Frontend/node\_modules/has-symbols/LICENSE\\
├─ hasown@2.0.2\\
│  ├─ licenses: MIT\\
│  ├─ repository: https://github.com/inspect-js/hasOwn\\
│  ├─ publisher: Jordan Harband\\
│  ├─ email: ljharb@gmail.com\\
│  ├─ path: ./Frontend/node\_modules/hasown\\
│  └─ licenseFile: ./Frontend/node\_modules/hasown/LICENSE\\
├─ hosted-git-info@7.0.2\\
│  ├─ licenses: ISC\\
│  ├─ repository: https://github.com/npm/hosted-git-info\\
│  ├─ publisher: GitHub Inc.\\
│  ├─ path: ./Frontend/node\_modules/hosted-git-info\\
│  └─ licenseFile: ./Frontend/node\_modules/hosted-git-info/LICENSE\\
├─ hpack.js@2.1.6\\
│  ├─ licenses: MIT\\
│  ├─ repository: https://github.com/indutny/hpack.js\\
│  ├─ publisher: Fedor Indutny\\
│  ├─ email: fedor@indutny.com\\
│  ├─ path: ./Frontend/node\_modules/hpack.js\\
│  └─ licenseFile: ./Frontend/node\_modules/hpack.js/README.md\\
├─ html-entities@2.5.2\\
│  ├─ licenses: MIT\\
│  ├─ repository: https://github.com/mdevils/html-entities\\
│  ├─ publisher: Marat Dulin\\
│  ├─ email: mdevils@yandex.ru\\
│  ├─ path: ./Frontend/node\_modules/html-entities\\
│  └─ licenseFile: ./Frontend/node\_modules/html-entities/LICENSE\\
├─ html-escaper@2.0.2\\
│  ├─ licenses: MIT\\
│  ├─ repository: https://github.com/WebReflection/html-escaper\\
│  ├─ publisher: Andrea Giammarchi\\
│  ├─ path: ./Frontend/node\_modules/html-escaper\\
│  └─ licenseFile: ./Frontend/node\_modules/html-escaper/LICENSE.txt\\
├─ htmlparser2@8.0.2\\
│  ├─ licenses: MIT\\
│  ├─ repository: https://github.com/fb55/htmlparser2\\
│  ├─ publisher: Felix Boehm\\
│  ├─ email: me@feedic.com\\
│  ├─ path: ./Frontend/node\_modules/htmlparser2\\
│  └─ licenseFile: ./Frontend/node\_modules/htmlparser2/LICENSE\\
├─ http-cache-semantics@4.1.1\\
│  ├─ licenses: BSD-2-Clause\\
│  ├─ repository: https://github.com/kornelski/http-cache-semantics\\
│  ├─ publisher: Kornel Lesiński\\
│  ├─ email: kornel@geekhood.net\\
│  ├─ url: https://kornel.ski/\\
│  ├─ path: ./Frontend/node\_modules/http-cache-semantics\\
│  └─ licenseFile: ./Frontend/node\_modules/http-cache-semantics/LICENSE\\
├─ http-deceiver@1.2.7\\
│  ├─ licenses: MIT\\
│  ├─ repository: https://github.com/indutny/http-deceiver\\
│  ├─ publisher: Fedor Indutny\\
│  ├─ email: fedor@indutny.com\\
│  ├─ path: ./Frontend/node\_modules/http-deceiver\\
│  └─ licenseFile: ./Frontend/node\_modules/http-deceiver/README.md\\
├─ http-errors@1.6.3\\
│  ├─ licenses: MIT\\
│  ├─ repository: https://github.com/jshttp/http-errors\\
│  ├─ publisher: Jonathan Ong\\
│  ├─ email: me@jongleberry.com\\
│  ├─ url: http://jongleberry.com\\
│  ├─ path: ./Frontend/node\_modules/serve-index/node\_modules/http-errors\\
│  └─ licenseFile: ./Frontend/node\_modules/serve-index/node\_modules/http-errors/LICENSE\\
├─ http-errors@2.0.0\\
│  ├─ licenses: MIT\\
│  ├─ repository: https://github.com/jshttp/http-errors\\
│  ├─ publisher: Jonathan Ong\\
│  ├─ email: me@jongleberry.com\\
│  ├─ url: http://jongleberry.com\\
│  ├─ path: ./Frontend/node\_modules/http-errors\\
│  └─ licenseFile: ./Frontend/node\_modules/http-errors/LICENSE\\
├─ http-parser-js@0.5.8\\
│  ├─ licenses: MIT\\
│  ├─ repository: https://github.com/creationix/http-parser-js\\
│  ├─ publisher: Tim Caswell\\
│  ├─ url: https://github.com/creationix\\
│  ├─ path: ./Frontend/node\_modules/http-parser-js\\
│  └─ licenseFile: ./Frontend/node\_modules/http-parser-js/LICENSE.md\\
├─ http-proxy-agent@7.0.2\\
│  ├─ licenses: MIT\\
│  ├─ repository: https://github.com/TooTallNate/proxy-agents\\
│  ├─ publisher: Nathan Rajlich\\
│  ├─ email: nathan@tootallnate.net\\
│  ├─ url: http://n8.io/\\
│  ├─ path: ./Frontend/node\_modules/http-proxy-agent\\
│  └─ licenseFile: ./Frontend/node\_modules/http-proxy-agent/LICENSE\\
├─ http-proxy-middleware@2.0.6\\
│  ├─ licenses: MIT\\
│  ├─ repository: https://github.com/chimurai/http-proxy-middleware\\
│  ├─ publisher: Steven Chim\\
│  ├─ path: ./Frontend/node\_modules/http-proxy-middleware\\
│  └─ licenseFile: ./Frontend/node\_modules/http-proxy-middleware/LICENSE\\
├─ http-proxy@1.18.1\\
│  ├─ licenses: MIT\\
│  ├─ repository: https://github.com/http-party/node-http-proxy\\
│  ├─ publisher: Charlie Robbins\\
│  ├─ email: charlie.robbins@gmail.com\\
│  ├─ path: ./Frontend/node\_modules/http-proxy\\
│  └─ licenseFile: ./Frontend/node\_modules/http-proxy/LICENSE\\
├─ https-proxy-agent@7.0.4\\
│  ├─ licenses: MIT\\
│  ├─ repository: https://github.com/TooTallNate/proxy-agents\\
│  ├─ publisher: Nathan Rajlich\\
│  ├─ email: nathan@tootallnate.net\\
│  ├─ url: http://n8.io/\\
│  ├─ path: ./Frontend/node\_modules/https-proxy-agent\\
│  └─ licenseFile: ./Frontend/node\_modules/https-proxy-agent/LICENSE\\
├─ human-signals@2.1.0\\
│  ├─ licenses: Apache-2.0\\
│  ├─ repository: https://github.com/ehmicky/human-signals\\
│  ├─ publisher: ehmicky\\
│  ├─ email: ehmicky@gmail.com\\
│  ├─ url: https://github.com/ehmicky\\
│  ├─ path: ./Frontend/node\_modules/human-signals\\
│  └─ licenseFile: ./Frontend/node\_modules/human-signals/LICENSE\\
├─ iban@0.0.14\\
│  ├─ licenses: MIT\\
│  ├─ repository: https://github.com/arhs/iban.js\\
│  ├─ publisher: Laurent VB\\
│  ├─ path: ./Frontend/node\_modules/iban\\
│  └─ licenseFile: ./Frontend/node\_modules/iban/LICENSE\\
├─ iconv-lite@0.4.24\\
│  ├─ licenses: MIT\\
│  ├─ repository: https://github.com/ashtuchkin/iconv-lite\\
│  ├─ publisher: Alexander Shtuchkin\\
│  ├─ email: ashtuchkin@gmail.com\\
│  ├─ path: ./Frontend/node\_modules/iconv-lite\\
│  └─ licenseFile: ./Frontend/node\_modules/iconv-lite/LICENSE\\
├─ iconv-lite@0.6.3\\
│  ├─ licenses: MIT\\
│  ├─ repository: https://github.com/ashtuchkin/iconv-lite\\
│  ├─ publisher: Alexander Shtuchkin\\
│  ├─ email: ashtuchkin@gmail.com\\
│  ├─ path: ./Frontend/node\_modules/source-map-loader/node\_modules/iconv-lite\\
│  └─ licenseFile: ./Frontend/node\_modules/source-map-loader/node\_modules/iconv-lite/LICENSE\\
├─ icss-utils@5.1.0\\
│  ├─ licenses: ISC\\
│  ├─ repository: https://github.com/css-modules/icss-utils\\
│  ├─ publisher: Glen Maddern\\
│  ├─ path: ./Frontend/node\_modules/icss-utils\\
│  └─ licenseFile: ./Frontend/node\_modules/icss-utils/LICENSE.md\\
├─ ieee754@1.2.1\\
│  ├─ licenses: BSD-3-Clause\\
│  ├─ repository: https://github.com/feross/ieee754\\
│  ├─ publisher: Feross Aboukhadijeh\\
│  ├─ email: feross@feross.org\\
│  ├─ url: https://feross.org\\
│  ├─ path: ./Frontend/node\_modules/ieee754\\
│  └─ licenseFile: ./Frontend/node\_modules/ieee754/LICENSE\\
├─ ignore-walk@6.0.5\\
│  ├─ licenses: ISC\\
│  ├─ repository: https://github.com/npm/ignore-walk\\
│  ├─ publisher: GitHub Inc.\\
│  ├─ path: ./Frontend/node\_modules/ignore-walk\\
│  └─ licenseFile: ./Frontend/node\_modules/ignore-walk/LICENSE\\
├─ ignore@5.3.1\\
│  ├─ licenses: MIT\\
│  ├─ repository: https://github.com/kaelzhang/node-ignore\\
│  ├─ publisher: kael\\
│  ├─ path: ./Frontend/node\_modules/ignore\\
│  └─ licenseFile: ./Frontend/node\_modules/ignore/LICENSE-MIT\\
├─ image-size@0.5.5\\
│  ├─ licenses: MIT\\
│  ├─ repository: https://github.com/image-size/image-size\\
│  ├─ publisher: netroy\\
│  ├─ email: aditya@netroy.in\\
│  ├─ url: http://netroy.in/\\
│  ├─ path: ./Frontend/node\_modules/image-size\\
│  └─ licenseFile: ./Frontend/node\_modules/image-size/LICENSE\\
├─ immutable@4.3.6\\
│  ├─ licenses: MIT\\
│  ├─ repository: https://github.com/immutable-js/immutable-js\\
│  ├─ publisher: Lee Byron\\
│  ├─ url: https://github.com/leebyron\\
│  ├─ path: ./Frontend/node\_modules/immutable\\
│  └─ licenseFile: ./Frontend/node\_modules/immutable/LICENSE\\
├─ import-fresh@3.3.0\\
│  ├─ licenses: MIT\\
│  ├─ repository: https://github.com/sindresorhus/import-fresh\\
│  ├─ publisher: Sindre Sorhus\\
│  ├─ email: sindresorhus@gmail.com\\
│  ├─ url: https://sindresorhus.com\\
│  ├─ path: ./Frontend/node\_modules/import-fresh\\
│  └─ licenseFile: ./Frontend/node\_modules/import-fresh/license\\
├─ imurmurhash@0.1.4\\
│  ├─ licenses: MIT\\
│  ├─ repository: https://github.com/jensyt/imurmurhash-js\\
│  ├─ publisher: Jens Taylor\\
│  ├─ email: jensyt@gmail.com\\
│  ├─ url: https://github.com/homebrewing\\
│  ├─ path: ./Frontend/node\_modules/imurmurhash\\
│  └─ licenseFile: ./Frontend/node\_modules/imurmurhash/README.md\\
├─ indent-string@4.0.0\\
│  ├─ licenses: MIT\\
│  ├─ repository: https://github.com/sindresorhus/indent-string\\
│  ├─ publisher: Sindre Sorhus\\
│  ├─ email: sindresorhus@gmail.com\\
│  ├─ url: sindresorhus.com\\
│  ├─ path: ./Frontend/node\_modules/indent-string\\
│  └─ licenseFile: ./Frontend/node\_modules/indent-string/license\\
├─ inflight@1.0.6\\
│  ├─ licenses: ISC\\
│  ├─ repository: https://github.com/npm/inflight\\
│  ├─ publisher: Isaac Z. Schlueter\\
│  ├─ email: i@izs.me\\
│  ├─ url: http://blog.izs.me/\\
│  ├─ path: ./Frontend/node\_modules/inflight\\
│  └─ licenseFile: ./Frontend/node\_modules/inflight/LICENSE\\
├─ inherits@2.0.3\\
│  ├─ licenses: ISC\\
│  ├─ repository: https://github.com/isaacs/inherits\\
│  ├─ path: ./Frontend/node\_modules/serve-index/node\_modules/inherits\\
│  └─ licenseFile: ./Frontend/node\_modules/serve-index/node\_modules/inherits/LICENSE\\
├─ inherits@2.0.4\\
│  ├─ licenses: ISC\\
│  ├─ repository: https://github.com/isaacs/inherits\\
│  ├─ path: ./Frontend/node\_modules/inherits\\
│  └─ licenseFile: ./Frontend/node\_modules/inherits/LICENSE\\
├─ ini@4.1.2\\
│  ├─ licenses: ISC\\
│  ├─ repository: https://github.com/npm/ini\\
│  ├─ publisher: GitHub Inc.\\
│  ├─ path: ./Frontend/node\_modules/ini\\
│  └─ licenseFile: ./Frontend/node\_modules/ini/LICENSE\\
├─ inquirer@8.2.6\\
│  ├─ licenses: MIT\\
│  ├─ repository: https://github.com/SBoudrias/Inquirer.js\\
│  ├─ publisher: Simon Boudrias\\
│  ├─ email: admin@simonboudrias.com\\
│  ├─ path: ./Frontend/node\_modules/@openapitools/openapi-generator-cli/node\_modules/inquirer\\
│  └─ licenseFile: ./Frontend/node\_modules/@openapitools/openapi-generator-cli/node\_modules/inquirer/LICENSE\\
├─ inquirer@9.2.15\\
│  ├─ licenses: MIT\\
│  ├─ repository: https://github.com/SBoudrias/Inquirer.js\\
│  ├─ publisher: Simon Boudrias\\
│  ├─ email: admin@simonboudrias.com\\
│  ├─ path: ./Frontend/node\_modules/inquirer\\
│  └─ licenseFile: ./Frontend/node\_modules/inquirer/LICENSE\\
├─ ip-address@9.0.5\\
│  ├─ licenses: MIT\\
│  ├─ repository: https://github.com/beaugunderson/ip-address\\
│  ├─ publisher: Beau Gunderson\\
│  ├─ email: beau@beaugunderson.com\\
│  ├─ url: https://beaugunderson.com/\\
│  ├─ path: ./Frontend/node\_modules/ip-address\\
│  └─ licenseFile: ./Frontend/node\_modules/ip-address/LICENSE\\
├─ ipaddr.js@1.9.1\\
│  ├─ licenses: MIT\\
│  ├─ repository: https://github.com/whitequark/ipaddr.js\\
│  ├─ publisher: whitequark\\
│  ├─ email: whitequark@whitequark.org\\
│  ├─ path: ./Frontend/node\_modules/proxy-addr/node\_modules/ipaddr.js\\
│  └─ licenseFile: ./Frontend/node\_modules/proxy-addr/node\_modules/ipaddr.js/LICENSE\\
├─ ipaddr.js@2.2.0\\
│  ├─ licenses: MIT\\
│  ├─ repository: https://github.com/whitequark/ipaddr.js\\
│  ├─ publisher: whitequark\\
│  ├─ email: whitequark@whitequark.org\\
│  ├─ path: ./Frontend/node\_modules/ipaddr.js\\
│  └─ licenseFile: ./Frontend/node\_modules/ipaddr.js/LICENSE\\
├─ is-arrayish@0.2.1\\
│  ├─ licenses: MIT\\
│  ├─ repository: https://github.com/qix-/node-is-arrayish\\
│  ├─ publisher: Qix\\
│  ├─ url: http://github.com/qix-\\
│  ├─ path: ./Frontend/node\_modules/is-arrayish\\
│  └─ licenseFile: ./Frontend/node\_modules/is-arrayish/LICENSE\\
├─ is-binary-path@2.1.0\\
│  ├─ licenses: MIT\\
│  ├─ repository: https://github.com/sindresorhus/is-binary-path\\
│  ├─ publisher: Sindre Sorhus\\
│  ├─ email: sindresorhus@gmail.com\\
│  ├─ url: sindresorhus.com\\
│  ├─ path: ./Frontend/node\_modules/is-binary-path\\
│  └─ licenseFile: ./Frontend/node\_modules/is-binary-path/license\\
├─ is-core-module@2.13.1\\
│  ├─ licenses: MIT\\
│  ├─ repository: https://github.com/inspect-js/is-core-module\\
│  ├─ publisher: Jordan Harband\\
│  ├─ email: ljharb@gmail.com\\
│  ├─ path: ./Frontend/node\_modules/is-core-module\\
│  └─ licenseFile: ./Frontend/node\_modules/is-core-module/LICENSE\\
├─ is-docker@2.2.1\\
│  ├─ licenses: MIT\\
│  ├─ repository: https://github.com/sindresorhus/is-docker\\
│  ├─ publisher: Sindre Sorhus\\
│  ├─ email: sindresorhus@gmail.com\\
│  ├─ url: https://sindresorhus.com\\
│  ├─ path: ./Frontend/node\_modules/is-docker\\
│  └─ licenseFile: ./Frontend/node\_modules/is-docker/license\\
├─ is-extglob@2.1.1\\
│  ├─ licenses: MIT\\
│  ├─ repository: https://github.com/jonschlinkert/is-extglob\\
│  ├─ publisher: Jon Schlinkert\\
│  ├─ url: https://github.com/jonschlinkert\\
│  ├─ path: ./Frontend/node\_modules/is-extglob\\
│  └─ licenseFile: ./Frontend/node\_modules/is-extglob/LICENSE\\
├─ is-fullwidth-code-point@3.0.0\\
│  ├─ licenses: MIT\\
│  ├─ repository: https://github.com/sindresorhus/is-fullwidth-code-point\\
│  ├─ publisher: Sindre Sorhus\\
│  ├─ email: sindresorhus@gmail.com\\
│  ├─ url: sindresorhus.com\\
│  ├─ path: ./Frontend/node\_modules/is-fullwidth-code-point\\
│  └─ licenseFile: ./Frontend/node\_modules/is-fullwidth-code-point/license\\
├─ is-glob@4.0.3\\
│  ├─ licenses: MIT\\
│  ├─ repository: https://github.com/micromatch/is-glob\\
│  ├─ publisher: Jon Schlinkert\\
│  ├─ url: https://github.com/jonschlinkert\\
│  ├─ path: ./Frontend/node\_modules/is-glob\\
│  └─ licenseFile: ./Frontend/node\_modules/is-glob/LICENSE\\
├─ is-interactive@1.0.0\\
│  ├─ licenses: MIT\\
│  ├─ repository: https://github.com/sindresorhus/is-interactive\\
│  ├─ publisher: Sindre Sorhus\\
│  ├─ email: sindresorhus@gmail.com\\
│  ├─ url: sindresorhus.com\\
│  ├─ path: ./Frontend/node\_modules/is-interactive\\
│  └─ licenseFile: ./Frontend/node\_modules/is-interactive/license\\
├─ is-lambda@1.0.1\\
│  ├─ licenses: MIT\\
│  ├─ repository: https://github.com/watson/is-lambda\\
│  ├─ publisher: Thomas Watson Steen\\
│  ├─ email: w@tson.dk\\
│  ├─ url: https://twitter.com/wa7son\\
│  ├─ path: ./Frontend/node\_modules/is-lambda\\
│  └─ licenseFile: ./Frontend/node\_modules/is-lambda/LICENSE\\
├─ is-number@7.0.0\\
│  ├─ licenses: MIT\\
│  ├─ repository: https://github.com/jonschlinkert/is-number\\
│  ├─ publisher: Jon Schlinkert\\
│  ├─ url: https://github.com/jonschlinkert\\
│  ├─ path: ./Frontend/node\_modules/is-number\\
│  └─ licenseFile: ./Frontend/node\_modules/is-number/LICENSE\\
├─ is-plain-obj@3.0.0\\
│  ├─ licenses: MIT\\
│  ├─ repository: https://github.com/sindresorhus/is-plain-obj\\
│  ├─ publisher: Sindre Sorhus\\
│  ├─ email: sindresorhus@gmail.com\\
│  ├─ url: https://sindresorhus.com\\
│  ├─ path: ./Frontend/node\_modules/is-plain-obj\\
│  └─ licenseFile: ./Frontend/node\_modules/is-plain-obj/license\\
├─ is-plain-object@2.0.4\\
│  ├─ licenses: MIT\\
│  ├─ repository: https://github.com/jonschlinkert/is-plain-object\\
│  ├─ publisher: Jon Schlinkert\\
│  ├─ url: https://github.com/jonschlinkert\\
│  ├─ path: ./Frontend/node\_modules/is-plain-object\\
│  └─ licenseFile: ./Frontend/node\_modules/is-plain-object/LICENSE\\
├─ is-stream@2.0.1\\
│  ├─ licenses: MIT\\
│  ├─ repository: https://github.com/sindresorhus/is-stream\\
│  ├─ publisher: Sindre Sorhus\\
│  ├─ email: sindresorhus@gmail.com\\
│  ├─ url: https://sindresorhus.com\\
│  ├─ path: ./Frontend/node\_modules/is-stream\\
│  └─ licenseFile: ./Frontend/node\_modules/is-stream/license\\
├─ is-unicode-supported@0.1.0\\
│  ├─ licenses: MIT\\
│  ├─ repository: https://github.com/sindresorhus/is-unicode-supported\\
│  ├─ publisher: Sindre Sorhus\\
│  ├─ email: sindresorhus@gmail.com\\
│  ├─ url: https://sindresorhus.com\\
│  ├─ path: ./Frontend/node\_modules/is-unicode-supported\\
│  └─ licenseFile: ./Frontend/node\_modules/is-unicode-supported/license\\
├─ is-what@3.14.1\\
│  ├─ licenses: MIT\\
│  ├─ repository: https://github.com/mesqueeb/is-what\\
│  ├─ publisher: Luca Ban - Mesqueeb\\
│  ├─ path: ./Frontend/node\_modules/is-what\\
│  └─ licenseFile: ./Frontend/node\_modules/is-what/LICENSE\\
├─ is-wsl@2.2.0\\
│  ├─ licenses: MIT\\
│  ├─ repository: https://github.com/sindresorhus/is-wsl\\
│  ├─ publisher: Sindre Sorhus\\
│  ├─ email: sindresorhus@gmail.com\\
│  ├─ url: sindresorhus.com\\
│  ├─ path: ./Frontend/node\_modules/is-wsl\\
│  └─ licenseFile: ./Frontend/node\_modules/is-wsl/license\\
├─ isarray@1.0.0\\
│  ├─ licenses: MIT\\
│  ├─ repository: https://github.com/juliangruber/isarray\\
│  ├─ publisher: Julian Gruber\\
│  ├─ email: mail@juliangruber.com\\
│  ├─ url: http://juliangruber.com\\
│  ├─ path: ./Frontend/node\_modules/isarray\\
│  └─ licenseFile: ./Frontend/node\_modules/isarray/README.md\\
├─ isbinaryfile@4.0.10\\
│  ├─ licenses: MIT\\
│  ├─ repository: https://github.com/gjtorikian/isBinaryFile\\
│  ├─ path: ./Frontend/node\_modules/isbinaryfile\\
│  └─ licenseFile: ./Frontend/node\_modules/isbinaryfile/LICENSE.txt\\
├─ isexe@2.0.0\\
│  ├─ licenses: ISC\\
│  ├─ repository: https://github.com/isaacs/isexe\\
│  ├─ publisher: Isaac Z. Schlueter\\
│  ├─ email: i@izs.me\\
│  ├─ url: http://blog.izs.me/\\
│  ├─ path: ./Frontend/node\_modules/isexe\\
│  └─ licenseFile: ./Frontend/node\_modules/isexe/LICENSE\\
├─ isexe@3.1.1\\
│  ├─ licenses: ISC\\
│  ├─ repository: https://github.com/isaacs/isexe\\
│  ├─ publisher: Isaac Z. Schlueter\\
│  ├─ email: i@izs.me\\
│  ├─ url: http://blog.izs.me/\\
│  ├─ path: ./Frontend/node\_modules/@npmcli/promise-spawn/node\_modules/isexe\\
│  └─ licenseFile: ./Frontend/node\_modules/@npmcli/promise-spawn/node\_modules/isexe/LICENSE\\
├─ isobject@3.0.1\\
│  ├─ licenses: MIT\\
│  ├─ repository: https://github.com/jonschlinkert/isobject\\
│  ├─ publisher: Jon Schlinkert\\
│  ├─ url: https://github.com/jonschlinkert\\
│  ├─ path: ./Frontend/node\_modules/isobject\\
│  └─ licenseFile: ./Frontend/node\_modules/isobject/LICENSE\\
├─ istanbul-lib-coverage@3.2.2\\
│  ├─ licenses: BSD-3-Clause\\
│  ├─ repository: https://github.com/istanbuljs/istanbuljs\\
│  ├─ publisher: Krishnan Anantheswaran\\
│  ├─ email: kananthmail-github@yahoo.com\\
│  ├─ path: ./Frontend/node\_modules/istanbul-lib-coverage\\
│  └─ licenseFile: ./Frontend/node\_modules/istanbul-lib-coverage/LICENSE\\
├─ istanbul-lib-instrument@5.2.1\\
│  ├─ licenses: BSD-3-Clause\\
│  ├─ repository: https://github.com/istanbuljs/istanbuljs\\
│  ├─ publisher: Krishnan Anantheswaran\\
│  ├─ email: kananthmail-github@yahoo.com\\
│  ├─ path: ./Frontend/node\_modules/istanbul-lib-instrument\\
│  └─ licenseFile: ./Frontend/node\_modules/istanbul-lib-instrument/LICENSE\\
├─ istanbul-lib-report@3.0.1\\
│  ├─ licenses: BSD-3-Clause\\
│  ├─ repository: https://github.com/istanbuljs/istanbuljs\\
│  ├─ publisher: Krishnan Anantheswaran\\
│  ├─ email: kananthmail-github@yahoo.com\\
│  ├─ path: ./Frontend/node\_modules/istanbul-lib-report\\
│  └─ licenseFile: ./Frontend/node\_modules/istanbul-lib-report/LICENSE\\
├─ istanbul-lib-source-maps@4.0.1\\
│  ├─ licenses: BSD-3-Clause\\
│  ├─ repository: https://github.com/istanbuljs/istanbuljs\\
│  ├─ publisher: Krishnan Anantheswaran\\
│  ├─ email: kananthmail-github@yahoo.com\\
│  ├─ path: ./Frontend/node\_modules/istanbul-lib-source-maps\\
│  └─ licenseFile: ./Frontend/node\_modules/istanbul-lib-source-maps/LICENSE\\
├─ istanbul-reports@3.1.7\\
│  ├─ licenses: BSD-3-Clause\\
│  ├─ repository: https://github.com/istanbuljs/istanbuljs\\
│  ├─ publisher: Krishnan Anantheswaran\\
│  ├─ email: kananthmail-github@yahoo.com\\
│  ├─ path: ./Frontend/node\_modules/istanbul-reports\\
│  └─ licenseFile: ./Frontend/node\_modules/istanbul-reports/LICENSE\\
├─ iterare@1.2.1\\
│  ├─ licenses: ISC\\
│  ├─ repository: https://github.com/felixfbecker/iterare\\
│  ├─ publisher: Felix Becker\\
│  ├─ email: felix.b@outlook.com\\
│  ├─ path: ./Frontend/node\_modules/iterare\\
│  └─ licenseFile: ./Frontend/node\_modules/iterare/LICENSE.txt\\
├─ jackspeak@2.3.6\\
│  ├─ licenses: BlueOak-1.0.0\\
│  ├─ repository: https://github.com/isaacs/jackspeak\\
│  ├─ publisher: Isaac Z. Schlueter\\
│  ├─ email: i@izs.me\\
│  ├─ path: ./Frontend/node\_modules/jackspeak\\
│  └─ licenseFile: ./Frontend/node\_modules/jackspeak/LICENSE.md\\
├─ jasmine-core@4.6.0\\
│  ├─ licenses: MIT\\
│  ├─ repository: https://github.com/jasmine/jasmine\\
│  ├─ path: ./Frontend/node\_modules/karma-jasmine/node\_modules/jasmine-core\\
│  └─ licenseFile: ./Frontend/node\_modules/karma-jasmine/node\_modules/jasmine-core/README.md\\
├─ jasmine-core@5.1.2\\
│  ├─ licenses: MIT\\
│  ├─ repository: https://github.com/jasmine/jasmine\\
│  ├─ path: ./Frontend/node\_modules/jasmine-core\\
│  └─ licenseFile: ./Frontend/node\_modules/jasmine-core/LICENSE\\
├─ jest-worker@27.5.1\\
│  ├─ licenses: MIT\\
│  ├─ repository: https://github.com/facebook/jest\\
│  ├─ path: ./Frontend/node\_modules/jest-worker\\
│  └─ licenseFile: ./Frontend/node\_modules/jest-worker/LICENSE\\
├─ jiti@1.21.0\\
│  ├─ licenses: MIT\\
│  ├─ repository: https://github.com/unjs/jiti\\
│  ├─ path: ./Frontend/node\_modules/jiti\\
│  └─ licenseFile: ./Frontend/node\_modules/jiti/LICENSE\\
├─ js-tokens@4.0.0\\
│  ├─ licenses: MIT\\
│  ├─ repository: https://github.com/lydell/js-tokens\\
│  ├─ publisher: Simon Lydell\\
│  ├─ path: ./Frontend/node\_modules/js-tokens\\
│  └─ licenseFile: ./Frontend/node\_modules/js-tokens/LICENSE\\
├─ js-yaml@3.14.1\\
│  ├─ licenses: MIT\\
│  ├─ repository: https://github.com/nodeca/js-yaml\\
│  ├─ publisher: Vladimir Zapparov\\
│  ├─ email: dervus.grim@gmail.com\\
│  ├─ path: ./Frontend/node\_modules/js-yaml\\
│  └─ licenseFile: ./Frontend/node\_modules/js-yaml/LICENSE\\
├─ js-yaml@4.1.0\\
│  ├─ licenses: MIT\\
│  ├─ repository: https://github.com/nodeca/js-yaml\\
│  ├─ publisher: Vladimir Zapparov\\
│  ├─ email: dervus.grim@gmail.com\\
│  ├─ path: ./Frontend/node\_modules/cosmiconfig/node\_modules/js-yaml\\
│  └─ licenseFile: ./Frontend/node\_modules/cosmiconfig/node\_modules/js-yaml/LICENSE\\
├─ jsbn@1.1.0\\
│  ├─ licenses: MIT\\
│  ├─ repository: https://github.com/andyperlitch/jsbn\\
│  ├─ publisher: Tom Wu\\
│  ├─ path: ./Frontend/node\_modules/jsbn\\
│  └─ licenseFile: ./Frontend/node\_modules/jsbn/LICENSE\\
├─ jsesc@0.5.0\\
│  ├─ licenses: MIT\\
│  ├─ repository: https://github.com/mathiasbynens/jsesc\\
│  ├─ publisher: Mathias Bynens\\
│  ├─ url: http://mathiasbynens.be/\\
│  ├─ path: ./Frontend/node\_modules/regjsparser/node\_modules/jsesc\\
│  └─ licenseFile: ./Frontend/node\_modules/regjsparser/node\_modules/jsesc/LICENSE-MIT.txt\\
├─ jsesc@2.5.2\\
│  ├─ licenses: MIT\\
│  ├─ repository: https://github.com/mathiasbynens/jsesc\\
│  ├─ publisher: Mathias Bynens\\
│  ├─ url: https://mathiasbynens.be/\\
│  ├─ path: ./Frontend/node\_modules/jsesc\\
│  └─ licenseFile: ./Frontend/node\_modules/jsesc/LICENSE-MIT.txt\\
├─ json-parse-even-better-errors@2.3.1\\
│  ├─ licenses: MIT\\
│  ├─ repository: https://github.com/npm/json-parse-even-better-errors\\
│  ├─ publisher: Kat Marchán\\
│  ├─ email: kzm@zkat.tech\\
│  ├─ path: ./Frontend/node\_modules/parse-json/node\_modules/json-parse-even-better-errors\\
│  └─ licenseFile: ./Frontend/node\_modules/parse-json/node\_modules/json-parse-even-better-errors/LICENSE.md\\
├─ json-parse-even-better-errors@3.0.2\\
│  ├─ licenses: MIT\\
│  ├─ repository: https://github.com/npm/json-parse-even-better-errors\\
│  ├─ publisher: GitHub Inc.\\
│  ├─ path: ./Frontend/node\_modules/json-parse-even-better-errors\\
│  └─ licenseFile: ./Frontend/node\_modules/json-parse-even-better-errors/LICENSE.md\\
├─ json-schema-traverse@0.4.1\\
│  ├─ licenses: MIT\\
│  ├─ repository: https://github.com/epoberezkin/json-schema-traverse\\
│  ├─ publisher: Evgeny Poberezkin\\
│  ├─ path: ./Frontend/node\_modules/webpack/node\_modules/json-schema-traverse\\
│  └─ licenseFile: ./Frontend/node\_modules/webpack/node\_modules/json-schema-traverse/LICENSE\\
├─ json-schema-traverse@1.0.0\\
│  ├─ licenses: MIT\\
│  ├─ repository: https://github.com/epoberezkin/json-schema-traverse\\
│  ├─ publisher: Evgeny Poberezkin\\
│  ├─ path: ./Frontend/node\_modules/json-schema-traverse\\
│  └─ licenseFile: ./Frontend/node\_modules/json-schema-traverse/LICENSE\\
├─ json5@2.2.3\\
│  ├─ licenses: MIT\\
│  ├─ repository: https://github.com/json5/json5\\
│  ├─ publisher: Aseem Kishore\\
│  ├─ email: aseem.kishore@gmail.com\\
│  ├─ path: ./Frontend/node\_modules/json5\\
│  └─ licenseFile: ./Frontend/node\_modules/json5/LICENSE.md\\
├─ jsonc-parser@3.2.1\\
│  ├─ licenses: MIT\\
│  ├─ repository: https://github.com/microsoft/node-jsonc-parser\\
│  ├─ publisher: Microsoft Corporation\\
│  ├─ path: ./Frontend/node\_modules/jsonc-parser\\
│  └─ licenseFile: ./Frontend/node\_modules/jsonc-parser/LICENSE.md\\
├─ jsonfile@4.0.0\\
│  ├─ licenses: MIT\\
│  ├─ repository: https://github.com/jprichardson/node-jsonfile\\
│  ├─ publisher: JP Richardson\\
│  ├─ email: jprichardson@gmail.com\\
│  ├─ path: ./Frontend/node\_modules/jsonfile\\
│  └─ licenseFile: ./Frontend/node\_modules/jsonfile/LICENSE\\
├─ jsonfile@6.1.0\\
│  ├─ licenses: MIT\\
│  ├─ repository: https://github.com/jprichardson/node-jsonfile\\
│  ├─ publisher: JP Richardson\\
│  ├─ email: jprichardson@gmail.com\\
│  ├─ path: ./Frontend/node\_modules/@openapitools/openapi-generator-cli/node\_modules/jsonfile\\
│  └─ licenseFile: ./Frontend/node\_modules/@openapitools/openapi-generator-cli/node\_modules/jsonfile/LICENSE\\
├─ jsonparse@1.3.1\\
│  ├─ licenses: MIT\\
│  ├─ repository: https://github.com/creationix/jsonparse\\
│  ├─ publisher: Tim Caswell\\
│  ├─ email: tim@creationix.com\\
│  ├─ path: ./Frontend/node\_modules/jsonparse\\
│  └─ licenseFile: ./Frontend/node\_modules/jsonparse/LICENSE\\
├─ karma-chrome-launcher@3.2.0\\
│  ├─ licenses: MIT\\
│  ├─ repository: https://github.com/karma-runner/karma-chrome-launcher\\
│  ├─ publisher: Vojta Jina\\
│  ├─ email: vojta.jina@gmail.com\\
│  ├─ path: ./Frontend/node\_modules/karma-chrome-launcher\\
│  └─ licenseFile: ./Frontend/node\_modules/karma-chrome-launcher/LICENSE\\
├─ karma-coverage@2.2.1\\
│  ├─ licenses: MIT\\
│  ├─ repository: https://github.com/karma-runner/karma-coverage\\
│  ├─ publisher: SATO taichi\\
│  ├─ email: ryushi@gmail.com\\
│  ├─ path: ./Frontend/node\_modules/karma-coverage\\
│  └─ licenseFile: ./Frontend/node\_modules/karma-coverage/LICENSE\\
├─ karma-jasmine-html-reporter@2.1.0\\
│  ├─ licenses: MIT\\
│  ├─ repository: https://github.com/dfederm/karma-jasmine-html-reporter\\
│  ├─ publisher: David Federman\\
│  ├─ email: david.federman@outlook.com\\
│  ├─ url: https://github.com/dfederm\\
│  ├─ path: ./Frontend/node\_modules/karma-jasmine-html-reporter\\
│  └─ licenseFile: ./Frontend/node\_modules/karma-jasmine-html-reporter/LICENSE\\
├─ karma-jasmine@5.1.0\\
│  ├─ licenses: MIT\\
│  ├─ repository: https://github.com/karma-runner/karma-jasmine\\
│  ├─ publisher: Vojta Jina\\
│  ├─ email: vojta.jina@gmail.com\\
│  ├─ path: ./Frontend/node\_modules/karma-jasmine\\
│  └─ licenseFile: ./Frontend/node\_modules/karma-jasmine/LICENSE\\
├─ karma-source-map-support@1.4.0\\
│  ├─ licenses: MIT\\
│  ├─ repository: https://github.com/tschaub/karma-source-map-support\\
│  ├─ publisher: Tim Schaub\\
│  ├─ url: http://tschaub.net/\\
│  ├─ path: ./Frontend/node\_modules/karma-source-map-support\\
│  └─ licenseFile: ./Frontend/node\_modules/karma-source-map-support/LICENSE\\
├─ karma@6.4.3\\
│  ├─ licenses: MIT\\
│  ├─ repository: https://github.com/karma-runner/karma\\
│  ├─ publisher: Vojta Jína\\
│  ├─ email: vojta.jina@gmail.com\\
│  ├─ path: ./Frontend/node\_modules/karma\\
│  └─ licenseFile: ./Frontend/node\_modules/karma/LICENSE\\
├─ kind-of@6.0.3\\
│  ├─ licenses: MIT\\
│  ├─ repository: https://github.com/jonschlinkert/kind-of\\
│  ├─ publisher: Jon Schlinkert\\
│  ├─ url: https://github.com/jonschlinkert\\
│  ├─ path: ./Frontend/node\_modules/kind-of\\
│  └─ licenseFile: ./Frontend/node\_modules/kind-of/LICENSE\\
├─ klona@2.0.6\\
│  ├─ licenses: MIT\\
│  ├─ repository: https://github.com/lukeed/klona\\
│  ├─ publisher: Luke Edwards\\
│  ├─ email: luke.edwards05@gmail.com\\
│  ├─ url: https://lukeed.com\\
│  ├─ path: ./Frontend/node\_modules/klona\\
│  └─ licenseFile: ./Frontend/node\_modules/klona/license\\
├─ launch-editor@2.6.1\\
│  ├─ licenses: MIT\\
│  ├─ repository: https://github.com/yyx990803/launch-editor\\
│  ├─ publisher: Evan You\\
│  ├─ path: ./Frontend/node\_modules/launch-editor\\
│  └─ licenseFile: ./Frontend/node\_modules/launch-editor/LICENSE\\
├─ less-loader@11.1.0\\
│  ├─ licenses: MIT\\
│  ├─ repository: https://github.com/webpack-contrib/less-loader\\
│  ├─ publisher: Johannes Ewald @jhnns\\
│  ├─ path: ./Frontend/node\_modules/less-loader\\
│  └─ licenseFile: ./Frontend/node\_modules/less-loader/LICENSE\\
├─ less@4.2.0\\
│  ├─ licenses: Apache-2.0\\
│  ├─ repository: https://github.com/less/less.js\\
│  ├─ publisher: Alexis Sellier\\
│  ├─ email: self@cloudhead.net\\
│  ├─ path: ./Frontend/node\_modules/less\\
│  └─ licenseFile: ./Frontend/node\_modules/less/LICENSE\\
├─ license-webpack-plugin@4.0.2\\
│  ├─ licenses: ISC\\
│  ├─ repository: https://github.com/xz64/license-webpack-plugin\\
│  ├─ publisher: S K\\
│  ├─ url: xz64\\
│  ├─ path: ./Frontend/node\_modules/license-webpack-plugin\\
│  └─ licenseFile: ./Frontend/node\_modules/license-webpack-plugin/LICENSE\\
├─ lines-and-columns@1.2.4\\
│  ├─ licenses: MIT\\
│  ├─ repository: https://github.com/eventualbuddha/lines-and-columns\\
│  ├─ publisher: Brian Donovan\\
│  ├─ email: brian@donovans.cc\\
│  ├─ path: ./Frontend/node\_modules/lines-and-columns\\
│  └─ licenseFile: ./Frontend/node\_modules/lines-and-columns/LICENSE\\
├─ loader-runner@4.3.0\\
│  ├─ licenses: MIT\\
│  ├─ repository: https://github.com/webpack/loader-runner\\
│  ├─ publisher: Tobias Koppers @sokra\\
│  ├─ path: ./Frontend/node\_modules/loader-runner\\
│  └─ licenseFile: ./Frontend/node\_modules/loader-runner/LICENSE\\
├─ loader-utils@2.0.4\\
│  ├─ licenses: MIT\\
│  ├─ repository: https://github.com/webpack/loader-utils\\
│  ├─ publisher: Tobias Koppers @sokra\\
│  ├─ path: ./Frontend/node\_modules/adjust-sourcemap-loader/node\_modules/loader-utils\\
│  └─ licenseFile: ./Frontend/node\_modules/adjust-sourcemap-loader/node\_modules/loader-utils/LICENSE\\
├─ loader-utils@3.2.1\\
│  ├─ licenses: MIT\\
│  ├─ repository: https://github.com/webpack/loader-utils\\
│  ├─ publisher: Tobias Koppers @sokra\\
│  ├─ path: ./Frontend/node\_modules/loader-utils\\
│  └─ licenseFile: ./Frontend/node\_modules/loader-utils/LICENSE\\
├─ locate-path@5.0.0\\
│  ├─ licenses: MIT\\
│  ├─ repository: https://github.com/sindresorhus/locate-path\\
│  ├─ publisher: Sindre Sorhus\\
│  ├─ email: sindresorhus@gmail.com\\
│  ├─ url: sindresorhus.com\\
│  ├─ path: ./Frontend/node\_modules/locate-path\\
│  └─ licenseFile: ./Frontend/node\_modules/locate-path/license\\
├─ locate-path@7.2.0\\
│  ├─ licenses: MIT\\
│  ├─ repository: https://github.com/sindresorhus/locate-path\\
│  ├─ publisher: Sindre Sorhus\\
│  ├─ email: sindresorhus@gmail.com\\
│  ├─ url: https://sindresorhus.com\\
│  ├─ path: ./Frontend/node\_modules/pkg-dir/node\_modules/locate-path\\
│  └─ licenseFile: ./Frontend/node\_modules/pkg-dir/node\_modules/locate-path/license\\
├─ lodash.debounce@4.0.8\\
│  ├─ licenses: MIT\\
│  ├─ repository: https://github.com/lodash/lodash\\
│  ├─ publisher: John-David Dalton\\
│  ├─ email: john.david.dalton@gmail.com\\
│  ├─ url: http://allyoucanleet.com/\\
│  ├─ path: ./Frontend/node\_modules/lodash.debounce\\
│  └─ licenseFile: ./Frontend/node\_modules/lodash.debounce/LICENSE\\
├─ lodash@4.17.21\\
│  ├─ licenses: MIT\\
│  ├─ repository: https://github.com/lodash/lodash\\
│  ├─ publisher: John-David Dalton\\
│  ├─ email: john.david.dalton@gmail.com\\
│  ├─ path: ./Frontend/node\_modules/lodash\\
│  └─ licenseFile: ./Frontend/node\_modules/lodash/LICENSE\\
├─ log-symbols@4.1.0\\
│  ├─ licenses: MIT\\
│  ├─ repository: https://github.com/sindresorhus/log-symbols\\
│  ├─ publisher: Sindre Sorhus\\
│  ├─ email: sindresorhus@gmail.com\\
│  ├─ url: https://sindresorhus.com\\
│  ├─ path: ./Frontend/node\_modules/log-symbols\\
│  └─ licenseFile: ./Frontend/node\_modules/log-symbols/license\\
├─ log4js@6.9.1\\
│  ├─ licenses: Apache-2.0\\
│  ├─ repository: https://github.com/log4js-node/log4js-node\\
│  ├─ path: ./Frontend/node\_modules/log4js\\
│  └─ licenseFile: ./Frontend/node\_modules/log4js/LICENSE\\
├─ lru-cache@10.2.2\\
│  ├─ licenses: ISC\\
│  ├─ repository: https://github.com/isaacs/node-lru-cache\\
│  ├─ publisher: Isaac Z. Schlueter\\
│  ├─ email: i@izs.me\\
│  ├─ path: ./Frontend/node\_modules/@npmcli/agent/node\_modules/lru-cache\\
│  └─ licenseFile: ./Frontend/node\_modules/@npmcli/agent/node\_modules/lru-cache/LICENSE\\
├─ lru-cache@5.1.1\\
│  ├─ licenses: ISC\\
│  ├─ repository: https://github.com/isaacs/node-lru-cache\\
│  ├─ publisher: Isaac Z. Schlueter\\
│  ├─ email: i@izs.me\\
│  ├─ path: ./Frontend/node\_modules/lru-cache\\
│  └─ licenseFile: ./Frontend/node\_modules/lru-cache/LICENSE\\
├─ lru-cache@6.0.0\\
│  ├─ licenses: ISC\\
│  ├─ repository: https://github.com/isaacs/node-lru-cache\\
│  ├─ publisher: Isaac Z. Schlueter\\
│  ├─ email: i@izs.me\\
│  ├─ path: ./Frontend/node\_modules/semver/node\_modules/lru-cache\\
│  └─ licenseFile: ./Frontend/node\_modules/semver/node\_modules/lru-cache/LICENSE\\
├─ magic-string@0.30.8\\
│  ├─ licenses: MIT\\
│  ├─ repository: https://github.com/rich-harris/magic-string\\
│  ├─ publisher: Rich Harris\\
│  ├─ path: ./Frontend/node\_modules/magic-string\\
│  └─ licenseFile: ./Frontend/node\_modules/magic-string/LICENSE\\
├─ make-dir@2.1.0\\
│  ├─ licenses: MIT\\
│  ├─ repository: https://github.com/sindresorhus/make-dir\\
│  ├─ publisher: Sindre Sorhus\\
│  ├─ email: sindresorhus@gmail.com\\
│  ├─ url: sindresorhus.com\\
│  ├─ path: ./Frontend/node\_modules/less/node\_modules/make-dir\\
│  └─ licenseFile: ./Frontend/node\_modules/less/node\_modules/make-dir/license\\
├─ make-dir@4.0.0\\
│  ├─ licenses: MIT\\
│  ├─ repository: https://github.com/sindresorhus/make-dir\\
│  ├─ publisher: Sindre Sorhus\\
│  ├─ email: sindresorhus@gmail.com\\
│  ├─ url: https://sindresorhus.com\\
│  ├─ path: ./Frontend/node\_modules/make-dir\\
│  └─ licenseFile: ./Frontend/node\_modules/make-dir/license\\
├─ make-fetch-happen@13.0.1\\
│  ├─ licenses: ISC\\
│  ├─ repository: https://github.com/npm/make-fetch-happen\\
│  ├─ publisher: GitHub Inc.\\
│  ├─ path: ./Frontend/node\_modules/make-fetch-happen\\
│  └─ licenseFile: ./Frontend/node\_modules/make-fetch-happen/LICENSE\\
├─ media-typer@0.3.0\\
│  ├─ licenses: MIT\\
│  ├─ repository: https://github.com/jshttp/media-typer\\
│  ├─ publisher: Douglas Christopher Wilson\\
│  ├─ email: doug@somethingdoug.com\\
│  ├─ path: ./Frontend/node\_modules/media-typer\\
│  └─ licenseFile: ./Frontend/node\_modules/media-typer/LICENSE\\
├─ memfs@3.5.3\\
│  ├─ licenses: Unlicense\\
│  ├─ repository: https://github.com/streamich/memfs\\
│  ├─ path: ./Frontend/node\_modules/memfs\\
│  └─ licenseFile: ./Frontend/node\_modules/memfs/LICENSE\\
├─ merge-descriptors@1.0.1\\
│  ├─ licenses: MIT\\
│  ├─ repository: https://github.com/component/merge-descriptors\\
│  ├─ publisher: Jonathan Ong\\
│  ├─ email: me@jongleberry.com\\
│  ├─ url: http://jongleberry.com\\
│  ├─ path: ./Frontend/node\_modules/merge-descriptors\\
│  └─ licenseFile: ./Frontend/node\_modules/merge-descriptors/LICENSE\\
├─ merge-stream@2.0.0\\
│  ├─ licenses: MIT\\
│  ├─ repository: https://github.com/grncdr/merge-stream\\
│  ├─ publisher: Stephen Sugden\\
│  ├─ email: me@stephensugden.com\\
│  ├─ path: ./Frontend/node\_modules/merge-stream\\
│  └─ licenseFile: ./Frontend/node\_modules/merge-stream/LICENSE\\
├─ merge2@1.4.1\\
│  ├─ licenses: MIT\\
│  ├─ repository: https://github.com/teambition/merge2\\
│  ├─ path: ./Frontend/node\_modules/merge2\\
│  └─ licenseFile: ./Frontend/node\_modules/merge2/LICENSE\\
├─ methods@1.1.2\\
│  ├─ licenses: MIT\\
│  ├─ repository: https://github.com/jshttp/methods\\
│  ├─ path: ./Frontend/node\_modules/methods\\
│  └─ licenseFile: ./Frontend/node\_modules/methods/LICENSE\\
├─ micromatch@4.0.5\\
│  ├─ licenses: MIT\\
│  ├─ repository: https://github.com/micromatch/micromatch\\
│  ├─ publisher: Jon Schlinkert\\
│  ├─ url: https://github.com/jonschlinkert\\
│  ├─ path: ./Frontend/node\_modules/micromatch\\
│  └─ licenseFile: ./Frontend/node\_modules/micromatch/LICENSE\\
├─ mime-db@1.52.0\\
│  ├─ licenses: MIT\\
│  ├─ repository: https://github.com/jshttp/mime-db\\
│  ├─ path: ./Frontend/node\_modules/mime-db\\
│  └─ licenseFile: ./Frontend/node\_modules/mime-db/LICENSE\\
├─ mime-types@2.1.35\\
│  ├─ licenses: MIT\\
│  ├─ repository: https://github.com/jshttp/mime-types\\
│  ├─ path: ./Frontend/node\_modules/mime-types\\
│  └─ licenseFile: ./Frontend/node\_modules/mime-types/LICENSE\\
├─ mime@1.6.0\\
│  ├─ licenses: MIT\\
│  ├─ repository: https://github.com/broofa/node-mime\\
│  ├─ publisher: Robert Kieffer\\
│  ├─ email: robert@broofa.com\\
│  ├─ url: http://github.com/broofa\\
│  ├─ path: ./Frontend/node\_modules/less/node\_modules/mime\\
│  └─ licenseFile: ./Frontend/node\_modules/less/node\_modules/mime/LICENSE\\
├─ mime@2.6.0\\
│  ├─ licenses: MIT\\
│  ├─ repository: https://github.com/broofa/mime\\
│  ├─ publisher: Robert Kieffer\\
│  ├─ email: robert@broofa.com\\
│  ├─ url: http://github.com/broofa\\
│  ├─ path: ./Frontend/node\_modules/mime\\
│  └─ licenseFile: ./Frontend/node\_modules/mime/LICENSE\\
├─ mimic-fn@2.1.0\\
│  ├─ licenses: MIT\\
│  ├─ repository: https://github.com/sindresorhus/mimic-fn\\
│  ├─ publisher: Sindre Sorhus\\
│  ├─ email: sindresorhus@gmail.com\\
│  ├─ url: sindresorhus.com\\
│  ├─ path: ./Frontend/node\_modules/mimic-fn\\
│  └─ licenseFile: ./Frontend/node\_modules/mimic-fn/license\\
├─ mini-css-extract-plugin@2.8.1\\
│  ├─ licenses: MIT\\
│  ├─ repository: https://github.com/webpack-contrib/mini-css-extract-plugin\\
│  ├─ publisher: Tobias Koppers @sokra\\
│  ├─ path: ./Frontend/node\_modules/mini-css-extract-plugin\\
│  └─ licenseFile: ./Frontend/node\_modules/mini-css-extract-plugin/LICENSE\\
├─ minimalistic-assert@1.0.1\\
│  ├─ licenses: ISC\\
│  ├─ repository: https://github.com/calvinmetcalf/minimalistic-assert\\
│  ├─ path: ./Frontend/node\_modules/minimalistic-assert\\
│  └─ licenseFile: ./Frontend/node\_modules/minimalistic-assert/LICENSE\\
├─ minimatch@3.1.2\\
│  ├─ licenses: ISC\\
│  ├─ repository: https://github.com/isaacs/minimatch\\
│  ├─ publisher: Isaac Z. Schlueter\\
│  ├─ email: i@izs.me\\
│  ├─ url: http://blog.izs.me\\
│  ├─ path: ./Frontend/node\_modules/minimatch\\
│  └─ licenseFile: ./Frontend/node\_modules/minimatch/LICENSE\\
├─ minimatch@9.0.4\\
│  ├─ licenses: ISC\\
│  ├─ repository: https://github.com/isaacs/minimatch\\
│  ├─ publisher: Isaac Z. Schlueter\\
│  ├─ email: i@izs.me\\
│  ├─ url: http://blog.izs.me\\
│  ├─ path: ./Frontend/node\_modules/@tufjs/models/node\_modules/minimatch\\
│  └─ licenseFile: ./Frontend/node\_modules/@tufjs/models/node\_modules/minimatch/LICENSE\\
├─ minimist@1.2.8\\
│  ├─ licenses: MIT\\
│  ├─ repository: https://github.com/minimistjs/minimist\\
│  ├─ publisher: James Halliday\\
│  ├─ email: mail@substack.net\\
│  ├─ url: http://substack.net\\
│  ├─ path: ./Frontend/node\_modules/minimist\\
│  └─ licenseFile: ./Frontend/node\_modules/minimist/LICENSE\\
├─ minipass-collect@2.0.1\\
│  ├─ licenses: ISC\\
│  ├─ repository: https://github.com/isaacs/minipass-collect\\
│  ├─ publisher: Isaac Z. Schlueter\\
│  ├─ email: i@izs.me\\
│  ├─ url: https://izs.me\\
│  ├─ path: ./Frontend/node\_modules/minipass-collect\\
│  └─ licenseFile: ./Frontend/node\_modules/minipass-collect/LICENSE\\
├─ minipass-fetch@3.0.5\\
│  ├─ licenses: MIT\\
│  ├─ repository: https://github.com/npm/minipass-fetch\\
│  ├─ publisher: GitHub Inc.\\
│  ├─ path: ./Frontend/node\_modules/minipass-fetch\\
│  └─ licenseFile: ./Frontend/node\_modules/minipass-fetch/LICENSE\\
├─ minipass-flush@1.0.5\\
│  ├─ licenses: ISC\\
│  ├─ repository: https://github.com/isaacs/minipass-flush\\
│  ├─ publisher: Isaac Z. Schlueter\\
│  ├─ email: i@izs.me\\
│  ├─ url: https://izs.me\\
│  ├─ path: ./Frontend/node\_modules/minipass-flush\\
│  └─ licenseFile: ./Frontend/node\_modules/minipass-flush/LICENSE\\
├─ minipass-json-stream@1.0.1\\
│  ├─ licenses: MIT\\
│  ├─ repository: https://github.com/npm/minipass-json-stream\\
│  ├─ publisher: Isaac Z. Schlueter\\
│  ├─ email: i@izs.me\\
│  ├─ url: https://izs.me\\
│  ├─ path: ./Frontend/node\_modules/minipass-json-stream\\
│  └─ licenseFile: ./Frontend/node\_modules/minipass-json-stream/LICENSE\\
├─ minipass-pipeline@1.2.4\\
│  ├─ licenses: ISC\\
│  ├─ publisher: Isaac Z. Schlueter\\
│  ├─ email: i@izs.me\\
│  ├─ url: https://izs.me\\
│  ├─ path: ./Frontend/node\_modules/minipass-pipeline\\
│  └─ licenseFile: ./Frontend/node\_modules/minipass-pipeline/LICENSE\\
├─ minipass-sized@1.0.3\\
│  ├─ licenses: ISC\\
│  ├─ repository: https://github.com/isaacs/minipass-sized\\
│  ├─ publisher: Isaac Z. Schlueter\\
│  ├─ email: i@izs.me\\
│  ├─ url: https://izs.me\\
│  ├─ path: ./Frontend/node\_modules/minipass-sized\\
│  └─ licenseFile: ./Frontend/node\_modules/minipass-sized/LICENSE\\
├─ minipass@3.3.6\\
│  ├─ licenses: ISC\\
│  ├─ repository: https://github.com/isaacs/minipass\\
│  ├─ publisher: Isaac Z. Schlueter\\
│  ├─ email: i@izs.me\\
│  ├─ url: http://blog.izs.me/\\
│  ├─ path: ./Frontend/node\_modules/minipass-flush/node\_modules/minipass\\
│  └─ licenseFile: ./Frontend/node\_modules/minipass-flush/node\_modules/minipass/LICENSE\\
├─ minipass@5.0.0\\
│  ├─ licenses: ISC\\
│  ├─ repository: https://github.com/isaacs/minipass\\
│  ├─ publisher: Isaac Z. Schlueter\\
│  ├─ email: i@izs.me\\
│  ├─ url: http://blog.izs.me/\\
│  ├─ path: ./Frontend/node\_modules/tar/node\_modules/minipass\\
│  └─ licenseFile: ./Frontend/node\_modules/tar/node\_modules/minipass/LICENSE\\
├─ minipass@7.1.1\\
│  ├─ licenses: ISC\\
│  ├─ repository: https://github.com/isaacs/minipass\\
│  ├─ publisher: Isaac Z. Schlueter\\
│  ├─ email: i@izs.me\\
│  ├─ url: http://blog.izs.me/\\
│  ├─ path: ./Frontend/node\_modules/minipass\\
│  └─ licenseFile: ./Frontend/node\_modules/minipass/LICENSE\\
├─ minizlib@2.1.2\\
│  ├─ licenses: MIT\\
│  ├─ repository: https://github.com/isaacs/minizlib\\
│  ├─ publisher: Isaac Z. Schlueter\\
│  ├─ email: i@izs.me\\
│  ├─ url: http://blog.izs.me/\\
│  ├─ path: ./Frontend/node\_modules/minizlib\\
│  └─ licenseFile: ./Frontend/node\_modules/minizlib/LICENSE\\
├─ mkdirp@0.5.6\\
│  ├─ licenses: MIT\\
│  ├─ repository: https://github.com/substack/node-mkdirp\\
│  ├─ publisher: James Halliday\\
│  ├─ email: mail@substack.net\\
│  ├─ url: http://substack.net\\
│  ├─ path: ./Frontend/node\_modules/mkdirp\\
│  └─ licenseFile: ./Frontend/node\_modules/mkdirp/LICENSE\\
├─ mkdirp@1.0.4\\
│  ├─ licenses: MIT\\
│  ├─ repository: https://github.com/isaacs/node-mkdirp\\
│  ├─ path: ./Frontend/node\_modules/tar/node\_modules/mkdirp\\
│  └─ licenseFile: ./Frontend/node\_modules/tar/node\_modules/mkdirp/LICENSE\\
├─ mrmime@2.0.0\\
│  ├─ licenses: MIT\\
│  ├─ repository: https://github.com/lukeed/mrmime\\
│  ├─ publisher: Luke Edwards\\
│  ├─ email: luke.edwards05@gmail.com\\
│  ├─ url: https://lukeed.com\\
│  ├─ path: ./Frontend/node\_modules/mrmime\\
│  └─ licenseFile: ./Frontend/node\_modules/mrmime/license\\
├─ ms@2.0.0\\
│  ├─ licenses: MIT\\
│  ├─ repository: https://github.com/zeit/ms\\
│  ├─ path: ./Frontend/node\_modules/compression/node\_modules/ms\\
│  └─ licenseFile: ./Frontend/node\_modules/compression/node\_modules/ms/license.md\\
├─ ms@2.1.2\\
│  ├─ licenses: MIT\\
│  ├─ repository: https://github.com/zeit/ms\\
│  ├─ path: ./Frontend/node\_modules/ms\\
│  └─ licenseFile: ./Frontend/node\_modules/ms/license.md\\
├─ ms@2.1.3\\
│  ├─ licenses: MIT\\
│  ├─ repository: https://github.com/vercel/ms\\
│  ├─ path: ./Frontend/node\_modules/send/node\_modules/ms\\
│  └─ licenseFile: ./Frontend/node\_modules/send/node\_modules/ms/license.md\\
├─ multicast-dns@7.2.5\\
│  ├─ licenses: MIT\\
│  ├─ repository: https://github.com/mafintosh/multicast-dns\\
│  ├─ publisher: Mathias Buus\\
│  ├─ url: @mafintosh\\
│  ├─ path: ./Frontend/node\_modules/multicast-dns\\
│  └─ licenseFile: ./Frontend/node\_modules/multicast-dns/LICENSE\\
├─ mute-stream@0.0.8\\
│  ├─ licenses: ISC\\
│  ├─ repository: https://github.com/isaacs/mute-stream\\
│  ├─ publisher: Isaac Z. Schlueter\\
│  ├─ email: i@izs.me\\
│  ├─ url: http://blog.izs.me/\\
│  ├─ path: ./Frontend/node\_modules/@openapitools/openapi-generator-cli/node\_modules/mute-stream\\
│  └─ licenseFile: ./Frontend/node\_modules/@openapitools/openapi-generator-cli/node\_modules/mute-stream/LICENSE\\
├─ mute-stream@1.0.0\\
│  ├─ licenses: ISC\\
│  ├─ repository: https://github.com/npm/mute-stream\\
│  ├─ publisher: GitHub Inc.\\
│  ├─ path: ./Frontend/node\_modules/mute-stream\\
│  └─ licenseFile: ./Frontend/node\_modules/mute-stream/LICENSE\\
├─ nanoid@3.3.7\\
│  ├─ licenses: MIT\\
│  ├─ repository: https://github.com/ai/nanoid\\
│  ├─ publisher: Andrey Sitnik\\
│  ├─ email: andrey@sitnik.ru\\
│  ├─ path: ./Frontend/node\_modules/nanoid\\
│  └─ licenseFile: ./Frontend/node\_modules/nanoid/LICENSE\\
├─ needle@3.3.1\\
│  ├─ licenses: MIT\\
│  ├─ repository: https://github.com/tomas/needle\\
│  ├─ publisher: Tomás Pollak\\
│  ├─ email: tomas@forkhq.com\\
│  ├─ path: ./Frontend/node\_modules/needle\\
│  └─ licenseFile: ./Frontend/node\_modules/needle/license.txt\\
├─ negotiator@0.6.3\\
│  ├─ licenses: MIT\\
│  ├─ repository: https://github.com/jshttp/negotiator\\
│  ├─ path: ./Frontend/node\_modules/negotiator\\
│  └─ licenseFile: ./Frontend/node\_modules/negotiator/LICENSE\\
├─ neo-async@2.6.2\\
│  ├─ licenses: MIT\\
│  ├─ repository: https://github.com/suguru03/neo-async\\
│  ├─ path: ./Frontend/node\_modules/neo-async\\
│  └─ licenseFile: ./Frontend/node\_modules/neo-async/LICENSE\\
├─ ngx-bootstrap@12.0.0\\
│  ├─ licenses: MIT\\
│  ├─ repository: https://github.com/valor-software/ngx-bootstrap\\
│  ├─ publisher: Dmitriy Shekhovtsov\\
│  ├─ email: valorkin@gmail.com\\
│  ├─ path: ./Frontend/node\_modules/ngx-bootstrap\\
│  └─ licenseFile: ./Frontend/node\_modules/ngx-bootstrap/LICENSE\\
├─ nice-napi@1.0.2\\
│  ├─ licenses: MIT\\
│  ├─ repository: https://github.com/addaleax/nice-napi\\
│  ├─ publisher: Anna Henningsen\\
│  ├─ email: anna@addaleax.net\\
│  ├─ path: ./Frontend/node\_modules/nice-napi\\
│  └─ licenseFile: ./Frontend/node\_modules/nice-napi/LICENSE\\
├─ node-addon-api@3.2.1\\
│  ├─ licenses: MIT\\
│  ├─ repository: https://github.com/nodejs/node-addon-api\\
│  ├─ path: ./Frontend/node\_modules/node-addon-api\\
│  └─ licenseFile: ./Frontend/node\_modules/node-addon-api/LICENSE.md\\
├─ node-fetch@2.7.0\\
│  ├─ licenses: MIT\\
│  ├─ repository: https://github.com/bitinn/node-fetch\\
│  ├─ publisher: David Frank\\
│  ├─ path: ./Frontend/node\_modules/node-fetch\\
│  └─ licenseFile: ./Frontend/node\_modules/node-fetch/LICENSE.md\\
├─ node-forge@1.3.1\\
│  ├─ licenses: (BSD-3-Clause OR GPL-2.0)\\
│  ├─ repository: https://github.com/digitalbazaar/forge\\
│  ├─ publisher: Digital Bazaar, Inc.\\
│  ├─ email: support@digitalbazaar.com\\
│  ├─ url: http://digitalbazaar.com/\\
│  ├─ path: ./Frontend/node\_modules/node-forge\\
│  └─ licenseFile: ./Frontend/node\_modules/node-forge/LICENSE\\
├─ node-gyp-build@4.8.1\\
│  ├─ licenses: MIT\\
│  ├─ repository: https://github.com/prebuild/node-gyp-build\\
│  ├─ publisher: Mathias Buus\\
│  ├─ url: @mafintosh\\
│  ├─ path: ./Frontend/node\_modules/node-gyp-build\\
│  └─ licenseFile: ./Frontend/node\_modules/node-gyp-build/LICENSE\\
├─ node-gyp@10.1.0\\
│  ├─ licenses: MIT\\
│  ├─ repository: https://github.com/nodejs/node-gyp\\
│  ├─ publisher: Nathan Rajlich\\
│  ├─ email: nathan@tootallnate.net\\
│  ├─ url: http://tootallnate.net\\
│  ├─ path: ./Frontend/node\_modules/node-gyp\\
│  └─ licenseFile: ./Frontend/node\_modules/node-gyp/LICENSE\\
├─ node-releases@2.0.14\\
│  ├─ licenses: MIT\\
│  ├─ repository: https://github.com/chicoxyzzy/node-releases\\
│  ├─ publisher: Sergey Rubanov\\
│  ├─ email: chi187@gmail.com\\
│  ├─ path: ./Frontend/node\_modules/node-releases\\
│  └─ licenseFile: ./Frontend/node\_modules/node-releases/LICENSE\\
├─ nopt@7.2.1\\
│  ├─ licenses: ISC\\
│  ├─ repository: https://github.com/npm/nopt\\
│  ├─ publisher: GitHub Inc.\\
│  ├─ path: ./Frontend/node\_modules/nopt\\
│  └─ licenseFile: ./Frontend/node\_modules/nopt/LICENSE\\
├─ normalize-package-data@6.0.1\\
│  ├─ licenses: BSD-2-Clause\\
│  ├─ repository: https://github.com/npm/normalize-package-data\\
│  ├─ publisher: GitHub Inc.\\
│  ├─ path: ./Frontend/node\_modules/normalize-package-data\\
│  └─ licenseFile: ./Frontend/node\_modules/normalize-package-data/LICENSE\\
├─ normalize-path@3.0.0\\
│  ├─ licenses: MIT\\
│  ├─ repository: https://github.com/jonschlinkert/normalize-path\\
│  ├─ publisher: Jon Schlinkert\\
│  ├─ url: https://github.com/jonschlinkert\\
│  ├─ path: ./Frontend/node\_modules/normalize-path\\
│  └─ licenseFile: ./Frontend/node\_modules/normalize-path/LICENSE\\
├─ normalize-range@0.1.2\\
│  ├─ licenses: MIT\\
│  ├─ repository: https://github.com/jamestalmage/normalize-range\\
│  ├─ publisher: James Talmage\\
│  ├─ email: james@talmage.io\\
│  ├─ url: github.com/jamestalmage\\
│  ├─ path: ./Frontend/node\_modules/normalize-range\\
│  └─ licenseFile: ./Frontend/node\_modules/normalize-range/license\\
├─ npm-bundled@3.0.1\\
│  ├─ licenses: ISC\\
│  ├─ repository: https://github.com/npm/npm-bundled\\
│  ├─ publisher: GitHub Inc.\\
│  ├─ path: ./Frontend/node\_modules/npm-bundled\\
│  └─ licenseFile: ./Frontend/node\_modules/npm-bundled/LICENSE\\
├─ npm-install-checks@6.3.0\\
│  ├─ licenses: BSD-2-Clause\\
│  ├─ repository: https://github.com/npm/npm-install-checks\\
│  ├─ publisher: GitHub Inc.\\
│  ├─ path: ./Frontend/node\_modules/npm-install-checks\\
│  └─ licenseFile: ./Frontend/node\_modules/npm-install-checks/LICENSE\\
├─ npm-normalize-package-bin@3.0.1\\
│  ├─ licenses: ISC\\
│  ├─ repository: https://github.com/npm/npm-normalize-package-bin\\
│  ├─ publisher: GitHub Inc.\\
│  ├─ path: ./Frontend/node\_modules/npm-normalize-package-bin\\
│  └─ licenseFile: ./Frontend/node\_modules/npm-normalize-package-bin/LICENSE\\
├─ npm-package-arg@11.0.1\\
│  ├─ licenses: ISC\\
│  ├─ repository: https://github.com/npm/npm-package-arg\\
│  ├─ publisher: GitHub Inc.\\
│  ├─ path: ./Frontend/node\_modules/npm-package-arg\\
│  └─ licenseFile: ./Frontend/node\_modules/npm-package-arg/LICENSE\\
├─ npm-packlist@8.0.2\\
│  ├─ licenses: ISC\\
│  ├─ repository: https://github.com/npm/npm-packlist\\
│  ├─ publisher: GitHub Inc.\\
│  ├─ path: ./Frontend/node\_modules/npm-packlist\\
│  └─ licenseFile: ./Frontend/node\_modules/npm-packlist/LICENSE\\
├─ npm-pick-manifest@9.0.0\\
│  ├─ licenses: ISC\\
│  ├─ repository: https://github.com/npm/npm-pick-manifest\\
│  ├─ publisher: GitHub Inc.\\
│  ├─ path: ./Frontend/node\_modules/npm-pick-manifest\\
│  └─ licenseFile: ./Frontend/node\_modules/npm-pick-manifest/LICENSE.md\\
├─ npm-registry-fetch@16.2.1\\
│  ├─ licenses: ISC\\
│  ├─ repository: https://github.com/npm/npm-registry-fetch\\
│  ├─ publisher: GitHub Inc.\\
│  ├─ path: ./Frontend/node\_modules/npm-registry-fetch\\
│  └─ licenseFile: ./Frontend/node\_modules/npm-registry-fetch/LICENSE.md\\
├─ npm-run-path@4.0.1\\
│  ├─ licenses: MIT\\
│  ├─ repository: https://github.com/sindresorhus/npm-run-path\\
│  ├─ publisher: Sindre Sorhus\\
│  ├─ email: sindresorhus@gmail.com\\
│  ├─ url: sindresorhus.com\\
│  ├─ path: ./Frontend/node\_modules/npm-run-path\\
│  └─ licenseFile: ./Frontend/node\_modules/npm-run-path/license\\
├─ nth-check@2.1.1\\
│  ├─ licenses: BSD-2-Clause\\
│  ├─ repository: https://github.com/fb55/nth-check\\
│  ├─ publisher: Felix Boehm\\
│  ├─ email: me@feedic.com\\
│  ├─ path: ./Frontend/node\_modules/nth-check\\
│  └─ licenseFile: ./Frontend/node\_modules/nth-check/LICENSE\\
├─ object-assign@4.1.1\\
│  ├─ licenses: MIT\\
│  ├─ repository: https://github.com/sindresorhus/object-assign\\
│  ├─ publisher: Sindre Sorhus\\
│  ├─ email: sindresorhus@gmail.com\\
│  ├─ url: sindresorhus.com\\
│  ├─ path: ./Frontend/node\_modules/object-assign\\
│  └─ licenseFile: ./Frontend/node\_modules/object-assign/license\\
├─ object-inspect@1.13.1\\
│  ├─ licenses: MIT\\
│  ├─ repository: https://github.com/inspect-js/object-inspect\\
│  ├─ publisher: James Halliday\\
│  ├─ email: mail@substack.net\\
│  ├─ url: http://substack.net\\
│  ├─ path: ./Frontend/node\_modules/object-inspect\\
│  └─ licenseFile: ./Frontend/node\_modules/object-inspect/LICENSE\\
├─ obuf@1.1.2\\
│  ├─ licenses: MIT\\
│  ├─ repository: https://github.com/indutny/offset-buffer\\
│  ├─ publisher: Fedor Indutny\\
│  ├─ email: fedor@indutny.com\\
│  ├─ path: ./Frontend/node\_modules/obuf\\
│  └─ licenseFile: ./Frontend/node\_modules/obuf/LICENSE\\
├─ on-finished@2.3.0\\
│  ├─ licenses: MIT\\
│  ├─ repository: https://github.com/jshttp/on-finished\\
│  ├─ path: ./Frontend/node\_modules/finalhandler/node\_modules/on-finished\\
│  └─ licenseFile: ./Frontend/node\_modules/finalhandler/node\_modules/on-finished/LICENSE\\
├─ on-finished@2.4.1\\
│  ├─ licenses: MIT\\
│  ├─ repository: https://github.com/jshttp/on-finished\\
│  ├─ path: ./Frontend/node\_modules/on-finished\\
│  └─ licenseFile: ./Frontend/node\_modules/on-finished/LICENSE\\
├─ on-headers@1.0.2\\
│  ├─ licenses: MIT\\
│  ├─ repository: https://github.com/jshttp/on-headers\\
│  ├─ publisher: Douglas Christopher Wilson\\
│  ├─ email: doug@somethingdoug.com\\
│  ├─ path: ./Frontend/node\_modules/on-headers\\
│  └─ licenseFile: ./Frontend/node\_modules/on-headers/LICENSE\\
├─ once@1.4.0\\
│  ├─ licenses: ISC\\
│  ├─ repository: https://github.com/isaacs/once\\
│  ├─ publisher: Isaac Z. Schlueter\\
│  ├─ email: i@izs.me\\
│  ├─ url: http://blog.izs.me/\\
│  ├─ path: ./Frontend/node\_modules/once\\
│  └─ licenseFile: ./Frontend/node\_modules/once/LICENSE\\
├─ onetime@5.1.2\\
│  ├─ licenses: MIT\\
│  ├─ repository: https://github.com/sindresorhus/onetime\\
│  ├─ publisher: Sindre Sorhus\\
│  ├─ email: sindresorhus@gmail.com\\
│  ├─ url: https://sindresorhus.com\\
│  ├─ path: ./Frontend/node\_modules/onetime\\
│  └─ licenseFile: ./Frontend/node\_modules/onetime/license\\
├─ open@8.4.2\\
│  ├─ licenses: MIT\\
│  ├─ repository: https://github.com/sindresorhus/open\\
│  ├─ publisher: Sindre Sorhus\\
│  ├─ email: sindresorhus@gmail.com\\
│  ├─ url: https://sindresorhus.com\\
│  ├─ path: ./Frontend/node\_modules/open\\
│  └─ licenseFile: ./Frontend/node\_modules/open/license\\
├─ ora@5.4.1\\
│  ├─ licenses: MIT\\
│  ├─ repository: https://github.com/sindresorhus/ora\\
│  ├─ publisher: Sindre Sorhus\\
│  ├─ email: sindresorhus@gmail.com\\
│  ├─ url: https://sindresorhus.com\\
│  ├─ path: ./Frontend/node\_modules/ora\\
│  └─ licenseFile: ./Frontend/node\_modules/ora/license\\
├─ os-tmpdir@1.0.2\\
│  ├─ licenses: MIT\\
│  ├─ repository: https://github.com/sindresorhus/os-tmpdir\\
│  ├─ publisher: Sindre Sorhus\\
│  ├─ email: sindresorhus@gmail.com\\
│  ├─ url: sindresorhus.com\\
│  ├─ path: ./Frontend/node\_modules/os-tmpdir\\
│  └─ licenseFile: ./Frontend/node\_modules/os-tmpdir/license\\
├─ p-limit@2.3.0\\
│  ├─ licenses: MIT\\
│  ├─ repository: https://github.com/sindresorhus/p-limit\\
│  ├─ publisher: Sindre Sorhus\\
│  ├─ email: sindresorhus@gmail.com\\
│  ├─ url: sindresorhus.com\\
│  ├─ path: ./Frontend/node\_modules/p-limit\\
│  └─ licenseFile: ./Frontend/node\_modules/p-limit/license\\
├─ p-limit@4.0.0\\
│  ├─ licenses: MIT\\
│  ├─ repository: https://github.com/sindresorhus/p-limit\\
│  ├─ publisher: Sindre Sorhus\\
│  ├─ email: sindresorhus@gmail.com\\
│  ├─ url: https://sindresorhus.com\\
│  ├─ path: ./Frontend/node\_modules/pkg-dir/node\_modules/p-limit\\
│  └─ licenseFile: ./Frontend/node\_modules/pkg-dir/node\_modules/p-limit/license\\
├─ p-locate@4.1.0\\
│  ├─ licenses: MIT\\
│  ├─ repository: https://github.com/sindresorhus/p-locate\\
│  ├─ publisher: Sindre Sorhus\\
│  ├─ email: sindresorhus@gmail.com\\
│  ├─ url: sindresorhus.com\\
│  ├─ path: ./Frontend/node\_modules/p-locate\\
│  └─ licenseFile: ./Frontend/node\_modules/p-locate/license\\
├─ p-locate@6.0.0\\
│  ├─ licenses: MIT\\
│  ├─ repository: https://github.com/sindresorhus/p-locate\\
│  ├─ publisher: Sindre Sorhus\\
│  ├─ email: sindresorhus@gmail.com\\
│  ├─ url: https://sindresorhus.com\\
│  ├─ path: ./Frontend/node\_modules/pkg-dir/node\_modules/p-locate\\
│  └─ licenseFile: ./Frontend/node\_modules/pkg-dir/node\_modules/p-locate/license\\
├─ p-map@4.0.0\\
│  ├─ licenses: MIT\\
│  ├─ repository: https://github.com/sindresorhus/p-map\\
│  ├─ publisher: Sindre Sorhus\\
│  ├─ email: sindresorhus@gmail.com\\
│  ├─ url: https://sindresorhus.com\\
│  ├─ path: ./Frontend/node\_modules/p-map\\
│  └─ licenseFile: ./Frontend/node\_modules/p-map/license\\
├─ p-retry@4.6.2\\
│  ├─ licenses: MIT\\
│  ├─ repository: https://github.com/sindresorhus/p-retry\\
│  ├─ publisher: Sindre Sorhus\\
│  ├─ email: sindresorhus@gmail.com\\
│  ├─ url: sindresorhus.com\\
│  ├─ path: ./Frontend/node\_modules/p-retry\\
│  └─ licenseFile: ./Frontend/node\_modules/p-retry/license\\
├─ p-try@2.2.0\\
│  ├─ licenses: MIT\\
│  ├─ repository: https://github.com/sindresorhus/p-try\\
│  ├─ publisher: Sindre Sorhus\\
│  ├─ email: sindresorhus@gmail.com\\
│  ├─ url: sindresorhus.com\\
│  ├─ path: ./Frontend/node\_modules/p-try\\
│  └─ licenseFile: ./Frontend/node\_modules/p-try/license\\
├─ pacote@17.0.6\\
│  ├─ licenses: ISC\\
│  ├─ repository: https://github.com/npm/pacote\\
│  ├─ publisher: GitHub Inc.\\
│  ├─ path: ./Frontend/node\_modules/pacote\\
│  └─ licenseFile: ./Frontend/node\_modules/pacote/LICENSE\\
├─ parent-module@1.0.1\\
│  ├─ licenses: MIT\\
│  ├─ repository: https://github.com/sindresorhus/parent-module\\
│  ├─ publisher: Sindre Sorhus\\
│  ├─ email: sindresorhus@gmail.com\\
│  ├─ url: sindresorhus.com\\
│  ├─ path: ./Frontend/node\_modules/parent-module\\
│  └─ licenseFile: ./Frontend/node\_modules/parent-module/license\\
├─ parse-json@5.2.0\\
│  ├─ licenses: MIT\\
│  ├─ repository: https://github.com/sindresorhus/parse-json\\
│  ├─ publisher: Sindre Sorhus\\
│  ├─ email: sindresorhus@gmail.com\\
│  ├─ url: https://sindresorhus.com\\
│  ├─ path: ./Frontend/node\_modules/parse-json\\
│  └─ licenseFile: ./Frontend/node\_modules/parse-json/license\\
├─ parse-node-version@1.0.1\\
│  ├─ licenses: MIT\\
│  ├─ repository: https://github.com/gulpjs/parse-node-version\\
│  ├─ publisher: Gulp Team\\
│  ├─ email: team@gulpjs.com\\
│  ├─ url: http://gulpjs.com/\\
│  ├─ path: ./Frontend/node\_modules/parse-node-version\\
│  └─ licenseFile: ./Frontend/node\_modules/parse-node-version/LICENSE\\
├─ parse5-html-rewriting-stream@7.0.0\\
│  ├─ licenses: MIT\\
│  ├─ repository: https://github.com/inikulin/parse5\\
│  ├─ publisher: Ivan Nikulin\\
│  ├─ email: ifaaan@gmail.com\\
│  ├─ url: https://github.com/inikulin\\
│  ├─ path: ./Frontend/node\_modules/parse5-html-rewriting-stream\\
│  └─ licenseFile: ./Frontend/node\_modules/parse5-html-rewriting-stream/LICENSE\\
├─ parse5-sax-parser@7.0.0\\
│  ├─ licenses: MIT\\
│  ├─ repository: https://github.com/inikulin/parse5\\
│  ├─ publisher: Ivan Nikulin\\
│  ├─ email: ifaaan@gmail.com\\
│  ├─ url: https://github.com/inikulin\\
│  ├─ path: ./Frontend/node\_modules/parse5-sax-parser\\
│  └─ licenseFile: ./Frontend/node\_modules/parse5-sax-parser/LICENSE\\
├─ parse5@7.1.2\\
│  ├─ licenses: MIT\\
│  ├─ repository: https://github.com/inikulin/parse5\\
│  ├─ publisher: Ivan Nikulin\\
│  ├─ email: ifaaan@gmail.com\\
│  ├─ url: https://github.com/inikulin\\
│  ├─ path: ./Frontend/node\_modules/parse5\\
│  └─ licenseFile: ./Frontend/node\_modules/parse5/LICENSE\\
├─ parseurl@1.3.3\\
│  ├─ licenses: MIT\\
│  ├─ repository: https://github.com/pillarjs/parseurl\\
│  ├─ path: ./Frontend/node\_modules/parseurl\\
│  └─ licenseFile: ./Frontend/node\_modules/parseurl/LICENSE\\
├─ path-exists@4.0.0\\
│  ├─ licenses: MIT\\
│  ├─ repository: https://github.com/sindresorhus/path-exists\\
│  ├─ publisher: Sindre Sorhus\\
│  ├─ email: sindresorhus@gmail.com\\
│  ├─ url: sindresorhus.com\\
│  ├─ path: ./Frontend/node\_modules/path-exists\\
│  └─ licenseFile: ./Frontend/node\_modules/path-exists/license\\
├─ path-exists@5.0.0\\
│  ├─ licenses: MIT\\
│  ├─ repository: https://github.com/sindresorhus/path-exists\\
│  ├─ publisher: Sindre Sorhus\\
│  ├─ email: sindresorhus@gmail.com\\
│  ├─ url: https://sindresorhus.com\\
│  ├─ path: ./Frontend/node\_modules/pkg-dir/node\_modules/path-exists\\
│  └─ licenseFile: ./Frontend/node\_modules/pkg-dir/node\_modules/path-exists/license\\
├─ path-is-absolute@1.0.1\\
│  ├─ licenses: MIT\\
│  ├─ repository: https://github.com/sindresorhus/path-is-absolute\\
│  ├─ publisher: Sindre Sorhus\\
│  ├─ email: sindresorhus@gmail.com\\
│  ├─ url: sindresorhus.com\\
│  ├─ path: ./Frontend/node\_modules/path-is-absolute\\
│  └─ licenseFile: ./Frontend/node\_modules/path-is-absolute/license\\
├─ path-key@3.1.1\\
│  ├─ licenses: MIT\\
│  ├─ repository: https://github.com/sindresorhus/path-key\\
│  ├─ publisher: Sindre Sorhus\\
│  ├─ email: sindresorhus@gmail.com\\
│  ├─ url: sindresorhus.com\\
│  ├─ path: ./Frontend/node\_modules/path-key\\
│  └─ licenseFile: ./Frontend/node\_modules/path-key/license\\
├─ path-parse@1.0.7\\
│  ├─ licenses: MIT\\
│  ├─ repository: https://github.com/jbgutierrez/path-parse\\
│  ├─ publisher: Javier Blanco\\
│  ├─ email: http://jbgutierrez.info\\
│  ├─ path: ./Frontend/node\_modules/path-parse\\
│  └─ licenseFile: ./Frontend/node\_modules/path-parse/LICENSE\\
├─ path-scurry@1.11.1\\
│  ├─ licenses: BlueOak-1.0.0\\
│  ├─ repository: https://github.com/isaacs/path-scurry\\
│  ├─ publisher: Isaac Z. Schlueter\\
│  ├─ email: i@izs.me\\
│  ├─ url: https://blog.izs.me\\
│  ├─ path: ./Frontend/node\_modules/path-scurry\\
│  └─ licenseFile: ./Frontend/node\_modules/path-scurry/LICENSE.md\\
├─ path-to-regexp@0.1.7\\
│  ├─ licenses: MIT\\
│  ├─ repository: https://github.com/component/path-to-regexp\\
│  ├─ path: ./Frontend/node\_modules/path-to-regexp\\
│  └─ licenseFile: ./Frontend/node\_modules/path-to-regexp/LICENSE\\
├─ path-to-regexp@3.2.0\\
│  ├─ licenses: MIT\\
│  ├─ repository: https://github.com/pillarjs/path-to-regexp\\
│  ├─ path: ./Frontend/node\_modules/@openapitools/openapi-generator-cli/node\_modules/path-to-regexp\\
│  └─ licenseFile: ./Frontend/node\_modules/@openapitools/openapi-generator-cli/node\_modules/path-to-regexp/LICENSE\\
├─ path-type@4.0.0\\
│  ├─ licenses: MIT\\
│  ├─ repository: https://github.com/sindresorhus/path-type\\
│  ├─ publisher: Sindre Sorhus\\
│  ├─ email: sindresorhus@gmail.com\\
│  ├─ url: sindresorhus.com\\
│  ├─ path: ./Frontend/node\_modules/path-type\\
│  └─ licenseFile: ./Frontend/node\_modules/path-type/license\\
├─ picocolors@1.0.1\\
│  ├─ licenses: ISC\\
│  ├─ repository: https://github.com/alexeyraspopov/picocolors\\
│  ├─ publisher: Alexey Raspopov\\
│  ├─ path: ./Frontend/node\_modules/picocolors\\
│  └─ licenseFile: ./Frontend/node\_modules/picocolors/LICENSE\\
├─ picomatch@2.3.1\\
│  ├─ licenses: MIT\\
│  ├─ repository: https://github.com/micromatch/picomatch\\
│  ├─ publisher: Jon Schlinkert\\
│  ├─ url: https://github.com/jonschlinkert\\
│  ├─ path: ./Frontend/node\_modules/micromatch/node\_modules/picomatch\\
│  └─ licenseFile: ./Frontend/node\_modules/micromatch/node\_modules/picomatch/LICENSE\\
├─ picomatch@4.0.1\\
│  ├─ licenses: MIT\\
│  ├─ repository: https://github.com/micromatch/picomatch\\
│  ├─ publisher: Jon Schlinkert\\
│  ├─ url: https://github.com/jonschlinkert\\
│  ├─ path: ./Frontend/node\_modules/picomatch\\
│  └─ licenseFile: ./Frontend/node\_modules/picomatch/LICENSE\\
├─ pify@4.0.1\\
│  ├─ licenses: MIT\\
│  ├─ repository: https://github.com/sindresorhus/pify\\
│  ├─ publisher: Sindre Sorhus\\
│  ├─ email: sindresorhus@gmail.com\\
│  ├─ url: sindresorhus.com\\
│  ├─ path: ./Frontend/node\_modules/pify\\
│  └─ licenseFile: ./Frontend/node\_modules/pify/license\\
├─ piscina@4.4.0\\
│  ├─ licenses: MIT\\
│  ├─ repository: https://github.com/piscinajs/piscina\\
│  ├─ publisher: James M Snell\\
│  ├─ email: jasnell@gmail.com\\
│  ├─ path: ./Frontend/node\_modules/piscina\\
│  └─ licenseFile: ./Frontend/node\_modules/piscina/LICENSE\\
├─ pkg-dir@7.0.0\\
│  ├─ licenses: MIT\\
│  ├─ repository: https://github.com/sindresorhus/pkg-dir\\
│  ├─ publisher: Sindre Sorhus\\
│  ├─ email: sindresorhus@gmail.com\\
│  ├─ url: https://sindresorhus.com\\
│  ├─ path: ./Frontend/node\_modules/pkg-dir\\
│  └─ licenseFile: ./Frontend/node\_modules/pkg-dir/license\\
├─ postcss-loader@8.1.1\\
│  ├─ licenses: MIT\\
│  ├─ repository: https://github.com/webpack-contrib/postcss-loader\\
│  ├─ publisher: Andrey Sitnik\\
│  ├─ email: andrey@sitnik.ru\\
│  ├─ path: ./Frontend/node\_modules/postcss-loader\\
│  └─ licenseFile: ./Frontend/node\_modules/postcss-loader/LICENSE\\
├─ postcss-media-query-parser@0.2.3\\
│  ├─ licenses: MIT\\
│  ├─ repository: https://github.com/dryoma/postcss-media-query-parser\\
│  ├─ publisher: dryoma\\
│  ├─ path: ./Frontend/node\_modules/postcss-media-query-parser\\
│  └─ licenseFile: ./Frontend/node\_modules/postcss-media-query-parser/README.md\\
├─ postcss-modules-extract-imports@3.1.0\\
│  ├─ licenses: ISC\\
│  ├─ repository: https://github.com/css-modules/postcss-modules-extract-imports\\
│  ├─ publisher: Glen Maddern\\
│  ├─ path: ./Frontend/node\_modules/postcss-modules-extract-imports\\
│  └─ licenseFile: ./Frontend/node\_modules/postcss-modules-extract-imports/LICENSE\\
├─ postcss-modules-local-by-default@4.0.5\\
│  ├─ licenses: MIT\\
│  ├─ repository: https://github.com/css-modules/postcss-modules-local-by-default\\
│  ├─ publisher: Mark Dalgleish\\
│  ├─ path: ./Frontend/node\_modules/postcss-modules-local-by-default\\
│  └─ licenseFile: ./Frontend/node\_modules/postcss-modules-local-by-default/LICENSE\\
├─ postcss-modules-scope@3.2.0\\
│  ├─ licenses: ISC\\
│  ├─ repository: https://github.com/css-modules/postcss-modules-scope\\
│  ├─ publisher: Glen Maddern\\
│  ├─ path: ./Frontend/node\_modules/postcss-modules-scope\\
│  └─ licenseFile: ./Frontend/node\_modules/postcss-modules-scope/LICENSE\\
├─ postcss-modules-values@4.0.0\\
│  ├─ licenses: ISC\\
│  ├─ repository: https://github.com/css-modules/postcss-modules-values\\
│  ├─ publisher: Glen Maddern\\
│  ├─ path: ./Frontend/node\_modules/postcss-modules-values\\
│  └─ licenseFile: ./Frontend/node\_modules/postcss-modules-values/LICENSE\\
├─ postcss-selector-parser@6.0.16\\
│  ├─ licenses: MIT\\
│  ├─ repository: https://github.com/postcss/postcss-selector-parser\\
│  ├─ path: ./Frontend/node\_modules/postcss-selector-parser\\
│  └─ licenseFile: ./Frontend/node\_modules/postcss-selector-parser/LICENSE-MIT\\
├─ postcss-value-parser@4.2.0\\
│  ├─ licenses: MIT\\
│  ├─ repository: https://github.com/TrySound/postcss-value-parser\\
│  ├─ publisher: Bogdan Chadkin\\
│  ├─ email: trysound@yandex.ru\\
│  ├─ path: ./Frontend/node\_modules/postcss-value-parser\\
│  └─ licenseFile: ./Frontend/node\_modules/postcss-value-parser/LICENSE\\
├─ postcss@8.4.35\\
│  ├─ licenses: MIT\\
│  ├─ repository: https://github.com/postcss/postcss\\
│  ├─ publisher: Andrey Sitnik\\
│  ├─ email: andrey@sitnik.ru\\
│  ├─ path: ./Frontend/node\_modules/postcss\\
│  └─ licenseFile: ./Frontend/node\_modules/postcss/LICENSE\\
├─ proc-log@3.0.0\\
│  ├─ licenses: ISC\\
│  ├─ repository: https://github.com/npm/proc-log\\
│  ├─ publisher: GitHub Inc.\\
│  ├─ path: ./Frontend/node\_modules/proc-log\\
│  └─ licenseFile: ./Frontend/node\_modules/proc-log/LICENSE\\
├─ proc-log@4.2.0\\
│  ├─ licenses: ISC\\
│  ├─ repository: https://github.com/npm/proc-log\\
│  ├─ publisher: GitHub Inc.\\
│  ├─ path: ./Frontend/node\_modules/make-fetch-happen/node\_modules/proc-log\\
│  └─ licenseFile: ./Frontend/node\_modules/make-fetch-happen/node\_modules/proc-log/LICENSE\\
├─ process-nextick-args@2.0.1\\
│  ├─ licenses: MIT\\
│  ├─ repository: https://github.com/calvinmetcalf/process-nextick-args\\
│  ├─ path: ./Frontend/node\_modules/process-nextick-args\\
│  └─ licenseFile: ./Frontend/node\_modules/process-nextick-args/license.md\\
├─ promise-inflight@1.0.1\\
│  ├─ licenses: ISC\\
│  ├─ repository: https://github.com/iarna/promise-inflight\\
│  ├─ publisher: Rebecca Turner\\
│  ├─ email: me@re-becca.org\\
│  ├─ url: http://re-becca.org/\\
│  ├─ path: ./Frontend/node\_modules/promise-inflight\\
│  └─ licenseFile: ./Frontend/node\_modules/promise-inflight/LICENSE\\
├─ promise-retry@2.0.1\\
│  ├─ licenses: MIT\\
│  ├─ repository: https://github.com/IndigoUnited/node-promise-retry\\
│  ├─ publisher: IndigoUnited\\
│  ├─ email: hello@indigounited.com\\
│  ├─ url: http://indigounited.com\\
│  ├─ path: ./Frontend/node\_modules/promise-retry\\
│  └─ licenseFile: ./Frontend/node\_modules/promise-retry/LICENSE\\
├─ proxy-addr@2.0.7\\
│  ├─ licenses: MIT\\
│  ├─ repository: https://github.com/jshttp/proxy-addr\\
│  ├─ publisher: Douglas Christopher Wilson\\
│  ├─ email: doug@somethingdoug.com\\
│  ├─ path: ./Frontend/node\_modules/proxy-addr\\
│  └─ licenseFile: ./Frontend/node\_modules/proxy-addr/LICENSE\\
├─ proxy-from-env@1.1.0\\
│  ├─ licenses: MIT\\
│  ├─ repository: https://github.com/Rob--W/proxy-from-env\\
│  ├─ publisher: Rob Wu\\
│  ├─ email: rob@robwu.nl\\
│  ├─ url: https://robwu.nl/\\
│  ├─ path: ./Frontend/node\_modules/proxy-from-env\\
│  └─ licenseFile: ./Frontend/node\_modules/proxy-from-env/LICENSE\\
├─ prr@1.0.1\\
│  ├─ licenses: MIT\\
│  ├─ repository: https://github.com/rvagg/prr\\
│  ├─ publisher: Rod Vagg\\
│  ├─ email: rod@vagg.org\\
│  ├─ url: https://github.com/rvagg\\
│  ├─ path: ./Frontend/node\_modules/prr\\
│  └─ licenseFile: ./Frontend/node\_modules/prr/LICENSE.md\\
├─ punycode@2.3.1\\
│  ├─ licenses: MIT\\
│  ├─ repository: https://github.com/mathiasbynens/punycode.js\\
│  ├─ publisher: Mathias Bynens\\
│  ├─ url: https://mathiasbynens.be/\\
│  ├─ path: ./Frontend/node\_modules/punycode\\
│  └─ licenseFile: ./Frontend/node\_modules/punycode/LICENSE-MIT.txt\\
├─ qjobs@1.2.0\\
│  ├─ licenses: MIT\\
│  ├─ repository: https://github.com/franck34/qjobs\\
│  ├─ publisher: Franck TABARY\\
│  ├─ path: ./Frontend/node\_modules/qjobs\\
│  └─ licenseFile: ./Frontend/node\_modules/qjobs/LICENCE\\
├─ qs@6.11.0\\
│  ├─ licenses: BSD-3-Clause\\
│  ├─ repository: https://github.com/ljharb/qs\\
│  ├─ path: ./Frontend/node\_modules/qs\\
│  └─ licenseFile: ./Frontend/node\_modules/qs/LICENSE.md\\
├─ queue-microtask@1.2.3\\
│  ├─ licenses: MIT\\
│  ├─ repository: https://github.com/feross/queue-microtask\\
│  ├─ publisher: Feross Aboukhadijeh\\
│  ├─ email: feross@feross.org\\
│  ├─ url: https://feross.org\\
│  ├─ path: ./Frontend/node\_modules/queue-microtask\\
│  └─ licenseFile: ./Frontend/node\_modules/queue-microtask/LICENSE\\
├─ randombytes@2.1.0\\
│  ├─ licenses: MIT\\
│  ├─ repository: https://github.com/crypto-browserify/randombytes\\
│  ├─ path: ./Frontend/node\_modules/randombytes\\
│  └─ licenseFile: ./Frontend/node\_modules/randombytes/LICENSE\\
├─ range-parser@1.2.1\\
│  ├─ licenses: MIT\\
│  ├─ repository: https://github.com/jshttp/range-parser\\
│  ├─ publisher: TJ Holowaychuk\\
│  ├─ email: tj@vision-media.ca\\
│  ├─ url: http://tjholowaychuk.com\\
│  ├─ path: ./Frontend/node\_modules/range-parser\\
│  └─ licenseFile: ./Frontend/node\_modules/range-parser/LICENSE\\
├─ raw-body@2.5.2\\
│  ├─ licenses: MIT\\
│  ├─ repository: https://github.com/stream-utils/raw-body\\
│  ├─ publisher: Jonathan Ong\\
│  ├─ email: me@jongleberry.com\\
│  ├─ url: http://jongleberry.com\\
│  ├─ path: ./Frontend/node\_modules/raw-body\\
│  └─ licenseFile: ./Frontend/node\_modules/raw-body/LICENSE\\
├─ read-package-json-fast@3.0.2\\
│  ├─ licenses: ISC\\
│  ├─ repository: https://github.com/npm/read-package-json-fast\\
│  ├─ publisher: GitHub Inc.\\
│  ├─ path: ./Frontend/node\_modules/read-package-json-fast\\
│  └─ licenseFile: ./Frontend/node\_modules/read-package-json-fast/LICENSE\\
├─ read-package-json@7.0.1\\
│  ├─ licenses: ISC\\
│  ├─ repository: https://github.com/npm/read-package-json\\
│  ├─ publisher: GitHub Inc.\\
│  ├─ path: ./Frontend/node\_modules/read-package-json\\
│  └─ licenseFile: ./Frontend/node\_modules/read-package-json/LICENSE\\
├─ readable-stream@2.3.8\\
│  ├─ licenses: MIT\\
│  ├─ repository: https://github.com/nodejs/readable-stream\\
│  ├─ path: ./Frontend/node\_modules/hpack.js/node\_modules/readable-stream\\
│  └─ licenseFile: ./Frontend/node\_modules/hpack.js/node\_modules/readable-stream/LICENSE\\
├─ readable-stream@3.6.2\\
│  ├─ licenses: MIT\\
│  ├─ repository: https://github.com/nodejs/readable-stream\\
│  ├─ path: ./Frontend/node\_modules/readable-stream\\
│  └─ licenseFile: ./Frontend/node\_modules/readable-stream/LICENSE\\
├─ readdirp@3.6.0\\
│  ├─ licenses: MIT\\
│  ├─ repository: https://github.com/paulmillr/readdirp\\
│  ├─ publisher: Thorsten Lorenz\\
│  ├─ email: thlorenz@gmx.de\\
│  ├─ url: thlorenz.com\\
│  ├─ path: ./Frontend/node\_modules/readdirp\\
│  └─ licenseFile: ./Frontend/node\_modules/readdirp/LICENSE\\
├─ reflect-metadata@0.1.13\\
│  ├─ licenses: Apache-2.0\\
│  ├─ repository: https://github.com/rbuckton/reflect-metadata\\
│  ├─ publisher: Ron Buckton\\
│  ├─ email: ron.buckton@microsoft.com\\
│  ├─ url: http://github.com/rbuckton\\
│  ├─ path: ./Frontend/node\_modules/@openapitools/openapi-generator-cli/node\_modules/reflect-metadata\\
│  └─ licenseFile: ./Frontend/node\_modules/@openapitools/openapi-generator-cli/node\_modules/reflect-metadata/LICENSE\\
├─ reflect-metadata@0.2.2\\
│  ├─ licenses: Apache-2.0\\
│  ├─ repository: https://github.com/rbuckton/reflect-metadata\\
│  ├─ publisher: Ron Buckton\\
│  ├─ email: ron.buckton@microsoft.com\\
│  ├─ url: http://github.com/rbuckton\\
│  ├─ path: ./Frontend/node\_modules/reflect-metadata\\
│  └─ licenseFile: ./Frontend/node\_modules/reflect-metadata/LICENSE\\
├─ regenerate-unicode-properties@10.1.1\\
│  ├─ licenses: MIT\\
│  ├─ repository: https://github.com/mathiasbynens/regenerate-unicode-properties\\
│  ├─ publisher: Mathias Bynens\\
│  ├─ url: https://mathiasbynens.be/\\
│  ├─ path: ./Frontend/node\_modules/regenerate-unicode-properties\\
│  └─ licenseFile: ./Frontend/node\_modules/regenerate-unicode-properties/LICENSE-MIT.txt\\
├─ regenerate@1.4.2\\
│  ├─ licenses: MIT\\
│  ├─ repository: https://github.com/mathiasbynens/regenerate\\
│  ├─ publisher: Mathias Bynens\\
│  ├─ url: https://mathiasbynens.be/\\
│  ├─ path: ./Frontend/node\_modules/regenerate\\
│  └─ licenseFile: ./Frontend/node\_modules/regenerate/LICENSE-MIT.txt\\
├─ regenerator-runtime@0.14.1\\
│  ├─ licenses: MIT\\
│  ├─ repository: https://github.com/facebook/regenerator/tree/main/packages/runtime\\
│  ├─ publisher: Ben Newman\\
│  ├─ email: bn@cs.stanford.edu\\
│  ├─ path: ./Frontend/node\_modules/regenerator-runtime\\
│  └─ licenseFile: ./Frontend/node\_modules/regenerator-runtime/LICENSE\\
├─ regenerator-transform@0.15.2\\
│  ├─ licenses: MIT\\
│  ├─ repository: https://github.com/facebook/regenerator/tree/main/packages/transform\\
│  ├─ publisher: Ben Newman\\
│  ├─ email: bn@cs.stanford.edu\\
│  ├─ path: ./Frontend/node\_modules/regenerator-transform\\
│  └─ licenseFile: ./Frontend/node\_modules/regenerator-transform/LICENSE\\
├─ regex-parser@2.3.0\\
│  ├─ licenses: MIT\\
│  ├─ repository: https://github.com/IonicaBizau/regex-parser.js\\
│  ├─ publisher: Ionică Bizău\\
│  ├─ email: bizauionica@gmail.com\\
│  ├─ url: https://ionicabizau.net\\
│  ├─ path: ./Frontend/node\_modules/regex-parser\\
│  └─ licenseFile: ./Frontend/node\_modules/regex-parser/LICENSE\\
├─ regexpu-core@5.3.2\\
│  ├─ licenses: MIT\\
│  ├─ repository: https://github.com/mathiasbynens/regexpu-core\\
│  ├─ publisher: Mathias Bynens\\
│  ├─ url: https://mathiasbynens.be/\\
│  ├─ path: ./Frontend/node\_modules/regexpu-core\\
│  └─ licenseFile: ./Frontend/node\_modules/regexpu-core/LICENSE-MIT.txt\\
├─ regjsparser@0.9.1\\
│  ├─ licenses: BSD-2-Clause\\
│  ├─ repository: https://github.com/jviereck/regjsparser\\
│  ├─ publisher: 'Julian Viereck'\\
│  ├─ email: julian.viereck@gmail.com\\
│  ├─ path: ./Frontend/node\_modules/regjsparser\\
│  └─ licenseFile: ./Frontend/node\_modules/regjsparser/LICENSE.BSD\\
├─ require-directory@2.1.1\\
│  ├─ licenses: MIT\\
│  ├─ repository: https://github.com/troygoode/node-require-directory\\
│  ├─ publisher: Troy Goode\\
│  ├─ email: troygoode@gmail.com\\
│  ├─ url: http://github.com/troygoode/\\
│  ├─ path: ./Frontend/node\_modules/require-directory\\
│  └─ licenseFile: ./Frontend/node\_modules/require-directory/LICENSE\\
├─ require-from-string@2.0.2\\
│  ├─ licenses: MIT\\
│  ├─ repository: https://github.com/floatdrop/require-from-string\\
│  ├─ publisher: Vsevolod Strukchinsky\\
│  ├─ email: floatdrop@gmail.com\\
│  ├─ url: github.com/floatdrop\\
│  ├─ path: ./Frontend/node\_modules/require-from-string\\
│  └─ licenseFile: ./Frontend/node\_modules/require-from-string/license\\
├─ requires-port@1.0.0\\
│  ├─ licenses: MIT\\
│  ├─ repository: https://github.com/unshiftio/requires-port\\
│  ├─ publisher: Arnout Kazemier\\
│  ├─ path: ./Frontend/node\_modules/requires-port\\
│  └─ licenseFile: ./Frontend/node\_modules/requires-port/LICENSE\\
├─ resolve-from@4.0.0\\
│  ├─ licenses: MIT\\
│  ├─ repository: https://github.com/sindresorhus/resolve-from\\
│  ├─ publisher: Sindre Sorhus\\
│  ├─ email: sindresorhus@gmail.com\\
│  ├─ url: sindresorhus.com\\
│  ├─ path: ./Frontend/node\_modules/import-fresh/node\_modules/resolve-from\\
│  └─ licenseFile: ./Frontend/node\_modules/import-fresh/node\_modules/resolve-from/license\\
├─ resolve-from@5.0.0\\
│  ├─ licenses: MIT\\
│  ├─ repository: https://github.com/sindresorhus/resolve-from\\
│  ├─ publisher: Sindre Sorhus\\
│  ├─ email: sindresorhus@gmail.com\\
│  ├─ url: sindresorhus.com\\
│  ├─ path: ./Frontend/node\_modules/resolve-from\\
│  └─ licenseFile: ./Frontend/node\_modules/resolve-from/license\\
├─ resolve-url-loader@5.0.0\\
│  ├─ licenses: MIT\\
│  ├─ repository: https://github.com/bholloway/resolve-url-loader\\
│  ├─ publisher: bholloway\\
│  ├─ path: ./Frontend/node\_modules/resolve-url-loader\\
│  └─ licenseFile: ./Frontend/node\_modules/resolve-url-loader/LICENSE\\
├─ resolve@1.22.8\\
│  ├─ licenses: MIT\\
│  ├─ repository: https://github.com/browserify/resolve\\
│  ├─ publisher: James Halliday\\
│  ├─ email: mail@substack.net\\
│  ├─ url: http://substack.net\\
│  ├─ path: ./Frontend/node\_modules/resolve\\
│  └─ licenseFile: ./Frontend/node\_modules/resolve/LICENSE\\
├─ restore-cursor@3.1.0\\
│  ├─ licenses: MIT\\
│  ├─ repository: https://github.com/sindresorhus/restore-cursor\\
│  ├─ publisher: Sindre Sorhus\\
│  ├─ email: sindresorhus@gmail.com\\
│  ├─ url: sindresorhus.com\\
│  ├─ path: ./Frontend/node\_modules/restore-cursor\\
│  └─ licenseFile: ./Frontend/node\_modules/restore-cursor/license\\
├─ retry@0.12.0\\
│  ├─ licenses: MIT\\
│  ├─ repository: https://github.com/tim-kos/node-retry\\
│  ├─ publisher: Tim Koschützki\\
│  ├─ email: tim@debuggable.com\\
│  ├─ url: http://debuggable.com/\\
│  ├─ path: ./Frontend/node\_modules/retry\\
│  └─ licenseFile: ./Frontend/node\_modules/retry/License\\
├─ retry@0.13.1\\
│  ├─ licenses: MIT\\
│  ├─ repository: https://github.com/tim-kos/node-retry\\
│  ├─ publisher: Tim Koschützki\\
│  ├─ email: tim@debuggable.com\\
│  ├─ url: http://debuggable.com/\\
│  ├─ path: ./Frontend/node\_modules/p-retry/node\_modules/retry\\
│  └─ licenseFile: ./Frontend/node\_modules/p-retry/node\_modules/retry/License\\
├─ reusify@1.0.4\\
│  ├─ licenses: MIT\\
│  ├─ repository: https://github.com/mcollina/reusify\\
│  ├─ publisher: Matteo Collina\\
│  ├─ email: hello@matteocollina.com\\
│  ├─ path: ./Frontend/node\_modules/reusify\\
│  └─ licenseFile: ./Frontend/node\_modules/reusify/LICENSE\\
├─ rfdc@1.3.1\\
│  ├─ licenses: MIT\\
│  ├─ repository: https://github.com/davidmarkclements/rfdc\\
│  ├─ publisher: David Mark Clements\\
│  ├─ email: david.clements@nearform.com\\
│  ├─ path: ./Frontend/node\_modules/rfdc\\
│  └─ licenseFile: ./Frontend/node\_modules/rfdc/LICENSE\\
├─ rimraf@3.0.2\\
│  ├─ licenses: ISC\\
│  ├─ repository: https://github.com/isaacs/rimraf\\
│  ├─ publisher: Isaac Z. Schlueter\\
│  ├─ email: i@izs.me\\
│  ├─ url: http://blog.izs.me/\\
│  ├─ path: ./Frontend/node\_modules/rimraf\\
│  └─ licenseFile: ./Frontend/node\_modules/rimraf/LICENSE\\
├─ rollup@4.17.2\\
│  ├─ licenses: MIT\\
│  ├─ repository: https://github.com/rollup/rollup\\
│  ├─ publisher: Rich Harris\\
│  ├─ path: ./Frontend/node\_modules/rollup\\
│  └─ licenseFile: ./Frontend/node\_modules/rollup/LICENSE.md\\
├─ run-async@2.4.1\\
│  ├─ licenses: MIT\\
│  ├─ repository: https://github.com/SBoudrias/run-async\\
│  ├─ publisher: Simon Boudrias\\
│  ├─ email: admin@simonboudrias.com\\
│  ├─ path: ./Frontend/node\_modules/@openapitools/openapi-generator-cli/node\_modules/run-async\\
│  └─ licenseFile: ./Frontend/node\_modules/@openapitools/openapi-generator-cli/node\_modules/run-async/LICENSE\\
├─ run-async@3.0.0\\
│  ├─ licenses: MIT\\
│  ├─ repository: https://github.com/SBoudrias/run-async\\
│  ├─ publisher: Simon Boudrias\\
│  ├─ email: admin@simonboudrias.com\\
│  ├─ path: ./Frontend/node\_modules/run-async\\
│  └─ licenseFile: ./Frontend/node\_modules/run-async/LICENSE\\
├─ run-parallel@1.2.0\\
│  ├─ licenses: MIT\\
│  ├─ repository: https://github.com/feross/run-parallel\\
│  ├─ publisher: Feross Aboukhadijeh\\
│  ├─ email: feross@feross.org\\
│  ├─ url: https://feross.org\\
│  ├─ path: ./Frontend/node\_modules/run-parallel\\
│  └─ licenseFile: ./Frontend/node\_modules/run-parallel/LICENSE\\
├─ rxjs@6.6.7\\
│  ├─ licenses: Apache-2.0\\
│  ├─ repository: https://github.com/reactivex/rxjs\\
│  ├─ publisher: Ben Lesh\\
│  ├─ email: ben@benlesh.com\\
│  ├─ path: ./Frontend/node\_modules/concurrently/node\_modules/rxjs\\
│  └─ licenseFile: ./Frontend/node\_modules/concurrently/node\_modules/rxjs/LICENSE.txt\\
├─ rxjs@7.8.1\\
│  ├─ licenses: Apache-2.0\\
│  ├─ repository: https://github.com/reactivex/rxjs\\
│  ├─ publisher: Ben Lesh\\
│  ├─ email: ben@benlesh.com\\
│  ├─ path: ./Frontend/node\_modules/rxjs\\
│  └─ licenseFile: ./Frontend/node\_modules/rxjs/LICENSE.txt\\
├─ safe-buffer@5.1.2\\
│  ├─ licenses: MIT\\
│  ├─ repository: https://github.com/feross/safe-buffer\\
│  ├─ publisher: Feross Aboukhadijeh\\
│  ├─ email: feross@feross.org\\
│  ├─ url: http://feross.org\\
│  ├─ path: ./Frontend/node\_modules/compression/node\_modules/safe-buffer\\
│  └─ licenseFile: ./Frontend/node\_modules/compression/node\_modules/safe-buffer/LICENSE\\
├─ safe-buffer@5.2.1\\
│  ├─ licenses: MIT\\
│  ├─ repository: https://github.com/feross/safe-buffer\\
│  ├─ publisher: Feross Aboukhadijeh\\
│  ├─ email: feross@feross.org\\
│  ├─ url: https://feross.org\\
│  ├─ path: ./Frontend/node\_modules/safe-buffer\\
│  └─ licenseFile: ./Frontend/node\_modules/safe-buffer/LICENSE\\
├─ safer-buffer@2.1.2\\
│  ├─ licenses: MIT\\
│  ├─ repository: https://github.com/ChALkeR/safer-buffer\\
│  ├─ publisher: Nikita Skovoroda\\
│  ├─ email: chalkerx@gmail.com\\
│  ├─ url: https://github.com/ChALkeR\\
│  ├─ path: ./Frontend/node\_modules/safer-buffer\\
│  └─ licenseFile: ./Frontend/node\_modules/safer-buffer/LICENSE\\
├─ sass-loader@14.1.1\\
│  ├─ licenses: MIT\\
│  ├─ repository: https://github.com/webpack-contrib/sass-loader\\
│  ├─ publisher: J. Tangelder\\
│  ├─ path: ./Frontend/node\_modules/sass-loader\\
│  └─ licenseFile: ./Frontend/node\_modules/sass-loader/LICENSE\\
├─ sass@1.71.1\\
│  ├─ licenses: MIT\\
│  ├─ repository: https://github.com/sass/dart-sass\\
│  ├─ publisher: Natalie Weizenbaum\\
│  ├─ email: nweiz@google.com\\
│  ├─ url: https://github.com/nex3\\
│  ├─ path: ./Frontend/node\_modules/sass\\
│  └─ licenseFile: ./Frontend/node\_modules/sass/LICENSE\\
├─ sax@1.3.0\\
│  ├─ licenses: ISC\\
│  ├─ repository: https://github.com/isaacs/sax-js\\
│  ├─ publisher: Isaac Z. Schlueter\\
│  ├─ email: i@izs.me\\
│  ├─ url: http://blog.izs.me/\\
│  ├─ path: ./Frontend/node\_modules/sax\\
│  └─ licenseFile: ./Frontend/node\_modules/sax/LICENSE\\
├─ schema-utils@3.3.0\\
│  ├─ licenses: MIT\\
│  ├─ repository: https://github.com/webpack/schema-utils\\
│  ├─ publisher: webpack Contrib\\
│  ├─ url: https://github.com/webpack-contrib\\
│  ├─ path: ./Frontend/node\_modules/webpack/node\_modules/schema-utils\\
│  └─ licenseFile: ./Frontend/node\_modules/webpack/node\_modules/schema-utils/LICENSE\\
├─ schema-utils@4.2.0\\
│  ├─ licenses: MIT\\
│  ├─ repository: https://github.com/webpack/schema-utils\\
│  ├─ publisher: webpack Contrib\\
│  ├─ url: https://github.com/webpack-contrib\\
│  ├─ path: ./Frontend/node\_modules/schema-utils\\
│  └─ licenseFile: ./Frontend/node\_modules/schema-utils/LICENSE\\
├─ select-hose@2.0.0\\
│  ├─ licenses: MIT\\
│  ├─ repository: https://github.com/indutny/select-hose\\
│  ├─ publisher: Fedor Indutny\\
│  ├─ email: fedor@indutny.com\\
│  ├─ path: ./Frontend/node\_modules/select-hose\\
│  └─ licenseFile: ./Frontend/node\_modules/select-hose/README.md\\
├─ selfsigned@2.4.1\\
│  ├─ licenses: MIT\\
│  ├─ repository: https://github.com/jfromaniello/selfsigned\\
│  ├─ publisher: José F. Romaniello\\
│  ├─ email: jfromaniello@gmail.com\\
│  ├─ url: http://joseoncode.com\\
│  ├─ path: ./Frontend/node\_modules/selfsigned\\
│  └─ licenseFile: ./Frontend/node\_modules/selfsigned/LICENSE\\
├─ semver@5.7.2\\
│  ├─ licenses: ISC\\
│  ├─ repository: https://github.com/npm/node-semver\\
│  ├─ publisher: GitHub Inc.\\
│  ├─ path: ./Frontend/node\_modules/less/node\_modules/semver\\
│  └─ licenseFile: ./Frontend/node\_modules/less/node\_modules/semver/LICENSE\\
├─ semver@6.3.1\\
│  ├─ licenses: ISC\\
│  ├─ repository: https://github.com/npm/node-semver\\
│  ├─ publisher: GitHub Inc.\\
│  ├─ path: ./Frontend/node\_modules/@babel/helper-compilation-targets/node\_modules/semver\\
│  └─ licenseFile: ./Frontend/node\_modules/@babel/helper-compilation-targets/node\_modules/semver/LICENSE\\
├─ semver@7.6.0\\
│  ├─ licenses: ISC\\
│  ├─ repository: https://github.com/npm/node-semver\\
│  ├─ publisher: GitHub Inc.\\
│  ├─ path: ./Frontend/node\_modules/semver\\
│  └─ licenseFile: ./Frontend/node\_modules/semver/LICENSE\\
├─ send@0.18.0\\
│  ├─ licenses: MIT\\
│  ├─ repository: https://github.com/pillarjs/send\\
│  ├─ publisher: TJ Holowaychuk\\
│  ├─ email: tj@vision-media.ca\\
│  ├─ path: ./Frontend/node\_modules/send\\
│  └─ licenseFile: ./Frontend/node\_modules/send/LICENSE\\
├─ serialize-javascript@6.0.2\\
│  ├─ licenses: BSD-3-Clause\\
│  ├─ repository: https://github.com/yahoo/serialize-javascript\\
│  ├─ publisher: Eric Ferraiuolo\\
│  ├─ email: edf@ericf.me\\
│  ├─ path: ./Frontend/node\_modules/serialize-javascript\\
│  └─ licenseFile: ./Frontend/node\_modules/serialize-javascript/LICENSE\\
├─ serve-index@1.9.1\\
│  ├─ licenses: MIT\\
│  ├─ repository: https://github.com/expressjs/serve-index\\
│  ├─ publisher: Douglas Christopher Wilson\\
│  ├─ email: doug@somethingdoug.com\\
│  ├─ path: ./Frontend/node\_modules/serve-index\\
│  └─ licenseFile: ./Frontend/node\_modules/serve-index/LICENSE\\
├─ serve-static@1.15.0\\
│  ├─ licenses: MIT\\
│  ├─ repository: https://github.com/expressjs/serve-static\\
│  ├─ publisher: Douglas Christopher Wilson\\
│  ├─ email: doug@somethingdoug.com\\
│  ├─ path: ./Frontend/node\_modules/serve-static\\
│  └─ licenseFile: ./Frontend/node\_modules/serve-static/LICENSE\\
├─ set-function-length@1.2.2\\
│  ├─ licenses: MIT\\
│  ├─ repository: https://github.com/ljharb/set-function-length\\
│  ├─ publisher: Jordan Harband\\
│  ├─ email: ljharb@gmail.com\\
│  ├─ path: ./Frontend/node\_modules/set-function-length\\
│  └─ licenseFile: ./Frontend/node\_modules/set-function-length/LICENSE\\
├─ setprototypeof@1.1.0\\
│  ├─ licenses: ISC\\
│  ├─ repository: https://github.com/wesleytodd/setprototypeof\\
│  ├─ publisher: Wes Todd\\
│  ├─ path: ./Frontend/node\_modules/serve-index/node\_modules/setprototypeof\\
│  └─ licenseFile: ./Frontend/node\_modules/serve-index/node\_modules/setprototypeof/LICENSE\\
├─ setprototypeof@1.2.0\\
│  ├─ licenses: ISC\\
│  ├─ repository: https://github.com/wesleytodd/setprototypeof\\
│  ├─ publisher: Wes Todd\\
│  ├─ path: ./Frontend/node\_modules/setprototypeof\\
│  └─ licenseFile: ./Frontend/node\_modules/setprototypeof/LICENSE\\
├─ shallow-clone@3.0.1\\
│  ├─ licenses: MIT\\
│  ├─ repository: https://github.com/jonschlinkert/shallow-clone\\
│  ├─ publisher: Jon Schlinkert\\
│  ├─ url: https://github.com/jonschlinkert\\
│  ├─ path: ./Frontend/node\_modules/shallow-clone\\
│  └─ licenseFile: ./Frontend/node\_modules/shallow-clone/LICENSE\\
├─ shebang-command@2.0.0\\
│  ├─ licenses: MIT\\
│  ├─ repository: https://github.com/kevva/shebang-command\\
│  ├─ publisher: Kevin Mårtensson\\
│  ├─ email: kevinmartensson@gmail.com\\
│  ├─ url: github.com/kevva\\
│  ├─ path: ./Frontend/node\_modules/shebang-command\\
│  └─ licenseFile: ./Frontend/node\_modules/shebang-command/license\\
├─ shebang-regex@3.0.0\\
│  ├─ licenses: MIT\\
│  ├─ repository: https://github.com/sindresorhus/shebang-regex\\
│  ├─ publisher: Sindre Sorhus\\
│  ├─ email: sindresorhus@gmail.com\\
│  ├─ url: sindresorhus.com\\
│  ├─ path: ./Frontend/node\_modules/shebang-regex\\
│  └─ licenseFile: ./Frontend/node\_modules/shebang-regex/license\\
├─ shell-quote@1.8.1\\
│  ├─ licenses: MIT\\
│  ├─ repository: https://github.com/ljharb/shell-quote\\
│  ├─ publisher: James Halliday\\
│  ├─ email: mail@substack.net\\
│  ├─ url: http://substack.net\\
│  ├─ path: ./Frontend/node\_modules/shell-quote\\
│  └─ licenseFile: ./Frontend/node\_modules/shell-quote/LICENSE\\
├─ side-channel@1.0.6\\
│  ├─ licenses: MIT\\
│  ├─ repository: https://github.com/ljharb/side-channel\\
│  ├─ publisher: Jordan Harband\\
│  ├─ email: ljharb@gmail.com\\
│  ├─ path: ./Frontend/node\_modules/side-channel\\
│  └─ licenseFile: ./Frontend/node\_modules/side-channel/LICENSE\\
├─ signal-exit@3.0.7\\
│  ├─ licenses: ISC\\
│  ├─ repository: https://github.com/tapjs/signal-exit\\
│  ├─ publisher: Ben Coe\\
│  ├─ email: ben@npmjs.com\\
│  ├─ path: ./Frontend/node\_modules/signal-exit\\
│  └─ licenseFile: ./Frontend/node\_modules/signal-exit/LICENSE.txt\\
├─ signal-exit@4.1.0\\
│  ├─ licenses: ISC\\
│  ├─ repository: https://github.com/tapjs/signal-exit\\
│  ├─ publisher: Ben Coe\\
│  ├─ email: ben@npmjs.com\\
│  ├─ path: ./Frontend/node\_modules/foreground-child/node\_modules/signal-exit\\
│  └─ licenseFile: ./Frontend/node\_modules/foreground-child/node\_modules/signal-exit/LICENSE.txt\\
├─ sigstore@2.3.0\\
│  ├─ licenses: Apache-2.0\\
│  ├─ repository: https://github.com/sigstore/sigstore-js\\
│  ├─ publisher: bdehamer@github.com\\
│  ├─ path: ./Frontend/node\_modules/sigstore\\
│  └─ licenseFile: ./Frontend/node\_modules/sigstore/LICENSE\\
├─ slash@4.0.0\\
│  ├─ licenses: MIT\\
│  ├─ repository: https://github.com/sindresorhus/slash\\
│  ├─ publisher: Sindre Sorhus\\
│  ├─ email: sindresorhus@gmail.com\\
│  ├─ url: https://sindresorhus.com\\
│  ├─ path: ./Frontend/node\_modules/slash\\
│  └─ licenseFile: ./Frontend/node\_modules/slash/license\\
├─ smart-buffer@4.2.0\\
│  ├─ licenses: MIT\\
│  ├─ repository: https://github.com/JoshGlazebrook/smart-buffer\\
│  ├─ publisher: Josh Glazebrook\\
│  ├─ path: ./Frontend/node\_modules/smart-buffer\\
│  └─ licenseFile: ./Frontend/node\_modules/smart-buffer/LICENSE\\
├─ socket.io-adapter@2.5.4\\
│  ├─ licenses: MIT\\
│  ├─ repository: https://github.com/socketio/socket.io-adapter\\
│  ├─ path: ./Frontend/node\_modules/socket.io-adapter\\
│  └─ licenseFile: ./Frontend/node\_modules/socket.io-adapter/LICENSE\\
├─ socket.io-parser@4.2.4\\
│  ├─ licenses: MIT\\
│  ├─ repository: https://github.com/socketio/socket.io-parser\\
│  ├─ path: ./Frontend/node\_modules/socket.io-parser\\
│  └─ licenseFile: ./Frontend/node\_modules/socket.io-parser/LICENSE\\
├─ socket.io@4.7.5\\
│  ├─ licenses: MIT\\
│  ├─ repository: https://github.com/socketio/socket.io\\
│  ├─ path: ./Frontend/node\_modules/socket.io\\
│  └─ licenseFile: ./Frontend/node\_modules/socket.io/LICENSE\\
├─ sockjs@0.3.24\\
│  ├─ licenses: MIT\\
│  ├─ repository: https://github.com/sockjs/sockjs-node\\
│  ├─ publisher: Marek Majkowski\\
│  ├─ path: ./Frontend/node\_modules/sockjs\\
│  └─ licenseFile: ./Frontend/node\_modules/sockjs/LICENSE\\
├─ socks-proxy-agent@8.0.3\\
│  ├─ licenses: MIT\\
│  ├─ repository: https://github.com/TooTallNate/proxy-agents\\
│  ├─ publisher: Nathan Rajlich\\
│  ├─ email: nathan@tootallnate.net\\
│  ├─ url: http://n8.io/\\
│  ├─ path: ./Frontend/node\_modules/socks-proxy-agent\\
│  └─ licenseFile: ./Frontend/node\_modules/socks-proxy-agent/LICENSE\\
├─ socks@2.8.3\\
│  ├─ licenses: MIT\\
│  ├─ repository: https://github.com/JoshGlazebrook/socks\\
│  ├─ publisher: Josh Glazebrook\\
│  ├─ path: ./Frontend/node\_modules/socks\\
│  └─ licenseFile: ./Frontend/node\_modules/socks/LICENSE\\
├─ source-map-js@1.2.0\\
│  ├─ licenses: BSD-3-Clause\\
│  ├─ repository: https://github.com/7rulnik/source-map-js\\
│  ├─ publisher: Valentin 7rulnik Semirulnik\\
│  ├─ email: v7rulnik@gmail.com\\
│  ├─ path: ./Frontend/node\_modules/source-map-js\\
│  └─ licenseFile: ./Frontend/node\_modules/source-map-js/LICENSE\\
├─ source-map-loader@5.0.0\\
│  ├─ licenses: MIT\\
│  ├─ repository: https://github.com/webpack-contrib/source-map-loader\\
│  ├─ publisher: Tobias Koppers @sokra\\
│  ├─ path: ./Frontend/node\_modules/source-map-loader\\
│  └─ licenseFile: ./Frontend/node\_modules/source-map-loader/LICENSE\\
├─ source-map-support@0.5.21\\
│  ├─ licenses: MIT\\
│  ├─ repository: https://github.com/evanw/node-source-map-support\\
│  ├─ path: ./Frontend/node\_modules/source-map-support\\
│  └─ licenseFile: ./Frontend/node\_modules/source-map-support/LICENSE.md\\
├─ source-map@0.6.1\\
│  ├─ licenses: BSD-3-Clause\\
│  ├─ repository: https://github.com/mozilla/source-map\\
│  ├─ publisher: Nick Fitzgerald\\
│  ├─ email: nfitzgerald@mozilla.com\\
│  ├─ path: ./Frontend/node\_modules/source-map-support/node\_modules/source-map\\
│  └─ licenseFile: ./Frontend/node\_modules/source-map-support/node\_modules/source-map/LICENSE\\
├─ source-map@0.7.4\\
│  ├─ licenses: BSD-3-Clause\\
│  ├─ repository: https://github.com/mozilla/source-map\\
│  ├─ publisher: Nick Fitzgerald\\
│  ├─ email: nfitzgerald@mozilla.com\\
│  ├─ path: ./Frontend/node\_modules/source-map\\
│  └─ licenseFile: ./Frontend/node\_modules/source-map/LICENSE\\
├─ spawn-command@0.0.2-1\\
│  ├─ licenses: MIT\\
│  ├─ repository: https://github.com/mmalecki/spawn-command\\
│  ├─ publisher: Maciej Małecki\\
│  ├─ email: me@mmalecki.com\\
│  ├─ path: ./Frontend/node\_modules/spawn-command\\
│  └─ licenseFile: ./Frontend/node\_modules/spawn-command/LICENSE\\
├─ spdx-correct@3.2.0\\
│  ├─ licenses: Apache-2.0\\
│  ├─ repository: https://github.com/jslicense/spdx-correct.js\\
│  ├─ path: ./Frontend/node\_modules/spdx-correct\\
│  └─ licenseFile: ./Frontend/node\_modules/spdx-correct/LICENSE\\
├─ spdx-exceptions@2.5.0\\
│  ├─ licenses: CC-BY-3.0\\
│  ├─ repository: https://github.com/kemitchell/spdx-exceptions.json\\
│  ├─ publisher: The Linux Foundation\\
│  ├─ path: ./Frontend/node\_modules/spdx-exceptions\\
│  └─ licenseFile: ./Frontend/node\_modules/spdx-exceptions/README.md\\
├─ spdx-expression-parse@3.0.1\\
│  ├─ licenses: MIT\\
│  ├─ repository: https://github.com/jslicense/spdx-expression-parse.js\\
│  ├─ publisher: Kyle E. Mitchell\\
│  ├─ email: kyle@kemitchell.com\\
│  ├─ url: https://kemitchell.com\\
│  ├─ path: ./Frontend/node\_modules/spdx-expression-parse\\
│  └─ licenseFile: ./Frontend/node\_modules/spdx-expression-parse/LICENSE\\
├─ spdx-license-ids@3.0.17\\
│  ├─ licenses: CC0-1.0\\
│  ├─ repository: https://github.com/jslicense/spdx-license-ids\\
│  ├─ publisher: Shinnosuke Watanabe\\
│  ├─ url: https://github.com/shinnn\\
│  ├─ path: ./Frontend/node\_modules/spdx-license-ids\\
│  └─ licenseFile: ./Frontend/node\_modules/spdx-license-ids/README.md\\
├─ spdy-transport@3.0.0\\
│  ├─ licenses: MIT\\
│  ├─ repository: https://github.com/spdy-http2/spdy-transport\\
│  ├─ publisher: Fedor Indutny\\
│  ├─ email: fedor@indutny.com\\
│  ├─ path: ./Frontend/node\_modules/spdy-transport\\
│  └─ licenseFile: ./Frontend/node\_modules/spdy-transport/README.md\\
├─ spdy@4.0.2\\
│  ├─ licenses: MIT\\
│  ├─ repository: https://github.com/indutny/node-spdy\\
│  ├─ publisher: Fedor Indutny\\
│  ├─ email: fedor.indutny@gmail.com\\
│  ├─ path: ./Frontend/node\_modules/spdy\\
│  └─ licenseFile: ./Frontend/node\_modules/spdy/README.md\\
├─ sprintf-js@1.0.3\\
│  ├─ licenses: BSD-3-Clause\\
│  ├─ repository: https://github.com/alexei/sprintf.js\\
│  ├─ publisher: Alexandru Marasteanu\\
│  ├─ email: hello@alexei.ro\\
│  ├─ url: http://alexei.ro/\\
│  ├─ path: ./Frontend/node\_modules/sprintf-js\\
│  └─ licenseFile: ./Frontend/node\_modules/sprintf-js/LICENSE\\
├─ sprintf-js@1.1.3\\
│  ├─ licenses: BSD-3-Clause\\
│  ├─ repository: https://github.com/alexei/sprintf.js\\
│  ├─ publisher: Alexandru Mărășteanu\\
│  ├─ email: hello@alexei.ro\\
│  ├─ path: ./Frontend/node\_modules/ip-address/node\_modules/sprintf-js\\
│  └─ licenseFile: ./Frontend/node\_modules/ip-address/node\_modules/sprintf-js/LICENSE\\
├─ ssri@10.0.6\\
│  ├─ licenses: ISC\\
│  ├─ repository: https://github.com/npm/ssri\\
│  ├─ publisher: GitHub Inc.\\
│  ├─ path: ./Frontend/node\_modules/ssri\\
│  └─ licenseFile: ./Frontend/node\_modules/ssri/LICENSE.md\\
├─ statuses@1.5.0\\
│  ├─ licenses: MIT\\
│  ├─ repository: https://github.com/jshttp/statuses\\
│  ├─ path: ./Frontend/node\_modules/statuses\\
│  └─ licenseFile: ./Frontend/node\_modules/statuses/LICENSE\\
├─ statuses@2.0.1\\
│  ├─ licenses: MIT\\
│  ├─ repository: https://github.com/jshttp/statuses\\
│  ├─ path: ./Frontend/node\_modules/http-errors/node\_modules/statuses\\
│  └─ licenseFile: ./Frontend/node\_modules/http-errors/node\_modules/statuses/LICENSE\\
├─ streamroller@3.1.5\\
│  ├─ licenses: MIT\\
│  ├─ repository: https://github.com/log4js-node/streamroller\\
│  ├─ publisher: Gareth Jones\\
│  ├─ email: gareth.nomiddlename@gmail.com\\
│  ├─ path: ./Frontend/node\_modules/streamroller\\
│  └─ licenseFile: ./Frontend/node\_modules/streamroller/LICENSE\\
├─ string-width@4.2.3\\
│  ├─ licenses: MIT\\
│  ├─ repository: https://github.com/sindresorhus/string-width\\
│  ├─ publisher: Sindre Sorhus\\
│  ├─ email: sindresorhus@gmail.com\\
│  ├─ url: sindresorhus.com\\
│  ├─ path: ./Frontend/node\_modules/string-width\\
│  └─ licenseFile: ./Frontend/node\_modules/string-width/license\\
├─ string-width@5.1.2\\
│  ├─ licenses: MIT\\
│  ├─ repository: https://github.com/sindresorhus/string-width\\
│  ├─ publisher: Sindre Sorhus\\
│  ├─ email: sindresorhus@gmail.com\\
│  ├─ url: https://sindresorhus.com\\
│  ├─ path: ./Frontend/node\_modules/@isaacs/cliui/node\_modules/string-width\\
│  └─ licenseFile: ./Frontend/node\_modules/@isaacs/cliui/node\_modules/string-width/license\\
├─ string\_decoder@1.1.1\\
│  ├─ licenses: MIT\\
│  ├─ repository: https://github.com/nodejs/string\_decoder\\
│  ├─ path: ./Frontend/node\_modules/hpack.js/node\_modules/string\_decoder\\
│  └─ licenseFile: ./Frontend/node\_modules/hpack.js/node\_modules/string\_decoder/LICENSE\\
├─ string\_decoder@1.3.0\\
│  ├─ licenses: MIT\\
│  ├─ repository: https://github.com/nodejs/string\_decoder\\
│  ├─ path: ./Frontend/node\_modules/string\_decoder\\
│  └─ licenseFile: ./Frontend/node\_modules/string\_decoder/LICENSE\\
├─ strip-ansi@6.0.1\\
│  ├─ licenses: MIT\\
│  ├─ repository: https://github.com/chalk/strip-ansi\\
│  ├─ publisher: Sindre Sorhus\\
│  ├─ email: sindresorhus@gmail.com\\
│  ├─ url: sindresorhus.com\\
│  ├─ path: ./Frontend/node\_modules/strip-ansi\\
│  └─ licenseFile: ./Frontend/node\_modules/strip-ansi/license\\
├─ strip-ansi@7.1.0\\
│  ├─ licenses: MIT\\
│  ├─ repository: https://github.com/chalk/strip-ansi\\
│  ├─ publisher: Sindre Sorhus\\
│  ├─ email: sindresorhus@gmail.com\\
│  ├─ url: https://sindresorhus.com\\
│  ├─ path: ./Frontend/node\_modules/@isaacs/cliui/node\_modules/strip-ansi\\
│  └─ licenseFile: ./Frontend/node\_modules/@isaacs/cliui/node\_modules/strip-ansi/license\\
├─ strip-final-newline@2.0.0\\
│  ├─ licenses: MIT\\
│  ├─ repository: https://github.com/sindresorhus/strip-final-newline\\
│  ├─ publisher: Sindre Sorhus\\
│  ├─ email: sindresorhus@gmail.com\\
│  ├─ url: sindresorhus.com\\
│  ├─ path: ./Frontend/node\_modules/strip-final-newline\\
│  └─ licenseFile: ./Frontend/node\_modules/strip-final-newline/license\\
├─ supports-color@5.5.0\\
│  ├─ licenses: MIT\\
│  ├─ repository: https://github.com/chalk/supports-color\\
│  ├─ publisher: Sindre Sorhus\\
│  ├─ email: sindresorhus@gmail.com\\
│  ├─ url: sindresorhus.com\\
│  ├─ path: ./Frontend/node\_modules/supports-color\\
│  └─ licenseFile: ./Frontend/node\_modules/supports-color/license\\
├─ supports-color@7.2.0\\
│  ├─ licenses: MIT\\
│  ├─ repository: https://github.com/chalk/supports-color\\
│  ├─ publisher: Sindre Sorhus\\
│  ├─ email: sindresorhus@gmail.com\\
│  ├─ url: sindresorhus.com\\
│  ├─ path: ./Frontend/node\_modules/critters/node\_modules/supports-color\\
│  └─ licenseFile: ./Frontend/node\_modules/critters/node\_modules/supports-color/license\\
├─ supports-color@8.1.1\\
│  ├─ licenses: MIT\\
│  ├─ repository: https://github.com/chalk/supports-color\\
│  ├─ publisher: Sindre Sorhus\\
│  ├─ email: sindresorhus@gmail.com\\
│  ├─ url: https://sindresorhus.com\\
│  ├─ path: ./Frontend/node\_modules/jest-worker/node\_modules/supports-color\\
│  └─ licenseFile: ./Frontend/node\_modules/jest-worker/node\_modules/supports-color/license\\
├─ supports-preserve-symlinks-flag@1.0.0\\
│  ├─ licenses: MIT\\
│  ├─ repository: https://github.com/inspect-js/node-supports-preserve-symlinks-flag\\
│  ├─ publisher: Jordan Harband\\
│  ├─ email: ljharb@gmail.com\\
│  ├─ path: ./Frontend/node\_modules/supports-preserve-symlinks-flag\\
│  └─ licenseFile: ./Frontend/node\_modules/supports-preserve-symlinks-flag/LICENSE\\
├─ symbol-observable@4.0.0\\
│  ├─ licenses: MIT\\
│  ├─ repository: https://github.com/blesh/symbol-observable\\
│  ├─ publisher: Ben Lesh\\
│  ├─ email: ben@benlesh.com\\
│  ├─ path: ./Frontend/node\_modules/symbol-observable\\
│  └─ licenseFile: ./Frontend/node\_modules/symbol-observable/license\\
├─ tapable@2.2.1\\
│  ├─ licenses: MIT\\
│  ├─ repository: https://github.com/webpack/tapable\\
│  ├─ publisher: Tobias Koppers @sokra\\
│  ├─ path: ./Frontend/node\_modules/tapable\\
│  └─ licenseFile: ./Frontend/node\_modules/tapable/LICENSE\\
├─ tar@6.2.1\\
│  ├─ licenses: ISC\\
│  ├─ repository: https://github.com/isaacs/node-tar\\
│  ├─ publisher: GitHub Inc.\\
│  ├─ path: ./Frontend/node\_modules/tar\\
│  └─ licenseFile: ./Frontend/node\_modules/tar/LICENSE\\
├─ terser-webpack-plugin@5.3.10\\
│  ├─ licenses: MIT\\
│  ├─ repository: https://github.com/webpack-contrib/terser-webpack-plugin\\
│  ├─ publisher: webpack Contrib Team\\
│  ├─ path: ./Frontend/node\_modules/terser-webpack-plugin\\
│  └─ licenseFile: ./Frontend/node\_modules/terser-webpack-plugin/LICENSE\\
├─ terser@5.29.1\\
│  ├─ licenses: BSD-2-Clause\\
│  ├─ repository: https://github.com/terser/terser\\
│  ├─ publisher: Mihai Bazon\\
│  ├─ email: mihai.bazon@gmail.com\\
│  ├─ url: http://lisperator.net/\\
│  ├─ path: ./Frontend/node\_modules/terser\\
│  └─ licenseFile: ./Frontend/node\_modules/terser/LICENSE\\
├─ test-exclude@6.0.0\\
│  ├─ licenses: ISC\\
│  ├─ repository: https://github.com/istanbuljs/test-exclude\\
│  ├─ publisher: Ben Coe\\
│  ├─ email: ben@npmjs.com\\
│  ├─ path: ./Frontend/node\_modules/test-exclude\\
│  └─ licenseFile: ./Frontend/node\_modules/test-exclude/LICENSE.txt\\
├─ through@2.3.8\\
│  ├─ licenses: MIT\\
│  ├─ repository: https://github.com/dominictarr/through\\
│  ├─ publisher: Dominic Tarr\\
│  ├─ email: dominic.tarr@gmail.com\\
│  ├─ url: dominictarr.com\\
│  ├─ path: ./Frontend/node\_modules/through\\
│  └─ licenseFile: ./Frontend/node\_modules/through/LICENSE.APACHE2\\
├─ thunky@1.1.0\\
│  ├─ licenses: MIT\\
│  ├─ repository: https://github.com/mafintosh/thunky\\
│  ├─ publisher: Mathias Buus Madsen\\
│  ├─ email: mathiasbuus@gmail.com\\
│  ├─ path: ./Frontend/node\_modules/thunky\\
│  └─ licenseFile: ./Frontend/node\_modules/thunky/LICENSE\\
├─ tmp@0.0.33\\
│  ├─ licenses: MIT\\
│  ├─ repository: https://github.com/raszi/node-tmp\\
│  ├─ publisher: KARASZI István\\
│  ├─ email: github@spam.raszi.hu\\
│  ├─ url: http://raszi.hu/\\
│  ├─ path: ./Frontend/node\_modules/tmp\\
│  └─ licenseFile: ./Frontend/node\_modules/tmp/LICENSE\\
├─ tmp@0.2.3\\
│  ├─ licenses: MIT\\
│  ├─ repository: https://github.com/raszi/node-tmp\\
│  ├─ publisher: KARASZI István\\
│  ├─ email: github@spam.raszi.hu\\
│  ├─ url: http://raszi.hu/\\
│  ├─ path: ./Frontend/node\_modules/karma/node\_modules/tmp\\
│  └─ licenseFile: ./Frontend/node\_modules/karma/node\_modules/tmp/LICENSE\\
├─ to-fast-properties@2.0.0\\
│  ├─ licenses: MIT\\
│  ├─ repository: https://github.com/sindresorhus/to-fast-properties\\
│  ├─ publisher: Sindre Sorhus\\
│  ├─ email: sindresorhus@gmail.com\\
│  ├─ url: sindresorhus.com\\
│  ├─ path: ./Frontend/node\_modules/to-fast-properties\\
│  └─ licenseFile: ./Frontend/node\_modules/to-fast-properties/license\\
├─ to-regex-range@5.0.1\\
│  ├─ licenses: MIT\\
│  ├─ repository: https://github.com/micromatch/to-regex-range\\
│  ├─ publisher: Jon Schlinkert\\
│  ├─ url: https://github.com/jonschlinkert\\
│  ├─ path: ./Frontend/node\_modules/to-regex-range\\
│  └─ licenseFile: ./Frontend/node\_modules/to-regex-range/LICENSE\\
├─ toidentifier@1.0.1\\
│  ├─ licenses: MIT\\
│  ├─ repository: https://github.com/component/toidentifier\\
│  ├─ publisher: Douglas Christopher Wilson\\
│  ├─ email: doug@somethingdoug.com\\
│  ├─ path: ./Frontend/node\_modules/toidentifier\\
│  └─ licenseFile: ./Frontend/node\_modules/toidentifier/LICENSE\\
├─ tr46@0.0.3\\
│  ├─ licenses: MIT\\
│  ├─ repository: https://github.com/Sebmaster/tr46.js\\
│  ├─ publisher: Sebastian Mayr\\
│  ├─ email: npm@smayr.name\\
│  └─ path: ./Frontend/node\_modules/tr46\\
├─ tree-kill@1.2.2\\
│  ├─ licenses: MIT\\
│  ├─ repository: https://github.com/pkrumins/node-tree-kill\\
│  ├─ publisher: Peteris Krumins\\
│  ├─ email: peteris.krumins@gmail.com\\
│  ├─ url: http://www.catonmat.net\\
│  ├─ path: ./Frontend/node\_modules/tree-kill\\
│  └─ licenseFile: ./Frontend/node\_modules/tree-kill/LICENSE\\
├─ tslib@1.14.1\\
│  ├─ licenses: 0BSD\\
│  ├─ repository: https://github.com/Microsoft/tslib\\
│  ├─ publisher: Microsoft Corp.\\
│  ├─ path: ./Frontend/node\_modules/concurrently/node\_modules/tslib\\
│  └─ licenseFile: ./Frontend/node\_modules/concurrently/node\_modules/tslib/LICENSE.txt\\
├─ tslib@2.6.2\\
│  ├─ licenses: 0BSD\\
│  ├─ repository: https://github.com/Microsoft/tslib\\
│  ├─ publisher: Microsoft Corp.\\
│  ├─ path: ./Frontend/node\_modules/tslib\\
│  └─ licenseFile: ./Frontend/node\_modules/tslib/LICENSE.txt\\
├─ tuf-js@2.2.1\\
│  ├─ licenses: MIT\\
│  ├─ repository: https://github.com/theupdateframework/tuf-js\\
│  ├─ publisher: bdehamer@github.com\\
│  ├─ path: ./Frontend/node\_modules/tuf-js\\
│  └─ licenseFile: ./Frontend/node\_modules/tuf-js/LICENSE\\
├─ type-fest@0.21.3\\
│  ├─ licenses: (MIT OR CC0-1.0)\\
│  ├─ repository: https://github.com/sindresorhus/type-fest\\
│  ├─ publisher: Sindre Sorhus\\
│  ├─ email: sindresorhus@gmail.com\\
│  ├─ url: https://sindresorhus.com\\
│  ├─ path: ./Frontend/node\_modules/type-fest\\
│  └─ licenseFile: ./Frontend/node\_modules/type-fest/license\\
├─ type-is@1.6.18\\
│  ├─ licenses: MIT\\
│  ├─ repository: https://github.com/jshttp/type-is\\
│  ├─ path: ./Frontend/node\_modules/type-is\\
│  └─ licenseFile: ./Frontend/node\_modules/type-is/LICENSE\\
├─ typed-assert@1.0.9\\
│  ├─ licenses: MIT\\
│  ├─ repository: https://github.com/elierotenberg/typed-assert\\
│  ├─ publisher: Elie Rotenberg\\
│  ├─ email: elie@rotenberg.io\\
│  ├─ path: ./Frontend/node\_modules/typed-assert\\
│  └─ licenseFile: ./Frontend/node\_modules/typed-assert/README.md\\
├─ typescript@5.4.5\\
│  ├─ licenses: Apache-2.0\\
│  ├─ repository: https://github.com/Microsoft/TypeScript\\
│  ├─ publisher: Microsoft Corp.\\
│  ├─ path: ./Frontend/node\_modules/typescript\\
│  └─ licenseFile: ./Frontend/node\_modules/typescript/LICENSE.txt\\
├─ ua-parser-js@0.7.37\\
│  ├─ licenses: MIT\\
│  ├─ repository: https://github.com/faisalman/ua-parser-js\\
│  ├─ publisher: Faisal Salman\\
│  ├─ email: f@faisalman.com\\
│  ├─ url: http://faisalman.com\\
│  ├─ path: ./Frontend/node\_modules/ua-parser-js\\
│  └─ licenseFile: ./Frontend/node\_modules/ua-parser-js/license.md\\
├─ uid@2.0.2\\
│  ├─ licenses: MIT\\
│  ├─ repository: https://github.com/lukeed/uid\\
│  ├─ publisher: Luke Edwards\\
│  ├─ email: luke.edwards05@gmail.com\\
│  ├─ url: https://lukeed.com\\
│  ├─ path: ./Frontend/node\_modules/uid\\
│  └─ licenseFile: ./Frontend/node\_modules/uid/license\\
├─ undici-types@5.26.5\\
│  ├─ licenses: MIT\\
│  ├─ repository: https://github.com/nodejs/undici\\
│  ├─ path: ./Frontend/node\_modules/undici-types\\
│  └─ licenseFile: ./Frontend/node\_modules/undici-types/README.md\\
├─ undici@6.11.1\\
│  ├─ licenses: MIT\\
│  ├─ repository: https://github.com/nodejs/undici\\
│  ├─ path: ./Frontend/node\_modules/undici\\
│  └─ licenseFile: ./Frontend/node\_modules/undici/LICENSE\\
├─ unicode-canonical-property-names-ecmascript@2.0.0\\
│  ├─ licenses: MIT\\
│  ├─ repository: https://github.com/mathiasbynens/unicode-canonical-property-names-ecmascript\\
│  ├─ publisher: Mathias Bynens\\
│  ├─ url: https://mathiasbynens.be/\\
│  ├─ path: ./Frontend/node\_modules/unicode-canonical-property-names-ecmascript\\
│  └─ licenseFile: ./Frontend/node\_modules/unicode-canonical-property-names-ecmascript/LICENSE-MIT.txt\\
├─ unicode-match-property-ecmascript@2.0.0\\
│  ├─ licenses: MIT\\
│  ├─ repository: https://github.com/mathiasbynens/unicode-match-property-ecmascript\\
│  ├─ publisher: Mathias Bynens\\
│  ├─ url: https://mathiasbynens.be/\\
│  ├─ path: ./Frontend/node\_modules/unicode-match-property-ecmascript\\
│  └─ licenseFile: ./Frontend/node\_modules/unicode-match-property-ecmascript/LICENSE-MIT.txt\\
├─ unicode-match-property-value-ecmascript@2.1.0\\
│  ├─ licenses: MIT\\
│  ├─ repository: https://github.com/mathiasbynens/unicode-match-property-value-ecmascript\\
│  ├─ publisher: Mathias Bynens\\
│  ├─ url: https://mathiasbynens.be/\\
│  ├─ path: ./Frontend/node\_modules/unicode-match-property-value-ecmascript\\
│  └─ licenseFile: ./Frontend/node\_modules/unicode-match-property-value-ecmascript/LICENSE-MIT.txt\\
├─ unicode-property-aliases-ecmascript@2.1.0\\
│  ├─ licenses: MIT\\
│  ├─ repository: https://github.com/mathiasbynens/unicode-property-aliases-ecmascript\\
│  ├─ publisher: Mathias Bynens\\
│  ├─ url: https://mathiasbynens.be/\\
│  ├─ path: ./Frontend/node\_modules/unicode-property-aliases-ecmascript\\
│  └─ licenseFile: ./Frontend/node\_modules/unicode-property-aliases-ecmascript/LICENSE-MIT.txt\\
├─ unique-filename@3.0.0\\
│  ├─ licenses: ISC\\
│  ├─ repository: https://github.com/npm/unique-filename\\
│  ├─ publisher: GitHub Inc.\\
│  ├─ path: ./Frontend/node\_modules/unique-filename\\
│  └─ licenseFile: ./Frontend/node\_modules/unique-filename/LICENSE\\
├─ unique-slug@4.0.0\\
│  ├─ licenses: ISC\\
│  ├─ repository: https://github.com/npm/unique-slug\\
│  ├─ publisher: GitHub Inc.\\
│  ├─ path: ./Frontend/node\_modules/unique-slug\\
│  └─ licenseFile: ./Frontend/node\_modules/unique-slug/LICENSE\\
├─ universalify@0.1.2\\
│  ├─ licenses: MIT\\
│  ├─ repository: https://github.com/RyanZim/universalify\\
│  ├─ publisher: Ryan Zimmerman\\
│  ├─ email: opensrc@ryanzim.com\\
│  ├─ path: ./Frontend/node\_modules/universalify\\
│  └─ licenseFile: ./Frontend/node\_modules/universalify/LICENSE\\
├─ universalify@2.0.1\\
│  ├─ licenses: MIT\\
│  ├─ repository: https://github.com/RyanZim/universalify\\
│  ├─ publisher: Ryan Zimmerman\\
│  ├─ email: opensrc@ryanzim.com\\
│  ├─ path: ./Frontend/node\_modules/@openapitools/openapi-generator-cli/node\_modules/universalify\\
│  └─ licenseFile: ./Frontend/node\_modules/@openapitools/openapi-generator-cli/node\_modules/universalify/LICENSE\\
├─ unpipe@1.0.0\\
│  ├─ licenses: MIT\\
│  ├─ repository: https://github.com/stream-utils/unpipe\\
│  ├─ publisher: Douglas Christopher Wilson\\
│  ├─ email: doug@somethingdoug.com\\
│  ├─ path: ./Frontend/node\_modules/unpipe\\
│  └─ licenseFile: ./Frontend/node\_modules/unpipe/LICENSE\\
├─ update-browserslist-db@1.0.16\\
│  ├─ licenses: MIT\\
│  ├─ repository: https://github.com/browserslist/update-db\\
│  ├─ publisher: Andrey Sitnik\\
│  ├─ email: andrey@sitnik.ru\\
│  ├─ path: ./Frontend/node\_modules/update-browserslist-db\\
│  └─ licenseFile: ./Frontend/node\_modules/update-browserslist-db/LICENSE\\
├─ uri-js@4.4.1\\
│  ├─ licenses: BSD-2-Clause\\
│  ├─ repository: https://github.com/garycourt/uri-js\\
│  ├─ publisher: Gary Court\\
│  ├─ email: gary.court@gmail.com\\
│  ├─ path: ./Frontend/node\_modules/uri-js\\
│  └─ licenseFile: ./Frontend/node\_modules/uri-js/LICENSE\\
├─ util-deprecate@1.0.2\\
│  ├─ licenses: MIT\\
│  ├─ repository: https://github.com/TooTallNate/util-deprecate\\
│  ├─ publisher: Nathan Rajlich\\
│  ├─ email: nathan@tootallnate.net\\
│  ├─ url: http://n8.io/\\
│  ├─ path: ./Frontend/node\_modules/util-deprecate\\
│  └─ licenseFile: ./Frontend/node\_modules/util-deprecate/LICENSE\\
├─ utils-merge@1.0.1\\
│  ├─ licenses: MIT\\
│  ├─ repository: https://github.com/jaredhanson/utils-merge\\
│  ├─ publisher: Jared Hanson\\
│  ├─ email: jaredhanson@gmail.com\\
│  ├─ url: http://www.jaredhanson.net/\\
│  ├─ path: ./Frontend/node\_modules/utils-merge\\
│  └─ licenseFile: ./Frontend/node\_modules/utils-merge/LICENSE\\
├─ uuid@8.3.2\\
│  ├─ licenses: MIT\\
│  ├─ repository: https://github.com/uuidjs/uuid\\
│  ├─ path: ./Frontend/node\_modules/uuid\\
│  └─ licenseFile: ./Frontend/node\_modules/uuid/LICENSE.md\\
├─ validate-npm-package-license@3.0.4\\
│  ├─ licenses: Apache-2.0\\
│  ├─ repository: https://github.com/kemitchell/validate-npm-package-license.js\\
│  ├─ publisher: Kyle E. Mitchell\\
│  ├─ email: kyle@kemitchell.com\\
│  ├─ url: https://kemitchell.com\\
│  ├─ path: ./Frontend/node\_modules/validate-npm-package-license\\
│  └─ licenseFile: ./Frontend/node\_modules/validate-npm-package-license/LICENSE\\
├─ validate-npm-package-name@5.0.1\\
│  ├─ licenses: ISC\\
│  ├─ repository: https://github.com/npm/validate-npm-package-name\\
│  ├─ publisher: GitHub Inc.\\
│  ├─ path: ./Frontend/node\_modules/validate-npm-package-name\\
│  └─ licenseFile: ./Frontend/node\_modules/validate-npm-package-name/LICENSE\\
├─ vary@1.1.2\\
│  ├─ licenses: MIT\\
│  ├─ repository: https://github.com/jshttp/vary\\
│  ├─ publisher: Douglas Christopher Wilson\\
│  ├─ email: doug@somethingdoug.com\\
│  ├─ path: ./Frontend/node\_modules/vary\\
│  └─ licenseFile: ./Frontend/node\_modules/vary/LICENSE\\
├─ vite@5.1.7\\
│  ├─ licenses: MIT\\
│  ├─ repository: https://github.com/vitejs/vite\\
│  ├─ publisher: Evan You\\
│  ├─ path: ./Frontend/node\_modules/vite\\
│  └─ licenseFile: ./Frontend/node\_modules/vite/LICENSE.md\\
├─ void-elements@2.0.1\\
│  ├─ licenses: MIT\\
│  ├─ repository: https://github.com/hemanth/void-elements\\
│  ├─ publisher: hemanth.hm\\
│  ├─ path: ./Frontend/node\_modules/void-elements\\
│  └─ licenseFile: ./Frontend/node\_modules/void-elements/LICENSE\\
├─ watchpack@2.4.0\\
│  ├─ licenses: MIT\\
│  ├─ repository: https://github.com/webpack/watchpack\\
│  ├─ publisher: Tobias Koppers @sokra\\
│  ├─ path: ./Frontend/node\_modules/watchpack\\
│  └─ licenseFile: ./Frontend/node\_modules/watchpack/LICENSE\\
├─ wbuf@1.7.3\\
│  ├─ licenses: MIT\\
│  ├─ repository: https://github.com/indutny/wbuf\\
│  ├─ publisher: Fedor Indutny\\
│  ├─ email: fedor@indutny.com\\
│  ├─ path: ./Frontend/node\_modules/wbuf\\
│  └─ licenseFile: ./Frontend/node\_modules/wbuf/README.md\\
├─ wcwidth@1.0.1\\
│  ├─ licenses: MIT\\
│  ├─ repository: https://github.com/timoxley/wcwidth\\
│  ├─ publisher: Tim Oxley\\
│  ├─ path: ./Frontend/node\_modules/wcwidth\\
│  └─ licenseFile: ./Frontend/node\_modules/wcwidth/LICENSE\\
├─ webidl-conversions@3.0.1\\
│  ├─ licenses: BSD-2-Clause\\
│  ├─ repository: https://github.com/jsdom/webidl-conversions\\
│  ├─ publisher: Domenic Denicola\\
│  ├─ email: d@domenic.me\\
│  ├─ url: https://domenic.me/\\
│  ├─ path: ./Frontend/node\_modules/webidl-conversions\\
│  └─ licenseFile: ./Frontend/node\_modules/webidl-conversions/LICENSE.md\\
├─ webpack-dev-middleware@5.3.4\\
│  ├─ licenses: MIT\\
│  ├─ repository: https://github.com/webpack/webpack-dev-middleware\\
│  ├─ publisher: Tobias Koppers @sokra\\
│  ├─ path: ./Frontend/node\_modules/webpack-dev-server/node\_modules/webpack-dev-middleware\\
│  └─ licenseFile: ./Frontend/node\_modules/webpack-dev-server/node\_modules/webpack-dev-middleware/LICENSE\\
├─ webpack-dev-middleware@6.1.2\\
│  ├─ licenses: MIT\\
│  ├─ repository: https://github.com/webpack/webpack-dev-middleware\\
│  ├─ publisher: Tobias Koppers @sokra\\
│  ├─ path: ./Frontend/node\_modules/webpack-dev-middleware\\
│  └─ licenseFile: ./Frontend/node\_modules/webpack-dev-middleware/LICENSE\\
├─ webpack-dev-server@4.15.1\\
│  ├─ licenses: MIT\\
│  ├─ repository: https://github.com/webpack/webpack-dev-server\\
│  ├─ publisher: Tobias Koppers @sokra\\
│  ├─ path: ./Frontend/node\_modules/webpack-dev-server\\
│  └─ licenseFile: ./Frontend/node\_modules/webpack-dev-server/LICENSE\\
├─ webpack-merge@5.10.0\\
│  ├─ licenses: MIT\\
│  ├─ repository: https://github.com/survivejs/webpack-merge\\
│  ├─ publisher: Juho Vepsalainen\\
│  ├─ email: bebraw@gmail.com\\
│  ├─ path: ./Frontend/node\_modules/webpack-merge\\
│  └─ licenseFile: ./Frontend/node\_modules/webpack-merge/LICENSE\\
├─ webpack-sources@3.2.3\\
│  ├─ licenses: MIT\\
│  ├─ repository: https://github.com/webpack/webpack-sources\\
│  ├─ publisher: Tobias Koppers @sokra\\
│  ├─ path: ./Frontend/node\_modules/webpack-sources\\
│  └─ licenseFile: ./Frontend/node\_modules/webpack-sources/LICENSE\\
├─ webpack-subresource-integrity@5.1.0\\
│  ├─ licenses: MIT\\
│  ├─ repository: https://github.com/waysact/webpack-subresource-integrity\\
│  ├─ publisher: Julian Scheid\\
│  ├─ email: julian@evergiving.com\\
│  ├─ path: ./Frontend/node\_modules/webpack-subresource-integrity\\
│  └─ licenseFile: ./Frontend/node\_modules/webpack-subresource-integrity/LICENSE\\
├─ webpack@5.90.3\\
│  ├─ licenses: MIT\\
│  ├─ repository: https://github.com/webpack/webpack\\
│  ├─ publisher: Tobias Koppers @sokra\\
│  ├─ path: ./Frontend/node\_modules/webpack\\
│  └─ licenseFile: ./Frontend/node\_modules/webpack/LICENSE\\
├─ websocket-driver@0.7.4\\
│  ├─ licenses: Apache-2.0\\
│  ├─ repository: https://github.com/faye/websocket-driver-node\\
│  ├─ publisher: James Coglan\\
│  ├─ email: jcoglan@gmail.com\\
│  ├─ url: http://jcoglan.com/\\
│  ├─ path: ./Frontend/node\_modules/websocket-driver\\
│  └─ licenseFile: ./Frontend/node\_modules/websocket-driver/LICENSE.md\\
├─ websocket-extensions@0.1.4\\
│  ├─ licenses: Apache-2.0\\
│  ├─ repository: https://github.com/faye/websocket-extensions-node\\
│  ├─ publisher: James Coglan\\
│  ├─ email: jcoglan@gmail.com\\
│  ├─ url: http://jcoglan.com/\\
│  ├─ path: ./Frontend/node\_modules/websocket-extensions\\
│  └─ licenseFile: ./Frontend/node\_modules/websocket-extensions/LICENSE.md\\
├─ whatwg-url@5.0.0\\
│  ├─ licenses: MIT\\
│  ├─ repository: https://github.com/jsdom/whatwg-url\\
│  ├─ publisher: Sebastian Mayr\\
│  ├─ email: github@smayr.name\\
│  ├─ path: ./Frontend/node\_modules/whatwg-url\\
│  └─ licenseFile: ./Frontend/node\_modules/whatwg-url/LICENSE.txt\\
├─ which@1.3.1\\
│  ├─ licenses: ISC\\
│  ├─ repository: https://github.com/isaacs/node-which\\
│  ├─ publisher: Isaac Z. Schlueter\\
│  ├─ email: i@izs.me\\
│  ├─ url: http://blog.izs.me\\
│  ├─ path: ./Frontend/node\_modules/which\\
│  └─ licenseFile: ./Frontend/node\_modules/which/LICENSE\\
├─ which@2.0.2\\
│  ├─ licenses: ISC\\
│  ├─ repository: https://github.com/isaacs/node-which\\
│  ├─ publisher: Isaac Z. Schlueter\\
│  ├─ email: i@izs.me\\
│  ├─ url: http://blog.izs.me\\
│  ├─ path: ./Frontend/node\_modules/cross-spawn/node\_modules/which\\
│  └─ licenseFile: ./Frontend/node\_modules/cross-spawn/node\_modules/which/LICENSE\\
├─ which@4.0.0\\
│  ├─ licenses: ISC\\
│  ├─ repository: https://github.com/npm/node-which\\
│  ├─ publisher: GitHub Inc.\\
│  ├─ path: ./Frontend/node\_modules/@npmcli/promise-spawn/node\_modules/which\\
│  └─ licenseFile: ./Frontend/node\_modules/@npmcli/promise-spawn/node\_modules/which/LICENSE\\
├─ wildcard@2.0.1\\
│  ├─ licenses: MIT\\
│  ├─ repository: https://github.com/DamonOehlman/wildcard\\
│  ├─ publisher: Damon Oehlman\\
│  ├─ email: damon.oehlman@gmail.com\\
│  ├─ path: ./Frontend/node\_modules/wildcard\\
│  └─ licenseFile: ./Frontend/node\_modules/wildcard/LICENSE\\
├─ wrap-ansi@6.2.0\\
│  ├─ licenses: MIT\\
│  ├─ repository: https://github.com/chalk/wrap-ansi\\
│  ├─ publisher: Sindre Sorhus\\
│  ├─ email: sindresorhus@gmail.com\\
│  ├─ url: sindresorhus.com\\
│  ├─ path: ./Frontend/node\_modules/wrap-ansi\\
│  └─ licenseFile: ./Frontend/node\_modules/wrap-ansi/license\\
├─ wrap-ansi@7.0.0\\
│  ├─ licenses: MIT\\
│  ├─ repository: https://github.com/chalk/wrap-ansi\\
│  ├─ publisher: Sindre Sorhus\\
│  ├─ email: sindresorhus@gmail.com\\
│  ├─ url: https://sindresorhus.com\\
│  ├─ path: ./Frontend/node\_modules/wrap-ansi-cjs\\
│  └─ licenseFile: ./Frontend/node\_modules/wrap-ansi-cjs/license\\
├─ wrap-ansi@8.1.0\\
│  ├─ licenses: MIT\\
│  ├─ repository: https://github.com/chalk/wrap-ansi\\
│  ├─ publisher: Sindre Sorhus\\
│  ├─ email: sindresorhus@gmail.com\\
│  ├─ url: https://sindresorhus.com\\
│  ├─ path: ./Frontend/node\_modules/@isaacs/cliui/node\_modules/wrap-ansi\\
│  └─ licenseFile: ./Frontend/node\_modules/@isaacs/cliui/node\_modules/wrap-ansi/license\\
├─ wrappy@1.0.2\\
│  ├─ licenses: ISC\\
│  ├─ repository: https://github.com/npm/wrappy\\
│  ├─ publisher: Isaac Z. Schlueter\\
│  ├─ email: i@izs.me\\
│  ├─ url: http://blog.izs.me/\\
│  ├─ path: ./Frontend/node\_modules/wrappy\\
│  └─ licenseFile: ./Frontend/node\_modules/wrappy/LICENSE\\
├─ ws@8.11.0\\
│  ├─ licenses: MIT\\
│  ├─ repository: https://github.com/websockets/ws\\
│  ├─ publisher: Einar Otto Stangvik\\
│  ├─ email: einaros@gmail.com\\
│  ├─ url: http://2x.io\\
│  ├─ path: ./Frontend/node\_modules/ws\\
│  └─ licenseFile: ./Frontend/node\_modules/ws/LICENSE\\
├─ ws@8.17.0\\
│  ├─ licenses: MIT\\
│  ├─ repository: https://github.com/websockets/ws\\
│  ├─ publisher: Einar Otto Stangvik\\
│  ├─ email: einaros@gmail.com\\
│  ├─ url: http://2x.io\\
│  ├─ path: ./Frontend/node\_modules/webpack-dev-server/node\_modules/ws\\
│  └─ licenseFile: ./Frontend/node\_modules/webpack-dev-server/node\_modules/ws/LICENSE\\
├─ y18n@5.0.8\\
│  ├─ licenses: ISC\\
│  ├─ repository: https://github.com/yargs/y18n\\
│  ├─ publisher: Ben Coe\\
│  ├─ email: bencoe@gmail.com\\
│  ├─ path: ./Frontend/node\_modules/y18n\\
│  └─ licenseFile: ./Frontend/node\_modules/y18n/LICENSE\\
├─ yallist@3.1.1\\
│  ├─ licenses: ISC\\
│  ├─ repository: https://github.com/isaacs/yallist\\
│  ├─ publisher: Isaac Z. Schlueter\\
│  ├─ email: i@izs.me\\
│  ├─ url: http://blog.izs.me/\\
│  ├─ path: ./Frontend/node\_modules/yallist\\
│  └─ licenseFile: ./Frontend/node\_modules/yallist/LICENSE\\
├─ yallist@4.0.0\\
│  ├─ licenses: ISC\\
│  ├─ repository: https://github.com/isaacs/yallist\\
│  ├─ publisher: Isaac Z. Schlueter\\
│  ├─ email: i@izs.me\\
│  ├─ url: http://blog.izs.me/\\
│  ├─ path: ./Frontend/node\_modules/semver/node\_modules/yallist\\
│  └─ licenseFile: ./Frontend/node\_modules/semver/node\_modules/yallist/LICENSE\\
├─ yargs-parser@20.2.9\\
│  ├─ licenses: ISC\\
│  ├─ repository: https://github.com/yargs/yargs-parser\\
│  ├─ publisher: Ben Coe\\
│  ├─ email: ben@npmjs.com\\
│  ├─ path: ./Frontend/node\_modules/karma/node\_modules/yargs-parser\\
│  └─ licenseFile: ./Frontend/node\_modules/karma/node\_modules/yargs-parser/LICENSE.txt\\
├─ yargs-parser@21.1.1\\
│  ├─ licenses: ISC\\
│  ├─ repository: https://github.com/yargs/yargs-parser\\
│  ├─ publisher: Ben Coe\\
│  ├─ email: ben@npmjs.com\\
│  ├─ path: ./Frontend/node\_modules/yargs-parser\\
│  └─ licenseFile: ./Frontend/node\_modules/yargs-parser/LICENSE.txt\\
├─ yargs@16.2.0\\
│  ├─ licenses: MIT\\
│  ├─ repository: https://github.com/yargs/yargs\\
│  ├─ path: ./Frontend/node\_modules/karma/node\_modules/yargs\\
│  └─ licenseFile: ./Frontend/node\_modules/karma/node\_modules/yargs/LICENSE\\
├─ yargs@17.7.2\\
│  ├─ licenses: MIT\\
│  ├─ repository: https://github.com/yargs/yargs\\
│  ├─ path: ./Frontend/node\_modules/yargs\\
│  └─ licenseFile: ./Frontend/node\_modules/yargs/LICENSE\\
├─ yocto-queue@1.0.0\\
│  ├─ licenses: MIT\\
│  ├─ repository: https://github.com/sindresorhus/yocto-queue\\
│  ├─ publisher: Sindre Sorhus\\
│  ├─ email: sindresorhus@gmail.com\\
│  ├─ url: https://sindresorhus.com\\
│  ├─ path: ./Frontend/node\_modules/yocto-queue\\
│  └─ licenseFile: ./Frontend/node\_modules/yocto-queue/license\\
└─ zone.js@0.14.5\\
├─ licenses: MIT\\
├─ repository: https://github.com/angular/angular\\
├─ publisher: Brian Ford\\
├─ path: ./Frontend/node\_modules/zone.js\\
└─ licenseFile: ./Frontend/node\_modules/zone.js/LICENSE\\


\end{document}
