\chapter{Reflektion Lernfortschritt}\label{ch:reflektion-lernfortschritt}


\section{Max Trautwein}


\section{Ayhan Yasar}


\section{Tobias Bührle}
Der größte Lernfortschritt in diesem Projekt und auch die größte Hürde, war für mich das
Kennen, Verstehen und Anwenden Lernen der verschiedenen Frameworks. Denn mit den
Basissprachen Java, TypeScript, HTML und CSS hatte ich zwar bereits Vorerfahrungen aus dem
Studium, allerdings habe ich zuvor noch nie wirklich mit derartigen Frameworks wie
Springboot, Angular oder Bootstrap gearbeitet. Es hat einige Zeit gebraucht, mich in die
Feinheiten jedes dieser Frameworks einzuarbeiten und ihre Funktionsweise zu verstehen.
Allerdings war dies dennoch eine gute Erfahrung für mich, da ich mich beim
Lernen dieser Sprachen schon des Öfteren gefragt habe, wie man nur damit vollständige
Applikationen aufbauen kann. Das Verwenden dieser Frameworks gab mir die Antwort darauf.
Auch die Verwendung von LaTeX als Tool zur Dokumentation im wissenschaftlichen Stil
musste ich erst verinnerlichen, worüber ich jedoch sehr froh bin, da ich davon ausgehe,
dass mir dies bei zukünftigen Projektarbeiten und vor allem meiner Bachelorthesis 
zugutekommen wird. Ein weiterer Teilbereich, in dem ich mein Wissen deutlich erweitern 
konnte, war das Deployment einer Applikation mithilfe von Docker. Da meine 
Entwicklungsumgebung
auf Windows läuft und der Windows-Support von Docker eher mangelhaft ist, gab es zwar
einige anfängliche Probleme, was jedoch dazu führte, dass ich mich deutlich mehr
mit der Funktionsweise von Docker beschäftigen musste und damit deutlich mehr darüber
gelernt habe. Auch beim Thema Authentifizierung und IT-Security habe ich vieles dazu
gelernt. Auch wenn ich bei der Implementierung der Nutzerauthentifizierung kaum beteiligt
war, so habe dennoch einiges über das Anfragen von geschützten backend Daten im
Frontend gelernt. Zu guter Letzt wäre da noch das Designen von \ac{UI} Elementen im
Frontend. Im letzten Semester habe ich bereits die Grundlagen des UI Design anhand der
Prototyping Software Figma gelernt, mit der auch unser erster \ac{UI} Prototyp entstanden
ist. Dieses Design nun jedoch mithilfe von Bootstrap und eigenem HTML und CSS Code
umzusetzen war eine größere Herausforderung, als ich zu Beginn annahm. Daher bin ich froh
über diese Erfahrung, da ich jetzt deutlich besser einschätzen kann, ob gewisse
Designentscheidungen überhaupt umsetzbar sind. Auch Probleme, welche in Figma keine Rolle
spielen, wie etwa das Skalieren von Größen bei veränderter Größe des Browserfensters sind
mir durch dieses Projekt erst bewusst geworden.

Alles in allem würde ich meinen Lernfortschritt also als hoch und einer Veranstaltung mit
10 \ac{ECTS} angemessen einstufen.