\chapter{Reflektion Lernfortschritt}\label{ch:reflektion-lernfortschritt}


\section{Max Trautwein}
Dieses Projekt hat mir die Gelegenheit geboten, mich vielen verscheidenden Technologien zu beschäftigen,
welche ich sonst vermutlich nicht so früh oder gar nicht in Betracht gezogen hätte.
Vor dem Projekt hatte ich hauptsächlich negative Erfahrungen mit Java gemacht,
jedoch verstehe ich nun, mit einer aktuellen Java Version sowie Spring Boot den Anreiz und nutzen.
Ich habe gelernt, wie ich eine OpenAPI Spezifikation erstelle und mithilfe der
ZalandoAPI Guidelines eine klare, sinnvolle und verständliche Schnittstelle definieren kann.
Mithilfe der OpenAPI Spezifikation und Generierung der Schnittstellen im Frontend sowie Backend
konnte neben der Zeitersparnisse auch Fehler vermieden werden.
Die MongoDB, welche wir im Backend zur Datenspeicherung nutzen, hat mir erlaubt,
erste Einblicke in NoSQL Datenbanken zu gewinnen.
Im Frontendbereich habe ich Angular sowie TypeScript detaillierter gelernt und fühle mir jetzt sicher,
diese in zukünftigen Projekten einzusetzen, um ein Frontend strukturierter und übersichtlicher zu gestalten.
Insbesondere habe ich einiges mehr Verständnis zu Async Funktionen für reaktive Funktionalitäten
unter Verwendung der RxJS Bibliothek erlangt.
Bei der Dokumentation habe ich mich näher mit LaTeX beschäftigt sowie auch eigene Funktionen definiert,
welche die Dokumentation erleichtert haben und zur Übersichtlichkeit beigetragen haben.
Selbst bei dem Deployment konnte ich neues Lernen, zum Beispiel habe ich von der Existenz und Bedeutung von Inodes gelernt,
welche vom Dateisystem benötigt werden, um Daten zu speichern, als diese auf der BW-Cloud Instanz ausgegangen sind.

Mich freut es, dass ich neben der ganzen Arbeit, welche dieses Projekt generiert hat,
viele verschiedenen Dinge im Detail lernen konnte.


\section{Ayhan Yasar}

In diesem Projekt habe ich viel gelernt. Da ich bisher noch nie an einem so großen Projekt beteiligt war, gab es enorme Hürden zu überwinden. 
Kenntnisse fehlten mir im Vergleich zu meinem Team, weshalb ich deutlich mehr aufzuholen hatte – mit Erfolg. 
Anfangs war ich übervorsichtig und habe zu viel Zeit in Recherche investiert, um möglichst fehlerfreie Commits zu tätigen. 
Im Laufe des Projekts wurde ich immer schneller und konnte mich langsam, aber sicher dem Tempo des Teams anpassen.

Zuvor hatte ich nie in Angular programmiert, aber dank meines Teams und zahlreicher Tutorials habe ich mir nun ein großes Verständnis angeeignet. 
Da sich die Struktur in Angular immer wieder ändert, musste ich intensiver recherchieren, da die meisten Tutorials veraltete Funktionen nutzten. 
Mein Verständnis für HTML und CSS ist deutlich sicherer geworden und basiert nicht mehr auf dem Versuch-und-Irrtum-Prinzip. 
Mit meinem jetzigen Know-how kann ich problemlos einen funktionsfähigen Login-Service und komplexe TypeScript-Module in ein Projekt integrieren.

Während des Projekts habe ich ein größeres Verständnis im Backend und in Datenbanken erlangt. 
Vor dem Projekt konnte ich Datenbanken und Backend-Code nicht auf Anhieb nachvollziehen. 
Unsere Dokumentation wurde mit LaTeX erstellt, wovon ich bisher nur gehört, mich aber nie aktiv damit beschäftigt hatte. 
Die Einbindung in Visual Studio Code war mühselig, und es kompiliert bisher immer noch nicht einwandfrei, was mich jedoch nicht daran hindert, weiterhin mit LaTeX zu arbeiten. 
Da LaTeX auch bei der Bachelorarbeit infrage kommt, war es sehr bereichernd, vorab damit zu arbeiten.

Alles in allem würde ich meinen Lernfortschritt als enorm hoch einstufen, da ich so viel nachzuholen hatte, um aktiv am Projekt zu arbeiten.

\section{Tobias Bührle}
Der größte Lernfortschritt in diesem Projekt und auch die größte Hürde, war für mich das
Kennen, Verstehen und Anwenden Lernen der verschiedenen Frameworks. Denn mit den
Basissprachen Java, TypeScript, HTML und CSS hatte ich zwar bereits Vorerfahrungen aus dem
Studium, allerdings habe ich zuvor noch nie wirklich mit derartigen Frameworks wie
Springboot, Angular oder Bootstrap gearbeitet. Es hat einige Zeit gebraucht, mich in die
Feinheiten jedes dieser Frameworks einzuarbeiten und ihre Funktionsweise zu verstehen.
Allerdings war dies dennoch eine gute Erfahrung für mich, da ich mich beim
Lernen dieser Sprachen schon des Öfteren gefragt habe, wie man nur damit vollständige
Applikationen aufbauen kann. Das Verwenden dieser Frameworks gab mir die Antwort darauf.
Auch die Verwendung von LaTeX als Tool zur Dokumentation im wissenschaftlichen Stil
musste ich erst verinnerlichen, worüber ich jedoch sehr froh bin, da ich davon ausgehe,
dass mir dies bei zukünftigen Projektarbeiten und vor allem meiner Bachelorthesis 
zugutekommen wird. Ein weiterer Teilbereich, in dem ich mein Wissen deutlich erweitern 
konnte, war das Deployment einer Applikation mithilfe von Docker. Da meine 
Entwicklungsumgebung
auf Windows läuft und der Windows-Support von Docker eher mangelhaft ist, gab es zwar
einige anfängliche Probleme, was jedoch dazu führte, dass ich mich deutlich mehr
mit der Funktionsweise von Docker beschäftigen musste und damit deutlich mehr darüber
gelernt habe. Auch beim Thema Authentifizierung und IT-Security habe ich vieles dazu
gelernt. Auch wenn ich bei der Implementierung der Nutzerauthentifizierung kaum beteiligt
war, so habe dennoch einiges über das Anfragen von geschützten backend Daten im
Frontend gelernt. Zu guter Letzt wäre da noch das Designen von \ac{UI} Elementen im
Frontend. Im letzten Semester habe ich bereits die Grundlagen des UI Design anhand der
Prototyping Software Figma gelernt, mit der auch unser erster \ac{UI} Prototyp entstanden
ist. Dieses Design nun jedoch mithilfe von Bootstrap und eigenem HTML und CSS Code
umzusetzen war eine größere Herausforderung, als ich zu Beginn annahm. Daher bin ich froh
über diese Erfahrung, da ich jetzt deutlich besser einschätzen kann, ob gewisse
Designentscheidungen überhaupt umsetzbar sind. Auch Probleme, welche in Figma keine Rolle
spielen, wie etwa das Skalieren von Größen bei veränderter Größe des Browserfensters sind
mir durch dieses Projekt erst bewusst geworden.

Alles in allem würde ich meinen Lernfortschritt also als hoch und einer Veranstaltung mit
10 \ac{ECTS} angemessen einstufen.