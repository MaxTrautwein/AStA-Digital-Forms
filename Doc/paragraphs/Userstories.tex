\chapter{Userstories}\label{ch:userstories}
Im Folgenden werden die Anwendungsfälle dieser Applikation anhand von Userstories präzisiert.
\section{Dockerized}
Als Kunde möchte ich, dass die Applikation in Docker Containern bereitgestellt werden kann, damit 
sichergestellt ist, dass die Anwendung in verschiedenen Umgebungen konsistent funktioniert.

\section{Technologie}
Als Kunde möchte ich, dass die Applikation auf bewährten Technologien basiert, damit eine hohe
Zuverlässigkeit und Stabilität, sowie eine gute Wartbarkeit gewährleistet ist.
\section{Datenbank}
Als Nutzer möchte ich, dass ich entscheiden kann, ob meine Daten in einer Datenbank der Applikation 
oder an einem anderen Ort gespeichert werden.
\section{Bedienung/Layout}
Als Nutzer möchte ich, dass die Bedienung der Applikation keine Fachkenntnisse voraussetzt, damit 
ich mich nicht lange einarbeiten muss.
\section{Persistenz}
Als Nutzer möchte ich, dass der Ausfüllfortschritt eines Antrags, geräteübergreifend gespeichert wird,
um es mir so zu ermöglichen, das Ausfüllen an einem anderen Gerät fortzusetzen.
\section{Login}
Als Nutzer möchte ich mich in der Applikation mit einem Nutzernamen bzw. E-Mail und einem Passwort
einloggen können, damit persönliche Daten gut geschützt sind.
\section{Antrag finden}
Als Nutzer möchte ich Zugriff auf eine Funktionalität haben, die mir hilft, den passenden 
Antrag zu finden, um so Zeit zu sparen und Verwechslungen zu vermeiden.
\section{Antragsbeschreibung}
Als Nutzer möchte ich, dass die Anträge neben einem Namen, auch über eine Beschreibung verfügen, damit
ich als Nutzer den Zweck des Antrags besser verstehen kann.
\section{Filter}
Als Nutzer möchte ich die Möglichkeit haben, Anträge nach bestimmten Kriterien zu filtern, 
damit ich die relevanten Anträge schneller finden kann.
\section{Antrags Kategorien}
Als Kunde möchte ich in der Lage sein, Anträge bestimmten Kategorien wie Beantragungen und 
Abrechnungen zuteilen zu können, um so Anträge besser organisieren und schneller auf benötigte 
Informationen zugreifen zu können.
\section{Favoriten}
Als Nutzer möchte ich in der Lage sein, Anträge als Favoriten zu markieren, um sie so schneller zu 
finden und schneller auf sie zugreifen zu können.
\section{Fertige Anträge}
Als Nutzer möchte ich in der Lage sein, von mir bereits fertig ausgefüllte Anträge zu überarbeiten, um 
so Zeit zu sparen.
\section{Automatisierung}
Als Nutzer möchte ich die Möglichkeit haben, die von mir hinterlegten persönlichen Daten zum 
automatischen Ausfüllen eines Antrags zu nutzen, um so viel Tipparbeit zu sparen.
\section{Überschreriben generierter Inhalte}
Als Nutzer möchte ich die Möglichkeit haben, von der Applikation automatisch eingegebene Informationen 
zu überschreiben, damit ich den Antrag möglichst effizient an meine Bedürfnisse anpassen kann.
\section{Import von Antragsdaten}
Als Nutzer möchte ich die Möglichkeit haben, Daten von einem Antrag in den jeweiligen 
Abrechnungsantrag zu übernehmen, um so Zeit und Arbeit zu sparen.
\section{Formular Aufbau}
Als Kunde möchte ich, dass die Anträge dynamisch generiert werden, um sie so schnell und 
unkompliziert anpassen zu können. Selbiges gilt für die Antragsbearbeitungs-Funktionalität.
\section{Vollständigkeitskontrolle}
Als Nutzer möchte ich, dass die Applikation mir Hinweise bezüglich fehlenden Unterlagen bzw. 
Informationen gibt, um das Einreichen eines fehlerhaften Antrags zu vermeiden.
\section{Anhangssystem}
Als Nutzer möchte ich die Möglichkeit haben, einem Antrag einen Anhang hinzuzufügen, um benötigte, 
zusätzliche Informationen bereitzustellen.
\section{Lieferschein}
Als Nutzer möchte ich die Möglichkeit haben, einen Lieferschein zu generieren, um sicherzustellen, 
dass alle erforderlichen Dokumente im Anhang vorhanden sind.
\section{Anhangsgenerator}
Als Nutzer möchte ich, dass bestimmte Informationen aus dem Antrag direkt dem Anhang hinzugefügt 
werden um eine besser Übersicht über den Antrag zu erhalten.
\section{PDF Generator}
Als Nutzer möchte ich in der Lage sein, den fertigen Antrag als PDF abspeichern zu können, damit man 
ihn ausdrucken und abheften kann.
\section{Routenberechnung}
Als Nutzer möchte ich Zugriff auf eine Funktionalität haben, welche mir die kürzeste Route zwischen 
mehreren Adressen anzeigt, um so die Routenplanung zu beschleunigen.
\section{Adressvervollständigung}
Als Nutzer möchte ich, dass beim Hinterlegen von Adressen in der Applikation, eine 
Adressvervollständigung mich bei der Eingabe unterstützt, um so Zeit zu sparen und die Richtigkeit der 
Adresse sicherzustellen.
\section{Reisekosten Helfer}
Als Nutzer möchte ich Zugriff auf eine Funktionalität haben, welche mir berechnet, wie viel für eine 
Reise erstattet wird. So kann ich sicherstellen, dass der Betrag für die Reise angemessen ist.