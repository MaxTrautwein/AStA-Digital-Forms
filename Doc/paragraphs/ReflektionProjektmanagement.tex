\chapter{Reflektion Projektmanagement}\label{ch:reflektion-projektmanagement}


\section{Max Trautwein}

Ich hatte zwar breites Erfahrung mit Teamarbeiten sowie dem Arbeiten in Teams aus dem Berufsalltag,
jedoch ist dieses Projekt das erste, in welchen ich auch Projektmanagement Tätlichkeiten übernehmen musste.
Letztendlich hatte ich zu Beginn des Projekts die Leistungsfähigkeit meines Teams falsch eingeschätzt und
damit mehr Requirements mit aufgenommen als realistisch war.
Ein weiterer Faktor, welchen ich nicht so gut eingeschätzt hatte, waren die Randbedingungen um das Projekt herum.
Zum einen sind da die verpflichtenden Abgaben, welche mehr Zeit in Anspruch genommen haben, als ich gedacht hätten
zum andren unsere Verpflichtungen an andere Module mit deren Laboren und Berichten.
Diese haben die verfügbare Zeit für die tatsächliche Entwicklung erheblich reduziert.\\
Ich bin überzeugt davon, dass ich in der Zukunft dies besser realisieren kann.

Rückblickend habe ich das Gefühl, dass die Beiträge in Zeitaufwand sowie Ergebnis nicht gleichmäßig verteilt sind,
obwohl es nie an Tätigkeiten gemangelt hat.
Ich bin mir unsicher, ob ich das nennenswert beeinflussen konnte.

\section{Ayhan Yasar}


\section{Tobias Bührle}
In Anbetracht der Tatsache, dass dies unser erstes Projekt in dieser Größenordnung war,
empfinde ich unser Projektmanagement als angemessen. Die Kommunikation über die jeweiligen
Kanäle wie Discord oder GitHub funktionierte weitestgehend ohne Probleme. Auch das
Arbeiten mit Git als Versionstool verlief nach kurzer Einarbeitungsphase ohne allzu große
Schwierigkeiten. Es war zu den allermeisten Zeiten klar ersichtlich, wer gerade an welchem
Feature arbeitet, sodass Transparenz gegeben war. Auch die Aufgabenverteilung in jedem
Sprint verlief ohne nennenswerte Konflikte und war immer harmonisch. Probleme gab es
allerdings immer wieder beim rechtzeitigen Fertigstellen von Features. Dies lässt sich
jedoch durch unsere mangelnde Erfahrung in derartigen Projektarbeiten begründen, welche
des Öfteren dazu führte, dass wir uns bei der Einschätzung von Zeitaufwand und Komplexität
irrten. Da wir in diesen Bereichen jedoch deutlich dazu gelernt haben, nahm dieses Problem
im Verlauf des Projekts ab.

