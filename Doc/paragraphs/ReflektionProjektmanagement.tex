\chapter{Reflektion Projektmanagement}\label{ch:reflektion-projektmanagement}


\section{Max Trautwein}


\section{Ayhan Yasar}


\section{Tobias Bührle}
In Anbetracht der Tatsache, dass dies unser erstes Projekt in dieser Größenordnung war,
empfinde ich unser Projektmanagement als angemessen. Die Kommunikation über die jeweiligen
Kanäle wie Discord oder GitHub funktionierte weitestgehend ohne Probleme. Auch das
Arbeiten mit Git als Versionstool verlief nach kurzer Einarbeitungsphase ohne allzu große
Schwierigkeiten. Es war zu den allermeisten Zeiten klar ersichtlich, wer gerade an welchem
Feature arbeitet, sodass Transparenz gegeben war. Auch die Aufgabenverteilung in jedem
Sprint verlief ohne nennenswerte Konflikte und war immer harmonisch. Probleme gab es
allerdings immer wieder beim rechtzeitigen Fertigstellen von Features. Dies lässt sich
jedoch durch unsere mangelnde Erfahrung in derartigen Projektarbeiten begründen, welche
des Öfteren dazu führte, dass wir uns bei der Einschätzung von Zeitaufwand und Komplexität
irrten. Da wir in diesen Bereichen jedoch deutlich dazu gelernt haben, nahm dieses Problem
im Verlauf des Projekts ab.

