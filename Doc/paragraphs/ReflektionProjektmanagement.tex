\chapter{Reflektion Projektmanagement}\label{ch:reflektion-projektmanagement}


\section{Max Trautwein}


\section{Ayhan Yasar}
Mein Hauptziel war die Entwicklung einer neuen, benutzerfreundlichen Möglichkeit AStA Formulare auszufüllen, die sowohl effizient als auch optisch ansprechend ist. 
Ich habe dieses Ziel erreicht, jedoch musste ich feststellen, dass ich langsamer als mein Team arbeite und musste mich doppelt so viel anstrengen, um mithalten zu können.
Die Planung des Projekts unter Scrum verlief immer pünktlich und es wurde engmaschig mit dem Team gearbeitet. Leider kam es ein paar Mal vor, dass ich die Frist eines Sprints nicht einhalten konnte, 
jedoch gab es meistens nur eine Verzögerung von 1-2 Tagen. 
Risiken wurden frühzeitig identifiziert und gemanagt, jedoch trat ein unerwartetes Problem mit der Integration eines Plugins auf.
Insgesamt erfüllt das Projekt die Qualitätsanforderungen und wurde von den Stakeholdern positiv bewertet. Die Benutzerfreundlichkeit und das Design entsprechen den Erwartungen, und die Rückmeldungen der Kunden sind positiv.


\section{Tobias Bührle}

