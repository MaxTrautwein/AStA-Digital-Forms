\chapter{Lizenzen}\label{ch:lizenzen}

In diesem Kapitel finden sich Informationen zu den verwendeten Lizenzen in dem Projekt.

\section{MIT}\label{sec:mit}
Die MIT Lizenz ist ein kurze und offene Open Source Lizenz mit nur minimalen Bedingungen.

\paragraph{Rechte}
\begin{itemize}
    \item kommerzielle Nutzung
    \item Weitergabe
    \item Anpassung
    \item private Nutzung und Modifikation
\end{itemize}

\paragraph{Bedingungen}
\begin{itemize}
    \item Die Lizenz und Urheberrechte müssen mit verteilt werden.
\end{itemize}

\paragraph{Limitierung}
\begin{itemize}
    \item Haftungsausschluss
    \item keinerlei Garantie
\end{itemize}

\section{Apache-2.0}\label{sec:apache-2.0}

Die Apache-2.0 ist eine Open Source Lizenz welche Nutzer vor Patentrechten schützt,
explizit Rechte auf Warenzeichen ausschließt und fordert, dass Änderungen dokumentiert werden.

\paragraph{Rechte}
\begin{itemize}
    \item kommerzielle Nutzung
    \item Weitergabe
    \item Anpassung
    \item Patent Nutzung % Nach meinem Verständnis schützt es vor Royalty Zahlungs Anforderungen
    \item private Nutzung und Modifikation
\end{itemize}

\paragraph{Bedingungen}
\begin{itemize}
    \item Die Lizenz und Urheberrechte müssen mit verteilt werden.
    \item Änderungen müssen dokumentiert werden.
\end{itemize}

\paragraph{Limitierung}
\begin{itemize}
    \item Haftungsausschluss
    \item keinerlei Garantie
    \item explizit keine Rechte auf Warenzeichen
\end{itemize}

\section{The PostgreSQL Licence}\label{sec:the-postgresql-licence}
%https://opensource.org/license/postgresql
% ähnlich MIT aber nicht Exakt
Eine eigene Open Source Lizenz von PostgreSQL.
Diese ähnelt der MIT Lizenz, ist jedoch nicht gleich im Wortlaut.

\paragraph{Rechte}
\begin{itemize}
    \item kommerzielle Nutzung
    \item Weitergabe
    \item Anpassung
    \item private Nutzung und Modifikation
\end{itemize}

\paragraph{Bedingungen}
\begin{itemize}
    \item Die Lizenz und Urheberrechte müssen mit verteilt werden.
\end{itemize}

\paragraph{Limitierung}
\begin{itemize}
    \item Haftungsausschluss
    \item keinerlei Garantie
\end{itemize}


\section{\acf{SSPL}}\label{sec:server-side-public-license}
Bei dieser Lizenz handelt es sich um eine modifizierte GNU AGPLv3.
Durch diese Anpassung wird die \ac{SSPL} nicht mehr als Open Source angesehen.\cite{osi-sspl}

\paragraph{Rechte}
\begin{itemize}
    \item kommerzielle Nutzung
    \item Weitergabe
    \item Anpassung
    \item Patent Nutzung
    \item private Nutzung und Modifikation
\end{itemize}
\paragraph{Bedingungen}
\begin{itemize}
    \item Anpassungen müssen unter derselben Lizenz bereitgestellt werden
    \item Änderungen Dokumentieren
    \item Die Lizenz und Urheberrechte müssen mit verteilt werden.
    \item Quellcode muss bei Weitergabe offengelegt werden
    \item wenn MongoDB als Service angeboten wird, dann muss der gesamte Quellcode unter der
    \ac{SSPL} frei zur Verfügung gestellt werden.
    Dies umfasst auch jegliche andere Software, welche benötigt wird, sodass ein Nutzer denselben Dienst anbieten kann.
\end{itemize}

\paragraph{Limitierung}
\begin{itemize}
    \item Haftungsausschluss
    \item keinerlei Garantie
\end{itemize}