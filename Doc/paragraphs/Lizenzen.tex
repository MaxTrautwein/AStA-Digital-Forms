\chapter{Lizenzen}

In diesem Kapitel finden sich informationen zu den Verwendeten Lizensen in dem Projekt.
% Kurtzbeschribungen hinzufügen und ausbauen

\section{MIT}\label{sec:mit}
Die MIT Lizenz ist ein kurze und offene Lizenz mit nur minimalen Bedingungen.

\paragraph{Rechte}
\begin{itemize}
    \item kommerzielle Nutzung
    \item Weitergabe
    \item Anpassung
    \item private Nutzung und Modifikation
\end{itemize}

\paragraph{Bedingungen}
\begin{itemize}
    \item Die Lizenz und Urheberrechte müssen mit verteilt werden.
\end{itemize}

\paragraph{Limitierung}
\begin{itemize}
    \item Haftungsausschluss
    \item keinerlei Garantie
\end{itemize}

\section{Apache-2.0}\label{sec:apache-2.0}

\paragraph{Rechte}
\begin{itemize}
    \item kommerzielle Nutzung
    \item Weitergabe
    \item Anpassung
    \item Patent Nutzung % Nach meinem Vertändniss schützt es vor Royalty Zahlungs anvorderungen
    \item private Nutzung und Modifikation
\end{itemize}

\paragraph{Bedingungen}
\begin{itemize}
    \item Die Lizenz und Urheberrechte müssen mit verteilt werden.
    \item Änderungen müssen dokumentiert werden.
\end{itemize}

\paragraph{Limitierung}
\begin{itemize}
    \item Haftungsausschluss
    \item keinerlei Garantie
    \item explizit keine rechte auf Warenzeichen
\end{itemize}

\section{The PostgreSQL Licence}\label{sec:the-postgresql-licence}
%https://opensource.org/license/postgresql
% ähnlich MIT aber nicht Exakt

\paragraph{Rechte}
\begin{itemize}
    \item kommerzielle Nutzung
    \item Weitergabe
    \item Anpassung
    \item private Nutzung und Modifikation
\end{itemize}

\paragraph{Bedingungen}
\begin{itemize}
    \item Die Lizenz und Urheberrechte müssen mit verteilt werden.
\end{itemize}

\paragraph{Limitierung}
\begin{itemize}
    \item Haftungsausschluss
    \item keinerlei Garantie
\end{itemize}


\section{Server Side Public License}\label{sec:server-side-public-license}
Bei dieser Lizenz handelt es sich um eine Modifizierte GNU AGPLv3.
Durch diese Anpassung wird die SSPL nicht mehr als OpenSource angesehen.\cite{osi-sspl}

\paragraph{Rechte}
\begin{itemize}
    \item kommerzielle Nutzung
    \item Weitergabe
    \item Anpassung
    \item Patent Nutzung
    \item private Nutzung und Modifikation
\end{itemize}
\paragraph{Bedingungen}
\begin{itemize}
    \item Anpassungen müssen unter derselben Lizenz bereitgestellt werden
    \item Änderungen Dokumentieren
    \item Die Lizenz und Urheberrechte müssen mit verteilt werden.
    \item Quellcode muss bei weitergabe offengelegt werden
    \item wenn MongoDB als Service angeboten wird, dann muss der gesamte Quellcode unter der
    SSPL frei zur verfügung gestellt werden.
    Dies umfasst auch alle andere Software, welche benötigt wird, sodass ein nutzer denselben Dienst anbieten kann.
\end{itemize}

\paragraph{Limitierung}
\begin{itemize}
    \item Haftungsausschluss
    \item keinerlei Garantie
\end{itemize}