\chapter{Lizenzen}\label{ch:lizenzen}

In diesem Kapitel finden sich Informationen zu den verwendeten Lizenzen in dem Projekt.

\section{MIT}\label{sec:mit}
Die MIT Lizenz ist ein kurze und offene Open Source Lizenz mit nur minimalen Bedingungen.

\paragraph{Rechte}
\begin{itemize}
    \item kommerzielle Nutzung
    \item Weitergabe
    \item Anpassung
    \item private Nutzung und Modifikation
\end{itemize}

\paragraph{Bedingungen}
\begin{itemize}
    \item Die Lizenz und Urheberrechte müssen mit verteilt werden.
\end{itemize}

\paragraph{Limitierung}
\begin{itemize}
    \item Haftungsausschluss
    \item keinerlei Garantie
\end{itemize}
Vgl. \cite{choosealicense-com}

\section{Apache-2.0}\label{sec:apache-2.0}

Die Apache-2.0 ist eine Open Source Lizenz welche Nutzer vor Patentrechten schützt,
explizit Rechte auf Warenzeichen ausschließt und fordert, dass Änderungen dokumentiert werden.

\paragraph{Rechte}
\begin{itemize}
    \item kommerzielle Nutzung
    \item Weitergabe
    \item Anpassung
    \item Patent Nutzung % Nach meinem Verständnis schützt es vor Royalty Zahlungs Anforderungen
    \item private Nutzung und Modifikation
\end{itemize}

\paragraph{Bedingungen}
\begin{itemize}
    \item Die Lizenz und Urheberrechte müssen mit verteilt werden.
    \item Änderungen müssen dokumentiert werden.
\end{itemize}

\paragraph{Limitierung}
\begin{itemize}
    \item Haftungsausschluss
    \item keinerlei Garantie
    \item explizit keine Rechte auf Warenzeichen
\end{itemize}
Vgl. \cite{choosealicense-com}

\section{The PostgreSQL Licence}\label{sec:the-postgresql-licence}
%https://opensource.org/license/postgresql
% ähnlich MIT aber nicht Exakt
Eine eigene Open Source Lizenz von PostgreSQL.
Diese ähnelt der MIT Lizenz, ist jedoch nicht gleich im Wortlaut.

\paragraph{Rechte}
\begin{itemize}
    \item kommerzielle Nutzung
    \item Weitergabe
    \item Anpassung
    \item private Nutzung und Modifikation
\end{itemize}

\paragraph{Bedingungen}
\begin{itemize}
    \item Die Lizenz und Urheberrechte müssen mit verteilt werden.
\end{itemize}

\paragraph{Limitierung}
\begin{itemize}
    \item Haftungsausschluss
    \item keinerlei Garantie
\end{itemize}


\section{\acf{SSPL}}\label{sec:server-side-public-license}
Bei dieser Lizenz handelt es sich um eine modifizierte GNU AGPLv3.
Durch diese Anpassung wird die \ac{SSPL} nicht mehr als Open Source angesehen.\cite{osi-sspl}

\paragraph{Rechte}
\begin{itemize}
    \item kommerzielle Nutzung
    \item Weitergabe
    \item Anpassung
    \item Patent Nutzung
    \item private Nutzung und Modifikation
\end{itemize}
\paragraph{Bedingungen}
\begin{itemize}
    \item Anpassungen müssen unter derselben Lizenz bereitgestellt werden
    \item Änderungen Dokumentieren
    \item Die Lizenz und Urheberrechte müssen mit verteilt werden.
    \item Quellcode muss bei Weitergabe offengelegt werden
    \item wenn MongoDB als Service angeboten wird, dann muss der gesamte Quellcode unter der
    \ac{SSPL} frei zur Verfügung gestellt werden.
    Dies umfasst auch jegliche andere Software, welche benötigt wird, sodass ein Nutzer denselben Dienst anbieten kann.
\end{itemize}

\paragraph{Limitierung}
\begin{itemize}
    \item Haftungsausschluss
    \item keinerlei Garantie
\end{itemize}
Vgl. \cite{choosealicense-com}

\section{GNU GPLv3}\label{sec:gnu-gplv3}
Die GNU GPLv3 Lizenz ist eine Weiterentwicklung der GNU GPLv2 Lizenz. In ihr wurden Schwachstellen der
GPLv2 Lizenz im Bezug auf Patentrechte behoben und eine Kompartibilität mit der weit verbreiteten
Apache-2.0 Lizenz integriert.

\paragraph{Rechte}
\begin{itemize}
    \item kommerzielle Nutzung
    \item Weitergabe
    \item Anpassung
    \item private Nutzung und Modifikation
    \item Patent Nutzung
\end{itemize}
\paragraph{Bedingungen}
\begin{itemize}
    \item Bei Veröffentlichung muss der Quellcode frei zugänglich sein.
    \item Die Lizenz und Urheberrechte müssen mit verteilt werden.
    \item Änderungen müssen dokumentiert werden.
    \item Änderungen müssen unter der selben Lizenz veröffentlicht werden.
\end{itemize}

\paragraph{Limitierung}
\begin{itemize}
    \item keinerlei Garantie
    \item eingeschränkte Haftung
\end{itemize}
Vgl. \cite{choosealicense-com}

\section{0BSD (Zero-Clause BSD)}\label{sec:0bsd-(zero-clause-bsd)}
Eine sehr offene Lizenz, welche die Haftung sowie Garantie der Entwickler ausschließt.
Keinerlei Bedingungen dafür.

\paragraph{Rechte}
\begin{itemize}
    \item Kopieren
    \item Anpassung
    \item Weitergabe
\end{itemize}
\paragraph{Bedingungen}
Keine.

\paragraph{Limitierung}
\begin{itemize}
    \item Haftungsausschluss
    \item keinerlei Garantie
\end{itemize}

Vgl. \cite{bsd-0-clause}

\section{BSD-3-Clause}\label{sec:bsd-3-clause}
Eine erweiterung der BSD-2 Lizenz mit dem expliziten zusatz, welcher Werbung sowie Befürwortungen
mit den Namen der Autoren untersagt.

\paragraph{Rechte}
\begin{itemize}
    \item Weitergabe
    \item Anpassung
\end{itemize}
\paragraph{Bedingungen}
\begin{itemize}
    \item Die Lizenz und Urheberrechte müssen mit verteilt werden.
\end{itemize}

\paragraph{Limitierung}
\begin{itemize}
    \item keine Namensnennung für Werbung
    \item Haftungsausschluss
    \item keinerlei Garantie
\end{itemize}

Vgl. \cite{bsd-3-clause}

\section{Eclipse Public License - v 2.0}\label{sec:eclipse-public-license---v-2.0}
Standard Lizenz für Projekte der Eclipse Foundation.

\paragraph{Rechte}
\begin{itemize}
    \item kommerzielle Nutzung
    \item Andere kompatible Lizenz
    \item Sekundäre Lizenz
\end{itemize}
\paragraph{Bedingungen}
\begin{itemize}
    \item Bei Veröffentlichung muss der Quellcode frei zugänglich sein.
    \item Die Lizenz und Urheberrechte müssen mit verteilt werden.
\end{itemize}

\paragraph{Limitierung}
\begin{itemize}
    \item keinerlei Garantie
    \item eingeschränkte Haftung
\end{itemize}

Vgl. \cite{epl-2}

\section{GNU GPL V2 mit GNU Classpath Exception}\label{sec:gnu-gpl-v2-mit-gnu-classpath-exception}
Die GNU GPL V2 ist die Vorgänger version der GNU GPLv3 siehe \refk{sec:gnu-gplv3}.
Wie schon dort beschrieben hat die V2 einige Probleme, welche mit der V3 behoben wurden.

Angesehen davon wird hier noch zusätzlich die Classpath Exception verwendet.
Diese Ermöglicht die nutzung mit anderen Lizenzen wie zum beispiel der Apache-2.0,
sofern diese module unabhängig agieren.

Vgl. \cite{gnu-classPath}

\paragraph{Rechte}
\begin{itemize}
    \item kommerzielle Nutzung
    \item Weitergabe
    \item Anpassung
    \item private Nutzung und Modifikation
\end{itemize}
\paragraph{Bedingungen}
\begin{itemize}
    \item Bei Veröffentlichung muss der Quellcode frei zugänglich sein.
    \item Die Lizenz und Urheberrechte müssen mit verteilt werden.
    \item Änderungen müssen dokumentiert werden.
    \item Änderungen müssen unter der selben Lizenz veröffentlicht werden.
\end{itemize}

\paragraph{Limitierung}
\begin{itemize}
    \item keinerlei Garantie
    \item eingeschränkte Haftung
\end{itemize}
Vgl. \cite{choosealicense-com}, \cite{gnu-why-upgrade-gplv3}


