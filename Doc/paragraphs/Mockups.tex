\chapter{\ac{UI} Design}\label{ch:ui-design}
Das \ac{UI} Design wird in mehreren Schritten erstellt und im Laufe des Projekts immer weiter
verfeinert. Am Anfang dieses Prozesses steht ein Low Fidelity Prototyp der Applikation.
Dieser dient dazu, das ungefähre Layout zu visualisieren und ein Gefühl für den Aufbau
des Frontends zu vermitteln. %eventuell etwas umgansprachlich schaffen -> vermitteln?%
Für die Erstellung dieser Prototypen wird die Prototyping- und
Designsoftware Figma \footnote{Software zum Erstellen von Softwareprototypen https://www.figma.com} verwendet.
%Figma mit fußnote oder Tool Liste?%

\section{Low Fidelity}\label{Low Fidelity}
\begin{figure}[h]
  \centering
    \includegraphics[width=1.0\textwidth]{Antragshelfer}
    \caption{Antragshelfer}\label{Antragshelfer}
\end{figure}
Das Layout des Prototyps lässt sich in drei Bereiche einteilen:
die Kopfzeile, den Hauptinhalt der Seite und die Fußzeile.

In der Kopfzeile befindet sich das \ac{AStA}-Logo, sowie eine Suchleiste um manuell nach
Anträgen zu suchen.
Auch der jeweiligen Nutzeraccount lässt sich über die Schaltfläche
am rechten Rand erreichen.
Außerdem befindet sich am linken Rand ein Burger-Menü,
welches das Hinzufügen von Funktionalitäten in zukünftigen Projekten erleichtern soll.

Der Hauptinhalt der Seite zeigt die Hauptfunktionalitäten unserer Anwendung und wird 
dynamisch generiert. Bei den Prototypen wird dies
durch ein statisches Layout simuliert.

Durch die Fußzeile soll der Nutzer die Möglichkeit erhalten, schnell und einfach auf
seine zuvor festgelegten Favoriten zugreifen zu können.
Was zu einer besseren User Experience führen soll.
Die Kopf- und Fußzeile der Applikation ist auf jeder Seite identisch.

Der Antragshelfer ist eine Hauptfunktionalität unserer Applikation deshalb fungiert
dieser als Landing-Page, wie in \refa{Antragshelfer} dargestellt.

Hier soll der Nutzer die ihm, von der Applikation gestellten, Fragen beantworten,
um so zum richtigen Antrag zu gelangen. %TODO Satz umvormoliren%
Die Beantwortung der Fragen erfolgt durch Klicken auf die verschiedenen Buttons. %Passt das so?%
Zur Übersicht, ist jeder Button mit einer Beschreibung versehen, welche genauere Informationen über die Auswahloption liefert.

\begin{figure}[h]
  \centering
    \includegraphics[width=1.0\textwidth]{Reisekostenhelper}
    \caption{Reisekostenhelfer}\label{Reisekostenhelfer}
\end{figure}


Beim Ausfüllen des Antrags hat der Nutzer die Möglichkeit über unterschiedliche Eingabemöglichkeiten,
wie Textfelder, Dropdown Menüs oder Checkboxen die benötigten Informationen einzugeben. Auch die Eingabe
über die integrierten \ac{API}s ist möglich, wie in \refa{Reisekostenhelfer} am Beispiel eines Reisekosten Antrags zu sehen.
Um dem Nutzer einen Überblick über den Fortschritt beim Ausfüllen des Antrags zu ermöglichen, befindet sich am oberen 
Rand eine Fortschrittsanzeige.
So soll der Nutzer darüber informiert werden, wie viele Schritte noch
nötig sind, bis der Antrag fertig bearbeitet ist. 


\section{Finales Design}\label{Finales Design}
Im Laufe des Development Prozesses wurde das Design fortwährend angepasst und auf Basis
von Kunden- und Nutzerfeedback weiterentwickelt. Am Ende dieses Prozesses steht nun 
dieses finale \ac{UI} Design.

\subsection{Login Seite}\label{Login Seite}
Die Login Seite wurde absichtlich mit wenigen \ac{UI} Elementen designed um den Nutzer 
nicht zu überfordern. Klickt der Nutzer auf den "Login" Button, wird er zu unserem 
Authentifizierungsdienst Keycloak weitergeleitet, wo er sich mit seinen Login Daten
authentifizieren kann. Ist dies erfolgreich, wird er auf die Hauptseite weitergeleitet.

\begin{figure}[h]
  \centering
    \includegraphics[width=1.0\textwidth]{Doc/images/Login Page.png}
    \caption{Login Seite}\label{Login Page}
\end{figure}

\subsection{Haupt Seite}\label{Haupt Seite}
Die Hauptseite orientiert sich am grundlegenden Aufbau stark am Low Fidelity Prototyp, 
wurde jedoch in bestimmten Details verfeinert. So wurde bei der Auswahl der Anträge eine 
Unterscheidung zwischen Anträgen und Abrechnungen ermöglicht. Zudem wurde das \ac{UI} 
Element der Fertigen Anträge vom unteren Ende der Hauptseite, in die aufklappbare Sidebar
verschoben um die Seite trotz mehr Details, übersichtlicher zu gestallten. Zu guter Letzt 
wurde die Darstellung der Favoriten verbessert und deren einzelne Bedienelemente farblich 
passend zu ihrer Funktion hervorgehoben.

\begin{figure}[h]
  \centering
    \includegraphics[width=1.0\textwidth]{Doc/images/Landing Page.png}
    \caption{Haupt Seite}\label{Hautp Seite}
\end{figure}

\pagebreak

\subsection{Eigene Daten}\label{Eigene Daten}
In dieser Seite, welche über das Burger Menü erreichbar ist, kann der Nutzer, die eigenen 
Daten hinterlegen, um diese später beim Ausfüllen eines Antrages nutzen zu können. Ahnlich 
wie der Login ist sie sehr schlicht gehalten, da für die simple Aufgabe der Dateneingabe ein
aufwendig gestalltetes \ac{UI} eher stöhrend wirken würde.

\begin{figure}[h]
  \centering
    \includegraphics[width=1.0\textwidth]{Doc/images/Own Data.png}
    \caption{Eigene Daten}\label{Eigene Daten}
\end{figure}

\pagebreak

\subsection{Eigene Anträge}\label{Eigene Anträge}

Auf diese Seite, welche ebenfalls über das Burger Menü erreichbar ist, finden sich alle 
fertigen sowie nicht fertigen Anträge, die der Nutzer bearbeitet hat. Wie zuvor auf der 
Haupt Seite erwähnt war dieses Feature zuvor am Boden der Haupt Seite zu finden, wurde 
jedoch auf eine eigene Seite verschoben. Hierr kann der Nutzer durch Klicken auf eine der 
Karten auf einen alten Antrag zugreifen. Der Fortschritt bleibt dabei erhalten. Das Design 
der Karten orientiert sich in seinen Grundzügen am Design der Antrags Elemente der 
Hauptseite.

\begin{figure}[h]
  \centering
    \includegraphics[width=1.0\textwidth]{Doc/images/Own Antraege.png}
    \caption{Eigene Anträge}\label{Eigene Daten}
\end{figure}

\pagebreak

\subsection{Antrag Ausfüllen}\label{Antrag Ausfüllen}

Hier kann der Nutzer die von ihm ausgewälten Anträge ausfüllen. Über die Sektionsanzeige 
oben ist es möglich schnell und unkompliziert zwischen den einzelnen Sektionen hin und her 
zu wechseln. Das wechseln an sich ist jedoch auch über die beiden Buttons am unteren Ende 
der Seite möglich. Auch dieses Design orientiert sich start am Prototypen und wurde nur 
verfeinert. Neu ist das Herz Icon, über welches man per Klick den momentanen Antrag zu den 
Favoriten hinzufügen kann. Diese Funktionalität wurde mit dem Ziel einer guten Usability 
hier implementiert, da das Hinzufügen von Favoriten über einen Klick auf ein Herz für 
den Nutzer bereits aus anderen Applikationen, wie Spotify, wohl vertraut ist.

\begin{figure}[h]
  \centering
    \includegraphics[width=1.0\textwidth]{Doc/images/Fill in Antrag.png}
    \caption{Ausfüllen eines Antrages}\label{Ausfüllen eines Antrages}
\end{figure}

\subsection{Allgemeines Design}\label{Allgemeines Design}
Zunächst ist anzumerken, dass die Farbgebung angepasst wurde. Dies geschah auf 
Kundenfeedbback, welches einen, vom AStA-Logo inspirierten Blau Ton, der orangenen 
Farbgebung vorzog. Auch die Farbabstufungen zwischen den einzelnen \ac{UI} Elementen 
wurden deutlich verstärkt, um ein Unterscheiden derselben zu vereinfachen.
Eine weitere, auf Basis von Kunden Feedback getroffene Design Entscheidung war es, die 
verschiedenen \ac{UI} Elemente wenn möglich zu verkleinern. Daher wurde beschlossen, die 
fertigen Anträge aus der Fußkomponente zu entfernen und stattdessen die über das Burger 
Menü erreichbare Sidebar hierfür zu verwenden.

