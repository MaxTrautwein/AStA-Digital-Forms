\chapter{UI Design}\label{ch:ui-design}
Das UI Design wird in mehreren Schritten erstellt und im Laufe des Projekts immer weiter
verfeinert. Am Anfang dieses Prozesses steht ein Low Fidelity Prototyp der Applikation.
Dieser dient dazu das ungefähre Layout zu visualisieren und ein Gefühl für den Aufbau
des Frontend zu schaffen. Für die Erstellung dieser Prototypen wird die Prototyping- und
Designsoftware Figma verwendet.

\section{Low Fidelity}\label{Low Fidelity}
\begin{figure}[h]
  \centering
    \includegraphics[width=1.0\textwidth]{Doc/images/Antragshelfer.png}
    \caption{Antragshelfer}
\end{figure}
Das Layout des Prototypen lässt sich in drei Bereiche einteilen. Die Kopfzeile, den
Hauptinhalt der Seite sowie die Fußzeile.

In der Kopfzeile befindet sich das AStA Logo, sowie eine Suchleiste um manuell nach
Anträgen zu suchen. Auch der jeweiligen Nutzerr Account lässt sich über die Schaltfläche
am rerchten Rand erreichen. Außerdem befindet sich am linken Rand ein Burger Menü,
welches das Hinzufügen von Funktionalitäten in zukünftigen Projekten erleichtern soll.

Der Hauptinhalt der Seite stellt die Hauptfunktionalitäten unserer Applikation dar und
ist daher in der fertigen Applikation dynamisch generiert. Bei den Prototypen wird dies
durch ein statisches Layout simuliert.

Durch die Fußzeile soll der Nutzer die Möglichkeit erhalten, schnell und einfach auf von
ihm zuvor festgelegten Anträge zugreifen zu können, was eine gute User Experience
gewährleisten soll. Zu erwähnen ist, dass sowohl Kopf- als auch Fußzeile auf jeder Seite
der Applikation identisch sind. Lediglich die Hauptinhalte unterscheiden sich.

Da der Antrags Helper die Hauptfunktionalität unserer Applikation darstellt, fungiert
dieser auch als Landing Page, also der Seite, welche der Nutzer direkt nach dem Login
sieht, wie in Abb.7.1 dargestellt. Hier soll der Nutzer durch Klicken mehrerer Buttons die
ihm, von der Applikation gestelllten, Fragen beantowrten, um so zum richtigen Antrag zu
gelangen

\begin{figure}[h]
  \centering
    \includegraphics[width=1.0\textwidth]{Doc/images/Reisekostenhelper.png}
    \caption{Reisekostenhelfer}
\end{figure}

