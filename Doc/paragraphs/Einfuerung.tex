\chapter{Einführung und Ziele}\label{ch:einfuhrung-und-ziele}

\section{Aufgabenstellung}\label{sec:aufgabenstellung}
Die Aufgabe besteht in der Entwicklung einer Web-Applikation, welche ein nahezu vollständig digitales
Ausfüllen von Anträgen ermöglicht. Dies soll eine Alternative zum bisherigen, anaolgen Ausfüllen der Anträge bieten und so zu einer vereinfachten und beschleunigten Verwaltung beitragen.
Dies beinhaltet das Entwerfen, Programmieren und Testen der Software, sowie eine ausführliche Planung und Dokumentation des gesamten Projekts.

\section{Qualitätsziele}\label{sec:qualitatsziele}
Die von uns entwickelte Software soll möglichst alle Merkmale hochqualitativer Software erfüllen, wie 
sie in der ISO/IEC 25010 festgehalten sind. Besonderes Augenmerk legen wir auf die Usability, 
Vollständigkeit, sowie leichte Portierbarkeit und Verlässlichkeit. Dieser Fokus erwächst aus der
Aufgabenstellung, da eine in der Verwaltung eingesetzte Software vor allem verlässlich und vollständig
sein muss. Da sie das Ausfüllen von Anträgen vereinfachen soll, ist zudem eine gute Usability und
Portierbarkeit nötig, um Anträge schnell, einfach und von überall aus ausfüllen zu können.

\section{Stakeholder}\label{sec:stakeholder}
Zu den Stakeholdern gehören sowohl Studenten der Hochschule als Endnutzer, als auch
der AStA als Kunde, das Entwicklerteam sowie der Projektbetreuer. Auch die Hochschule 
gehört zu unseren Stakeholdern da sie, durch das Setzen der Rahmenbedingungen, Einfluss auf das 
Projekt hat.