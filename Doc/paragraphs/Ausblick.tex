\chapter{Ausblick}\label{ch:ausblick}

Da die Aufgabe unserer Applikation darin besteht, das Ausfüllen von Anträgen so einfach 
und schnell wie möglich zu machen, besteht das größte Potential für Weiterentwicklungen
im Themenbereich der User Experience.

\section{\ac{UI} Design}\label{sec: ui design}
Das momentane \ac{UI} Design beruht auf Kunden und Nutzerfeedback, welches wir in
Interviews erhalten haben. Zukünftig könnte es von Vorteil sein, solche Interviews
erneut und mit mehr Probanden durchzuführen, da so eine differenziertere Analyse
des Feedbacks und damit das Erstellen eines besseren, auf den Kunden zugeschnittenen \ac{UI}
Designs ermöglicht werden kann.

\section{Formular Editor}\label{sec: formular editor}
Auch das Hinzufügen eines Editors für die Erstellung und Manipulation von 
Formular-Templates und der daraus resultierenden PDF wäre eine sinnvolle Erweiterung. Denn 
momentan ist zwar gut dokumentiert, wie neue Anträge in die Applikation eingebunden werden 
können, dies erfordert jedoch das direkte Manipulieren von Dateien in der Applikation 
und ist daher von nicht fachkundigen Personen nicht durchführbar. Ein visueller Editor, der 
beispielsweise nach einem "Drag and Drop" Prinzip funktionieren könnte, würde diese Lücke 
in der Usability der Applikation schließen, sodass auch IT fremde Nutzer, Anträge 
erstellen, verändern oder löschen können, was die allgemeine User Experience deutlich 
verbessern würde.

\section{Wiederverwenden von Antragsdaten}\label{sec: wiederverwenden von Antragsdaten}
Eine weitere potenzielle Verbesserung der User Experience bestünde in der Möglichkeit, 
Daten, welche in einem bestimmten Antrag verwendet wurden, direkt in ein 
Abrechnungsformular zu übernehmen. Dies könnte durch eine neue Entität in der Datenbank, 
oder durch Import einer passenden Datei ermöglicht werden, ähnlich wie es bereits im 
Autofill-Feature der Fall ist. Dadurch ließe sich beim Ausfüllen einer Abrechnung deutlich 
mehr Zeit sparen, was eine bessere User Experience zur Folge hätte.

\section{Maps \ac{API}}\label{sec: maps api}
Um das Ausfüllen von Anträgen wie der Reisekostenerstattung weiter zu vereinfachen, würde
sich die Integration einer Karten-\ac{API} anbieten. Wenn man also beispielsweise im
Antrag angeben muss, von wo man startet, wo man hinfährt, und welche Zwischenstopps man
macht, könnte dies durch einfache Klicks auf einer interaktiven Karte passieren. Zusätzlich
könnten weitere Daten wie voraussichtliche Treibstoffkosten, voraussichtliche Reisedauer
und andere wichtige Daten auf Basis dieser Funktion errechnet, und direkt in den Antrag
integriert werden. Das Speichern von Adressen, um sie in zukünftigen Anträgen schneller
auswählen zu können, wäre ebenfalls sinnvoll.
Auch ein Anhängen der Karte mit visualisierter Route an die generierte
PDF als Beleg der Daten, wäre möglich. So ließe sich das mühsame Eingeben von Adressen
und die damit einhergehende Fehleranfälligkeit umgehen. Dies würde es 
ermöglichen, den Antrag schneller und einfacher auszufüllen und würde damit die User
Experience deutlich verbessern.

\section{Verwalten von Accounts}\label{sec: verwalten von Accounts}
Momentan müssen die Nutzeraccounts der Applikation vollständig von einem fachkundigen
Administrator verwaltet werden. Dies schließt das Erstellen neuer Accounts, sowie das
Ändern von Passwörtern und Nutzernamen ein. Es wäre jedoch wünschenswert, wenn jeder
Nutzer über ein geeignetes \ac{UI} die Möglichkeit hätte, die eigenen Login Daten selbst
zu verwalten, um so zum einen den Administrator zu entlasten und zum anderen selbst mehr
Einfluss auf den eigenen Account zu haben. Dies würde die User Experience deutlich
verbessern, da zum Ändern der Logindaten bzw. zum Erstellen eines Accounts keine
Konsultation des Administrators mehr nötig ist, was den Prozess deutlich vereinfacht und
beschleunigt.

\section{Login mit anderen Accounts}\label{sec: login mit anderen Accounts}
Da davon auszugehen ist, dass die Nutzer bereits über einen Hochschulaccount verfügen,
wäre es für eine gute Usability von Vorteil, wenn man sich nicht nur über den
applikationseigenen Account bei Keycloak, sondern auch über den Hochschulaccount
anmelden könnte. So könnte man das zeitintensive Erstellen und Verwalten eines
weiteren Accounts vermeiden.

\section{Umstellung der verwendeten Datenbank}\label{umstellung der verwendeten datenbank}
Die Applikation könnte dahingehend erweitert werden, dass sie es dem Nutzer ermöglicht, 
selbst zu entscheiden, ob seine Daten auf der applikationinternen Datenbank, oder auf einer 
Nextcloud Instanz gespeichert werden sollen. 


\section{Formularfelder}\label{sec: formularfelder}
Ein Ausbau der möglichen Eingabefelder beim Ausfüllen eines Antrags wäre für die
User Experience ebenfalls zuträglich, da diese dem Nutzer im Idealfall bereits durch ihr
Layout und Design vermitteln, welche Daten hier eingegeben werden sollen. Außerdem können
so unterschiedlichere Arten von Eingaben in der Applikation realisiert werden, was wiederum
die Flexibilität der Applikation steigern würde.

