\chapter{Ausblick}\label{ch:ausblick}

Da die Aufgabe unserer Applikation darin besteht, das Ausfüllen von Anträgen so einfach 
und schnell wie möglich zu machen, besteht das größte Potential für Weiterentwicklungen
im Themenbereich der User Experience.

\section{\ac{UI} Design}\label{sec: ui design}
Das momentane \ac{UI} Design beruht auf Kunden und Nutzerfeedback, welches wir in 
Interviews erhalten haben. Zukünftig könnte es von Vorteil sein, solche Interviews 
erneut und mit mehr Probanten durchzuführen, da so eine differenziertere Anaylse
des Feedback und damit das Erstellen eines besseren, auf den Kunden zugeschnittenes \ac{UI} 
Designs ermöglicht werden kann.%ref auf Interviews?

\section{Formular Editor}\label{sec: formular editor}
Auch das Hinzufügen eines Editors für die Erstellung und Manipulation von 
Formular-Templates und der daraus resultierenden PDF wäre eine sinnvolle Erweiterung. Denn 
momentan ist zwar gut dokumentiert, wie neue Anträge in die Applikation eingebunden werden 
können, dies erfordert jedoch das direkte Manipulieren von Dateien in der Applikation 
und ist daher von nicht Fachkundigen Personen nicht durchführbar. Ein visueller Editor, der 
beispielsweise nach einem "Drag and Drop" Prinzip funktionieren könnte, würde diese Lücke 
in der Usability der Applikation schließen, sodass auch IT fremde Nutzer, Anträge 
erstellen, verändern oder löschen können, was die allgemeine Userexperience deutlich 
verbessern würde.