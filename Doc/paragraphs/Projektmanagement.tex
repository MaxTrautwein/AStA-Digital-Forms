\chapter{Projektmanagement}\label{ch:projektmanagement}

\section{Methode}\label{sec:methode}

In diesem Projekt wird mit der agilen Arbeitsweise Scrum gearbeitet.\\
Mit Scrum ist es möglich Fehler bereits in der Ausführung zu erkennen und eine schnelle Lösung zu finden.
Scrum ist iterativ und agil. Das bedeutet es gibt wiederholende Sprint-Zyklen.
Ein Sprint Zyklus unter Scrum besteht aus: Sprint Planung, Sprint, Sprint Review und der Retrospektive.

\section{Lizenz}\label{sec:lizenz}

Der Code wird unter der GNU General Public License v3.0 veröffentlicht.\\
Die Lizenz ist eine Open-Source Lizenz, die die Innovation und Zusammenarbeit fördert.\\
Die GNU GPL v3.0 beschränkt die Haftungs- und Gewährleistungspflichten der Entwickler, aber sie lässt den Code für Nutzer und andere Entwickler frei zugänglich.

\section{Github Flow}\label{sec:github-flow}

Das Projekt wird mit einem GitHub-Flow realisiert.\\
In einem GitHub-Flow wird für jede Teilaufgabe eine eigene Branch erstellt, da diese eine Einsicht der Problemlösung erfordert.
Dadurch wird verhindert, dass versehentlich fehlerhafter Code im Master landet. Nach erfolgreicher Einsicht des Codes kann dieser auf den Master gemerged werden \dah{} Die Lösung ist valide und kann in den vorerst finalen Code integriert werden.

\section{Definition of Done}\label{sec:dod}

Der Begriff Definition of Done kommt aus einer agilen Arbeitsweise wie Scrum.\\
Die Definition of Done beschreibt Kriterien, die abgedeckt sein müssen, damit ein Task als abgeschlossen gilt.
Die Kriterien dieses Projekts für einen Task bestehen zusammengefasst aus einem voll funktionsfähigen Code, sinnvoll kommentierten Code, Erstellung von Tests, erfolgreichen Tests und einer vollständigen Dokumentation des Codes.\\
Der Code muss, die in den Requirements beschriebenen Funktionen lauffähig beinhalten.\\
Komplexer Code muss sinnvoll und nachvollziehbar kommentiert sein.\\
Für jeden Task muss es einen Test geben. Ein Test kann auch mehrere Tasks abdecken. Alle Tests müssen bestanden sein.
Das letzte Kriterium wäre eine vollständige Dokumentation des Codes, welche die Vorgehensweise für die Lösung begründet.