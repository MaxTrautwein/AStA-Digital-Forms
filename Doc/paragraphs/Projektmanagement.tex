\chapter{Projektmanagement}\label{ch:projektmanagement}
Die Struktur unseres Projektmanagements lässt sich in die vier Teilgebiete Methode, Lizenz, Github-Flow
und Definition of Done unterteilen.
\section{Methode}\label{sec:methode}

In diesem Projekt wird mit der agilen Arbeitsweise Scrum gearbeitet.\\
Mit Scrum ist es möglich Fehler bereits in der Ausführung zu erkennen und eine schnelle Lösung zu finden.
Scrum ist iterativ und agil.
Das bedeutet es gibt wiederholende Sprint-Zyklen.
Ein Sprint Zyklus unter Scrum besteht aus: Sprint Planung, Sprint, Sprint Review und der Retrospektive.

\section{Lizenz}\label{sec:lizenz}

Der Code wird unter der GNU General Public License v3.0 veröffentlicht.\\
Die Lizenz ist eine Open-Source Lizenz, die die Innovation und Zusammenarbeit fördert.\\
Die GNU GPL v3.0\footnote{https://www.gnu.org/licenses/gpl-3.0.en.html}
beschränkt die Haftungs- und Gewährleistungspflichten der Entwickler, aber sie lässt den Code für Nutzer und andere Entwickler frei zugänglich.

\section{Github-Flow}\label{sec:github-flow}

Das Projekt wird mit einem GitHub-Flow realisiert.\\
In einem GitHub-Flow wird für jede Teilaufgabe eine eigene Branch erstellt, da diese eine Einsicht der Problemlösung erfordert.
Dadurch wird verhindert, dass versehentlich fehlerhafter Code im Master-Branch landet.
Nach erfolgreicher Einsicht des Codes durch eine Merge Request kann dieser auf die Master-Branch gemerged werden.
Dies validiert die Lösung, sodass diese in den vorerst finalen Code integriert werden kann.
%Eventuell dah teil überdenken eventeull 2 sätze? -> Done%

\section{Definition of Done}\label{sec:dod}

Der Begriff Definition of Done kommt aus einer agilen Arbeitsweise.\\
Die Definition of Done beschreibt Kriterien, die erfüllt sein müssen, damit ein Task als abgeschlossen gilt.
Die Kriterien für dieses Projekts sind:
\begin{itemize}
    \item Funktionsfähiger Code
    \item Sinnvoll kommentierter Code
    \item Erfolgreiches Testen des Codes\\ \textit{Für jeden Task muss es einen Test geben.
    \\Ein Test kann auch mehrere Tasks abdecken. Alle Tests müssen bestanden sein.}
    \item Vollständige Dokumentation des Codes
    \item Erfüllung des Requirements
    
\end{itemize}