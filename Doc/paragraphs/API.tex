\chapter{\ac{API}s}\label{ch:apis}
In diesem Kapitel wird detailierter auf die Verwendung von verschiedenen \ac{API}s in unserem Projekt
eingegangen.

\section{Guidelines}
In dieses Projekt orientiert sich im Bezug auf \ac{API}s an den Zalando API Guidelines\footnote{https://opensource.zalando.com/restful-api-guidelines/}. 
Diese werden, so weit sie für das Projekt förderlich und sinnvoll sind, bei der Implementierung 
berücksichtigt.
\section{Verwendete \ac{API}s}
In unserem Projekt finden vollgende \ac{API}s, mit den jeweiligen Endpunkten anwendung:

\section{Endpoints}
Folgende Endpunkte sind in der verwendeten API ansprechbar.

\subsection{GET /templates}

Dieser API Endpunkt liefert alle in der Datenbank gespeicherten Templates zurück. Diese
 werden genutzt, um die Buttons auf der Main Page dynamisch zu generieren.
%Dieser Abschnitt wird weiter ergänzt, wenn die Implementierung der Applikation voranschreitet.

\section{Schemata}
Die Schemata, die die API verwendet, sehen wie folgt aus.

\subsection{FormElement}