\chapter{\ac{API}s}\label{ch:apis}
In diesem Kapitel wird detailierter auf die Verwendung von verschiedenen \ac{API}s in unserem Projekt
eingegangen.

\section{Guidelines}
In dieses Projekt orientiert sich im Bezug auf \ac{API}s an den Zalando API Guidelines\footnote{https://opensource.zalando.com/restful-api-guidelines/}. 
Diese werden, so weit sie für das Projekt förderlich und sinnvoll sind, bei der Implementierung 
berücksichtigt.

\section{Verwendete \ac{API}s}
Um Daten in unerer Datenbank zu speichern und abzufragen, wird die OpenAPI Specification 
verwendet. Diese bietet die Möglichkeit, die Integration der verschiedenen Bestandteile der 
\ac{API}, nach den Zalando Guidelines automatisch auf Basis einer im yaml Format hinterlegten 
Beschreibung der gewünschten \ac{API} zu generieren. So kann Fehlern im Developmentprozess 
vorgebeugt und der Prozess an sich beschleunigt werden.

\section{Endpoints}
Folgende Endpunkte sind in der verwendeten \ac{API} ansprechbar.

\subsection{GET /templates}

Dieser \ac{API} Endpunkt liefert alle in der Datenbank gespeicherten Templates zurück. Diese
 werden genutzt, um die Buttons auf der Main Page dynamisch zu generieren. Da hierfür nicht 
 alle Informationen über die Templates benötigt werden, sind in der Antwort nicht alle 
 Informationen über die einzelnen Templates enthalten, da sie hier nicht benötigt werden.

 \subsection{GET /templates/{templateId}}
 Dieser Endpunkt gibt ein bestimmtes Template zurück. Im Vergleich zum Ergebnis des vorherigen 
 Endpunkts werden jedoch deutlich mehr Details zurückgegeben.

\section{Schemata}
Im Folgenden werden die verwendeten Schemata grob beschrieben. Für Detailiertere Informationen 
wird auf den folgenden Anhang verwiesen.

\subsection{FormElement}
Das FormElement Schema beinhaltet alle Daten, die für ein einzelnes, dynamisch generiertes 
Eingabeelement benötigt wird.

\subsection{FormSection}
Das Schema der FormSection beinhaltet Daten, die für die dynamische Generierung einer Sektion
für das Ausfüllen eines Antrags, benötigt werden. Dazu zählt eine Liste der FormElements in 
dieser Section, sowie eine Reihenfolge, um die Sections ordnen zu können.

\subsection{Attachment}
Das Attachment Schema dient der Organisation und Orrdnung der dem Antrag beigefügten Dateien.

\subsection{Form}
Das Form Schema entspricht in seinen Daten, einem komplettenn Antrag und besteht aus mehreren 
FormSections, die wiederrum aus mehreren FormElements bestehen, sowie potentiellen Attachments.

\section{Detailierte Informtionen}
Im Folgenden ist der exakte Aufbau der Endpunkte und der Schemata unserer API dokumentiert.
\includepdf[pages=-, landscape=true]{test.pdf}
%Dieser Abschnitt wird weiter ergänzt, wenn die Implementierung der Applikation voranschreitet.