\chapter{Randbedingungen}\label{ch:randbedingungen}


\section{Technische Randbedingungen}\label{sec:technische-randbedingungen}

\section{Organisatorische Randbedingungen}\label{sec:organisatorische-randbedingungen}

Das Projekt \workTitel wird als Teil des Moduls \workTyp durchgeführt.
Dies Zieht einige organisatorische Randbedingungen mit sich.\\
Das Projekt wurden nach der Bekanntgabe der Einteilungen am 15.03.2024 Begonnen und muss 25.06.2024 Abgeschlossen sein.
Da dem Modul mit 10 \ac{ECTS} Punkten bewertet wird, liegt der Arbeitsaufwand pro Teammitglied bei \(30\,Stunden * 10 \ac{ECTS} = 300\,Stunden\) also insgesamt 900 Stunden.
Davon abzuziehen sind noch die Zwei verpflichtenden Seminare "Teambildung und Konfliktlösung" sowie "Präsentation und Disputation".
Dementsprechend sind auch die requirements für das Projekt einzugrenzen.\\

Das Modul fordert einige vordefinierte Abgaben ein, welche teils der agilen Vorgehensweise welche angestrebt wird wiedersprechen.
Diese führt dazu, dass nicht immer alle teile der angeilen Methodik befolgt werden können.\\

Die Tatsache selbst, dass es sich hierbei um ein Projekt handelt, welches während des Studiums durchgeführt wird, hat seine einflüssen.
Neben dem Projekt sollten die Teammitglieder natürlich auch nicht ihre anderen Module vernachlässigen, dies schränk unter anderem die verfügbarkeit ein.
Ferner belegen nicht immer alle Teammitglieder dieselben module was die Terminfindung / verfügbarkeit komplexer macht.