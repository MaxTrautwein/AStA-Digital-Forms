\chapter{Aufwandsschätzung}\label{ch:aufwandsschatzung}
% Helper Command to format entrys in the Table %
\newcommand{\trschaetzung}[3]{\rowcolor{lightgray}\multicolumn{1}{|l|}{\textbf{#1}} & \textbf{#2} \\* \multicolumn{2}{|l|}{\begin{tabular}[c]{@{}l@{}}#3\end{tabular}} \\ \hline}

\section{Gruppenleistungen}\label{sec:gruppenleistungen}

% Add Text Explaining this Category %

\begin{longtable}{|lr|}
    \hline
    \rowcolor{darkgray}\multicolumn{1}{|l|}{\textbf{Aufgabe}} & \textbf{\begin{tabular}[c]{@{}c@{}}Zeitschätzung\\\(Stunden\)\end{tabular}} \\ \hline
    \endhead
    \trschaetzung{Betreuer Meeting 1}{1.5}{Das Erste Treffen mit unserem Betreuer}
    \trschaetzung{Kunden Treffen 1}{1}{Das Erste Traffen mit Unseren Kunden}
    \trschaetzung{Regeltreffen}{6}{Aufsummirte Zeit der geplanten Regelmeetings mit den Kunden sowie dem Betreuer}
    \trschaetzung{Präsentation und Disputation}{8}{Pflicht Seminar am Samstag den 13.04.2024}
    \trschaetzung{Teambildung und Konfliktlösung}{8}{Pflicht Seminar am Samstag den 6.04.2024}
    \trschaetzung{Präsentation}{1}{Präsentations Zeit}
    \trschaetzung{Präsentation vorbrereiten}{12}{Vorbereitungszeit auf die Präsentationen.
        \\Umfasst das Erstellen, verfeinern sowie Einproben.}
    \trschaetzung{Angular Testen}{5}{Sipke um sich mit Angular vertraut zu machne und zu prüfen ob das
        \\Team damit effektiv arbeiten kann}
    \trschaetzung{Erste Team Festlegungen}{1}{Interne Besprechung grundlegenden Themen wie:
        \\Lizens, Github-flow, Repository Struktur, Dokumentations Format}
    \trschaetzung{Aufwandsschätzung}{8}{Zeit zum Schätzen den Aufwandes für alle einzelenen Tasks
        \\Enthält auch Zeit welche zur entwiklung von Requierments aufgewendet wurde\\ um Tasks besser aufzuteilen.}
    \trschaetzung{Internes Treffen 30.3}{2}{Beschluss über Angular Wahl für das Frontend,
        \\Terminfindung für Agile Regelmeetings im Team,
        \\erste descussionen zur technologie wahl fürs Backend sowie der Datenbank}
    \trschaetzung{Sprint-Plan}{12}{Aufsummirte Zeit um Sprintplannings durchzuführen}
    \trschaetzung{Protokoll}{6}{Aufsummirte Zeit zum aufbereiten der Protokolle zu Meetings}
    \trschaetzung{Reflektion Projektmanagement}{3}{Zeit für die Reflektion über das Projektmanagement am ende des Projekts}
    \trschaetzung{Reflektion Lernfortschritt \& Doku}{4}{Zeit für die Reflektion über den Lernfortschritt am ende des Projekts}
    \trschaetzung{Spike \ac{PDF} Generator wahl}{3}{Spike um die Optionen zur automatisireten Generation von \ac{PDF} dokumenten zu evaluieren.}
    \trschaetzung{Style Guyde Auswhal}{3}{Festlegung von Interen Style Richtlinien für besssere Code Qualität und Verständniss}
\end{longtable}

\section{Individualleistungen}\label{sec:individualleistungen}

% Add Text explaining that Category %

\begin{longtable}{|lr|}
    \hline
    \rowcolor{darkgray}\multicolumn{1}{|l|}{\textbf{Aufgabe}} & \textbf{\begin{tabular}[c]{@{}c@{}}Zeitschätzung\\\(Stunden\)\end{tabular}} \\ \hline
    \endhead
    \trschaetzung{Docker Compose Setup}{6}{Follständiges Docker Compose Setup für alle Module\\Sowie Veifikation auf allen Team Systemen}
    \trschaetzung{Klickdummie bauen und testen}{15}{Entwickeln Testen eines Dummies der Applikation\\
    dies soll fühzeitig probleme aufdecken und die Qualität der \ac{UI} verbessern}
    \trschaetzung{Interviews Fragen}{4.5}{Erstellen von Interview Fragen and die verschidenen Stakeholder.}
    \trschaetzung{Interviews Durchfüren}{6}{Interview Durführung und Dokumentation}
    \trschaetzung{User Storys entwerfen}{9}{Userstorys entwikeln}
    \trschaetzung{Doku Meilenstein 1 - \ac{UI} Design}{6}{Entwurf erster Mockups, Dokumentation dazu}
    \trschaetzung{Doku Meilenstein 1 - Archiketur}{7.5}{Architiktur Entworf mit Dokumentation}
    \trschaetzung{Doku Meilenstein 1 - Projektmanagment}{6}{Dokumentation zu dem eigenen Projektmanagement}
    \trschaetzung{Doku Meilenstein 1 - Randbedingungen}{3}{Dokumentation der Technischen sowie Organistatorischen Randbedingungen}
    \trschaetzung{Doku Meilenstein 1 - Einführung und Ziele}{6}{Dokumenation der Aufgabenstellung sowie der Ziele}
    \trschaetzung{Repo Init}{1}{Inizialisrung  des GitHub Repositorys}
    \trschaetzung{Doku Init}{0.5}{Inizialisrung der LaTeX Dokumentationsvorlage}
    \trschaetzung{Datenbankmodell}{6}{Erstellung schlüssigen eines Datenbankmodells}
    \trschaetzung{Softwarearchitektur inklusive unterschiedlicher (logischer) Sichten}{15}{Detailirte Softwarearchitektur auf den Tieferen schichten}
    \trschaetzung{Doku Meilenstein 2 - Verwendete Technologien / Frameworks}{6}{Dokumenation zu den Gewählten Technoligen sowie Frameworks}
    \trschaetzung{Verwendete Schnittstellentechnologie}{6}{Dokumenattion der Schnittstellen und deren Technologie}
    \trschaetzung{Installations- und Administrationshandbuch}{15}{Detailirtes Installations- und Administrationshandbuch\\
    Entält erklärungen zu den Konigurations Dateien}
    \trschaetzung{Aufteilung des Teams}{3}{Klare dokumentation welches Teammitglid welche Tätichkeit übernommen hat}
    \trschaetzung{Doku Meilenstein X - Reflektion Projektmanagement}{6}{Dokumentation der eigenen Reflektion über das Projektmanagement}
    \trschaetzung{Lizenzen: verwendete Lizenzen (Fremdcode: Frameworks, Libraries)}{6}{Dokumentation aller verwendeten Lizenzen}
    \trschaetzung{Ausblick}{6}{Dokumenation des Aublicks am ende Des Projekts.\\Was sind nächste schritte? Was kann weiter verbessert werden? Anwendung?}
    \trschaetzung{Favoriten}{21}{Favoriten System mit \ac{GUI},\\Einstellungen zur Automatischen Erkennung und Automatik Funktion}
    \trschaetzung{Antrags Beschribungen Erstellen}{9}{Erstellung von Sinfollen Bescheibungen der Aktuellen Anträge}
    \trschaetzung{Vollständigkeitskontrolle}{9}{TODO}
    \trschaetzung{Kategorisieren und Taggen Von Anträgen}{9}{System zum Kategorisiren und Taggen der Anträge in der Konfiguration}
    \trschaetzung{Filter System}{12}{TODO}
    \trschaetzung{Auswhals Helfer - Konfigurations System}{30}{TODO}
    \trschaetzung{Formular Felder Kompatibilität markieren}{6}{TODO}
    \trschaetzung{Formular Fortschritt Speichern}{6}{TODO}
    \trschaetzung{Formular Fortschritt Laden}{12}{TODO}
    \trschaetzung{Hinweis System}{3}{TODO}
    \trschaetzung{Keycloak Einrichten}{4.5}{TODO}
    \trschaetzung{Keycloak Anmeldung Frontend}{6}{TODO}
    \trschaetzung{Keycloak Verifikation}{6}{TODO}
    \trschaetzung{Backend Datenbank Interface}{12}{TODO}
    \trschaetzung{Formular Aoutofill Option}{12}{TODO}
    \trschaetzung{Routenberechnung \ac{API}}{30}{TODO}
    \trschaetzung{Adressvervollständigung \ac{API}}{15}{TODO}
    \trschaetzung{Formular \ac{PDF} Vorlagen Erstellen}{36}{TODO}
    \trschaetzung{link 2 \ac{PDF} Generator}{6}{TODO}
    \trschaetzung{Routen Plan \ac{PDF} Gen}{9}{TODO}
    \trschaetzung{Anhangs Manager}{9}{TODO}
    \trschaetzung{Anhangs Lieferschin}{3}{TODO}
    \trschaetzung{Dynamischer Reisekosten helper}{12}{TODO}
    \trschaetzung{Konzept Dynamiche konfigurations Layout}{15}{TODO}
    \trschaetzung{Config to Frontend Layout link System}{24}{TODO}
    \trschaetzung{Datums und Uhrzeit Feld}{6}{TODO}
    \trschaetzung{Genric Text}{3}{TODO}
    \trschaetzung{Adress Feld}{3}{TODO}
    \trschaetzung{Feld Gruppirung Auto Gen}{30}{TODO}
    \trschaetzung{Geld Feld}{3}{TODO}
    \trschaetzung{Boolean Feld}{3}{TODO}
    \trschaetzung{Tabllen Abrechnungs Feld}{12}{TODO}
    \trschaetzung{IBAN Feld}{12}{TODO}
    \trschaetzung{Von Bis Datumsfeld}{6}{TODO}
    \trschaetzung{FS-WE Kosten Kategorie Element}{12}{TODO}
    \trschaetzung{Telnemer ListenElement}{6}{TODO}
    \trschaetzung{Generisches Text Listen Element}{6}{TODO}
    \trschaetzung{Weiterführendes UI Design}{21}{TODO}
    \trschaetzung{Doku Meilenstein1 Zeitpanung TextForm}{6}{TODO}
\end{longtable}\label{tab:table}