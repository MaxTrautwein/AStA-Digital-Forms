\chapter{Aufwandsschätzung}\label{ch:aufwandsschatzung}
% Helper Command to format entrys in the Table %
\newcommand{\trschaetzung}[3]{\rowcolor{lightgray}\multicolumn{1}{|l|}{\textbf{#1}} & \textbf{#2} \\* \multicolumn{2}{|l|}{\begin{tabular}[c]{@{}l@{}}#3\end{tabular}} \\ \hline}

\section{Aufwandsschätzung}\label{sec:aufwandsschatzung}

In einer Aufwandsschätzung wird offengelegt, wie viel Zeit das Projekt benötigt und wie lange einzelne Tasks benötigen.
Eine Aufwandsschätzung ist eine grobe Schätzung und meist ungenau. Diese wird benötigt, um vorab einen Überblick zu haben.\\
In diesem Projekt wird zwischen Gruppenarbeiten und einzelnen Tasks unterschieden. Einzelne Tasks umfassen die Arbeit, die eine Person aufwendet, um den Task abzuschließen und eine Schätzung erfolgt dabei mit der 1:3 Regel. Es wird zwischen dem rohen Aufwand für das schreiben des Codes, dem Testing und der Dokumentation unterschieden, daher wird die Zeit für den rohen Aufwand mit dem Faktor 3 multipliziert.\\
Bei Gruppenarbeiten wird die Zeit angegeben, bei der alle Gruppenmitglieder beteiligt sind und ausschliesslich nur in der Gruppe durchgeführt werden können. 

\section{Gruppenleistungen}\label{sec:gruppenleistungen}

% Add Text Explaining this Category %

\begin{longtable}{|lr|}
    \hline
    \rowcolor{darkgray}\multicolumn{1}{|l|}{\textbf{Aufgabe}} & \textbf{\begin{tabular}[c]{@{}c@{}}Zeitschätzung\\\(Stunden\)\end{tabular}} \\ \hline
    \endhead
    \trschaetzung{Betreuer Meeting 1}{1.5}{Das Erste Treffen mit unserem Betreuer}
    \trschaetzung{Kunden Treffen 1}{1}{Das Erste Traffen mit Unseren Kunden}
    \trschaetzung{Regeltreffen}{6}{Aufsummirte Zeit der geplanten Regelmeetings mit den Kunden sowie dem Betreuer}
    \trschaetzung{Präsentation und Disputation}{8}{Pflicht Seminar am Samstag den 13.04.2024}
    \trschaetzung{Teambildung und Konfliktlösung}{8}{Pflicht Seminar am Samstag den 6.04.2024}
    \trschaetzung{Präsentation}{1}{Präsentations Zeit}
    \trschaetzung{Präsentation vorbrereiten}{12}{Vorbereitungszeit auf die Präsentationen.
        \\Umfasst das Erstellen, verfeinern sowie Einproben.}
    \trschaetzung{Angular Testen}{5}{Sipke um sich mit Angular vertraut zu machne und zu prüfen ob das
        \\Team damit effektiv arbeiten kann}
    \trschaetzung{Erste Team Festlegungen}{1}{Interne Besprechung grundlegenden Themen wie:
        \\Lizens, Github-flow, Repository Struktur, Dokumentations Format}
    \trschaetzung{Aufwandsschätzung}{8}{Zeit zum Schätzen den Aufwandes für alle einzelenen Tasks
        \\Enthält auch Zeit welche zur entwiklung von Requierments aufgewendet wurde\\ um Tasks besser aufzuteilen.}
    \trschaetzung{Internes Treffen 30.3}{2}{Beschluss über Angular Wahl für das Frontend,
        \\Terminfindung für Agile Regelmeetings im Team,
        \\erste descussionen zur technologie wahl fürs Backend sowie der Datenbank}
    \trschaetzung{Sprint-Plan}{12}{Aufsummirte Zeit um Sprintplannings durchzuführen}
    \trschaetzung{Protokoll}{6}{Aufsummirte Zeit zum aufbereiten der Protokolle zu Meetings}
    \trschaetzung{Reflektion Projektmanagement}{3}{Zeit für die Reflektion über das Projektmanagement am ende des Projekts}
    \trschaetzung{Reflektion Lernfortschritt \& Doku}{4}{Zeit für die Reflektion über den Lernfortschritt am ende des Projekts}
    \trschaetzung{Spike \ac{PDF} Generator wahl}{3}{Spike um die Optionen zur automatisireten Generation von \ac{PDF} dokumenten zu evaluieren.}
    \trschaetzung{Style Guyde Auswhal}{3}{Festlegung von Interen Style Richtlinien für besssere Code Qualität und Verständniss}
\end{longtable}

\section{Individualleistungen}\label{sec:individualleistungen}

% Add Text explaining that Category %

\begin{longtable}{|lr|}
    \hline
    \rowcolor{darkgray}\multicolumn{1}{|l|}{\textbf{Aufgabe}} & \textbf{\begin{tabular}[c]{@{}c@{}}Zeitschätzung\\\(Stunden\)\end{tabular}} \\ \hline
    \endhead
    \trschaetzung{Docker Compose Setup}{6}{Follständiges Docker Compose Setup für alle Module
        \\Sowie Veifikation auf allen Team Systemen}
    \trschaetzung{Klickdummie bauen und testen}{15}{Entwickeln Testen eines Dummies der Applikation
        \\dies soll fühzeitig probleme aufdecken und die Qualität der \ac{UI} verbessern}
    \trschaetzung{Interviews Fragen}{4.5}{Erstellen von Interview Fragen and die verschidenen Stakeholder.}
    \trschaetzung{Interviews Durchfüren}{6}{Interview Durführung und Dokumentation}
    \trschaetzung{User Storys entwerfen}{9}{Userstorys entwikeln}
    \trschaetzung{Doku Meilenstein 1 - \ac{UI} Design}{6}{Entwurf erster Mockups, Dokumentation dazu}
    \trschaetzung{Doku Meilenstein 1 - Archiketur}{7.5}{Architiktur Entworf mit Dokumentation}
    \trschaetzung{Doku Meilenstein 1 - Projektmanagment}{6}{Dokumentation zu dem eigenen Projektmanagement}
    \trschaetzung{Doku Meilenstein 1 - Randbedingungen}{3}{Dokumentation der Technischen sowie Organistatorischen Randbedingungen}
    \trschaetzung{Doku Meilenstein 1 - Einführung und Ziele}{6}{Dokumenation der Aufgabenstellung sowie der Ziele}
    \trschaetzung{Repo Init}{1}{Inizialisrung  des GitHub Repositorys}
    \trschaetzung{Doku Init}{0.5}{Inizialisrung der LaTeX Dokumentationsvorlage}
    \trschaetzung{Datenbankmodell}{6}{Erstellung schlüssigen eines Datenbankmodells}
    \trschaetzung{Softwarearchitektur inklusive unterschiedlicher (logischer) Sichten}{15}{Detailirte Softwarearchitektur auf den Tieferen schichten}
    \trschaetzung{Doku Meilenstein 2 - Verwendete Technologien / Frameworks}{6}{Dokumenation zu den Gewählten Technoligen sowie Frameworks}
    \trschaetzung{Verwendete Schnittstellentechnologie}{6}{Dokumenattion der Schnittstellen und deren Technologie}
    \trschaetzung{Installations- und Administrationshandbuch}{15}{Detailirtes Installations- und Administrationshandbuch\\
    Entält erklärungen zu den Konigurations Dateien}
    \trschaetzung{Aufteilung des Teams}{3}{Klare dokumentation welches Teammitglid welche Tätichkeit übernommen hat}
    \trschaetzung{Doku Meilenstein X - Reflektion Projektmanagement}{6}{Dokumentation der eigenen Reflektion über das Projektmanagement}
    \trschaetzung{Lizenzen: verwendete Lizenzen (Fremdcode: Frameworks, Libraries)}{6}{Dokumentation aller verwendeten Lizenzen}
    \trschaetzung{Ausblick}{6}{Dokumenation des Aublicks am ende Des Projekts.
        \\Was sind nächste schritte? Was kann weiter verbessert werden? Anwendung?}
    \trschaetzung{Favoriten}{21}{Favoriten System mit \ac{GUI},
        \\Einstellungen zur Automatischen Erkennung und Automatik Funktion}
    \trschaetzung{Antrags Beschribungen Erstellen}{9}{Erstellung von Sinfollen Bescheibungen der Aktuellen Anträge}
    \trschaetzung{Vollständigkeitskontrolle}{9}{Funktion um zu Prüfen ob der Antrag follständig ausgefüllt wurden
        \\ Sowie sinfolle anzweige was noch aussteht}
    \trschaetzung{Kategorisieren und Taggen Von Anträgen}{9}{System zum Kategorisiren und Taggen der Anträge in der Konfiguration}
    \trschaetzung{Filter System}{12}{\ac{GUI} Sytem zum Filtern von Antrgägen basiren auf bestehendn Kategorien und Tags.}
    \trschaetzung{Auswhals Helfer - Konfigurations System}{30}{Auswhal Sytem zum Finden von dem Passenden Antrag.
        \\Umfasst Entwiklung der Fragen und Beschribungen sowie des \ac{UI}.
        \\Entwiklung von Konfigurationssystem um dies auch bei geänderten Anträgen zu ermöglichen.}
    \trschaetzung{Formular Felder Kompatibilität markieren}{6}{Um Inhalten von Anträgen auf deren Abrechnung zu übrtragen
        \\Müssen kompatible felder in der Konfiguration markirt werden.}
    \trschaetzung{Formular Fortschritt Speichern}{6}{Speichern dess Fortschritts inehalb eines Antrags
        \\Umgabng mit Anhängen}
    \trschaetzung{Formular Fortschritt Laden}{12}{\ac{GUI} und Sytem zum Laden von Gespeicherten Anträgen
        \\Umfasst unfertige sowie alte komplette Anträge}
    \trschaetzung{Hinweis System}{3}{Generelles System um Nutzer auf Zusamenhänge hinzuweisen
        \\\zb Nötige begründung bei wahl einer Teuereren Option}
    \trschaetzung{Keycloak Einrichten}{4.5}{Konfigurations zeit für Keycloak.
        \\Generell, Testuser, Richtlinien sowie dem Anmeldeflow}
    \trschaetzung{Keycloak Anmeldung Frontend}{6}{Managment im Frontend um einen JWT Token von Kekyclok für den User zu Erhalten}
    \trschaetzung{Keycloak Verifikation}{6}{Backend Logik um den JWT Token zu verifizieren und einem Nutzer zuzuordnen}
    \trschaetzung{Backend Datenbank Interface}{12}{Entwiklung von grundlegendem Interface mit der Datenbank.}
    \trschaetzung{Formular Autofill Option}{12}{Automatischen Ausfüllen von felden nach spezifikation des Nutzers
        \\Nutzer hat kontrolle darüber was zur weiderverwendung aufgehoben wird
        \\Sowie kontrolle über was eingefügt wird}
    \trschaetzung{Routenberechnung \ac{API}}{30}{Routenberechnung über gegebene Punkte, Kilomenter angaben für Teilstrecken
        \\Visuelle representation der Strecke in Grafik}
    \trschaetzung{Adressvervollständigung \ac{API}}{15}{Adressverfollständigungs funktionalität}
    \trschaetzung{Formular \ac{PDF} Vorlagen Erstellen}{36}{Vorlagen für die PDF Generation Erstellen.
        \\Für alle derzeitigen Anträge wobei das ergebniss dem derzeitigen Stand gleichen soll.
        \\Prüfung von Edge Cases mit viel und wenig Inhalt}
    \trschaetzung{link 2 \ac{PDF} Generator}{6}{Aus einem Link auf eine Website automatisch einen \ac{PDF} Anhang generieren.}
    \trschaetzung{Routen Plan \ac{PDF} Gen}{9}{Die Routenplanung in \ac{PDF} form festhalten.
        \\Darstellung von verchidenen Routen Typen und verschidensten Zwischenstopps}
    \trschaetzung{Anhangs Manager}{9}{Anhangs Management Funktion für einen Antrag.}
    \trschaetzung{Anhangs Lieferschin}{3}{Dynamische Erstellung und Beritstellung des Lieferscheins}
    \trschaetzung{Dynamischer Reisekosten helper}{12}{System zum berechnen von Resekosten.
        \\Dabei ist auf flexibilität des Systems für verschidene berechnungsarten zu achten.
        \\Parameter sollen möglichts einfach anpassbar sein}
    \trschaetzung{Konzept Dynamiche konfigurations Layout}{15}{Grundkonzept für die Dynamiche Konfiguration
        \\Planlastige aufgabe um vor und Nachteile für alle teilkomponenten zu berücksichtigen}
    \trschaetzung{Config to Frontend Layout link System}{24}{konfiguations Systemkomponente die das Layout im Frontend Steuret
        \\Notwendig um Anträge Tazächlich konfiguriebar zu machnen.
        \\Erfordert einge zusammenarbeit mit dem \ac{UI} design sowie sinfolle Parameter wahl}
    \trschaetzung{Datums und Uhrzeit Feld}{6}{Gnereiches Feld zum auswählen von Datum und oder Uhrzeit
        \\Nähere einstellung zu was und unter welchen bedingungen über die Konfiguration}
    \trschaetzung{Genric Text}{3}{Einzel oder mehrzeiliges Textfeld}
    \trschaetzung{Adress Feld}{3}{Feld für die Eingabe von Addressen}
    \trschaetzung{Feld Gruppirung Auto Gen}{30}{Für die umsezung bestimmter Funktionalitäten müssen die eingaben von
        \\Feldenrn in einer Gruppe betrachtet werden.
        \\Eine sinfolle Umsezung hierzu erforderd bedacht.}
    \trschaetzung{Geld Feld}{3}{Eingabefeld zu Notieren von Geldbeträgen}
    \trschaetzung{Boolean Feld}{3}{Eingabefeld in verschdenen formen welches ein Boolean wärt enthält}
    \trschaetzung{Tabllen Abrechnungs Feld}{12}{Generisches Abrechnungs Feld mit verschiedenen Zusammenhängen:
    \\Bescheibung, Summe abzüglich aller Nicht ersttbarer Teile sowie des passenden Belegs}
    \trschaetzung{IBAN Feld}{12}{IBAN Eingabe Feld mit Richtigskeit Prüfung}
    \trschaetzung{Von Bis Datumsfeld}{6}{datums Feld welches eine klar von bis Logik implementirt
        \\Option zur konfiguration von wechen Maximaldauern
        \\Wahlweise mit und ohne Uhrzeit}
    \trschaetzung{FS-WE Kosten Kategorie Element}{12}{Spezialisirtes Abrechnungsfeld für Fachschaftswochenenden
        \\Aüfschlüsselung und erhalt der bestehenden Kategorien }
    \trschaetzung{Telnemer ListenElement}{6}{Element für die Erstellung von Teilnehmerlisten}
    \trschaetzung{Generisches Text Listen Element}{6}{Generisches Listen Elemnet}
    \trschaetzung{Weiterführendes \ac{UI} Design}{21}{Weiterewntwiklung des \ac{UI} Designs über den Klickdummie hinaus}
    \trschaetzung{Doku Meilenstein1 Zeitpanung TextForm}{6}{Die Zeitplanung welche im Team erstellt wurden,
        \\in die Dokumentation mit erklärenden Beschreibungen einfügen.
        \\Übertragen der erstellten Requierments}
\end{longtable}\label{tab:table}