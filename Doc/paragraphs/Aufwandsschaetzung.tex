\chapter{Aufwandsschätzung}\label{ch:aufwandsschatzung}
% Helper Command to format entrys in the Table %
\newcommand{\trschaetzung}[3]{\rowcolor{lightgray}\textbf{#1} & \textbf{#2} \\* \multicolumn{2}{|l|}{\begin{tabular}[c]{@{}l@{}}#3\end{tabular}} \\ \hline}

\section{Definition}\label{sec:aufwandsschatzungdef}

In einer Aufwandsschätzung wird offengelegt, wie viel Zeit das Projekt benötigt und wie lange einzelne Tasks benötigen.
Eine Aufwandsschätzung ist eine grobe Schätzung und meist ungenau. Diese wird benötigt, um vorab einen Überblick zu haben.
In diesem Projekt wird zwischen Gruppenarbeiten und einzelnen Tasks unterschieden. 

\section{Gruppenleistungen}\label{sec:gruppenleistungen}

Bei Gruppenarbeiten wird die Zeit angegeben, bei der \textbf{alle} Gruppenmitglieder beteiligt sind und ausschliesslich nur in der Gruppe durchgeführt werden können. 

\begin{longtable}{|lr|}
    \hline
    \rowcolor{gray}\textbf{Aufgabe} & \textbf{\begin{tabular}[c]{@{}c@{}}Zeitschätzung\\\(Stunden\)\end{tabular}} \\ \hline
    \endhead
    \trschaetzung{Betreuer Meeting 1}{1.5}{Das Erste Treffen mit unserem Betreuer}
    \trschaetzung{Kunden Treffen 1}{1}{Das Erste Treffen mit Unseren Kunden}
    \trschaetzung{Regeltreffen}{6}{Aufsummirte Zeit der geplanten Regelmeetings mit den Kunden sowie dem Betreuer}
    \trschaetzung{Präsentation und Disputation}{8}{Pflicht Seminar am Samstag den 13.04.2024}
    \trschaetzung{Teambildung und Konfliktlösung}{8}{Pflicht Seminar am Samstag den 6.04.2024}
    \trschaetzung{Präsentation}{1}{Präsentations Zeit}
    \trschaetzung{Präsentation vorbereiten}{12}{Vorbereitungszeit auf die Präsentationen.
    \\Umfasst das Erstellen, Verfeinern sowie Einproben.}
    \trschaetzung{Angular Testen}{5}{Sipke um sich mit Angular vertraut zu machen und zu prüfen ob das
    \\Team damit effektiv arbeiten kann}
    \trschaetzung{Erste Team Festlegungen}{1}{Interne Besprechung grundlegenden Themen wie:
    \\Lizenz, Github-flow, Repository Struktur, Dokumentations Format}
    \trschaetzung{Aufwandsschätzung}{8}{Zeit zum Schätzen den Aufwandes für alle einzelnen Tasks
    \\Enthält auch Zeit, welche zur Entwicklung von Requirements aufgewendet wurde\\ um Tasks besser aufzuteilen.}
    \trschaetzung{Internes Treffen 30.3}{2}{Beschluss über Angular Wahl für das Frontend,
        \\Terminfindung für Agile Regelmeetings im Team,
        \\erste diskussionen zur technologie wahl fürs Backend sowie der Datenbank}
    \trschaetzung{Sprint-Plan}{12}{Aufsummierte Zeit um Sprint Plannings durchzuführen}
    \trschaetzung{Protokoll}{6}{Aufsummierte Zeit zum aufbereiten der Protokolle zu Meetings}
    \trschaetzung{Reflektion Projektmanagement}{3}{Zeit für die Reflektion über das Projektmanagement am ende des Projekts}
    \trschaetzung{Reflektion Lernfortschritt \& Doku}{4}{Zeit für die Reflektion über den Lernfortschritt am ende des Projekts}
    \trschaetzung{Spike \ac{PDF} Generator wahl}{3}{Spike um die Optionen zur automatisierten Generation von \ac{PDF} dokumenten zu evaluieren.}
    \trschaetzung{Style Guide Auswahl}{3}{Festlegung von Internen Style Richtlinien für bessere Codequalität und Verständnis}
\end{longtable}

\newpage\section{Individualleistungen}\label{sec:individualleistungen}

Individualleistungen umfassen die Arbeit, die überwiegend \textbf{eine} Person aufwendet, um den Task abzuschließen und eine Schätzung erfolgt dabei mit der 1:3 Regel. Es wird zwischen dem rohen Aufwand für das schreiben des Codes, dem Testing und der Dokumentation unterschieden, daher wird die Zeit für den rohen Aufwand mit dem Faktor 3 multipliziert.


\begin{longtable}{|lr|}
    \hline
    \rowcolor{gray}\textbf{Aufgabe} & \textbf{\begin{tabular}[c]{@{}c@{}}Zeitschätzung\\\(Stunden\)\end{tabular}} \\ \hline
    \endhead
    \trschaetzung{Docker Compose Setup}{6}{Vollständiges Docker Compose Setup für alle Module
    \\Sowie Verifikation auf allen Team Systemen}
    \trschaetzung{Klickdummie bauen und testen}{15}{Entwickeln Testen eines Dummies der Applikation
    \\dies soll frühzeitig probleme aufdecken und die Qualität der \ac{UI} verbessern}
    \trschaetzung{Interviews Fragen}{4.5}{Erstellen von Interview Fragen an die verschidenen Stakeholder.}
    \trschaetzung{Interviews Durchführen}{6}{Interview Durchführung und Dokumentation}
    \trschaetzung{User Storys entwerfen}{9}{Userstorys entwikeln}
    \trschaetzung{Doku Meilenstein 1 - \ac{UI} Design}{6}{Entwurf erster Mockups, Dokumentation dazu}
    \trschaetzung{Doku Meilenstein 1 - Archiketur}{7.5}{Architiktur Entworf mit Dokumentation}
    \trschaetzung{Doku Meilenstein 1 - Projektmanagment}{6}{Dokumentation zu dem eigenen Projektmanagement}
    \trschaetzung{Doku Meilenstein 1 - Randbedingungen}{3}{Dokumentation der Technischen sowie Organisatorische Randbedingungen}
    \trschaetzung{Doku Meilenstein 1 - Einführung und Ziele}{6}{Dokumentation der Aufgabenstellung sowie der Ziele}
    \trschaetzung{Repo Init}{1}{Inizialisrung  des GitHub repositories}
    \trschaetzung{Doku Init}{0.5}{Initialisierung der LaTeX Dokumentationsvorlage}
    \trschaetzung{Datenbankmodell}{6}{Erstellung schlüssigen eines Datenbankmodells}
    \trschaetzung{Softwarearchitektur inklusive unterschiedlicher (logischer) Schichten}{15}{Detaillierte Softwarearchitektur auf den Tieferen schichten}
    \trschaetzung{Doku Meilenstein 2 - Verwendete Technologien / Frameworks}{6}{Dokumentation zu den Gewählten Technologien sowie Frameworks}
    \trschaetzung{Verwendete Schnittstellentechnologie}{6}{Dokumentation der Schnittstellen und deren Technologie}
    \trschaetzung{Installations- und Administrationshandbuch}{15}{Detailirtes Installations- und Administrationshandbuch\\
    Enthält erklärungen zu den Konfigurations Dateien}
    \trschaetzung{Aufteilung des Teams}{3}{Klare dokumentation welches Teammitglied welche Tätigkeit übernommen hat}
    \trschaetzung{Doku Meilenstein X - Reflektion Projektmanagement}{6}{Dokumentation der eigenen Reflektion über das Projektmanagement}
    \trschaetzung{Lizenzen: verwendete Lizenzen (Fremdcode: Frameworks, Libraries)}{6}{Dokumentation aller verwendeten Lizenzen}
    \trschaetzung{Ausblick}{6}{Dokumentation des Ausblicks am ende Des Projekts.
    \\Was sind die nächsten Schritte? Was kann weiter verbessert werden? Anwendung?}
    \trschaetzung{Favoriten}{21}{Favoriten System mit \ac{GUI},
        \\Einstellungen zur Automatischen Erkennung und Automatik Funktion}
    \trschaetzung{Antrags Beschreibungen Erstellen}{9}{Erstellung von Sinnvollen Beschreibungen der Aktuellen Anträge}
    \trschaetzung{Vollständigkeitskontrolle}{9}{Funktion um zu Prüfen ob der Antrag vollständig ausgefüllt wurden
    \\ Sowie sinnvolle anzeige was noch aussteht}
    \trschaetzung{Kategorisieren und Taggen Von Anträgen}{9}{System zum Kategorisieren und Taggen der Anträge in der Konfiguration}
    \trschaetzung{Filter System}{12}{\ac{GUI} System zum Filtern von Anträgen basieren auf bestehenden Kategorien und Tags.}
    \trschaetzung{Auswhals Helfer - Konfigurations System}{30}{Auswahlsystem zum Finden von dem Passenden Antrag.
    \\Umfasst die Entwicklung der Fragen und Beschreibungen sowie des \ac{UI}.
    \\Entwicklung von Konfigurationssystem, um dies auch bei geänderten Anträgen zu ermöglichen.}
    \trschaetzung{Formular Felder Kompatibilität markieren}{6}{Um Inhalten von Anträgen auf deren Abrechnung zu übertragen
    \\Müssen kompatible Felder in der Konfiguration markiert werden.}
    \trschaetzung{Formular Fortschritt Speichern}{6}{Speichern des Fortschritts innerhalb eines Antrags
    \\Umgang mit Anhängen}
    \trschaetzung{Formular Fortschritt Laden}{12}{\ac{GUI} und System zum Laden von Gespeicherten Aufträgen
    \\Umfasst unfertige sowie alte komplette Anträge}
    \trschaetzung{Hinweis System}{3}{Generelles System um Nutzer auf Zusammenhänge hinzuweisen
    \\\zB Nötige begründung bei wahl einer Teureren Option}
    \trschaetzung{Keycloak Einrichten}{4.5}{Konfigurations zeit für Keycloak.
    \\Generell, Testuser, Richtlinien sowie dem Anmelde Flow}
    \trschaetzung{Keycloak Anmeldung Frontend}{6}{Managment im Frontend um einen JWT Token von Keycloak für den User zu Erhalten}
    \trschaetzung{Keycloak Verifikation}{6}{Backend Logik um den JWT Token zu verifizieren und einem Nutzer zuzuordnen}
    \trschaetzung{Backend Datenbank Interface}{12}{Entwiklung von grundlegendem Interface mit der Datenbank.}
    \trschaetzung{Formular Auto Fill Option}{12}{Automatischen Ausfüllen von feldern nach spezifikation des Nutzers
    \\Nutzer hat kontrolle darüber was zur wiederverwendung aufgehoben wird
    \\Sowie kontrolle über was eingefügt wird}
    \trschaetzung{Routenberechnung \ac{API}}{30}{Routenberechnung über gegebene Punkte, Kilometerangaben für Teilstrecken
    \\Visuelle repräsentation der Strecke in Grafik}
    \trschaetzung{Adressvervollständigung \ac{API}}{15}{Adressverfollständigungs funktionalität}
    \trschaetzung{Formular \ac{PDF} Vorlagen Erstellen}{36}{Vorlagen für die PDF Generation Erstellen.
    \\Für alle derzeitigen Anträge, wobei das Ergebnis dem derzeitigen Stand gleichen soll.
    \\Prüfung von Edge Cases mit viel und wenig Inhalt}
    \trschaetzung{link 2 \ac{PDF} Generator}{6}{Aus einem Link auf eine Website automatisch einen \ac{PDF} Anhang generieren.}
    \trschaetzung{Routen Plan \ac{PDF} Gen}{9}{Die Routenplanung in \ac{PDF} form festhalten.
    \\Darstellung von verschiedenen Routen Typen und verschiedensten Zwischenstopps}
    \trschaetzung{Anhangs Manager}{9}{Anhangs Management Funktion für einen Antrag.}
    \trschaetzung{Anhangs Lieferschein}{3}{Dynamische Erstellung und Bereitstellung des Lieferscheins}
    \trschaetzung{Dynamischer Reisekosten-Helper}{12}{System zum Berechnen von Reisekosten.
    \\Dabei ist auf die Flexibilität des Systems für verschiedene Berechnungsarten zu achten.
    \\Parameter sollten möglichst einfach anpassbar sein}
    \trschaetzung{Konzept dynamisches Konfigurationslayout}{15}{Grundkonzept für die dynamische Konfiguration
    \\Planungsintensive Aufgabe, da Vor- und Nachteile für alle Teilkomponenten berücksichtigt werden müssen}
    \trschaetzung{Config to Frontend Layout link System}{24}{Konfigurations Systemkomponente die das Layout im Frontend steuert
    \\Notwendig um Anträge konfigurierbar zu machen.
    \\Erfordert enge Zusammenarbeit mit dem \ac{UI} Design sowie eine sinnvolle Parameterwahl}
    \trschaetzung{Datums und Uhrzeit Feld}{6}{Generisches Feld zum Auswählen von Datum und oder Uhrzeit.
    \\Nähere Einstellung durch die Konfiguration}
    \trschaetzung{Genric Text}{3}{Einzel oder mehrzeiliges Textfeld}
    \trschaetzung{Adress Feld}{3}{Feld für die Eingabe von Adressen}
    \trschaetzung{Feld Gruppierung autogeneriert}{30}{Für die Umsetzung bestimmter Funktionalitäten müssen die Eingaben von
    \\Feldern in einer Gruppe betrachtet werden.
    \\Eine sinnvolle Umsetzung hiervon ist komplex.}
    \trschaetzung{Geld Feld}{3}{Eingabefeld zum Erfassen von Geldbeträgen}
    \trschaetzung{Boolean Feld}{3}{Eingabefeld in verschiedenen Formen welches einen boolschen Wert enthält}
    \trschaetzung{Tabllen Abrechnungs Feld}{12}{Generisches Abrechnungs Feld mit verschiedenen Zusammenhängen:
    \\Beschreibung, Summe abzüglich aller nicht erstattbarer Teile, sowie des passenden Belegs}
    \trschaetzung{IBAN Feld}{12}{IBAN Eingabefeld mit Richtigkeitsprüfung}
    \trschaetzung{Von-Bis Datumsfeld}{6}{Datumsfeld, welches eine klare Von-Bis-Logik implementiert.
    \\Option zur Konfiguration der Maximaldauer
    \\Wahlweise mit oder ohne Uhrzeit}
    \trschaetzung{FS-WE Kostenkategorie Element}{12}{Spezialisiertes Abrechnungsfeld für Fachschaftswochenenden
    \\Aufschlüsselung und erhalt der bestehenden Kategorien }
    \trschaetzung{Telnemer ListenElement}{6}{Element für die Erstellung von Teilnehmerlisten}
    \trschaetzung{Generisches Text Listen Element}{6}{Generisches Listen Element}
    \trschaetzung{Weiterführendes \ac{UI} Design}{21}{Weiterentwicklung des \ac{UI} Designs über den Klickdummie hinaus}
    \trschaetzung{Doku Meilenstein 1 Zeitplanung Text Form}{6}{Die Zeitplanung, welche im Team erstellt wurde,
        \\in die Dokumentation mit erklärenden Beschreibungen einfügen.
        \\Übertragen der erstellten Requirements}
\end{longtable}\label{tab:table}