\chapter{Architektur}\label{ch:architektur}

\section{Überblick}\label{sec:uberblick}
Wie in \refa{fig:HighLevelArch} zu sehen ist, wird hier eine klassische Web-App Architektur angewendet,
welche um einen Authentifizierungsdienst erweitert wird.\\
Das Frontend übernimmt die Aufgabe der Darstellung und Interaktion mit dem benutzer.
Dabei interagiert dieses mit dem Backend über eine \ac{REST} Schnittstelle sowie dem Authentifizierungsdienst.\\
Das Backend stellt dem Frontend Daten berit und interagiert mit einem Datenspeicher.\\
Der Authentifizierungsdienst ist ein externer Dienst der verwendet wir, um die Nutzer Accounts zu managen.
Über diesen erhält das Frontend einen \ac{JWT} Token, welcher den Nutzer gegenüber dem Backend Authentifizieren kann.
Dazu stellt das Backend mithilfe des \ac{JWT} Tokens eine entsprechende anfrage an den Authentifizierungsdienst.

\begin{figure}[h]
    \includesvg[width=10cm]{images/HighLevelArch}
    \caption{Architektur überblick}\label{fig:HighLevelArch}
\end{figure}

\section{Fokus: Backend}\label{sec:fokus:-backend}

Während die innere Struktur des Frontends stark vom verwendeten Framework abhängt und an diese Stelle nicht näher beschrieben werden kann,
ist es für das Backend durchaus möglich.
\refa{fig:BackendArch} Zeigt den strukturellen Aufbau des Backends Näher, dabei

\begin{figure}[h]
    \includesvg[width=15cm]{images/BackendArch}
    \caption{Backend Architektur}\label{fig:BackendArch}
\end{figure}