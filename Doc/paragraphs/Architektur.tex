\chapter{Architektur}\label{ch:architektur}
Wie in \refa{fig:HighLevelArch} zu sehen ist, wird hier eine Web-App Architektur angewendet,
welche um einen Authentifizierungsdienst erweitert wird.\\
Das Frontend übernimmt die Aufgabe der Darstellung und Interaktion mit dem Benutzer.
Dabei interagiert dieses mit dem Backend über eine \ac{REST} Schnittstelle, sowie dem Authentifizierungsdienst.\\
Das Backend stellt dem Frontend Daten bereit und interagiert mit einem Datenspeicher.\\
Der Authentifizierungsdienst ist ein externer Dienst, um die Nutzer Accounts zu managen.
Über diesen erhält das Frontend einen \ac{JWT}, welcher den Nutzer gegenüber dem Backend authentifiziert.
Dazu stellt das Backend eine entsprechende Anfrage an den Authentifizierungsdienst mit dem \ac{JWT}.

\begin{figure}[h]
    \centering
    \includesvg[width=8cm]{images/HighLevelArch}
    \caption{Architektur überblick}\label{fig:HighLevelArch}
\end{figure}

\section{Frontend}\label{sec:frontend}

Die innere Struktur des Frontends ist stark vom verwendeten Framework abhängig.
Daher ist an diese Stelle keine näher Beschreibung möglich.

\section{Backend}\label{sec:fokus:-backend}

\refa{fig:BackendArch} Zeigt den strukturellen Aufbau des Backends Näher, dabei kann die grafik in vier zonen unterteilt werden:
\begin{itemize}
    \item Links (grün): Kommunikation mit Webdiensten sowie dem eigenen Frontend.
    \item Unten (gelb): Interface mit Datenspeicher.
    \item Rechts (blau): zugriff auf Konfigurationsdateien.
    \item Zentral (lila): die Verarbeitung aller Teilkomponenten
\end{itemize}

\begin{figure}[h]
    \includesvg[width=17cm]{images/BackendArch}
    \caption{Backend Architektur}\label{fig:BackendArch}
\end{figure}

Die Kommunikation mit den Webdiensten lässt sich in zwei Teilmodule, unterteilen:
\begin{itemize}
    \item \ac{REST} \ac{API}
    \item Authentifizierungsprüfung
\end{itemize}
Dabei stellt die \ac{REST} \ac{API} eine klare Schnittstelle bereit und greift auf die Authentifizierungsprüfung zu,
um die Berechtigungen des Nutzers zu prüfen.

Das Interface mit den Datenspeichern entkoppelt die Anwendung sowie deren Daten von dem verwendeten Speicherdienst.
Dabei wird zwischen "Core Daten" welche für die funktionalität zwingend notwendig benötigt werden
und "User Daten" welche lediglich die bedingung sowie handhabung des Nutzers verbessern.\\
"Core Daten" werden dabei immer auf einer eigenen Datenbank gespeichert.
Hierbei ist die Datenbank Technologie selbst durch das Datenbank Interface entkoppelt.\\
"User Daten" hingegen könnten an anderer Stelle gespeichert werden.
Diese Möglichkeit wird hier zwar eindeutig berücksichtigt, jedoch wird ein dementsprechendes Interface nicht explizit entwickelt.

Prozess Konfigurationen beinhalten all jene, welchen den Fluss und Inhalt eines Antrags sowie dessen Export beschreiben.
Diese weisen eine hohe Komplexität auf und können nicht unbedingt ohne weiteres bearbeitet werden.
Regel Konfigurationen hingegen sind einfach gestaltet und sollten nach einer Erklärung aus der Anleitung leicht anpassbar sein.
Diese enthalten beispielsweise Parameter zur Berechnung von Reisekosten.

Zentral steht die Verarbeitung der außenstehenden Module.
Dabei werden die Informationen der Konfigurationsdateien ausgewertet und kombiniert.
