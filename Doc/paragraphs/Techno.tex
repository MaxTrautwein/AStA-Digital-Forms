\chapter{Technologien & Frameworks}\label{ch:technologien-&-frameworks}

In diesem Kapitel sind die verwendeten Technologien und Frameworks benannt, erklärt
sowie begründet, weshalb diese verwendet werden.

\section{Angular}\label{sec:angular}

Im Frontend wir Angular für die komponentenbasierte Darstellung verwendet.
Da in Angular die einzelnen Komponenten direkt in solche unterteilt sind,
eignet es sich für den angestrebten modularen Aufbau besonders.

Angular ist eine auf \gls{TypeScript} basierende Entwicklungsplattform, die folgende Funktionalitäten umfasst:
\begin{itemize}
    \item Komponenten basiertes Framework
    \item Eine Vielzahl von stark integrierten Bibliotheken für Routing, Formularmanagement, Client-Server Kommunikation
    \item Entwicklertools zum Entwickeln, Testen und Updaten des Codes
\end{itemize}
\cite{about-angular}

Angular ist MIT lizenziert.
Für mehr Details siehe \refk{sec:mit}.

\subsection{Direkte Abhängigkeiten}

\paragraph{zone.js}
Ist Teil von Angular und somit ebenfalls MIT lizenziert.
Für mehr Details siehe \refk{sec:mit}.

\paragraph{angular-oauth2-oidc}
OAuth 2 Bibliothek für Angular.
Ist MIT lizenziert.
Für mehr Details siehe \refk{sec:mit}.

\paragraph{bootstrap / ngx-bootstrap}
Styling Framework, welches MIT lizenziert ist.
Für mehr Details siehe \refk{sec:mit}.

\paragraph{cors}
Für Optionen in Zusammenhang mit CORS.
Ist MIT lizenziert.
Für mehr Details siehe \refk{sec:mit}.

\paragraph{rxjs}
Bibliothek für reaktiven JavaScript Code
Ist Apache-2.0 Lizenziert.
Für mehr Details siehe \refk{sec:apache-2.0}.

\paragraph{TypeScript}
Ist Apache-2.0 Lizenziert.
Für mehr Details siehe \refk{sec:apache-2.0}.

\paragraph{tslib}
Laufzeit Bibliothek für TypeScript
Ist unter 0BSD(Zero-Clause BSD) lizenziert.
Für mehr Details siehe \refk{sec:0bsd-(zero-clause-bsd)}.


\paragraph{jasmine-core}
Type Script Test Bibliothek.
Ist MIT lizenziert.
Für mehr Details siehe \refk{sec:mit}.

\paragraph{Karma}
Testbibliothek, welche von Angular verwendet wird.\\
Mittlerweile ist diese Deprecated, ein Alternative wird von Angular gesucht.

Besteht aus:
\begin{itemize}
    \item karma
    \item karma-chrome-launcher
    \item karma-coverage
    \item karma-jasmine
    \item karma-jasmine-html-reporter
\end{itemize}

Alle Teile sind MIT lizenziert.
Für mehr Details siehe \refk{sec:mit}.


\subsection{Alle Abhängigkeiten}
Eine vollständige Liste aller Frontend Abhängigkeiten findet sich im Anhang unter \refk{sec:frontend---abhangigkeitsbaum}.
Diese sind wie folgt lizenziert:
\begin{itemize}
    \item 672x MIT
    \item 103x ISC
    \item 27x Apache-2.0
    \item 23x BSD-2-Clause
    \item 16x BSD-3-Clause
    \item 2x Unlicense
    \item 2x 0BSD
    \item 1x BlueOak-1.0.0
    \item 1x Python-2.0
    \item 1x CC-BY-4.0
    \item 1x CC-BY-3.0
    \item 1x BSD-3-Clause / GPL-2.0
    \item 1x CC0-1.0:
    \item 1x MIT / CC0-1.0
\end{itemize}



\section{Spring Boot}\label{sec:spring-boot}

Im Backend kommt die bewährte Technologie Spring Boot zu Einsatz.

Spring ermöglicht es Java schneller, leichter und sicherer zu programmieren.
Aufgrund seiner Geschwindigkeit, Einfachheit und Produktivität gilt Spring als am meisten geschätztes Java-Framework weltweit.
\cite{about-springboot}

Spring Boot ist Apache-2.0 Lizenziert.
Für mehr Details siehe \refk{sec:apache-2.0}.

\subsection{Direkte Abhängigkeiten}\label{subsec:direkte-abhanigkeiten}

\paragraph{Spring Boot Starters}
\begin{itemize}
    \item Thymeleaf
    \item Data MongoDB
    \item Web
    \item Web Services
    \item Test
    \item OAuth2 Resource Server
\end{itemize}
Sind Teil von Spring Boot und demnach ebenfalls Apache-2.0 Lizenziert.
Für mehr Details siehe \refk{sec:apache-2.0}.

\paragraph{lombok}
Erspart Boilerplate Code und ist MIT lizenziert.
Für mehr Details siehe \refk{sec:mit}.\\
Verwendet Komponenten, welch unter MIT und der BSD-3-Clause lizenziert sind.
Für mehr Details siehe \refk{sec:bsd-3-clause} und \refk{sec:mit}.

\paragraph{commons-lang3}
Java Utility welche Apache-2.0 Lizenziert ist.
Für mehr Details siehe \refk{sec:apache-2.0}.

\paragraph{OpenAPI}
\begin{itemize}
    \item JsonNullable Jackson Module
    \item swagger
    \item OpenAPI Generator
\end{itemize}
Ist Apache-2.0 Lizenziert.
Für mehr Details siehe \refk{sec:apache-2.0}.

\paragraph{jakarta.validation-api}
Bean Validation Modell.
Ist Apache-2.0 Lizenziert.
Für mehr Details siehe \refk{sec:apache-2.0}.

\paragraph{jakarta.annotation-api}
Jakarta Annotationen API.
Ist primär unter der EPL 2.0 (Eclipse Public License 2.0)
Sowie sekundär unter der GNU General Public License, Version 2 with the GNU Classpath Exception
Für mehr Details siehe \refk{sec:eclipse-public-license---v-2.0} und \refk{sec:gnu-gpl-v2-mit-gnu-classpath-exception}.

\paragraph{de.flapdoodle.embed.mongo}
Eine Embedded MongoDB für Test.
Ist Apache-2.0 Lizenziert.
Für mehr Details siehe \refk{sec:apache-2.0}.

\subsection{Alle Abhängigkeiten}
Ein vollständiger Abhängigkeitsbaum ist im Anhang unter \refk{sec:backend-abhangigkeitsbaum} aufgeführt.

\section{Keycloak}\label{sec:keycloak}

Keycloak dient als einfache Möglichkeit die Applikation abzusichern, ohne selbst Passwörter zu managen.
Zusätzlich wird \gls{Benutzer-Foederation}, starke Authentifizierung sowie eine feine Autorisierung bereitgestellt.
\cite{about-keycloak}

Keycloak ist Apache-2.0 Lizenziert.
Für mehr Details siehe \refk{sec:apache-2.0}.

\section{Postgres}\label{sec:postgres}

Postgres dient in diesem Projekt als Datenspeicher für Keycloak.

PostgreSQL ist ein leistungsfähiges, objektrelationales Open-Source-Datenbanksystem.
Es verwendet eine erweiterte \ac{SQL} Syntax und verfügt über zahlreiche Features,
die es ermöglichen, selbst die komplexesten Daten-Workloads sicher zu speichern und zu skalieren.
\cite{about-postgres}

Postgres steht unter der The PostgreSQL Licence.
Für mehr Details siehe \refk{sec:the-postgresql-licence}.


\section{MongoDB}\label{sec:mongodb}

MongoDB wird zum Speichern der Anträge sowie der Verknüpfung zu den Keycloak Accounts genutzt.
Im Gegensatz zu den klassischen, relationalen Datenbanken zählt MongoDB als Dokumenten basierte NoSQL Datenbank.
Dabei werden Information nicht streng auf einzelne Tabellen verteilt, sondern in Form von \ac{JSON} Objekten in Dokumenten abgelegt.

MongoDB steht unter der \acl{SSPL} zur Verfügung.
Für mehr Details siehe \refk{sec:server-side-public-license}.

\section{Docker}\label{sec:docker}

Docker bietet eine unkomplizierte Option, Container zu konfigurieren und zu teilen.
Dabei kann auf Datenbanken mit Basis Images zugegriffen werden, welche bei Bedarf
einfach in einem Dockerfile angepasst werden können.

Docker selbst ist nicht Open Source, jedoch für die meisten kostenlos verfügbar.
Davon ausgeschlossen sind lediglich Firmen mit mehr als 250 Mitarbeitern oder mindestens 10 Millionen Dollar Einkommen.

\section{Docker Compose}\label{sec:docker-compose}

Docker Compose ist eine Erweiterung von Docker.
Diese ermöglicht es direkt mehrere Container in einer Datei zu managen.
Ein weiterer Vorteil stellt die einfache Konfiguration von Docker-Secrets dar, wodurch \ua Passwörter sicher übergeben werden können.

Docker Compose ist Apache-2.0 Lizenziert.
Für mehr Details siehe \refk{sec:apache-2.0}.

