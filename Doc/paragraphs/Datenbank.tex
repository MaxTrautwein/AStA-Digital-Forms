\chapter{Datenbank}
Da in diesem Projekt eine NoSQL Datenbank verwendet wird, gibt es kein Datenbanklayout im klassischen Sinne.
Regeln für die Strukturierung von Daten wie Normalformen bei relationalen Datenbanken gibt es bei MongoDB nicht.

Dennoch gibt es einige Punkte, die zu beachten sind:
\begin{itemize}
    \item Limit von 16 \ac{MB} Pro Dokument
    \item Einbetten ist bei NoSQL Datenbanken zu bevorzugen, jedoch nicht immer passend
    \item Arrays sollten nicht unendlich wachsen können
\end{itemize}

Es folgt die vorläufige Struktur für die verschiedenen Dokumente.

\section{Userdata / Autofill}
In diesem Dokument werden Daten vom Nutzer gespeichert, um diese für die Autofill Funktion zu verwenden.
\begin{lstlisting}[label={lst:lstlistingusers}]
    // userData Document
    {
      "_id": {"$oid": string},
      "IBAN": string,
      "_class": "de.PSWTM.DigitalForms.model.UserData",
      "adress": string,
      "creditInstitute": string,
      "email": string,
      "firstName": string,
      "name": string,
      "userId": string
    }


\end{lstlisting}

\section{Template \ac{PDF}}
Verbindet Templates mit deren \ac{PDF} Vorlagen.
\begin{lstlisting}[label={lst:lstlistingauto}]
    // templatePDF Document
    {
      "_id": {"$oid": string},
      "_class": "de.PSWTM.DigitalForms.Model.TemplatePDF",
      "formId": string,
      "templatePdf": string
    }

\end{lstlisting}

\section{Template Gruppe}
Verlinkt Formularvorlagen miteinander.
Erlaubt die Gruppierung eines Antrags mit ein oder mehreren dazugehörigen Abrechnungen.
\begin{lstlisting}[label={lst:templateGroup}]
    // templateGroup Document
    {
      "_id": {"$oid": string},
      "_class": "de.PSWTM.DigitalForms.model.TemplateGroup",
      "antragId": string,
      "description": string,
      "reasons": [string, string],
      "rechnungen": [string, string],
      "titel": string
    }

\end{lstlisting}


\section{Favoriten}
Speichert die Favoriten eines Nutzers.
\begin{lstlisting}[label={lst:favourite}]
    // favourite Document
    {
      "_id": {"$oid": string},
      "_class": "de.PSWTM.DigitalForms.model.Favourite",
      "formId": string,
      "owner": string
    }

\end{lstlisting}


\section{Form}\label{sec:form}
Das Form-Dokument definiert die Struktur, in welcher Formulare gespeichert werden.
Hierbei werden verschiedene \gls{enum}s verwendet, welche im Folgenden näher beschrieben werden.


'TypeEnum' beschreibt die verschiedenen Feldtypen, welche dargestellt werden sollen.
\begin{lstlisting}[label={lst:TypeEnum}]
    public enum TypeEnum {
        text,
        address,
        iban,
        date,
        money,
        TextMultiLine,
        bool
    }
\end{lstlisting}


'CategoryEnum' ermöglicht die Unterteilung eines Formulars in fest definierte Kategorien
\begin{lstlisting}[label={lst:CategoryEnum}]
    public enum CategoryEnum {
        Antrag,
        Abrechnung
    }
\end{lstlisting}


'RequiredEnum' teilt Anhänge in einen von 3 Typen ein.
Dabei steht die Option zwischen, immer notwendig, eventuell notwendig, je nach bedingung notwendig.
\begin{lstlisting}[label={lst:RequiredEnum}]
    public enum RequiredEnum {
        always,
        user,
        conditional
    }
\end{lstlisting}


Das Form-Dokument selbst ist im Folgenden beschrieben.
Der Boolean "Template" dient hierbei zur Unterscheidung zwischen
Form-Dokumenten, welche beschreiben, wie das Formular dem Nutzer
dargestellt werden soll, sowie Form-Dokumenten, welche Daten enthalten.

Sollte es sich nicht um eine Vorlage zum Ausfüllen handeln, so werden nur die notwendigen Informationen erfasst.
\begin{lstlisting}[label={lst:lstlistingdoc}]
    {
      "_id": {"$oid": string},
      "_class": "de.PSWTM.DigitalForms.model.Form",
      "template": bool,
      "titel": string,
      "category": CategoryEnum,
      "description": string,
      "form": [
        {
          "order": integer,
          "section": string,
          "items": [
            {
              "_id": string,
              "description": string,
              "type": TypeEnum,
              "value": string
            },
            {
              "_id": string,
              "description": string,
              "type": TypeEnum,
              "value": string
            }
          ]
        },
        {
          "order": integer,
          "section": string,
          "items": [
            ...
          ]
        },
        {
          ...
        }
      ],
      "attachments": [
        {
          "_id": string,
          "description": string,
          "required": RequiredEnum,
          "conditionRef": string,
          "conditionRefVal": string,
          "help": string
        }
      ]
    }

\end{lstlisting}