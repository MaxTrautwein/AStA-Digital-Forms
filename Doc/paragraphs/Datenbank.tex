\chapter{Datenbank}
Da wir in unserem Projekt eine NoSQL Datenbank verwenden gibt es für un kein Datenbanklayout im klassischen Sinnen.
Regeln für die Strukturierung von Daten wie Normalformen bei Relationalen Datenbanken gibt es bei MongoDB nicht.

Dennoch gibt es einige Punkte die zu beachten sind:
\begin{itemize}
    \item Limit von 16\ac{MB} Pro Dokument
    \item Einbetten ist bei NoSQL Datenbanken zu bevorzugen, jedoch nicht immer passend
    \item Arrays sollten nicht unendlich wachsen können
\end{itemize}

Im Folgenden zeigen wir unsere vorläufige Structure für die verschiedenen Dokumente.
Der genaue inhalt wird angefügt, sobald diese bekannt werden.

\section{User}

\begin{lstlisting}[label={lst:lstlistingusers}]
    // Users Document
    {
    id: string
    settings:
        {
            ...
        }
    }

\end{lstlisting}

\section{Autofill}
\begin{lstlisting}[label={lst:lstlistingauto}]
    // Autofill Document
    {
    Userid: ref
    pairs:
        [
            {
                key:value
            },
            {
            ...
            }
        ]
    }

\end{lstlisting}

\section{Document}
\begin{lstlisting}[label={lst:lstlistingdoc}]
    // Document Document
    {
    Userid: ref
    status: string
    data:   [
                {
                    ref: string
                    val: object
                },
                {
                ...
                }
            ]
    }

\end{lstlisting}