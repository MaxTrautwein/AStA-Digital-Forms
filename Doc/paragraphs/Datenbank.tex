\chapter{Datenbank}
Da in diesem Projekt eine NoSQL Datenbank verwendet wird, gibt es kein Datenbanklayout im klassischen Sinne.
Regeln für die Strukturierung von Daten wie Normalformen bei Relationalen Datenbanken gibt es bei MongoDB nicht.

Dennoch gibt es einige Punkte die zu beachten sind:
\begin{itemize}
    \item Limit von 16\ac{MB} Pro Dokument
    \item Einbetten ist bei NoSQL Datenbanken zu bevorzugen, jedoch nicht immer passend
    \item Arrays sollten nicht unendlich wachsen können
\end{itemize}

Es folgt die vorläufige Struktur für die verschiedenen Dokumente.
Der genaue Inhalt wird angefügt, sobald dieser bekannt wird.

\section{User}

\begin{lstlisting}[label={lst:lstlistingusers}]
    // Users Document
    {
    id: string
    settings:
        {
            ...
        }
    }

\end{lstlisting}

\section{Autofill}
\begin{lstlisting}[label={lst:lstlistingauto}]
    // Autofill Document
    {
    Userid: ref
    pairs:
        [
            {
                key:value
            },
            {
            ...
            }
        ]
    }

\end{lstlisting}

\section{Document}
\begin{lstlisting}[label={lst:lstlistingdoc}]
    // Document Document
    {
    Userid: ref
    status: string
    data:   [
                {
                    ref: string
                    val: object
                },
                {
                ...
                }
            ]
    }

\end{lstlisting}